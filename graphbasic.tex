\special{pdf: out 3 << /Title 
(Grundlagen) 
/Dest [ @thispage /FitH @ypos ] >>}
\subsection{Grundlagen}
Die Theorie der Graphen ist heute zu einem unverzichtbaren Bestandteil
der Informatik geworden. Viele Probleme, wie z.B.~das Verlegen von
Leiterbahnen auf einer Platine, die Modellierung von Netzwerken oder
die Lösung von Routingproblemen in Verkehrsnetzen benutzen Graphen
oder Algorithmen, die Graphen als Datenstruktur verwenden. Auch schon
bekannte Datenstrukturen wie Listen und Bäume können als Graphen
aufgefasst werden. All dies gibt einen Anhaltspunkt, dass die
Graphentheorie eine sehr zentrale Rolle für die Informatik spielt und
vielfältige Anwendungen hat. In diesem Kontext ist es wichtig zu
bemerken, dass der Begriff des Graphen in der Informatik \emph{nicht}
im Sinne von Graph einer Funktion gebraucht wird, sondern wie folgt
definiert ist:

\begin{definition}
Ein \dindex{gerichteter Graph}\index{Graph!gerichtet} $G = (V,E)$ ist
ein Paar, das aus einer Menge von \dindex{Knoten} $V$ und einer Menge
von \dindex{Kanten} $E \subseteq V \times V$
(\dindex{Kantenrelation}\index{Relation!Kante}) besteht. Eine Kante
$k = (u,v)$ aus $E$ kann als Verbindung zwischen den Knoten $u,v \in
V$ aufgefasst werden. Aus diesem Grund nennt man $u$
auch \dindex{Startknoten}\index{Knoten!Start} und
$v$ \dindex{Endknoten}\index{Knoten!End}. Zwei Knoten, die durch eine
Kante verbunden sind, heißen auch \dindex{benachbart}
oder \dindex{adjazent}.

Ein Graph $H = (V', E')$ mit $V' \subseteq V$ und $E' \subset E$ heißt
\dindex{Untergraph} von $G$.
\end{definition}

Ein Graph $(V,E)$ heißt \dindex{endlich} \gdw die Menge der Knoten $V$
endlich ist. Obwohl man natürlich auch unendliche Graphen betrachten
kann, werden wir uns in diesem Abschnitt nur mit endlichen Graphen
beschäftigen, da diese für den Informatiker von großem Nutzen sind.

\bigskip

Da wir eine Kante $(u,v)$ als Verbindung zwischen den Knoten $u$ und
$v$ interpretieren können, bietet es sich an, Graphen durch Diagramme
darzustellen. Dabei wird die Kante $(u,v)$ durch einen Pfeil von $u$
nach $v$ dargestellt. Drei Beispiele für eine bildliche Darstellung
von gerichteten Graphen finden sich in Abbildung \ref{gGraphen}.

\special{pdf: out 3 << /Title 
(Einige Eigenschaften von Graphen) 
/Dest [ @thispage /FitH @ypos ] >>}
\subsection{Einige Eigenschaften von Graphen}
Der Graph in Abbildung \ref{gGraphen}(c) hat eine besondere
Eigenschaft, denn offensichtlich kann man die Knotenmenge $V_{1c}
= \set{0,1,2,3,4,5,6,7,8}$ in zwei disjunkte Teilmengen $V_{1c}^l
= \set{0,1,2,3}$ und $V_{1c}^r = \set{4,5,6,7,8}$ so aufteilen, dass
keine Kante zwischen zwei Knoten aus $V_{1c}^l$ oder $V_{1c}^r$
verläuft.

\begin{definition}
Ein Graph $G = (V,E)$ heißt \dindex{bipartit}, wenn gilt:
\begin{enumerate}
%
\item Es gibt zwei Teilmengen $V^l$ und $V^r$ von $V$ mit $V =
V^l \cup V^r$, $V^l \cap V^r = \emptyset$ und 
%
\item für jede Kante $(u,v) \in E$ gilt $u \in V^l$ und $v \in V^r$.
%
\end{enumerate}
\end{definition}

Bipartite Graphen haben viele Anwendungen, weil man jede binäre
Relation $R \subseteq A \times B$ mit $A \cap B = \emptyset$ ganz
natürlich als bipartiten Graph auffassen kann, dessen Kanten von
Knoten aus $A$ zu Knoten aus $B$ laufen.

\begin{example}
Gegeben sei ein bipartiter Graph $G = (V,E)$ mit $V = V^F \cup V^M$
und $V^F \cap V^M = \emptyset$. Die Knoten aus $V^F$ symbolisieren
Frauen und $V^M$ symbolisiert eine Menge von Männern. Kann sich eine
Frau vorstellen einen Mann zu heiraten, so wird der entsprechende
Knoten aus $V^F$ mit dem passenden Knoten aus $V^M$ durch eine Kante
verbunden.  Eine \dindex{Heirat} ist nun eine Kantenmenge $H \subseteq
E$, so dass keine zwei Kanten aus $H$ einen gemeinsamen Knoten
besitzen. Das \dindex{Heiratsproblem} ist nun die Aufgabe für $G$ eine
Heirat $H$ zu finden, so dass alle Frauen heiraten können, d.h.~es ist
das folgende Problem zu lösen:

\goodbreak
\prob{MARRIAGE}{%
Bipartiter Graph $G = (V,E)$ mit $V = V^F \cup V^M$ und $V^F \cap V^M
= \emptyset$}{%
Eine Heirat $H$ mit $\cnt H = \cnt V^F$
}

Im Beispielgraphen \ref{gGraphen}(c) gibt es keine Lösung für das
Heiratsproblem, denn für die Knoten ($\triangleq$ Kandidatinnen) $2$ und
$3$ existieren nicht ausreichend viele Partner, d.h.~keine Heirat in
diesem Graphen enthält zwei Kanten die sowohl $2$ als auch $3$ als
Startknoten haben.

\medskip

Obwohl dieses Beispiel auf den ersten Blick nur von untergeordneter
Bedeutung erscheint, kann man es auf eine Vielfalt von Anwendungen
übertragen. Immer wenn die Elemente zweier disjunkter Mengen durch
eine Beziehung verbunden sind, kann man dies als bipartiten Graphen
auffassen. Sollen nun die Bedürfnisse der einen Menge völlig
befriedigt werden, so ist dies wieder ein Heiratsproblem. Beispiele
mit mehr praktischem Bezug finden sich u.a.~bei Beziehungen zwischen
Käufern und Anbietern.
\end{example}

\begin{figure}[h]
\centering
\subfloat[Ein gerichteter Graph mit $5$
Knoten]{\includegraphics[scale=1.1]{graphex1}}
\hspace*{2em}
\subfloat[Ein planarer gerichteter
Graph mit $5$ Knoten]{\includegraphics[scale=1.1]{graphex3}}
\hspace*{2em}
\subfloat[Ein gerichteter bipartiter 
Graph]{\includegraphics[scale=1.1]{graphex2}}
\caption{Beispiele für gerichtete Graphen}
\label{gGraphen}
\end{figure}
Oft beschränken wir uns auch auf eine Unterklasse von Graphen, bei
denen die Kanten keine "`Richtung"' haben (siehe
Abbildung \ref{ugGraphen}) und einfach durch eine Verbindungslinie
symbolisiert werden können:

\begin{definition}
Sei $G=(V,E)$ ein Graph. Ist die Kantenrelation
$E$ \dindex{symmetrisch}, d.h.~gibt es zu jeder Kante $(u,v) \in E$
auch eine Kante $(v,u) \in E$ (siehe auch Abschnitt \ref{PropRel}),
dann bezeichnen wir $G$ als \dindex{ungerichteten
Graphen}\index{Graph!ungerichtet} oder kurz als \dindex{Graph}.
\end{definition}

Es ist praktisch, die Kanten $(u,v)$ und $(v,u)$ eines ungerichteten
Graphen als Menge $\set{u,v}$ mit zwei Elementen aufzufassen. Diese
Vorgehensweise führt zu einem kleinen technischen Problem. Eine Kante
$(u,u)$ mit gleichem Start- und Endknoten nennen wir, entsprechend der
intuitiven Darstellung eines Graphens als Diagramm, \dindex{Schleife}.
Wandelt man nun solch eine Kante in eine Menge um, so würde nur eine
einelementige Menge entstehen. Aus diesem Grund legen wir fest, dass
ungerichtete Graphen \dindex{schleifenfrei} sind.

\begin{definition}
Der (ungerichtete) Graph $K = (V,E)$ heißt \dindex{vollständig}, wenn
für alle $u,v \in V$ mit $u \neq v$ auch $(u,v) \in E$ gilt,
d.h.~jeder Knoten des Graphen ist mit allen anderen Knoten
verbunden. Ein Graph $O=(V,\emptyset)$ ohne Kanten wird
als \dindex{Nullgraph}\index{Graph!Null} bezeichnet.
\end{definition}
Mit dieser Definition ergibt sich, dass die Graphen in
Abbildung \ref{ugGraphen}(a) und Abbildung \ref{ugGraphen}(b)
vollständig sind. Der Nullgraph $(V,\emptyset)$ ist Untergraph jedes
beliebigen Graphen $(V,E)$. Diese Definitionen lassen sich natürlich
auch analog auf gerichtete Graphen übertragen.

\begin{figure}
\centering
\subfloat[Vollständiger ungerichteter Graph 
$K_{16}$]{\includegraphics[scale=0.55]{kclique}}
\hfill
\subfloat[Vollständiger ungerichteter Graph
$K_{20}$]{\includegraphics[scale=0.55]{kclique3}}
\subfloat[Zufälliger Graph mit $32$ Knoten]{\includegraphics[scale=0.55]{random}}
\hfill
\subfloat[Regulärer Graph mit Grad $3$]{\includegraphics[scale=0.55]{moebius}}
\caption{Beispiele für ungerichtete Graphen}
\label{ugGraphen}
\end{figure}

\begin{definition}
Sei $G = (V,E)$ ein gerichteter Graph und $v \in V$ ein beliebiger
Knoten. Der \dindex{Ausgrad} von $v$ (kurz:
$\mathrm{outdeg}(v)$\index{outdeg=$\mathrm{outdeg}(v)$}) ist dann 
die Anzahl der Kanten in $G$, die $v$ als Startknoten haben. Analog 
ist der \dindex{Ingrad} von $v$ 
(kurz: $\mathrm{indeg}(v)$\index{indeg=$\mathrm{indeg}(v)$}) die 
Anzahl der Kanten in $G$, die $v$ als Endknoten haben.

Bei ungerichteten Graphen gilt für jeden Knoten
$\mathrm{outdeg}(v)=\mathrm{indeg}(v)$. Aus diesem Grund schreiben wir
kurz $\mathrm{deg}(v)$ und bezeichnen dies als \emph{Grad von
$v$}\index{Grad}. 
Ein Graph $G$ heißt \dindex{regulär} \gdw alle
Knoten von $G$ den gleichen Grad haben.
\end{definition}
Die Diagramme der Graphen in den Abbildungen \ref{gGraphen}
und \ref{ugGraphen} haben die Eigenschaft, dass sich einige Kanten
schneiden. Es stellt sich die Frage, ob man diese Diagramme auch so
zeichnen kann, dass keine Überschneidungen auftreten. Diese
Eigenschaft von Graphen wollen wir durch die folgende Definition
festhalten:
\begin{definition}
Ein Graph $G$ heißt \dindex{planar}, wenn sich sein Diagramm ohne
Überschneidungen zeichnen läßt.
\end{definition}

\begin{example}
Der Graph in Abbildung \ref{gGraphen}(a) ist, wie man leicht
nachprüfen kann, planar, da die Diagramme aus
Abbildung \ref{gGraphen}(a) und \ref{gGraphen}(b) den gleichen Graphen
repräsentieren.
\end{example}
Auch planare Graphen haben eine anschauliche Bedeutung. Der Schaltplan
einer elektronischen Schaltung kann als Graph aufgefasst werden. Die
Knoten entsprechen den Stellen an denen die Bauteile aufgelötet werden
müssen, und die Kanten entsprechen den Leiterbahnen auf der
Platine. In diesem Zusammenhang bedeutet planar, ob man die
Leiterbahnen kreuzungsfrei verlegen kann, d.h.~ob es möglich ist, eine
Platine zu fertigen, die mit einer Kupferschicht auskommt. In der
Praxis kommen oft Platinen mit mehreren Schichten zum Einsatz
("`Multilayer-Platine"'). Ein Grund dafür kann sein, dass der
"`Schaltungsgraph"' nicht planar war und deshalb mehrere Schichten
benötigt werden. Da Platinen mit mehreren Schichten in der Fertigung
deutlich teurer sind als solche mit einer Schicht, hat die
Planaritätseigenschaft von Graphen somit auch unmittelbare finanzielle
Auswirkungen.

\special{pdf: out 3 << /Title 
(Wege, Kreise, Wälder und Bäume) 
/Dest [ @thispage /FitH @ypos ] >>}
\subsection{Wege, Kreise, Wälder und Bäume}

\begin{definition}
\label{GraphWeg}
Sei $G=(V,E)$ ein Graph und $u,v \in V$. Eine Folge von Knoten
$\enu{u}{0}{l} \in V$ mit $u = u_0$, $v = u_l$ und $(u_i,u_{i+1}) \in
E$ für $0 \le i \le l - 1$ heißt \emph{Weg von $u$ nach $v$ der Länge
$l$}\index{Weg}. Der Knoten $u$
wird \dindex{Startknoten}\index{Knoten!Start} und $v$
wird \dindex{Endknoten}\index{Knoten!End} des Wegs genannt.

Ein Weg, bei dem Start- und Endknoten gleich sind,
heißt \dindex{geschlossener Weg}\index{Weg!geschlossen}. Ein
geschlossener Weg, bei dem kein Knoten außer dem Startknoten mehrfach
enthalten ist, wird \dindex{Kreis} genannt.
\end{definition}
Mit Definition \ref{GraphWeg} wird klar, dass der Graph in
Abbildung \ref{gGraphen}(a) den Kreis $1,2,3,\dots,\allowbreak 5,1$ mit
Startknoten $1$ hat.

\begin{definition}
Sei $G=(V,E)$ ein Graph. Zwei Knoten $u,v \in V$
heißen \dindex{zusammenhängend}, wenn es einen Weg von $u$ nach $v$
gibt. Der Graph $G$ heißt \dindex{zusammenhängend}, wenn jeder Knoten
von $G$ mit jedem anderen Knoten von $G$ zusammenhängt. 

Sei $G'$ ein zusammenhängender Untergraph von $G$ mit einer besonderen
Eigenschaft: Nimmt man einen weiteren Knoten von $G$ zu $G'$ hinzu,
dann ist der neu entstandene Graph nicht mehr zusammenhängend, d.h.~es
gibt keinen Weg zu diesem neu hinzugekommenen Knoten. Solch einen
Untergraph nennt man \dindex{Zusammenhangskomponente}.
\end{definition}
Offensichtlich sind die Graphen in den
Abbildungen \ref{gGraphen}(a), \ref{ugGraphen}(a), \ref{ugGraphen}(b)
und \ref{ugGraphen}(d) zusammenhängend und haben genau eine
Zusammenhangskomponente. Man kann sich sogar leicht überlegen, dass
die Eigenschaft "`$u$ hängt mit $v$"' zusammen
eine \emph{Äquivalenzrelation} (siehe Abschnitt \ref{PropRel})
darstellt.

Mit Hilfe der Definition des geschlossenen Wegs lässt sich nun der
Begriff der Bäume definieren, die eine sehr wichtige Unterklasse der
Graphen darstellen.
\goodbreak
\begin{definition}
Ein Graph $G$ heißt
\begin{itemize}
%
\item \dindex{Wald}, wenn es keinen geschlossenen Weg mit Länge $\ge
1$ in $G$ gibt und 
%
\item \dindex{Baum}, wenn $G$ ein zusammenhängender Wald ist,
d.h.~wenn er nur genau eine Zusammenhangskomponente hat.
%
\end{itemize}
\end{definition}

\begin{figure}
\begin{center}
\includegraphics[scale=0.5]{wald.eps}
\end{center}
\caption{Ein Wald mit zwei Bäumen}
\label{wald}
\end{figure}

\special{pdf: out 3 << /Title 
(Die Repräsentation von Graphen und einige Algorithmen) 
/Dest [ @thispage /FitH @ypos ] >>}
\subsection{Die Repräsentation von Graphen und einige Algorithmen}
Nachdem Graphen eine große Bedeutung sowohl in der praktischen als
auch in der theoretischen Informatik erlangt haben, stellt sich noch
die Frage, wie man Graphen effizient als Datenstruktur in einem
Computer ablegt. Dabei soll es möglich sein, Graphen effizient zu
speichern und zu manipulieren. 

Die erste Idee, Graphen als dynamische Datenstrukturen zu
repräsentieren, scheitert an dem relativ ineffizienten Zugriff auf die
Knoten und Kanten bei dieser Art der Darstellung. Sie ist nur von
Vorteil, wenn ein Graph nur sehr wenige Kanten enthält. Die folgende
Methode der Speicherung von Graphen hat sich als effizient erwiesen
und ermöglicht auch die leichte Manipulation des Graphens:
\begin{definition}
Sei $G=(V,E)$ ein gerichteter Graph mit $V = \set{\enu{v}{1}{n}}$. Wir
definieren eine $n \times n$ Matrix $A_G =(a_{i,j})_{1 \le i,j, \le
n}$ durch
\begin{displaymath}
a_{i,j} =
\left\{
\begin{array}{ll}
1,& \text{ falls $(v_i, v_j) \in E$}\\
0,& \text{ sonst}
\end{array}
\right.
\end{displaymath}
Die so definierte Matrix $A_G$ mit Einträgen aus der Menge $\set{0,1}$
heißt \dindex{Adjazenzmatrix} von $G$.
\end{definition}

\begin{example}
Für den gerichteten Graphen aus Abbildung \ref{gGraphen}(a) ergibt sich die
folgende Adjazenzmatrix:
\begin{displaymath}
A_{G_5} =
\left(
\begin{array}{ccccc}
0 & 1 & 0 & 0 & 0\\ 
0 & 0 & 1 & 0 & 0\\
0 & 0 & 0 & 1 & 0\\
0 & 0 & 0 & 0 & 1\\
1 & 0 & 0 & 0 & 0
\end{array}
\right)
\end{displaymath}
Die Adjazenzmatrix eines ungerichteten Graphen erkennt man daran, dass
sie spiegelsymmetrisch zu Diagonale von links oben nach rechts unten
ist (die Kantenrelation\index{Kantenrelation} ist symmetrisch) und
dass die Diagonale aus $0$-Einträgen besteht (der Graphen hat keine
Schleifen). Für den vollständigen Graphen $K_{16}$ aus
Abbildung \ref{ugGraphen}(a) ergibt sich offensichtlich die folgende
Adjazenzmatrix:
\begin{displaymath}
A_{K_{16}} =
\left(
\begin{array}{cccccccccccccccc}
0 & 1 & 1 & 1 & 1 & 1 & 1 & 1 & 1 & 1 & 1 & 1 & 1 & 1 & 1 & 1\\ 
1 & 0 & 1 & 1 & 1 & 1 & 1 & 1 & 1 & 1 & 1 & 1 & 1 & 1 & 1 & 1\\ 
1 & 1 & 0 & 1 & 1 & 1 & 1 & 1 & 1 & 1 & 1 & 1 & 1 & 1 & 1 & 1\\ 
1 & 1 & 1 & 0 & 1 & 1 & 1 & 1 & 1 & 1 & 1 & 1 & 1 & 1 & 1 & 1\\ 
1 & 1 & 1 & 1 & 0 & 1 & 1 & 1 & 1 & 1 & 1 & 1 & 1 & 1 & 1 & 1\\ 
1 & 1 & 1 & 1 & 1 & 0 & 1 & 1 & 1 & 1 & 1 & 1 & 1 & 1 & 1 & 1\\ 
1 & 1 & 1 & 1 & 1 & 1 & 0 & 1 & 1 & 1 & 1 & 1 & 1 & 1 & 1 & 1\\ 
1 & 1 & 1 & 1 & 1 & 1 & 1 & 0 & 1 & 1 & 1 & 1 & 1 & 1 & 1 & 1\\ 
1 & 1 & 1 & 1 & 1 & 1 & 1 & 1 & 0 & 1 & 1 & 1 & 1 & 1 & 1 & 1\\ 
1 & 1 & 1 & 1 & 1 & 1 & 1 & 1 & 1 & 0 & 1 & 1 & 1 & 1 & 1 & 1\\ 
1 & 1 & 1 & 1 & 1 & 1 & 1 & 1 & 1 & 1 & 0 & 1 & 1 & 1 & 1 & 1\\ 
1 & 1 & 1 & 1 & 1 & 1 & 1 & 1 & 1 & 1 & 1 & 0 & 1 & 1 & 1 & 1\\ 
1 & 1 & 1 & 1 & 1 & 1 & 1 & 1 & 1 & 1 & 1 & 1 & 0 & 1 & 1 & 1\\ 
1 & 1 & 1 & 1 & 1 & 1 & 1 & 1 & 1 & 1 & 1 & 1 & 1 & 0 & 1 & 1\\ 
1 & 1 & 1 & 1 & 1 & 1 & 1 & 1 & 1 & 1 & 1 & 1 & 1 & 1 & 0 & 1\\ 
1 & 1 & 1 & 1 & 1 & 1 & 1 & 1 & 1 & 1 & 1 & 1 & 1 & 1 & 1 & 0
\end{array}
\right)
\end{displaymath}
\end{example}
Mit Hilfe der Adjazenzmatrix und Algorithmus \ref{Reach} kann man
leicht berechnen, ob ein Weg von einem Knoten $u$ zu einem Knoten $v$
existiert. Mit einer ganz ähnlichen Idee kann man auch leicht die
Anzahl der Zusammenhangskomponenten berechnen (siehe
Algorithmus \ref{Kompo}). Dieser Algorithmus markiert die Knoten der
einzelnen Zusammenhangskomponenten auch mit unterschiedlichen
"`Farben"', die hier durch Zahlen repräsentiert werden.

\restylealgo{ruled}
\begin{algorithm}
\caption{Erreichbarkeit in Graphen}
\label{Reach}
\KwData{Ein Graph $G=(V,E)$ und zwei Knoten $u,v \in V$}
\KwResult{\texttt{true} wenn es einen Weg von $u$ nach $v$
gibt, \texttt{false} sonst}
\BlankLine

markiert = \texttt{true}\;
markiere Startknoten $u \in V$\;

\BlankLine

\While{(markiert)}{

markiert = \texttt{false}\;

\For{(alle markierten Knoten $w \in V$)}{

\If{($w \in V$ ist adjazent zu einem unmarkierten Knoten $w' \in V$)}{
markiere Knoten $w'$\;
markiert = \texttt{true}\;
}

}

}

\eIf{($v$ ist markiert)}{
\Return \texttt{true}\;
}{
\Return \texttt{false}\;
}

\printsemicolon
\end{algorithm}

\restylealgo{ruled}
\begin{algorithm}
\caption{Zusammenhangskomponenten}
\label{Kompo}
\KwData{Ein Graph $G=(V,E)$}
\KwResult{Anzahl der Zusammenhangskomponenten von $G$}
\BlankLine
\printsemicolon

kFarb = 0\;

\BlankLine

\While{(es gibt einen unmarkierten Knoten $u \in V$)}{

kFarb++\;
markiere $u \in V$ mit kFarb\;
\BlankLine
\printsemicolon

markiert=\texttt{true}\;

\While{(markiert)}{

markiert=\texttt{false}\;
\BlankLine
\printsemicolon

\For{(alle mit kFarb markierten Knoten $v \in V$)}{
\If{($v \in V$ ist adjazent zu einem unmarkierten Knoten $v' \in V$)}{
markiere Knoten $v' \in V$ mit kFarb\;
markiert=\texttt{true}\;
}
}
}
}
\Return kFarb.
\end{algorithm}

\begin{definition}
Sei $G = (V,E)$ ein ungerichteter Graph. Eine Funktion der Form
$f \colon V \rightarrow \set{\range{1}{k}}$
heißt \dindex{Farbung=$k$-Färbung} des Graphen $G$. Anschaulich
ordnet die Funktion $f$ jedem Knoten eine von $k$ verschiedenen Farben
zu, die hier durch die Zahlen $\range{1}{k}$ symbolisiert werden. Eine
Färbung heißt \emph{verträglich}\index{Färbung!verträglich}, wenn für
alle Kanten $(u,v) \in E$ gilt, dass $f(u) \neq f(v)$, d.h.~zwei
adjazente Knoten werden nie mit der gleichen Farbe markiert.
\end{definition}

\ifkomplex
Für viele Probleme der Graphentheorie gibt es mit hoher
Wahrscheinlichkeit keinen effizienten Algorithmus. Mehr Informationen
zu diesem Thema finden sich in Abschnitt \ref{KomplexSect}.
\else
Auch das Färbbarkeitsproblem spielt in der Praxis der Informatik eine
wichtige Rolle. Ein Beispiel dafür ist die Planung eines
Mobilfunknetzes. Dabei werden die Basisstationen eines Mobilfunknetzes
als Knoten eines Graphen repräsentiert. Zwei Knoten werden mit einer
Kante verbunden, wenn Sie geographisch so verteilt sind, dass sie sich
beim Senden auf der gleichen Frequenz gegenseitig stören
können. Existiert eine verträgliche $k$-Färbung für diesen Graphen, so
ist es möglich, ein störungsfreies Mobilfunktnetz mit $k$
verschiedenen Funkfrequenzen aufzubauen. Dabei entsprechen die Farben
den verfügbaren Frequenzen. Bei der Planung eines solchen
Mobilfunknetzes ist also das folgende Problem zu lösen:

\goodbreak
\dprob{$k\mathrm{COL}$}{
Ein ungerichteter Graph $G$ und eine Zahl $k \in \N$.
}
{
Gibt es eine verträgliche Färbung von $G$ mit $k$ Farben?
}
Dieses Problem gehört zu einer (sehr großen) Klasse von (praktisch
relevanten) Problemen, für die bis heute keine effizienten Algorithmen
bekannt sind (Stichwort: \NP-Vollständigkeit). Vielfältige Ergebnisse 
der Theoretischen Informatik zeigen sogar, dass man nicht hoffen darf, 
dass ein schneller Algorithmus zur Lösung des Färbbarkeitsproblems existiert.
\fi
