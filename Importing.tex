\section{Importing}


In order to use the definitions from the previous chapter we are going to compile the file 
\texttt{Basics.v} and import the compiles version in the current file \texttt{Induction.v}.
(In case the book \cite{PACGGHSY} is used and downloaded as an archive, the following steps can be skipped.)   

\subsection{Building Coq Libraries}

First of all, a Coq project, called {\itshape \_CoqProject}, is created.  
It is going to map the current directory ``\texttt{.}'' to the directory \texttt{lf} wherein the source files are kept.
Using the CoqIDE, Proof General or executing Coq via the command line the compiling and building process differs.
The Proof General using reader is refereed to the literature \cite[Section, Induction, Proof by Induction]{PACGGHSY}.\\

Create a file called \texttt{\_coqProject} within the working directory which contains the file \texttt{Basics.v}.
Within this file add the line 

\begin{lstlisting}[caption = \lstinline!naming a library!]
-Q .LF
\end{lstlisting}

which declares the library's name \texttt{LF.Basics}. 
To make the executable \texttt{Basics.vo}-file out of the \texttt{Basics.v} multiple options exist.

Using the CoqIDE, the file \texttt{Basics.v} should be opened, and the button in the compile menu  ``{\itshape Compile Buffer}'' is clicked.
If using the command line the file \texttt{Basics.vo} can be build or compiled using the Coq  make-file utility.
By the way, the {\itshape Coq-makefile utility } is installed together with Coq.
Type the following console-command:
\begin{lstlisting}[language=bash, caption = \lstinline!coq-makefile!, label = lst:coq-makefile]
coq-makefile -f _CoqProject*.v -o Makefile
\end{lstlisting}

In addition run this command, whenever files to the working directory \texttt{lf} were added or removed.
To compile \texttt{Basics.v}, call either on of the following commands 
\begin{lstlisting}[language=bash, caption = \lstinline!make!, label = lst:make]
make (* builds the complete directory *)
make Basics.vo (* builds Basics.vo *)
\end{lstlisting}
Note that, \lstinline!make! compiles and calculates dependencies automatically. 
The Coq-compiler, which is called {\itshape coqc} can be called by: 
\begin{lstlisting}[language=bash, caption =\lstinline!coqc!, label = lst:coqc]
coqc -Q.LF Basics.v
\end{lstlisting}
But remember always to prefer \texttt{make} over compiling, because of the dependencies.\\
 
To include the compiled chapter \texttt{Basics.vo} into the source code of the chapter \texttt{Induction.v} it is written
\begin{lstlisting}[language = bash, caption = \lstinline!Require Export!, label = lst:RequireExport]
From LF Require Export Basics.
\end{lstlisting}
into the first line of the \texttt{Induction.v}-file.

 

\subsection{Potential Troubles}

If an error arises which complains about missing identifiers in the file the "load path" for Coq might not be set up correctly.
Checking the loaded path by the command
\begin{lstlisting}[language = bash, caption = checking the loaded path, label = lst:PrintLoadPath]
Print LoadPath.  
\end{lstlisting} 
might be helpful. The error message
\begin{lstlisting}[language = bash, caption = possible  version error, label = lst:possibleVersionError ]
  Compiled library Foo makes inconsistent assumptions over library Bar
\end{lstlisting}
might be generated, because incompatible versions of Coq are installed on a machine.
In particular,  CoqIDE and Proof General can use different versions of Coq for compiling. 

To resolve these issues build the files again.\\

Receiving the error message 
\begin{lstlisting}[language = bash, caption= possible compiling error, label=lst:PossibleCompilingError]
Unable to locate library Basics.
\end{lstlisting}
might be caused if two libraries are dependent and \texttt{Bar} was modified and compiled singly.
To overcome this issue, build \texttt{Basics.vo} again.
In case to many files are affected build everything again.
\begin{lstlisting}[language=bash, caption = rebuilding libraries, label = lst:RebuildingLibraries]
Make clean, make
\end{lstlisting}

Using the CoqIDE and running \texttt{coqc} in the command line might cause inconsistency. 
A workaround of this issue is always using the {\itshape make} button from the CoqIDE and never calling the compiler directly.
