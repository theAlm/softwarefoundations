\subsection{The Kaiser's Scheduler}

% refernces for this section

%@misc{KBK,
%author = {R. Kaiser and K. Beckmann and R. Kröger},
%title = {Echtzeitplanung},
%howpublished = {Handouts},
%note = {found online at \url{https://www.cs.hs-rm.de/~kaiser/1919_ezv/%6_Scheduling-handout.pdf} acessed at the January 07th 2020}, 
%}

%@book{L,
%title = {Real-time systems},
%author= {J. W.S. Liu},
%publisher = {Prentice-Hall, Inc.},
%isbn = {0-13-099651-3}, 
%year = {2000},
%}


%@phdthesis{K,
%  author       = {R. Kasier}, 
%  title        = {Virtualisierung von Mehrprozessorsystemen mit
%Echtzeitanwendungen},
%  school       = {Universität Koblenz-Landau},
%  year         = 2009,
%  month        = 2,
%  day          = 11,
% howpublished = {PHD Thesis},
%}

%@phdthesis{B,
%  author       = {G. Bollella}, 
%  title        = {Slotted Priorities: Supporting Real-Time Computing Within General-Purpose Operating Systems},
%  school       = {Chapel Hill},
%  year         = 1997,
% howpublished = {PHD Thesis},
%}



In this section we want to investigate the encapsulated EDF-scheduler (earliest dedline first) from the schedulig procedure by \cite{K}. \\

Let's recap \glspl{real-time system} as in \cite{KBK}.
The process is defined as a sequential execution of a program on a processor. The execution ends after a finite number of steps. 
Therefore it corresponds to a finite execution of machine commands and is not separable. \\
A process is called periodic if it should be restated after a certain time called {\itshape the period}. Otherwise a process is called aperiodic or sporadic.\\

Furthermore, whenever as process is said to be {\itshape non-preemptive} the execution may not be interrupted between the beginning and ending of the process. 
It is called {\itshape preemptive} if it may be interrupted after any instruction.\\  
 
The slotted prirority modell from \cite{B} with  sporadic and periodic processes (\cite{K}).
Due to \cite{B} the major requirment is saied to be as in the follwing:

\begin{quote}
	`If a system intsleves the execution of a real-time and non-real-time thread in alternate intervals the intervals in which real-time threads execute are scheduled to be in every $l$ time unit, then it must be ensured that the intervall begin at time $t$ where $kl \leq t \leq kl+\epsilon \quad \forall
 \epsilon \geq 0$'.
\end{quote}

Moreover there is this requirement $\mathcal{B}$.
\begin{quote}
	For $L>\epsilon$ (for a suitable $\epsilon$) for which the real-time thread schedule has asserted a real-time thread $\tau$ to be expectet on the CPU, there must be a function of the method by which the minimum number of CPU cycles avialable to execute the instructions of $\tau$ can be determined.
\end{quote}
   
\begin{align}
\text{Let} \quad  & \delta_p \in \mathbb{N} \quad \text{be a periodic disruptive process and}\\
 &\delta_s  \mathbb{N} \quad \text{an asporadic disruptive process.}\\
 &\text{Let} \quad P =:{1, ..., n} \subset \mathbb{N}^n \text{be an disruptable process.}  \\
 &\text{Let} \quad  \delta p_i, i = 1,..,n \quad  \text{denote the period and}  \\
 &\text{Let} \quad  \delta e_i, i = 1,..,n \quad  \text{denote the execution time}.  
\end{align}   

Moreover we define the slotted priority modell by Bollea with Kaiser's sporadic and periodic disruptive process \cite[] with a discrete time in contrast to these autors.
This chioce is due to the real-time model in \cite[] and \cite[p.12] the exclusion model.

\begin{definition}
	We represent the time the as $\mathbf{N}$.
\end{definition}

According to  \cite{}
\begin{definition}
Let $P_i\in \mathcal{P}$ be a process and 
\begin{equation}
P^j_i = ( \Delta p_i, r_i) \in \mathbb{N}_o^2,  
\end{equation}
where $\Delta p_i$ is called the persiod and $r_i$ the residum. 
\end{definition}
Furthermore, time controlled process allocation due to \cite{B} and \cite[p. 34]{K} is defined as.

\begin{definition}
Let $\sigma: \mathbf{N} \times \mathbf{N}_0 \rightarrow \mathbf{N}_0$ be a function called the time schedule or implementation plan.
\begin{enumerate}
\item $\forall t \in \mathbb{R}^+$ we have $\sigma(t)\in \{1,2,...,n\})$
\item \begin{align*}
		&\text{If} \quad \sigma(t) = k > 0  \text{we say the process} P_k \text{is working at time}t.\\
		&\text{If} \quad \sigma(t)= 0 \quad  \text{the process is not working. Then the idel-process of the operating system is working.} 	
	\end{align*}
\end{enumerate}
\end{definition}

\begin{remark}
Note that $\sigma$ is a step-function. 
\end{remark}

\begin{lemma}
 $\sigma$ is well-defined.
\end{lemma}
\begin{proof}
Recall the definition of a map \ref{}. We have:
\end{proof}

