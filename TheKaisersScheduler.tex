\subsection{The Kaiser's Scheduler}

% refernces for this section

%@misc{KBK,
%author = {R. Kaiser and K. Beckmann and R. Kröger},
%title = {Echtzeitplanung},
%howpublished = {Handouts},
%note = {found online at \url{https://www.cs.hs-rm.de/~kaiser/1919_ezv/%6_Scheduling-handout.pdf} acessed at the January 07th 2020}, 
%}

%@book{L,
%title = {Real-time systems},
%author= {J. W.S. Liu},
%publisher = {Prentice-Hall, Inc.},
%isbn = {0-13-099651-3}, 
%year = {2000},
%}


%@phdthesis{K,
%  author       = {R. Kasier}, 
%  title        = {Virtualisierung von Mehrprozessorsystemen mit
%Echtzeitanwendungen},
%  school       = {Universität Koblenz-Landau},
%  year         = 2009,
%  month        = 2,
%  day          = 11,
% howpublished = {PHD Thesis},
%}

%@phdthesis{B,
%  author       = {G. Bollella}, 
%  title        = {Slotted Priorities: Supporting Real-Time Computing Within General-Purpose Operating Systems},
%  school       = {Chapel Hill},
%  year         = 1997,
% howpublished = {PHD Thesis},
%}



In this section we want to investigate the encapsulated EDF-scheduler (earliest dedline first) from the schedulig procedure by \cite{K}. \\

Let's recap \glspl{real-time system} as in \cite{KBK}.
The {\itshape process} is defined as a sequential execution of a program on a processor. 
The execution ends after a finite number of steps. 
Therefore it corresponds to a finite execution of machine commands and is not separable.\\
A process is called {\itshape periodic} if it should be restated after a certain time called {\itshape the period}. 
Otherwise a process is called {\itshape aperiodic} or {\itshape sporadic}.\\

Furthermore, whenever as process is said to be {\itshape non-preemptive} the execution may not be interrupted between the beginning and ending of the process. 
It is called {\itshape preemptive} if it may be interrupted after any instruction.\\  
 
The slotted prirority modell from \cite{B} with  sporadic and periodic processes (\cite{K}).
Due to \cite{B} the major requirement is saied to be as in the following:

\begin{quote}
	`If a system itselves the execution of a real-time and non-real-time thread in alternate intervals the intervals in which real-time threads execute are scheduled to be in every $l$ time unit, then it must be ensured that the intervall begin at time $t$ where $kl \leq t \leq kl+\epsilon \quad \forall
 \epsilon \geq 0$'.
\end{quote}

Moreover there is this requirement $\mathcal{B}$.
\begin{quote}
For $L>\epsilon$ (for a suitable $\epsilon$) for which the real-time thread schedule has asserted a real-time thread $\tau$ to be expectet on the CPU, there must be a function of the method by which the minimum number of CPU cycles avialable to execute the instructions of $\tau$ can be determined.
\end{quote}

\paragraph{Model of a sporadic disriptive process}
ommited because of a mighty redudandeny

\paragraph*{Encapsulation of a scheduled process}
 
The scheduler found  in \cite{K} is going to be classified for the scope of implementation. It is an encapsulated EDF (earlies-deadline-first) process with scheduling inside the represenatative process. Let
      
\begin{align}
 & \delta_p \in \mathbb{N} \quad \text{be a periodic disruptive process and}\\
 	&\delta_s \in  \mathbb{N} \quad \text{be an asporadic disruptive process.}\\
	 &\text{Let} \quad P := \{1, \cdots, n \} \subset \mathbb{N}^n \quad \text{be an disruptable process.}  \\
 	&\text{Let} \quad  \delta p_i, \quad \text{for} \quad i = 1,..,n \quad  \text{denote the period and}  \\
 	&\delta e_i \quad \text{for} \quad  i = 1,..,n \quad  \text{denote the execution time}.  
\end{align}   

<<<<<<< HEAD
Moreover, we define the slotted priority modell by Bollea with Kaiser's sporadic and periodic disruptive process with a discrete time in contrast to these autors.
This chioce is due to the real-time model from \cite{PROSA_schedubility_analysis} and the real-time kernel as in  \cite[chp. 5.3]{B}. 
=======
Moreover we define the slotted priority model by Bollea with Kaiser's sporadic and periodic disruptive process \cite[] with a discrete time in contrast to these autors.
This choice is due to the real-time model in \cite[] and \cite[p.12] the exclusion model.
>>>>>>> ad4cdc49bc1e49740d01e638de2de04d33a35f25

\begin{definition}
	Time is $\mathbb{N}$.
\end{definition}

According to \cite[]{} due to the establishing the exclusion model we denote by

\begin{equation}
\sigma: \mathbb{N}^n \times \mathbb{N}^n\longrightarrow \mathbb{N} 
\end{equation}
a scheduler satisfying an EDF-regulation \ref{}. TODO: Add this condition.


\begin{definition}
Let $P_i\in \mathcal{P}$ be a process and 
\begin{equation}
P^j_i = ( \Delta p_i, r_i) \in \mathbb{N}_o^2,  
\end{equation}
<<<<<<< HEAD
where $\Delta p_i$ is called the period and $r_i$ the residum. 
=======
where $\Delta p_i$ is called the period and $r_i$ the residuum. 
>>>>>>> ad4cdc49bc1e49740d01e638de2de04d33a35f25
\end{definition}
Furthermore, time controlled process allocation due to \cite{B} and \cite[p. 34]{K} is defined as in the following.

\begin{definition}
Let $\sigma: \mathbb{N} \times \mathbb{N}_0 \rightarrow \mathbb{N}_0$ be a function called the time schedule or implementation plan.
\begin{enumerate}
\item $\forall t \in \mathbb{R}^+$ we have $\sigma(t)\in \{1,2,...,n\}$
\item \begin{equation}
		\sigma(t) =
		\begin{cases}
			k > 0 & \text{we say the process $P_k$ is working at time $t$.}\\
			0 & \text{the process is not working.}\\
			  &\quad  \text{Then the idele-process$i$ of the operating system is working.}
		\end{cases}       
\end{equation}
\end{enumerate}
\end{definition}

A depute process is given by $ N = \{1, \cdots, n\}\subset \mathbf{N}$. 

Let $P \subset \{1, \cdots, n\} \subset S$.
$S$ is executed in a periodically repeated timeslot such that $\delta p_{sv}$ is the periodlength and $\delta e_{sv}$ is the size of the time slot.
The remaining period time is contains the  disruptive process $\delta_p$ and $\delta_s$.

We are writing
\begin{align}
	\Delta p_{sv} &\text{for the period length} \\
	\Delta e_s &\text{for the executiontime of the sporatic process and}\\ 
	\Delta e_p &\text{for the executiontime of a periodic task.}
\end{align}

\paragraph{The slotted priority model by Bollela}

Within this model we are having a two-staged procces hirachie. 
The global scheduler is purly time-controlled.
Its' period is given by $\Delta p_{glob}$ and its's to execution task time slot is still to remain $\Delta e_{time}$.
The representative process is the local scheduler. 
It is non interruptable, periodic disruptive process.
We have
\begin{align}
\Delta e_p &:= \Delta p_{glob} - \Delta e\\
\Delta p_{p} &:= \Delta p_{glob}, 
\end{align}
where $Delta p_{p}$ denotes the period of the desruptive process.

\paragraph{The sporadic disruptive process}
As in \cite{K} the sporadic desruptive process with cummulated calculation time $\Delta e$ satisfies
\begin{equation}
\lvert{\Delta p_{glob}}\rvert \leq \lvert{\Delta_{glob}} \rvert. 
\end{equation}
With the above assumptions we have
\begin{theorem}
With the assumption of the schedubility criterium 
\begin{multline}
\forall t \geq \mathbf{N} \quad \text{we have}\\
\sum_{i=1}^n \lfloor \frac{t}{\Delta p_i} \rfloor \Delta e_i \leq \lfloor \frac{t}\Delta p_{sv} \Delta e_{sv} \rfloor 
\end{multline}
\end{theorem}

\begin{remark}
Note that $\sigma$ is a step-function. 
\end{remark}

\begin{lemma}
 $\sigma$ is well-defined.
\end{lemma}
\begin{proof}
Recall the definition of a map \ref{}. We have:
\end{proof}

