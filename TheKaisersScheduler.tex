\subsection{The Kaiser's Scheduler}


In this section we want to investigate the encapeseled EDF-scheduler (earliest dedline first) from the schedulig procedure by \cite{K}. \\

Let's recap \glspl{real-time system} as in \cite{KBK}.
The process is defined as a sequential execution of a program on a processor. The execution ends after a finite number of steps. 
Therefore it corresponds to a finite execution of machine commands and is not separable. \\
A process is called periodic if it should be restated after a certain time called {\itshape the period}. Otherwise a process is called aperiodic or sporadic.\\

Furthermore whenever as process is said to be  non-preemptive the execution may not be interrupted between the beginning and ending of the process. It is called preemptive if it may be interrupted after any instruction.\\  
 
The slotted prirority modell from \cite{B} with \cite[]{K} sporadic and periodic Störprozess.
Due to \cite[Requirements]{B} the major requirment is saied to be.

`If a system intstves the execution of a real-time and non-real-time thread in altenate intervalls the intervalls in which real-time threads execute are scheduled to be in every $l$ time unit, then it must be ensured that the intervall begin at time $t$ wehere $kl \leq t \leq kl+\epsilon \forall
 \epsilon \geq 0$'.
  	    
   
   
   
