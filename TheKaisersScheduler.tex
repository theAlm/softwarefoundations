\section{Schedubility Analysis Modeling}

% refernces for this section

%@misc{KBK,
%author = {R. Kaiser and K. Beckmann and R. Kröger},
%title = {Echtzeitplanung},
%howpublished = {Handouts},
%note = {found online at \url{https://www.cs.hs-rm.de/~kaiser/1919_ezv/%6_Scheduling-handout.pdf} acessed at the January 07th 2020}, 
%}

%@book{L,
%title = {Real-time systems},
%author= {J. W.S. Liu},
%publisher = {Prentice-Hall, Inc.},
%isbn = {0-13-099651-3}, 
%year = {2000},
%}


%@phdthesis{K,
%  author       = {R. Kasier}, 
%  title        = {Virtualisierung von Mehrprozessorsystemen mit
%Echtzeitanwendungen},
%  school       = {Universität Koblenz-Landau},
%  year         = 2009,
%  month        = 2,
%  day          = 11,
% howpublished = {PHD Thesis},
%}

%@phdthesis{B,
%  author       = {G. Bollella}, 
%  title        = {Slotted Priorities: Supporting Real-Time Computing Within General-Purpose Operating Systems},
%  school       = {Chapel Hill},
%  year         = 1997,
% howpublished = {PHD Thesis},
%}



In this section we want to investigate the encapsulated EDF-scheduler (earliest deadline first) from the schedulig procedure by \cite{K}. \\

Let's recap \glspl{real-time system} as in \cite{KBK}.
The {\itshape process} is defined as a sequential execution of a program on a processor. 
The execution ends after a finite number of steps. 
Therefore it corresponds to a finite execution of machine commands and is not separable.\\
\begin{definition}
	A process is called {\itshape periodic} if it should be restated after a certain time called {\itshape the period}. 
	Otherwise a process is called {\itshape aperiodic} or {\itshape sporadic}.


Furthermore, whenever as process is said to be {\itshape non-preemptive} the execution may not be interrupted between the beginning and ending of the process. 
It is called {\itshape preemptive} if it may be interrupted after any instruction.
\end{definition}  
 
The slotted prirority modell from \cite{B} with  sporadic and periodic processes (\cite{K}).
Due to \cite{B97} the major requirement is saied to be as in the following:

\begin{quote}
	`If a system itselves the execution of a real-time and non-real-time thread in alternate intervals the intervals in which real-time threads execute are scheduled to be in every $l$ time unit, then it must be ensured that the intervall begin at time $t$ where $kl \leq t \leq kl+\epsilon \quad \forall
 \epsilon \geq 0$'.
\end{quote}

Moreover there is this requirement $\mathcal{B}$.
\begin{quote}
`For $L>\epsilon$ (for a suitable $\epsilon$) for which the real-time thread schedule has asserted a real-time thread $\tau$ to be expectet on the CPU, there must be a function of the method by which the minimum number of CPU cycles avialable to execute the instructions of $\tau$ can be determined.'
\end{quote}

\paragraph{Model of a sporadic disriptive process}
ommited because of a mighty redudandeny

\paragraph*{Encapsulation of a scheduled process}
 
The scheduler found  in \cite{K} is going to be classified for the scope of implementation. It is an encapsulated EDF (earlies-deadline-first) process with scheduling inside the represenatative process. Let
      
\begin{align}
 & \delta_p \in \mathbb{N} \quad \text{be a periodic disruptive process and}\\
 	&\delta_s \in  \mathbb{N} \quad \text{be an asporadic disruptive process.}\\
	 &\text{Let} \quad P := \{1, \cdots, n \} \subset \mathbb{N}^n \quad \text{be an disruptable process.}  \\
 	&\text{Let} \quad  \delta p_i, \quad \text{for} \quad i = 1,..,n \quad  \text{denote the period and}  \\
 	&\delta e_i \quad \text{for} \quad  i = 1,..,n \quad  \text{denote the execution time}.  
\end{align}   

Moreover, we define the slotted priority modell by Bollea with Kaiser's sporadic and periodic disruptive process with a discrete time in contrast to these autors.
This chioce is due to the real-time model from \cite{PROSA_schedubility_analysis} and the real-time kernel as in  \cite[chp. 5.3]{B} and the exclusion model  \cite[p.12]{B}.

\begin{definition}
	Time is $\mathbb{N}$.
\end{definition}

According to \cite[]{} due to the establishing the exclusion model we denote by

\begin{equation}
\sigma: \mathbb{N}^n \times \mathbb{N}^n\longrightarrow \mathbb{N} 
\end{equation}
a scheduler satisfying an EDF-regulation \ref{}. TODO: Add this condition.


\begin{definition}
Let $P_i\in \mathcal{P}$ be a process and 
\begin{equation}
P^j_i = ( \Delta p_i, r_i) \in \mathbb{N}_o^2,  
\end{equation}
where $\Delta p_i$ is called the period and $r_i$ the residuum. 
\end{definition}
Furthermore, time controlled process allocation due to \cite{B} and \cite[p. 34]{K} is defined as in the following.

\begin{definition}
Let $\sigma: \mathbb{N} \times \mathbb{N}_0 \rightarrow \mathbb{N}_0$ be a function called the time schedule or implementation plan.
\begin{enumerate}
\item $\forall t \in \mathbb{R}^+$ we have $\sigma(t)\in \{1,2,...,n\}$
\item \begin{equation}
		\sigma(t) =
		\begin{cases}
			k > 0 & \text{we say the process $P_k$ is working at time $t$.}\\
			0 & \text{the process is not working.}\\
			  &\quad  \text{Then the idele-process$i$ of the operating system is working.}
		\end{cases}       
\end{equation}
\end{enumerate}
\end{definition}

A depute process is given by $ N = \{1, \cdots, n\}\subset \mathbb{N}$. 

Let $P \subset \{1, \cdots, n\} \subset S$.
$S$ is executed in a periodically repeated timeslot such that $\delta p_{sv}$ is the periodlength and $\delta e_{sv}$ is the size of the time slot.
The remaining period time is contains the  disruptive process $\delta_p$ and $\delta_s$.

We are writing
\begin{align}
	\Delta p_{sv} &\text{ for the period length} \\
	\Delta e_s &\text{ for the executiontime of the sporatic process and}\\ 
	\Delta e_p &\text{ for the executiontime of a periodic task.}
\end{align}

\paragraph{The slotted priority model by Bollela}

Within this model we are having a two-staged procces hirachie. 
The global scheduler is purly time-controlled.
Its' period is given by $\Delta p_{glob}$ and its's to execution task time slot is still to remain $\Delta e_{time}$.
The representative process is the local scheduler. 
It is non interruptable, periodic disruptive process.
We have
\begin{align}
\Delta e_p &:= \Delta p_{glob} - \Delta e\\
\Delta p_{p} &:= \Delta p_{glob}, 
\end{align}
where $\Delta p_{p}$ denotes the period of the desruptive process.

\paragraph{The sporadic disruptive process}
As in \cite{K} the sporadic des
ptive process with cummulated calculation time $\Delta e$ satisfies
\begin{equation}
\Delta p_{p} \leq \lvert{\Delta p_{glob}} \rvert. 
\end{equation}
With the above assumptions we have

\begin{theorem}
\end{theorem}

\begin{remark}
Note that $\sigma$ is a step-function. 
\end{remark}

\begin{lemma}
 $\sigma$ is well-defined.
\end{lemma}
\begin{proof}
Recall the definition of a map \ref{}. We have:
\end{proof}

\subsection{The Akra-Bazzi Theorem}
\label{subsection:The Akra-Bazzi Theorem}
The optimality of algorithms is a very important researche area in many fields.
One may ask if an algorithm was best to choosen to solve the given problem.
But first of all, we have to determine the problem carefully in which sense it should be optimal.There are some theorems which determine an steight-forward solution to this optimality problem.\\
For example there is this very powerfull theorem names Akra-Bazzi \cite{AB98}.
It is a general solution for linar devide-and-conquer recurrences of the of the form

 \begin{equation}
 u_n = \sum_{i}^k a_i u _{\floor{\frac{n}{b_i}}} + g(n).
 \end{equation}
 
 Linear devide-and-conquer recurrecences of the form 
 \begin{equation}
 u_n(t) = 
 	\begin{cases}
			u_0 > 0 & n= 0 \\
			\sum_{i=1}^k a_i u_{\floor{\frac{n}{b_i}}} \ + g(n) \quad &n  \geq 1 
	\end{cases}
 \end{equation}
 where
 \begin{itemize}
 \item $u_0, a_i \in \mathbb{R}^{\star+}$, $\sum_{i=1}^k a_i \geq 1$.
 \item $b_{i}, k \in \mathbb{N}, b_i \geq 2, k\geq 1$
 \item $g(x)$ is defined for real values $x$, and is bounded, positive an nondecreasing function $\forall x\geq 0$.
 \item $\forall c > 1, \exists x_1, k_1 >0$ such that $g(\frac{x}{c}) \geq k_1 g(x), \forall x \geq x_1$. 
 \end{itemize}
 
 \begin{theorem}[Akra and Bazzi]
 If $p_0$ is the characteristic of the real solution of the characteristic equation
 \begin{equation}
 \sum_{i=1}^k a_i b_i ^{-p} = 1
 \end{equation}
 (which always exists and is uniquely positive), then 
 \begin{equation}
 u_n = \Theta(n^{p_0}) + \Theta( n^{p_0} \int_{n_1}^n \frac{g(u)}{u^{p_0+1}} du
 \end{equation}
 for $n1$ large enough. In particualr,
 \begin{itemize}
 \item If $\exists \epsilon >0$ such that $g(x)=\mathcal{O}(x^{p_0-\epsilon})$, then $u_n = \Theta (n^{p_0})$.
 \item If $\exists \epsilon > 0$ such that $g(x) = \Omega(x^{p_0 + \epsilon})$ and $g(x)/ x^{p_0+\epsilon}$ is a non decreasing function, then $u_n = \Theta(g(n))$.
 \item If $g(x) = \Theta(x^{p_0})$, then $u_{n} = \Theta(n^p_0 log n)$. 
 \end{itemize}
 \end{theorem}
 
 \end{itemize}
  There is this application of the theorem given by \cite[Theorem 4]{AB98}.
 \begin{theorem}
 Let $f(x)$ be a function defined as in Threorem1. Let $p_0$ be the unique solution of the characteristic equation. Then,
 \begin{enumerate}
% \item If $\exists \epsilon > 0 $ such that $g(x) = \Mathcal{O}(x^{p_0-\epsilon})$, then $f(x) = \Theta(x^{p_0}).$
 \item If $\exists \epsilon > 0$ such that $g(x) = \Omega( x^p_{0}+\epsilon)$ and $g(x)/ x^{p_0+\epsilon})$ is a non decreasing function, then $f(x)=\Theta(g(x)).$
 \item If $g(x) = \Theta(x^{p_{0}}$ then $f(x) = \Theta( x^p_{0} logx)).$   
 \end{enumerate}
 \end{theorem}
