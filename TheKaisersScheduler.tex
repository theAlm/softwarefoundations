\subsection{The Kaiser's Scheduler}


In this section we want to investigate the encapeseled scheduler from the schedulig procedure by \ref[] Robert Kaiser. 

Let's recap {real-time-systems} as in \cite{KBK}.

The process is defined as a sequential execution of a programm on a processor. The execution ends after a finite number of steps. 
Therefore it correconds to a finite execuition of machine commands and is not seperble. \\
A process is called periodic if it should be restared after a certain time called the period. Otherwise a process is called periodic or sporadic.\\

Non-preemptive the execution my not be interuppted between the beginning and ending of the process. It is called preemptive if it may be interrupted after any instruction.\\  
 
The slotted prirority Modell from [Bollela] with [] sporadic and periodic Störprozess.
We define  	    
   
   
   
