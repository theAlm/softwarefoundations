\documentclass[11pt, a4paper, twoside]{scrartcl}

\usepackage{german}
\usepackage{ae}
\usepackage{aecompl}
\usepackage[german]{algorithm2e}
\usepackage[latin1]{inputenc}
\usepackage{amssymb}
\usepackage{amsfonts}
\usepackage{amsmath}
\usepackage{graphicx}
\usepackage{geometry}
\usepackage{fancyhdr}
\usepackage{script}
\usepackage{multicol}
\usepackage{color}
\usepackage{makeidx}
\usepackage{enumerate}
\usepackage[dvipdfm,
            bookmarksopen=true,
            bookmarks=false,
	    pdfpagemode=UseOutlines,
            pdftitle={Einige elementare Begriffe und Schreibweisen der Mathematik},
            pdfauthor={Steffen Reith},
            pdfcreator={Steffen Reith}
           ]{hyperref}
\usepackage[nofancy]{svninfo}
\usepackage{wrapfig}
\usepackage{subfigure}

% Include global settings
% Special settings for discrete mathematics
\newif\ifdiscretemath
\discretemathfalse


% Baue einen Index
\makeindex

% Setze bibliography style
\bibliographystyle{alpha}

% Seitenaufteilung
\geometry{top=3.0cm, left=2.5cm, right=2.5cm, bottom=3.0cm}

% Seitenstil (Fu�- und Kopfzeilen)
\pagestyle{headings}
\pagestyle{fancy}

% Suchpfad f�r Graphiken
\graphicspath{{pics/}}

\renewcommand{\sectionmark}[1]{\markboth{\bf\boldmath\thesection\ #1}{}}
\renewcommand{\subsectionmark}[1]{\markright{\thesubsection\ #1}}
\lhead[\fancyplain{}{\thepage}]{\fancyplain{}{\rightmark}}
\chead[\fancyplain{}{}]{\fancyplain{}{}}
\rhead[\fancyplain{}{\leftmark}]{\fancyplain{}{\thepage}}
\lfoot[\fancyplain{}{}]{\fancyplain{}{}}
\cfoot[\fancyplain{}{}]{\fancyplain{}{}}
\rfoot[\fancyplain{}{}]{\fancyplain{}{}}

\fancypagestyle{plain}{
    \fancyhead{}
    \renewcommand{\headrulewidth}{0pt}
}

% Aufz�hlung mit (i), (ii), (iii), (iv) usw.
%\renewcommand{\theenumi}{\roman{enumi}}
%\renewcommand{\labelenumi}{(\theenumi)}

% Definitionen f�r Titelseite
\newcommand{\quotename}{\textit}
\newcommand{\emphasize}{\emph}

\newcommand{\docutyp}{Skript}
\newcommand{\lecture}{Einige elementare mathematische Begriffe und
  Schreibweisen in der Informatik}
\newcommand{\docudate}{Sommersemester 2007}

\newcommand{\institution}{{\Large Fachhochschule Wiesbaden}\\
                          Fachbereich Design Informatik Medien}
\newcommand{\lecturer}{Prof.~Dr.~Steffen Reith}
\newcommand{\lectureremail}{reith@informatik.fh-wiesbaden.de}

\svnInfo $Id$

\newcommand{\writer}{Steffen Reith}
\newcommand{\reviser}{Steffen Reith}
\newcommand{\email}{reith@informatik.fh-wiesbaden.de}
\newcommand{\writtendate}{August 2006}

\begin{document}

\begin{titlepage}
        \ifpdf
        \special{pdf: out 1 << /Title 
        (Einige elementare mathematische Begriffe und Schreibweisen in
          der Informatik) 
        /Dest [ @thispage /FitH @ypos ] >>}
        \fi	
        \vspace{60pt}
	\begin{center}
		\vspace{20pt}
		\textbf{\Large {\docutyp}} \\
			
		\vspace{30pt}
		\textbf{\huge {\lecture}} \\
			
		\vspace{20pt}
		\textbf{\docudate} \\

		\vspace{20pt}
		\textbf{\lecturer} \\
		\textbf{\lectureremail}
		\\
		\vspace{120pt}
		{\institution} \\
		
		\vfill			
		\vspace{20pt}
		\begin{tabular}[t]{rl}
			Erstellt von: & {\writer} \\
                        Zuletzt �berarbeitet von: & {\reviser} \\
			Email: & {\email}\\
			Erste Version vollendet: & {\writtendate} \\
			Version: & {\svnInfoRevision} \\
			Date: & {\svnInfoDate} \\
		\end{tabular}
	\end{center}
	\newpage
\end{titlepage}

\pagenumbering{roman}

\special{pdf: out 2 << /Title 
(Inhaltsverzeichnis) 
/Dest [ @thispage /FitH @ypos ] >>}
\tableofcontents

\newpage
	
% Haupt-Textteil
\pagenumbering{arabic}

%%%%%%%%%%%%%%%%%%%%%%%%% Sektion 1 %%%%%%%%%%%%%%%%%%%%%%%%%%%%%%%

% Einige Informationen �ber Mengen
\section{Grundlagen und Schreibweisen}

\subsection{Mengen}Es ist sehr schwer den fundamentalen Begriff der Menge mathematisch exakt
zu definieren. Aus diesem Grund soll uns hier die von Cantor im Jahr $1895$ gegebene
Erklärung genügen, da sie für unsere Zwecke völlig ausreichend ist:

\begin{definition}[Georg Cantor (\cite{Ca85})]
Unter einer ,\dindex{Menge}' verstehen wir jede Zusammenfassung $M$ von 
bestimmten wohlunterschiedenen Objecten $m$ unsrer Anschauung oder 
unseres Denkens (welche die ,\dindex{Elemente}' von $M$ genannt werden) zu 
einem Ganzen\footnote{Diese Zitat entspricht der originalen
Schreibweise von Cantor.}. 
\end{definition}

\noindent Für die Formulierung "`genau dann wenn"' verwenden wir im Folgenden
die Abkürzung \gdw\index{gdw=gdw.} um Schreibarbeit zu sparen.

\subsubsection{Die Elementbeziehung und die Enthaltenseinsrelation}
Sehr oft werden einfache große lateinische Buchstaben wie $N$, $M$, $A$, $B$ oder $C$ 
als Symbole für Mengen verwendet und kleine Buchstaben für die Elemente einer Menge.
Mengen von Mengen notiert man gerne mit kalligraphischen Buchstaben wie $\mathcal{A}$, 
$\mathcal{B}$ oder $\mathcal{M}$.
\begin{definition}
\label{InclSet}
Sei $M$ eine beliebige Menge, dann ist
\begin{itemize}
%
\item $a \in M$ \gdw\ $a$ ist ein Element der Menge
$M$\index{$\in$},
%
\item $a \not\in M$ \gdw\ $a$ ist kein Element der Menge $M$\index{$\not\in$},
%
\item $M \subseteq N$ \gdw\ aus $a \in M$ folgt $a \in N$ ($M$ ist
\dindex{Teilmenge}\index{Menge!Teil-} von $N$)\index{$\subseteq$},
%
\item $M \not\subseteq N$ \gdw\ es gilt nicht $M \subseteq
N$. Gleichwertig: es gibt ein $a \in M$ mit $a \not\in N$ ($M$ ist
keine Teilmenge von $N$)\index{$\not\subseteq$} und
%
\item $M \subset N$ \gdw\ es gilt $M \subseteq N$ und $M \not= N$ ($M$ ist
echte Teilmenge von $N$)\index{$\subset$}.
%
\end{itemize}
Statt $a \in M$ schreibt man auch $M \ni a$\index{$\ni$}, was in
einigen Fällen zu einer deutlichen Vereinfachung der Notation führt.
\end{definition}

\subsubsection{Definition spezieller Mengen}
\label{MengenDef}
Spezielle Mengen können auf verschiedene Art und Weise definiert
werden, wie z.B.
\begin{itemize}
%
\item durch Angabe von Elementen:\ So ist $\set{\enu{a}{1}{n}}$ die Menge,
die aus den Elementen $\enu{a}{1}{n}$ besteht, oder
%
\item durch eine Eigenschaft $E$:\ Dabei ist $\set{a \mid E(a)}$ die Menge
aller Elemente $a$, die die Eigenschaft\footnote{Die Eigenschaft $E$
kann man dann auch als \dindex{Prädikat} bezeichnen.} $E$ besitzen.
%
\end{itemize}
Alternativ zu der Schreibweise $\set{a \mid E(a)}$ wird auch oft 
$\set{a \colon E(a)}$ verwendet.

\goodbreak
\begin{example}
\label{ex:numbers}
Mengen, die durch die Angabe von Elementen definiert sind:
\begin{itemize}
%
\item $\mathbb{B} \eqd \set{0,1}$
%
\item $\N \eqd \set{0, 1, 2, 3, 4, 5, 6, 7, 8, \dots}$ (Menge der \dindex{natürlichen Zahlen}\index{Zahlen!natürlich})\index{$\N$}
%
\item $\Z \eqd \set{\dots, -4, -3, -2, -1, 0, 1, 2, 3, 4, \dots}$ (Menge der \dindex{ganzen Zahlen}\index{Zahlen!ganz})\index{$\Z$}
%
\item $2\Z \eqd \set{0, \pm 2, \pm 4, \pm 6, \pm 8, \dots}$ (Menge der geraden ganzen Zahlen)\index{$2\Z$}
%
\item $\PRIM \eqd \set{2, 3, 5, 7, 11, 13, 17, 19, \dots}$ (Menge der \dindex{Primzahlen})\index{$\PRIM$}
%
\end{itemize}
\end{example}

\begin{example}
Mengen, die durch eine Eigenschaft $E$ definiert sind:
\begin{itemize}
%
\item $\set{n \mid n \in \N \text{ und $n$ ist durch $3$ teilbar}}$
%
\item $\set{n \mid n \in \N \text{ und $n$ ist Primzahl und $n \le 40$}}$
%
\item $\emptyset \eqd \set{a \mid a \not= a}$ (die leere Menge)\index{$\emptyset$}
%
\end{itemize}
\end{example}

Aus Definition \ref{InclSet} ergibt sich, dass die leere Menge (Schreibweise: $\emptyset$) Teilmenge jeder Menge ist. Dabei ist zu beachten, dass 
$\set{\emptyset} \not= \emptyset$\index{$\set{\emptyset}$} gilt, denn $\set{\emptyset}$ 
enthält \emph{ein} Element (die leere Menge) und $\emptyset$ enthält \emph{kein} Element.

\subsubsection{Operationen auf Mengen}
\label{OpSetSect}

\begin{definition}
\label{OpSet}
Seien $A$ und $B$ beliebige Mengen, dann ist
\begin{itemize}
%
\item $A \cap B \eqd \set{a \mid a \in A \text{ und } a \in B}$
(\dindex{Schnitt}\index{Menge!Schnitt-} von $A$ und $B$)\index{$\cap$},
%
\item $A \cup B \eqd \set{a \mid a \in A \text{ oder } a \in B}$
(\dindex{Vereinigung}\index{Menge!Vereinigung-} von $A$ und $B$)\index{$\cup$},
%
\item $A \setminus B \eqd \set{a \mid a \in A \text{ und } a
\not\in B}$ (\dindex{Differenz}\index{Menge!Differenz-} von $A$ und $B$)\index{$\setminus$},
%
\item $\overline{A} \eqd M \setminus A$ (\dindex{Komplement}\index{Menge!Komplement-} von $A$ 
bezüglich einer festen Grundmenge $M$)\index{$\overline{A}$} und
%
\item $\PowerSet{A} \eqd \set{B \mid B \subseteq A}$ 
(\dindex{Potenzmenge}\index{Menge!Potenz-} von $A$)\index{$\PowerSet{A}$}.
%
\end{itemize}
Zwei Mengen $A$ und $B$ mit $A \cap B = \emptyset$ nennt
man \dindex{disjunkt}.

\begin{example}
Sei $A = \set{2, 3, 5, 7}$ und $B = \set{1, 2, 4 , 6}$, dann ist $A
\cap B = \set{2}$, $A \cup B = \set{1, 2, 3, 4, 5, 6, \allowbreak 7}$ und $A
\setminus B = \set{3, 5, 7}$. Wählen wir als Grundmenge die
natürlichen Zahlen, also $M = \N$, dann ist $\overline{A} = \set{n \in \N
\mid n \not= 2 \text{ und } n \not= 3 \text{ und } n
\not= 5 \text{ und } n \not= 7} = \set{1, 4, 6, 8, 9, 10, 11,
\dots}$. 

Als Potenzmenge der Menge $A$ ergibt sich die folgende Menge von Mengen
von natürlichen Zahlen $\PowerSet{A}
= \set{\emptyset,\allowbreak \set{2},\allowbreak \set{3},
\allowbreak \set{5},\allowbreak \set{7}, \allowbreak \set{2,3},
\set{2,5}, \allowbreak \set{2,7}, \allowbreak \set{3,5}, \allowbreak\set{3,7},
\allowbreak \set{5,7},\allowbreak \set{2,\allowbreak 3,5},\allowbreak
\set{2,3,7},\allowbreak \set{2,5,\allowbreak 7}, \allowbreak \set{3,\allowbreak 5,\allowbreak 7},\set{2,3,5,7}}$. 

Offensichtlich ist die Menge $\set{0,2,4,6,8, \dots }$ der geraden
natürlichen Zahlen und die Menge $\set{1,3,5,7,9, \dots }$ der
ungeraden natürlichen Zahlen disjunkt.
\end{example}
\end{definition}

\subsubsection{Gesetze für Mengenoperationen}
\label{SetOpSect}
Für die klassischen Mengenoperationen gelten die folgenden Beziehungen:
\begin{displaymath}
\begin{array}{rcll}
A \cap B &=& B \cap A & \text{Kommutativgesetz für den Schnitt}\\
A \cup B &=& B \cup A & \text{Kommutativgesetz für die Vereinigung}\\
A \cap (B \cap C) &=& (A \cap B) \cap C & \text{Assoziativgesetz für
den Schnitt}\\
A \cup (B \cup C) &=& (A \cup B) \cup C & \text{Assoziativgesetz für
die Vereinigung}\\
A \cap (B \cup C) &=& (A \cap B) \cup (A \cap C) & \text{Distributivgesetz}\\
A \cup (B \cap C) &=& (A \cup B) \cap (A \cup C) & \text{Distributivgesetz}\\
A \cap A &=& A & \text{Duplizitätsgesetz für den Schnitt}\\
A \cup A &=& A & \text{Duplizitätsgesetz für die Vereinigung}\\
A \cap (A \cup B) &=& A & \text{Absorptionsgesetz}\\
A \cup (A \cap B) &=& A & \text{Absorptionsgesetz}\\
\overline{A \cap B} &=& (\overline{A} \cup \overline{B}) &
\text{de-Morgansche Regel}\\
\overline{A \cup B} &=& (\overline{A} \cap \overline{B}) &
\text{de-Morgansche Regel}\\
\overline{\overline{A}} &=& A & \text{Gesetz des doppelten Komplements}
\end{array}
\end{displaymath}
Die "`de-Morganschen Regeln"' wurden nach dem englischen
Mathematiker \textsc{Augustus De Morgan}\footnote{\textborn $1806$ in
Madurai, Tamil Nadu, Indien - \textdied $1871$ in London, England}
benannt.

Als Abkürzung schreibt man statt $X_1 \cup X_2 \cup \dots \cup X_n$
(bzw.~$X_1 \cap X_2 \cap \dots \cap X_n$) einfach $\bigcup\limits_{i=1}^n X_i$
(bzw.~$\bigcap\limits_{i=1}^n X_i$). Möchte man alle Mengen $X_i$ mit
$i \in \N$ schneiden (bzw.~vereinigen), so schreibt man kurz
$\bigcap\limits_{i \in \N} X_i$ (bzw.~$\bigcup\limits_{i \in \N} X_i$).

\goodbreak

Oft benötigt man eine Verknüpfung von zwei Mengen, eine solche
Verknüpfung wird allgemein wie folgt definiert:

\begin{definition}["`Verknüpfung von Mengen"']
Seien $A$ und $B$ zwei Mengen und "`$\odot$"' eine beliebige
Verknüpfung zwischen den Elementen dieser Mengen, dann definieren wir
\begin{displaymath}
A \odot B \eqd \set{a \odot b \mid a \in A \text{ und } b \in B}.
\end{displaymath}
\end{definition}

\begin{example}
Die Menge $3\Z = \set{0, \pm 3, \pm 6, \pm 9, \dots}$ enthält alle
Vielfachen\footnote{Eigentlich müsste man statt $3\Z$ die Notation
$\set{3}\Z$ verwenden. Dies ist allerdings unüblich.} von $3$, damit
ist $3\Z + \set{1} = \set{1,\allowbreak 4,\allowbreak -2,\allowbreak
7,\allowbreak -5, 10, -8, \dots}$. Die Menge $3\Z + \set{1}$ schreibt
man kurz oft auch als $3\Z + 1$, wenn klar ist, was mit dieser
Abkürzung gemeint ist.
\end{example}

\subsubsection{Tupel (Vektoren) und das Kreuzprodukt}
Seien $A, A_1, \dots , A_n$ im folgenden Mengen, dann bezeichnet

\begin{itemize}
  % 
  \item $(\enu{a}{1}{n}) \eqd$ die Elemente $\enu{a}{1}{n}$ in genau dieser
  festgelegten \emph{Reihenfolge} und z.B.~$(3,2) \not= (2,3)$. Wir
  sprechen von einem $n$-Tupel\index{Tupel}\index{Tupel=$n$-Tupel}.
  % 
  \item $A_1 \times A_2 \times \dots \times
  A_n \eqd \set{(\enu{a}{1}{n}) \mid a_1 \in A_1, a_2 \in A_2, \dots
  ,a_n \in A_n }$ (Kreuzprodukt der Mengen $A_1, A_2, \dots ,
  A_n$)\index{Kreuzprodukt}\index{$\times$},
  %
  \item $A^n \eqd \underbrace{A \times A \times \dots \times
  A}_{n\text{-mal}}$ ($n$-faches Kreuzprodukt der Menge $A$)\index{$A^n$} und
  %
  \item speziell gilt $A^1 = \set{(a) \mid a \in A}$.
  %
\end{itemize}
Wir nennen $2$-Tupel auch \emph{Paare}\index{Paar}, $3$-Tupel
auch \dindex{Tripel}, $4$-Tupel auch \dindex{Quadrupel} und $5$-Tupel 
\dindex{Quintupel}. Bei $n$-Tupeln ist, im Gegensatz zu Mengen, eine 
Reihenfolge vorgegeben, d.h.~es gilt z.B.~immer $\set{a,b} = \set{b,a}$, aber 
im Allgemeinen $(a,b) \not= (b,a)$.

\begin{example}
Sei $A = \set{1, 2, 3}$ und $B = \set{a, b, c}$, dann bezeichnet das
Kreuzprodukt von $A$ und $B$ die Menge von Paaren $A \times B =
\set{(1,a), (1,b), (1,c), (2,a), (2,b), (2,c),\allowbreak (3,\allowbreak a), (3,b), (3,c)}$.
\end{example}

\subsubsection{Die Anzahl von Elementen in Mengen}
\label{cntSet}
Sei $A$ eine Menge, die endlich viele Elemente\footnote{Solche Mengen
werden als \dindex{endliche Mengen}\index{Menge!endliche} bezeichnet.}
enthält, dann ist
\begin{displaymath}
\cnt A \eqd \text{Anzahl der Elemente in der Menge $A$}.
\end{displaymath}
\noindent Beispielsweise ist $\cnt \set{4,7,9} = 3$. Mit dieser Definition gilt

\begin{itemize}
%
\item $\cnt(A^n) = (\cnt A)^n$\index{$\cnt$},
%
\item $\cnt \PowerSet{A} = 2^{\cnt A}$,
%
\item $\cnt A + \cnt B = \cnt(A \cup B) + \cnt (A \cap B)$ und
%
\item $\cnt A = \cnt (A \setminus B) + \cnt(A \cap B)$.
%
\end{itemize}

\subsection{Relationen und Funktionen}

\subsubsection{Eigenschaften von Relationen}
\label{PropRel}

Seien $\enu{A}{1}{n}$ beliebige Mengen, dann ist $R$ eine
\emph{$n$-stellige Relation}\index{Relation} \gdw 
\ $R \subseteq A_1 \times A_2 \times \dots \times A_n$. Eine
zweistellige Relation nennt man auch \dindex{binäre
Relation}\index{Relation!binär}. Oft werden auch Relationen
$R \subseteq A^n$ betrachtet, diese bezeichnet man dann als
$n$-stellige Relation über der Menge $A$.

\begin{definition}
Sei $R$ eine zweistellige Relation über $A$, dann ist $R$
\begin{itemize}
%
\item \dindex{reflexiv} \gdw \ $(a,a) \in R$ für alle $a \in A$,
%
\item \dindex{symmetrisch} \gdw \ aus $(a,b) \in R$ folgt $(b,a) \in R$,
%
\item \dindex{antisymmetrisch} \gdw \ aus $(a,b) \in R$ und $(b,a) \in R$ folgt $a =
b$,
%
\item \dindex{transitiv} \gdw \ aus $(a,b) \in R$ und $(b,c) \in R$ folgt $(a,c)
\in R$ und 
%
\item \dindex{linear} \gdw \ es gilt immer $(a,b) \in R$ oder $(b, a) \in R$.
%
\item Wir nennen $R$ eine \dindex{Halbordnung} \gdw $R$ ist reflexiv,
antisymmetrisch und transitiv,
%
\item eine \dindex{Ordnung} \gdw $R$ ist eine lineare Halbordnung und
%
\item eine \emph{Äquivalenzrelation}\index{Aquivalenzrelation=Äquivalenzrelation} 
\gdw $R$ reflexiv, transitiv und symmetrisch ist.
%
\end{itemize}
\end{definition}

\begin{example}
Die Teilmengenrelation "`$\subseteq$"' auf allen Teilmengen von $\Z$ ist
eine Halbordnung, aber keine Ordnung. 
\end{example}
\goodbreak

\begin{example}
Wir schreiben $a \equiv b \mod
n$, falls es eine ganze Zahl $q$ gibt, für die $a - b = q n$ gilt. Für $n \ge 2$
ist die Relation $R_n(a,b) \eqd \set{(a,b) \mid a \equiv b \mod n} \subseteq
\Z^2$ eine Äquivalenzrelation.
\end{example}

\subsubsection{Eigenschaften von Funktionen}
\label{PropFunc}
Seien $A$ und $B$ beliebige Mengen. $f$ ist eine \dindex{Funktion} von $A$ nach
$B$ (Schreibweise: $f \colon A \rightarrow B$) \gdw \ $f \subseteq A \times
B$ und für jedes $a \in A$ gibt es \emph{höchstens} ein $b \in B$ mit
$(a, b) \in f$. Ist also $(a,b) \in f$, so schreibt man $f(a) =
b$. Ebenfalls gebrächlich ist die Notation $a \mapsto b$.

\begin{remark}
Unsere Definition von Funktion umfasst auch mehrstellige
Funktionen. Seien $C$ und $B$ Mengen und $A = C^n$ das $n$-fache
Kreuzprodukt von $C$. Die Funktion $f \colon A \rightarrow B$ ist dann
eine $n$-stellige Funktion, denn sie bildet $n$-Tupel aus $C^n$ auf Elemente
aus $B$ ab.
\end{remark}

\begin{definition}
Sei $f$ eine $n$-stellige Funktion. Möchte man die Funktion $f$
benutzen, aber keine Namen für die Argumente vergeben, so
schreibt man auch 
\begin{displaymath}
f(\underbrace{\cdot, \cdot, \ldots , \cdot}_{\text{$n$-mal}})
\end{displaymath}
Ist also der Namen des Arguments einer einstelligen Funktion $g(x)$
für eine Betrachtung unwichtig, so kann man
$g(\cdot)$ \index{$f(\cdot)$} schreiben, um anzudeuten, dass $g$
einstellig ist, ohne dies weiter zu erwähnen.
\end{definition}

Sei nun $R \subseteq A_1 \times A_2 \times \dots \times A_n$ eine
$n$-stellige Relation, dann definieren wir $P^n_R \colon A_1 \times
A_2 \times \dots \times A_n \rightarrow \set{0,1}$ wie folgt:

\begin{displaymath}
P^n_R(\enu{x}{1}{n}) \eqd 
\left\{
\begin{array}{rl}
1,& \text{ falls $(\enu{x}{1}{n}) \in R$}\\
0,& \text{ sonst} 
\end{array}
\right.
\end{displaymath}
Eine solche ($n$-stellige) Funktion, die "`anzeigt"', ob ein Element 
aus $A_1 \times A_2 \times \dots \times A_n$ entweder zu $R$ gehört 
oder nicht, nennt man ($n$-stelliges) \dindex{Prädikat}.

\begin{example}
Sei $\mathbb{P} \eqd \set{n \in \N \mid \text{$n$ ist Primzahl}}$, dann
ist $\mathbb{P}$ eine $1$-stellige Relation über den natürlichen Zahlen. 
Das Prädikat $P^1_{\mathbb{P}}(n)$ liefert für eine natürliche Zahl
$n$ genau dann $1$, wenn $n$ eine Primzahl ist.
\end{example}

Ist für ein Prädikat $P^n_R$ sowohl die Relation $R$ als auch die
Stelligkeit $n$ aus dem Kontext klar, dann schreibt man auch kurz $P$
oder verwendet das Relationensymbol $R$ als Notation für das Prädikat
$P^n_R$. 

\bigskip

\noindent Nun legen wir zwei spezielle Funktionen fest, die oft sehr
hilfreich sind:
\begin{definition}
\label{floorceil}
Sei $\alpha \in \R$ eine beliebige reelle Zahl, dann gilt
\begin{itemize}
%
\item $\lceil \alpha \rceil \eqd \text{die kleinste ganze Zahl, die größer
oder gleich $\alpha$ ist}$ ($\triangleq$ "`Aufrunden"') \index{$\lceil \cdot \rceil$}
%
\item $\lfloor \alpha \rfloor \eqd \text{die größte ganze Zahl, die kleiner
oder gleich $\alpha$ ist}$ ($\triangleq$ "`Abrunden"') \index{$\lfloor \cdot \rfloor$}
%
\end{itemize}
\end{definition}

\begin{definition}
Für eine beliebige Funktion $f$ legen wir fest:
\begin{itemize}
%
\item Der \dindex{Definitionsbereich} von $f$ ist $D_f \eqd
\set{a \mid \text{es gibt ein $b$ mit $f(a) = b$}}$.
%
\item Der \dindex{Wertebereich} von $f$ ist $W_f \eqd
\set{b \mid \text{es gibt ein $a$ mit $f(a) = b$}}$.
%
\item Die Funktion $f \colon A \rightarrow B$ ist \dindex{total} \gdw $D_f
= A$.
% 
\item Die Funktion $f \colon A \rightarrow B$ heißt \dindex{surjektiv} \gdw $W_f = B$.
%
\item Die Funktion $f$ heißt \dindex{injektiv} (oder
eineindeutig\footnote{Achtung: Dieser Begriff wird manchmal
unterschiedlich, je nach Autor, in den Bedeutungen "`bijektiv"' oder
"`injektiv"' verwendet.}) \gdw\ immer wenn $f(a_1)\allowbreak =
f(a_2)$ gilt auch $a_1 = a_2$.
%
\item Die Funktion $f$ heißt \dindex{bijektiv} \gdw $f$ ist injektiv und surjektiv.
\end{itemize}
\end{definition}
Mit Hilfe der Kontraposition (siehe Abschnitt \ref{KontraPos}) kann
man für die Injektivität alternativ auch zeigen, dass immer wenn
$a_1 \not= a_2$, dann muss auch $f(a_1) \not= f(a_2)$ gelten.

\begin{example}
Sei die Funktion $f \colon \N \rightarrow \Z$ durch $f(n) = (-1)^n
\lceil \frac{n}{2} \rceil$ gegeben. Die Funktion $f$ ist surjektiv,
denn $f(0) = 0, f(1) = -1, f(2) = 1, f(3) = -2, f(4) = 2, \dots$, d.h.~die 
ungeraden natürlichen Zahlen werden auf die negativen ganzen Zahlen 
abgebildet, die geraden Zahlen aus $\N$ werden auf die positiven
ganzen Zahlen abgebildet und deshalb ist $W_f = \Z$.

Weiterhin ist $f$ auch injektiv, denn aus\footnote{Für die Definition
der Funktion $\lceil \cdot \rceil$ siehe Definition \ref{floorceil}.}
$(-1)^{a_1} \lceil \frac{a_1}{2} \rceil = (-1)^{a_2}
\lceil \frac{a_2}{2} \rceil$ folgt, dass entweder $a_1$ und $a_2$
gerade oder $a_1$ und $a_2$ ungerade, denn sonst würden auf der linken
und rechten Seite der Gleichung unterschiedliche Vorzeichen
auftreten. Ist $a_1$ gerade und $a_2$ gerade, dann gilt
$\lceil \frac{a_1}{2} \rceil = \lceil \frac{a_2}{2} \rceil$ und auch
$a_1 = a_2$. Sind $a_1$ und $a_2$ ungerade, dann gilt
$-\lceil \frac{a_1}{2} \rceil = -\lceil \frac{a_2}{2} \rceil$, woraus
auch folgt, dass $a_1 = a_2$.
%
Damit ist die Funktion $f$ bijektiv. Weiterhin ist $f$ auch total,
d.h.~$D_f = \N$.
\end{example}

\begin{definition}
Unter einem $n$-stelligen \dindex{Operator} $f$ (auf der Menge $Y$) versteht man in der Mathematik eine Funktion der Form $f \colon Y^n \rightarrow Y$.  Einfache Beispiele für zweistellige Operatoren sind der Additions- oder Multiplikationsoperator.
\end{definition}

\subsubsection{Hüllenoperatoren}

\begin{definition}
Sei $X$ eine Menge. Ein einstelliger Operator $\Psi \colon \PowerSet{X} \rightarrow \PowerSet{X}$ heißt \dindex{Hüllenoperator}\index{Operator!Hüllen}, wenn er die folgenden drei Eigenschaften erfüllt:

\begin{description}
%
\item[Einbettung:] für alle $A \in \PowerSet{X}$ gilt $A \subseteq \Psi(A))$
%
\item[Monotonie:] für alle $A,B \in \PowerSet{X}$ mit $A \subseteq B$ folgt $\Psi(A) \subseteq \Psi(B)$
%
\item[Abgeschlossenheit:] für alle $A \in \PowerSet{X}$ gilt $\Psi(\Psi(A)) = \Psi(A)$
%
\end{description}
\end{definition}

Aufgrund der Monotonieeigenschaft eines Hüllenoperators kann man bei der Abgeschlossenheit die Eigenschaft $\Psi(\Psi(A)) = \Psi(A)$ auch durch $\Psi(\Psi(A)) \subseteq \Psi(A)$ ersetzen. In der Informatik spielen Hüllenoperatoren eine große Rolle. Gute Beispiele hierfür sind z.B.~die \dindex{transitive Hülle} (vgl.~Computergraphik) oder die Kleene-Hülle (vgl.~Formale Sprachen). 

\ifdiscretemath
%
% Remove the subsubsection
%
\else

\subsubsection{Permutationen}
\label{Permutationen}
Sei $S$ eine beliebige endliche Menge, dann heißt eine bijektive Funktion $\pi$ der Form 
$\pi \colon S \rightarrow S$ \dindex{Permutation}\index{$\pi$}. Das bedeutet, dass die
Funktion $\pi$ Elemente aus $S$ wieder auf Elemente aus $S$ abbildet,
wobei für jedes $b \in S$ ein $a \in S$ mit $f(a) = b$ existiert
(Surjektivität) und falls $f(a_1) = f(a_2)$ gilt, dann ist $a_1 = a_2$
(Injektivität).

\begin{remark}
 Man kann den Permutationsbegriff auch auf unendliche Mengen erweitern, aber besonders häufig werden in der Informatik \emph{Permutationen von endlichen Mengen} benötigt. Aus diesem Grund sollen hier nur endliche Mengen $S$ betrachtet werden.
\end{remark}

Sei nun $S = \set{\range{1}{n}}$ (eine endliche Menge) und
$\pi \colon \set{\range{1}{n}} \rightarrow \set{\range{1}{n}}$ eine
Permutation. Permutationen dieser Art kann man sehr anschaulich mit
Hilfe einer Matrix aufschreiben:

\begin{displaymath}
\pi = \left( 
\begin{array}{cccc}
1 & 2 & \dots & n\\
\pi(1) & \pi(2) & \dots & \pi(n)
\end{array}
\right)
\end{displaymath}
Durch diese Notation wird klar, dass das Element $1$ der Menge $S$
durch das Element $\pi(1)$ ersetzt wird, das Element $2$ wird mit
$\pi(2)$ vertauscht und allgemein das Element $i$ durch $\pi(i)$ für
$1 \le i \le n$. In der zweiten Zeile dieser Matrixnotation findet
sich also \emph{jedes} (Surjektivität) Element der Menge $S$
genau \emph{einmal} (Injektivität).

\begin{example}
Sei $S = \set{\range{1}{3}}$ eine Menge mit drei Elementen. Dann gibt
es, wie man ausprobieren kann, genau $6$ Permutationen von $S$:

\begin{displaymath}
\begin{array}{rlrlrl}
\pi_1 &= \left( 
\begin{array}{ccc}
1 & 2 & 3\\
1 & 2 & 3
\end{array}
\right)
&
\pi_2 &= \left( 
\begin{array}{ccc}
1 & 2 & 3\\
1 & 3 & 2
\end{array}
\right)
&
\pi_3 &= \left( 
\begin{array}{ccc}
1 & 2 & 3\\
2 & 1 & 3
\end{array}
\right)\\[\bigskipamount]
%
\pi_4 &= \left( 
\begin{array}{ccc}
1 & 2 & 3\\
2 & 3 & 1
\end{array}
\right)
&
\pi_5 &= \left( 
\begin{array}{ccc}
1 & 2 & 3\\
3 & 1 & 2
\end{array}
\right)
&
\pi_6 &= \left( 
\begin{array}{ccc}
1 & 2 & 3\\
3 & 2 & 1
\end{array}
\right)\\
\end{array}
\end{displaymath}
\end{example}

\begin{theorem}
Sei $S$ eine endliche Menge mit $n = |S|$, dann gibt es genau $n!$
(Fakultät) verschiedene Permutationen von $S$.
\end{theorem}

\begin{proof}
Jede Permutation $\pi$ der Menge $S$ von $n$ Elementen kann als Matrix
der Form
\begin{displaymath}
\pi = \left( 
\begin{array}{cccc}
1 & 2 & \dots & n\\
\pi(1) & \pi(2) & \dots & \pi(n)
\end{array}
\right)
\end{displaymath}
aufgeschrieben werden. Damit ergibt sich die Anzahl der Permutationen
von $S$ durch die Anzahl der verschiedenen zweiten Zeilen solcher
Matrizen. In jeder solchen Zeile muss jedes der $n$ Elemente von $S$
genau einmal vorkommen, da $\pi$ eine bijektive Abbildung ist,
d.h.~wir haben für die erste Position der zweiten Zeile der
Matrixdarstellung genau $n$ verschiedene Möglichkeiten, für die zweite
Position noch $n - 1$ und für die dritte noch $n-2$. Für die $n$-te
Position bleibt nur noch $1$ mögliches Element aus $S$
übrig\footnote{Dies kann man sich auch als die Anzahl der
verschiedenen Möglichkeiten vorstellen, die bestehen, wenn man aus
einer Urne mit $n$ numerierten Kugeln alle Kugeln \emph{ohne}
Zurücklegen nacheinander zieht.}. Zusammengenommen haben wir also
$n \cdot (n - 1) \cdot (n - 2) \cdot (n - 3) \multdots 2 \cdot
1 = n!$ verschiedene mögliche Permutationen der Menge $S$.
\qed
\end{proof}

\fi


% Graphentheorie
\ifpdf
\special{pdf: out 2 << /Title 
(Graphen und Graphenalgorithmen) 
/Dest [ @thispage /FitH @ypos ] >>}
\fi
\section{Graphen und Graphenalgorithmen}
\ifpdf
\special{pdf: out 3 << /Title 
(Grundlagen) 
/Dest [ @thispage /FitH @ypos ] >>}
\fi
\subsection{Grundlagen}
Die Theorie der Graphen ist heute zu einem unverzichtbaren Bestandteil
der Informatik geworden. Viele Probleme wie z.B.~das Verlegen von
Leiterbahnen auf einer Platine, die Modellierung von Netzwerken oder
die L�sung von Routingproblemen in Verkehrsnetzen benutzen Graphen
oder Algorithmen, die Graphen als Datenstruktur verwenden. Auch schon
bekannte Datenstrukturen wie Listen und B�umen k�nnen als Graphen
aufgefasst werden. All dies gibt einen Anhaltspunkt, dass die
Graphentheorie eine sehr zentrale Rolle f�r die Informatik spielt und
vielf�ltige Anwendungen hat. In diesem Kontext ist es wichtig zu
bemerken, dass der Begriff des Graphen in der Informatik \emph{nicht}
im Sinne von Graph einer Funktion gebraucht wird, sondern wie folgt
definiert ist:

\begin{definition}
Ein \dindex{gerichteter Graph}\index{Graph!gerichtet} $G = (V,E)$ ist
ein Paar, das aus einer Menge von \dindex{Knoten} $V$ und einer Menge
von \dindex{Kanten} $E \subseteq V \times V$
(\dindex{Kantenrelation}\index{Relation!Kante}) besteht. Eine Kante
$k = (u,v)$ aus $E$ kann als Verbindung zwischen den Knoten $u,v \in
V$ aufgefasst werden. Aus diesem Grund nennt man $u$
auch \dindex{Startknoten}\index{Knoten!Start} und
$v$ \dindex{Endknoten}\index{Knoten!End}. Zwei Knoten, die durch eine
Kante verbunden sind, hei�en auch \dindex{benachbart}
oder \dindex{adjazent}.

Ein Graph $H = (V', E')$ mit $V' \subseteq V$ und $E' \subset E$ hei�t
\dindex{Untergraph} von $G$.
\end{definition}

Ein Graph $(V,E)$ hei�t \dindex{endlich} \gdw die Menge der Knoten $V$
endlich ist. Obwohl man nat�rlich auch unendliche Graphen betrachten
kann, werden wir uns in diesem Abschnitt nur mit endlichen Graphen
besch�ftigen, da diese f�r den Informatiker von gro�em Nutzen sind.

\bigskip

Da wir eine Kante $(u,v)$ als Verbindung zwischen den Knoten $u$ und
$v$ interpretieren k�nnen, bietet es sich an Graphen durch Diagramme
darzustellen. Dabei wird die Kante $(u,v)$ durch einen Pfeil von $u$
nach $v$ dargestellt. Drei Beispiele f�r eine bildliche Darstellung
von Graphen finden sich in Abbildung \ref{gGraphen}.

\ifpdf
\special{pdf: out 3 << /Title 
(Einige Eigenschaften von Graphen) 
/Dest [ @thispage /FitH @ypos ] >>}
\fi
\subsection{Einige Eigenschaften von Graphen}
Der Graph in Abbildung \ref{gGraphen}(c) hat eine besondere
Eigenschaft, denn offensichtlich kann man die Knotenmenge $V_{1c}
= \set{0,1,2,3,4,5,6,7,8}$ in zwei disjunkte Teilmengen $V_{1c}^l
= \set{0,1,2,3}$ und $V_{1c}^r = \set{4,5,6,7,8}$ so aufteilen, dass
keine Kante zwischen zwei Knoten aus $V_{1c}^l$ oder $V_{1c}^r$
verl�uft.

\begin{definition}
Ein Graph $G = (V,E)$ hei�t \dindex{bipartit}, wenn gilt:
\begin{enumerate}
%
\item Es gibt zwei Teilmengen $V^l$ und $V^r$ von $V$ mit $V =
V^l \cup V^r$ und $V^l \cap V^r = \emptyset$ und 
%
\item f�r jede Kante $(u,v) \in E$ gilt $u \in V^l$ und $v \in V^r$.
%
\end{enumerate}
\end{definition}

Bipartite Graphen haben viele Anwendungen, weil man jede bin�re
Relation $R \subseteq A \times B$ nat�rlich als bipartiten Graph
auffassen kann, dessen Kanten von Knoten aus $A$ zu Knoten aus $B$
laufen.

\begin{example}
Gegeben sei ein bipartiter Graph $G = (V,E)$ mit $V = V^F \cup V^M$
und $V^F \cap V^M = \emptyset$. Die Knoten aus $V^F$ symbolisieren
Frauen und $V^M$ symbolisiert eine Menge von M�nnern. Kann sich eine
Frau vorstellen einen Mann zu heiraten, so wird der entsprechende
Knoten aus $V^F$ mit dem passenden Knoten aus $V^M$ durch eine Kante
verbunden.  Eine \dindex{Heirat} ist nun eine Kantenmenge $H \subseteq
E$, so dass keine zwei Kanten aus $H$ einen gemeinsamen Knoten
besitzen. Das \dindex{Heiratsproblem} ist nun die Aufgabe f�r $G$ eine
Heirat $H$ zu finden, so dass alle Frauen heiraten k�nnen, d.h.~es ist
das folgende Problem zu l�sen:

\goodbreak
\prob{MARRIAGE}{%
Bipartiter Graph $G = (V,E)$ mit $V = V^F \cup V^M$ und $V^F \cap V^M
= \emptyset$}{%
Eine Heirat $H$ mit $\cnt H = \cnt V^F$
}

Im Beispielgraphen \ref{gGraphen}(c) gibt es keine L�sung f�r das
Heiratsproblem, denn f�r die Knoten ($\triangleq$ Kandidatinnen) $2$ und
$3$ existieren nicht ausreichend viele Partner, d.h.~keine Heirat in
diesem Graphen enth�lt zwei Kanten die sowohl $2$ als auch $3$ als
Startknoten haben.

\medskip

Obwohl dieses Beispiel auf den ersten Blick nur von untergeordneter
Bedeutung erscheint, kann man es auf eine Vielfalt von Anwendungen
�bertragen. Immer wenn die Elemente zweier disjunkter Mengen durch
eine Beziehung verbunden sind, kann man dies als bipartiten Graphen
auffassen. Sollen nun die Bed�rfnisse der einen Menge v�llig
befriedigt werden, so ist dies wieder ein Heiratsproblem. Beispiele
mit mehr praktischem Bezug finden sich u.a.~bei Beziehungen zwischen
K�ufern und Anbietern.
\end{example}

\begin{figure}
\centering
\subfigure[Ein gerichteter Graph mit $5$
Knoten]{\includegraphics[scale=1.2]{graphex1.eps}}
\hspace*{2em}
\subfigure[Ein planarer gerichteter
Graph mit $5$ Knoten]{\includegraphics[scale=1.2]{graphex3.eps}}
\hspace*{2em}
\subfigure[Ein gerichteter bipartiter 
Graph]{\includegraphics[scale=1.2]{graphex2.eps}}
\caption{Beispiele f�r gerichtete Graphen}
\label{gGraphen}
\end{figure}

Oft beschr�nken wir uns auch auf eine Unterklasse von Graphen, bei
denen die Kanten keine "`Richtung"' haben (siehe
Abbildung \ref{ugGraphen}) und einfach durch eine Verbindungslinie
symbolisiert werden k�nnen:

\begin{definition}
Sei $G=(V,E)$ ein Graph. Ist die Kantenrelation
$E$ \dindex{symmetrisch}, d.h.~gibt es zu jeder Kante $(u,v) \in E$
auch eine Kante $(v,u) \in E$ (siehe auch Abschnitt \ref{PropRel}),
dann bezeichnen wir $G$ als \dindex{ungerichteten
Graphen}\index{Graph!ungerichtet} oder kurz als \dindex{Graph}.
\end{definition}

Es ist praktisch die Kanten $(u,v)$ und $(v,u)$ eines ungerichteten
Graphen als Menge $\set{u,v}$ mit zwei Elementen aufzufassen. Diese
Vorgehensweise f�hrt zu einem kleinen technischen Problem. Eine Kante
$(u,u)$ mit gleichem Start- und Endknoten nennen wir, entsprechend der
intuitiven Darstellung eines Graphens als Diagramm, \dindex{Schleife}.
Wandelt man nun solch eine Kante in eine Menge um, so w�rde nur eine
einelementige Menge entstehen. Aus diesem Grund legen wir fest, dass
ungerichtete Graphen \dindex{schleifenfrei} sind.

\begin{definition}
Der (ungerichtete) Graph $K = (V,E)$ hei�t \dindex{vollst�ndig}, wenn
f�r alle $u,v \in V$ mit $u \neq v$ auch $(u,v) \in E$ gilt,
d.h.~jeder Knoten des Graphen ist mit allen anderen Knoten
verbunden. Ein Graph $O=(V,\emptyset)$ ohne Kanten wird
als \dindex{Nullgraph}\index{Graph!Null} bezeichnet.
\end{definition}
Mit dieser Definition ergibt sich, dass die Graphen in
Abbildung \ref{ugGraphen}(a) und Abbildung \ref{ugGraphen}(b)
vollst�ndig sind. Der Nullgraph $(V,\emptyset)$ ist Untergraph jedes
beliebigen Graphen $(V,E)$. Diese Definitionen lassen sich nat�rlich
auch analog auf gerichtete Graphen �bertragen.

\begin{figure}
\centering
\subfigure[Vollst�ndiger ungerichteter Graph 
$K_{16}$]{\includegraphics[scale=0.7]{kclique.eps}}
\hfill
\subfigure[Vollst�ndiger ungerichteter Graph
$K_{20}$]{\includegraphics[scale=0.7]{kclique3.eps}}
\subfigure[Zuf�lliger Graph mit $32$ Knoten]{\includegraphics[scale=0.7]{random.eps}}
\hfill
\subfigure[Regul�rer Graph mit Grad $3$]{\includegraphics[scale=0.7]{moebius.eps}}
\caption{Beispiele f�r ungerichtete Graphen}
\label{ugGraphen}
\end{figure}

\begin{definition}
Sei $G = (V,E)$ ein gerichteter Graph und $v \in V$ ein beliebiger
Knoten. Der \dindex{Ausgrad} von $v$ (kurz:
$\mathrm{outdeg}(v)$\index{$\mathrm{outdeg}(v)$}) ist dann die Anzahl
der Kanten in $G$, die $v$ als als Startknoten haben. Analog ist
der \dindex{Ingrad} von $v$ 
(kurz: $\mathrm{indeg}(v)$\index{$\mathrm{indeg}(v)$}) die Anzahl der Kanten
in $G$, die $v$ als Endknoten haben. 

Bei ungerichteten Graphen gilt f�r jeden Knoten
$\mathrm{outdeg}(v)=\mathrm{indeg}(v)$. Aus diesem Grund schreiben wir
kurz $\mathrm{deg}(v)$ und bezeichnen dies als \emph{Grad von
$v$}\index{Grad}. 
Ein Graph $G$ hei�t \dindex{regul�r} \gdw alle
Knoten von $G$ den gleichen Grad haben.
\end{definition}
Die Diagramme der Graphen in den Abbildungen \ref{gGraphen}
und \ref{ugGraphen} haben Eigenschaft, dass sich einige Kanten
schneiden. Es stellt sich die Frage, ob man diese Diagramme auch so
zeichnen kann, dass keine �berschneidungen auftreten. Diese
Eigenschaft von Graphen wollen wir durch die folgende Definition
festhalten:
\begin{definition}
Ein Graph $G$ hei�t \dindex{planar}, wenn sich sein Diagramm ohne
�berschneidungen zeichnen l��t.
\end{definition}

\begin{example}
Der Graph in Abbildung \ref{gGraphen}(a) ist, wie man leicht
nachpr�fen kann, planar, da die Diagramme aus
Abbildung \ref{gGraphen}(a) und \ref{gGraphen}(b) den gleichen Graphen
repr�sentieren.
\end{example}
Auch planare Graphen haben eine anschauliche Bedeutung. Der Schaltplan
einer elektronischen Schaltung kann als Graph aufgefasst werden. Die
Knoten entsprechen den Stellen an denen die Bauteile aufgel�tet werden
m�ssen, und die Kanten entsprechen den Leiterbahnen auf der
Platine. In diesem Zusammenhang bedeutet planar nun, ob man die
Leiterbahnen kreuzungsfrei verlegen kann, d.h.~ob es m�glich ist, eine
Platine zu fertigen, die mit einer Kupferschicht auskommt. In der
Praxis kommen oft Platinen mit mehreren Schichten zum Einsatz
("`Multilayer-Platine"'). Ein Grund daf�r kann sein, dass der
"`Schaltungsgraph"' nicht planar war und deshalb mehrere Schichten
ben�tigt werden. Da Platinen mit mehreren Schichten in der Fertigung
deutlich teurer sind als solche mit einer Schicht, hat sich
Planarit�tseigenschaft von Graphen also auch unmittelbare finanzielle
Auswirkungen.

\ifpdf
\special{pdf: out 2 << /Title 
(Wege, Kreise, W�lder und B�ume) 
/Dest [ @thispage /FitH @ypos ] >>}
\fi
\subsection{Wege, Kreise, W�lder und B�ume}

\begin{definition}
Sei $G=(V,E)$ ein Graph und $u,v \in V$. Eine Folge von Knoten
$\enu{u}{0}{l} \in V$ mit $u = u_0$, $v = u_l$ und $(u_i,u_{i+1}) \in
E$ f�r $0 \le i \le l - 1$ hei�t \emph{Weg von $u$ nach $v$ der L�nge
$l$}\index{Weg}. Der Knoten $u$
wird \dindex{Startknoten}\index{Knoten!Start} und $v$
wird \dindex{Endknoten}\index{Knoten!End} des Wegs genannt.

Ein Weg bei dem Start- und Endknoten gleich sind,
hei�t \dindex{geschlossener Weg}\index{Weg!geschlossen}. Ein
geschlossener Weg, bei dem kein Knoten au�er dem Startknoten mehrfach
enthalten ist, wird \dindex{Kreis} genannt.
\end{definition}
Mit dieser Definition wird klar, dass der Graph in
Abbildung \ref{gGraphen}(a) den Kreis $1,2,3,\dots,5,1$ mit
Startknoten $1$ hat.

\begin{definition}
Sei $G=(V,E)$ ein Graph. Zwei Knoten $u,v \in V$
hei�en \dindex{zusammenh�ngend}, wenn es einen Weg von $u$ nach $v$
gibt. Der Graph $G$ hei�t \dindex{zusammenh�ngend}, wenn jeder Knoten
von $G$ mit jedem anderen Knoten von $G$ zusammenh�ngt. 

Sei $G'$ ein zusammenh�ngender Untergraph von $G$ mit einer besonderen
Eigenschaft: Nimmt man einen weiteren Knoten von $G$ zu $G'$ hinzu,
dann ist der neu entstandene Graph nicht mehr zusammenh�ngend, d.h.~es
gibt keinen Weg zu diesem neu hinzugekommenen Knoten. Solch einen
Untergraph nennt man \dindex{Zusammenhangskomponente}.
\end{definition}
Offensichtlich sind die Graphen in den
Abbildungen \ref{gGraphen}(a), \ref{ugGraphen}(a), \ref{ugGraphen}(b)
und \ref{ugGraphen}(d) zusammenh�ngend und sie haben genau eine
Zusammenhangskomponente. Man kann sich sogar leicht �berlegen, dass
die Eigenschaft $u$ h�ngt mit $v$ zusammen
eine \emph{�quivalenzrelation} (siehe Abschnitt \ref{PropRel})
darstellt.

Mit Hilfe der Definition des geschlossenen Wegs l��t sich nun der
Begriff der B�ume definieren, die eine sehr wichtige Unterklasse der
Graphen darstellen.
\goodbreak
\begin{definition}
Ein Graph $G$ hei�t
\begin{itemize}
%
\item \dindex{Wald}, wenn es keinen geschlossenen Weg mit L�nge $\ge
1$ in $G$ gibt und 
%
\item \dindex{Baum}, wenn $G$ ein zusammenh�ngender Wald ist,
d.h.~wenn er nur genau eine Zusammenhangskomponente hat.
%
\end{itemize}
\end{definition}

\ifpdf
\special{pdf: out 2 << /Title 
(Die Repr�sentation von Graphen und einige Algorithmen) 
/Dest [ @thispage /FitH @ypos ] >>}
\fi
\subsection{Die Repr�sentation von Graphen und einige Algorithmen}
Nachdem Graphen eine gro�e Bedeutung sowohl in der praktischen als
auch in der theoretischen Informatik erlangt haben, stellt sich noch
die Frage, wie man Graphen effizient als Datenstruktur in einem
Computer ablegt. Dabei soll es m�glich sein Graphen effizient zu
speichern und zu manipulieren. 

Die erste Idee, Graphen als dynamische Datenstrukturen zu
repr�sentieren, scheitert an dem relativ ineffizienten Zugriff auf die
Knoten und Kanten bei dieser Art der Darstellung. Sie ist nur von
Vorteil, wenn ein Graph nur sehr wenige Kanten enth�lt. Die folgende
Methode der Speicherung von Graphen hat sich als effizient erwiesen
und erm�glicht auch die leichte Manipulation des Graphens:
\begin{definition}
Sei $G=(V,E)$ ein gerichteter Graph mit $V = \set{\enu{v}{1}{n}}$. Wir
definieren eine $n \times n$ Matrix $A_G =(a_{i,j})_{1 \le i,j, \le
n}$ durch
\begin{displaymath}
a_{i,j} =
\left\{
\begin{array}{ll}
1,& \text{ falls $(v_i, v_j) \in E$}\\
0,& \text{ sonst}
\end{array}
\right.
\end{displaymath}
Die so definierte Matrix $A_G$ mit Eintr�gen aus der Menge $\set{0,1}$
hei�t \dindex{Adjazenzmatrix} von $G$.
\end{definition}

\begin{example}
F�r den gerichteten Graphen aus Abbildung \ref{gGraphen}(a) ergibt sich die
folgende Adjazenzmatrix:
\begin{displaymath}
A_{G_{1a}} =
\left(
\begin{array}{ccccc}
0 & 1 & 0 & 0 & 0\\ 
0 & 0 & 1 & 0 & 0\\
0 & 0 & 0 & 1 & 0\\
0 & 0 & 0 & 0 & 1\\
1 & 0 & 0 & 0 & 0
\end{array}
\right)
\end{displaymath}
Die Adjazenzmatrix eines ungerichteten Graphen erkennt man daran, dass
sie spiegelsymmetrisch zu Diagonale von links oben nach rechts unten
ist (die Kantenrelation\index{Kantenrelation} ist symmetrisch) und
dass die Diagonale aus $0$-Eintr�gen besteht (der Graphen hat keine
Schleifen). F�r den vollst�ndigen Graphen $K_{16}$ aus
Abbildung \ref{ugGraphen}(a) ergibt sich offensichtlich die folgende Adjazenzmatrix:
\begin{displaymath}
A_{G_{1a}} =
\left(
\begin{array}{cccccccccccccccc}
0 & 1 & 1 & 1 & 1 & 1 & 1 & 1 & 1 & 1 & 1 & 1 & 1 & 1 & 1 & 1\\ 
1 & 0 & 1 & 1 & 1 & 1 & 1 & 1 & 1 & 1 & 1 & 1 & 1 & 1 & 1 & 1\\ 
1 & 1 & 0 & 1 & 1 & 1 & 1 & 1 & 1 & 1 & 1 & 1 & 1 & 1 & 1 & 1\\ 
1 & 1 & 1 & 0 & 1 & 1 & 1 & 1 & 1 & 1 & 1 & 1 & 1 & 1 & 1 & 1\\ 
1 & 1 & 1 & 1 & 0 & 1 & 1 & 1 & 1 & 1 & 1 & 1 & 1 & 1 & 1 & 1\\ 
1 & 1 & 1 & 1 & 1 & 0 & 1 & 1 & 1 & 1 & 1 & 1 & 1 & 1 & 1 & 1\\ 
1 & 1 & 1 & 1 & 1 & 1 & 0 & 1 & 1 & 1 & 1 & 1 & 1 & 1 & 1 & 1\\ 
1 & 1 & 1 & 1 & 1 & 1 & 1 & 0 & 1 & 1 & 1 & 1 & 1 & 1 & 1 & 1\\ 
1 & 1 & 1 & 1 & 1 & 1 & 1 & 1 & 0 & 1 & 1 & 1 & 1 & 1 & 1 & 1\\ 
1 & 1 & 1 & 1 & 1 & 1 & 1 & 1 & 1 & 0 & 1 & 1 & 1 & 1 & 1 & 1\\ 
1 & 1 & 1 & 1 & 1 & 1 & 1 & 1 & 1 & 1 & 0 & 1 & 1 & 1 & 1 & 1\\ 
1 & 1 & 1 & 1 & 1 & 1 & 1 & 1 & 1 & 1 & 1 & 0 & 1 & 1 & 1 & 1\\ 
1 & 1 & 1 & 1 & 1 & 1 & 1 & 1 & 1 & 1 & 1 & 1 & 0 & 1 & 1 & 1\\ 
1 & 1 & 1 & 1 & 1 & 1 & 1 & 1 & 1 & 1 & 1 & 1 & 1 & 0 & 1 & 1\\ 
1 & 1 & 1 & 1 & 1 & 1 & 1 & 1 & 1 & 1 & 1 & 1 & 1 & 1 & 0 & 1\\ 
1 & 1 & 1 & 1 & 1 & 1 & 1 & 1 & 1 & 1 & 1 & 1 & 1 & 1 & 1 & 0
\end{array}
\right)
\end{displaymath}
\end{example}
Mit Hilfe der Adjazenzmatrix und Algorithmus \ref{Reach} kann man
leicht berechnen, ob ein Weg von einem Knoten $u$ zu einem Knoten $v$
existiert. Mit einer ganz �hnlichen Idee kann man auch leicht die
Anzahl der Zusammenhangskomponenten berechnen (siehe
Algorithmus \ref{Kompo}). Dieser Algorithmus markiert die Knoten der
einzelnen Zusammenhangskomponenten auch mit unterschiedlichen
"`Farben"', die hier durch Zahlen repr�sentiert werden.

\restylealgo{ruled}
\begin{algorithm}
\caption{Erreichbarkeit in Graphen}
\label{Reach}
\KwData{Ein Graph $G=(V,E)$ und zwei Knoten $u,v \in V$}
\KwResult{\texttt{true} wenn es einen Weg von $u$ nach $v$
gibt, \texttt{false} sonst}
\BlankLine
\dontprintsemicolon

markiert = \texttt{true}\;
markiere Startknoten $u \in V$\;

\BlankLine
\dontprintsemicolon

\While{(markiert)}{

markiert = \texttt{false}\;

\For{(alle markierten Knoten $w \in V$)}{

\If{($w \in V$ ist adjazent zu einem unmarkierten Knoten $w' \in V$)}{
markiere Knoten $w'$\;
markiert = \texttt{true}\;
}

}

}

\eIf{($v$ ist markiert)}{
\Return \texttt{true}\;
}{
\Return \texttt{false}\;
}

\printsemicolon
\end{algorithm}


\restylealgo{ruled}
\begin{algorithm}
\caption{Zusammenhangskomponenten}
\label{Kompo}
\KwData{Ein Graph $G=(V,E)$}
\KwResult{Anzahl der Zusammenhangskomponenten von $G$}
\BlankLine
\dontprintsemicolon

kFarb = 0\;

\BlankLine
\dontprintsemicolon

\While{(es gibt einen unmarkierten Knoten $u \in V$)}{

kFarb++\;
markiere $u \in V$ mit kFarb\;
\BlankLine
\dontprintsemicolon

markiert=\texttt{true}\;

\While{(markiert)}{

markiert=\texttt{false}\;
\BlankLine
\dontprintsemicolon

\For{(alle mit kFarb markierten Knoten $v \in V$)}{

\If{($v \in V$ ist adjazent zu einem unmarkierten Knoten $v' \in V$)}{
markiere Knoten $v' \in V$ mit kFarb\;
markiert=\texttt{true}\;
}

}

}

}

\Return kFarb\;
\printsemicolon
\end{algorithm}

\begin{definition}
Sei $G = (V,E)$ ein ungerichteter Graph. Eine Funktion $f \colon
V \rightarrow \set{\range{1}{k}}$
hei�t \dindex{$k$-F�rbung}\index{F�rbung} des Graphen $G$. Anschaulich
ordnet die Funktion $f$ jedem Knoten eine von $k$ verschiedenen Farben
zu, die hier durch die Zahlen $\range{1}{k}$ symbolisiert werden. Eine
F�rbung hei�t \emph{vertr�glich}\index{F�rbung!vertr�glich}, wenn f�r
alle Kanten $(u,v) \in E$ gilt, dass $f(u) \neq f(v)$, d.h.~zwei
adjazente Knoten werden nie mit der gleichen Farbe markiert.
\end{definition}

Auch das F�rbbarkeitsproblem spielt in der Praxis der Informatik eine
wichtige Rolle. Ein Beispiel daf�r ist die Planung eines
Mobilfunknetzes. Dabei werden die Basisstationen eines Mobilfunknetzes
als Knoten eines Graphen repr�sentiert. Zwei Knoten werden mit einer
Kante verbunden, wenn Sie geographisch so verteilt sind, dass sie sich
beim Senden auf der gleichen Frequenz gegenseitig st�ren
k�nnen. Existiert eine vertr�gliche $k$-F�rbung f�r diesen Graphen, so
ist es m�glich, ein st�rungsfreies Mobilfunktnetz mit $k$
verschiedenen Funkfrequenzen aufzubauen. Dabei entsprechen die Farben
den verf�gbaren Frequenzen. Bei der Planung eines solchen
Mobilfunknetzes ist also das folgende Problem zu l�sen:
\dprob{COLORABILIY}{
Ein ungerichteter Graph $G$ und eine Zahl $k \in \N$.
}
{
Gibt es eine vertr�gliche F�rbung von $G$ mit $k$ Farben?
}
Dieses Problem geh�rt zu einer (sehr gro�en) Klasse von (praktisch
relevanten) Problemen, f�r die bis heute keine effizienten Algorithmen
bekannt sind (Stichwort: \NP-Vollst�ndigkeit). Vielf�ltige Ergebnisse
der Theoretischen Informatik zeigen sogar, dass man nicht hoffen darf,
dass ein schneller Algorithmus zur L�sung des F�rbbarkeitsproblems
existiert.


% Einige grundlegende Beweistechniken
\special{pdf: out 2 << /Title 
(Einige formale Grundlagen von Beweistechniken) 
/Dest [ @thispage /FitH @ypos ] >>}
\section{Einige formale Grundlagen von Beweistechniken}
Praktisch arbeitende Informatiker glauben oft v�llig ohne (formale)
Beweistechniken auskommen zu k�nnen. Dabei meinen sie sogar, dass
formale Beweise keinerlei Berechtigung in der Praxis der Informatik
haben und bezeichnen solches Wissen als "`in der Praxis irrelevantes
Zeug, das nur von und f�r seltsame Wissenschaftler erfunden
wurde"'. Oft stellen sie sich auf den Standpunkt, dass die Korrektheit
von Programmen und Algorithmen durch "`Lassen wir es doch mal laufen
und probieren es aus!"' ($\triangleq$ Testen) belegt werden
k�nne. Diese Einstellung zeigt sich oft auch darin, dass Programme mit
Hilfe einer IDE schnell "`testweise"' �bersetzt werden, in der
Hoffnung oder (schlimmer) in der �berzeugung, dass ein �bersetzbares
Programm immer auch semantisch korrekt sei.

Theoretiker, die sich mit den Grundlagen der Informatik besch�ftigen,
vertreten oft den Standpunkt, dass die Korrektheit \emph{jedes}
Programms rigoros \emph{bewiesen} werden muss. Wahrscheinlich ist die
Position zwischen diesen beiden Extremen richtig, denn zum einen ist
der formale Beweis von (gro�en) Programmen oft nicht praktikabel (oder
m�glich) und zum anderen kann das Testen mit einer (relativ kleinen)
Menge von Eingaben sicherlich nicht belegen, dass ein Programm
vollst�ndig den Spezifikationen entspricht. Im praktischen Einsatz ist
es dann oft mit Eingaben konfrontiert, die zu einer fehlerhaften
Reaktion f�hren oder es sogar abst�rzen\footnote{Dies wird
eindrucksvoll durch viele Softwarepakete und verbreitete
Betriebssysteme im PC-Umfeld belegt.} lassen. Bei einfacher
Anwendersoftware sind solche Fehler �rgerlich, aber oft zu
verschmerzen. Bei sicherheitskritischer Software (z.B.~bei der
Regelung von Atomkraftwerken, Airbags und Bremssystemen in Autos, in
der Medizintechnik oder bei der Steuerung von Raumsonden) gef�hrden
solche Fehler menschliches Leben oder f�hren zu extrem hohen
finanziellen Verlusten und m�ssen deswegen unbedingt vermieden werden.

F�r den Praktiker bringen Kenntnisse �ber formale Beweise aber noch
andere Vorteile. Viele Beweise beschreiben direkt den zur L�sung
ben�tigten Algorithmus, d.h.~eigentlich wird die Richtigkeit einer
Aussage durch die (implizite) Angabe eines Algorithmus gezeigt. Aber
es gibt noch einen anderen Vorteil. Ist der umzusetzende Algorithmus
komplex (z.B.~aufgrund einer komplizierten Schleifenstruktur oder
einer verschachtelten Rekursion), so ist es unwahrscheinlich, eine
korrekte Implementation an den Kunden liefern zu k�nnen, ohne die
Hintergr�nde ($\triangleq$ Beweis) verstanden zu haben. All dies
zeigt, dass auch ein praktischer Informatiker Einblicke in
Beweistechniken haben solle. Interessanterweise zeigt die Erfahrung im
praktischen Umfeld auch, dass solches (theoretisches) Wissen �ber die
Hintergr�nde oft zu klarer strukturierten und effizienteren Programmen
f�hrt.

Aus diesen Gr�nden sollen in diesem Abschnitt einige grundlegende
Beweistechniken mit Hilfe von Beispielen vorgestellt werden.

\special{pdf: out 3 << /Title 
(Direkte Beweise)
/Dest [ @thispage /FitH @ypos ] >>}
\subsection{Direkte Beweise}
Um einen direkten Beweis zu f�hren, m�ssen wir, beginnend von einer
initialen Aussage ($\triangleq$ Hypothese), durch Angabe einer Folge
von (richtigen) Zwischenschritten zu der zu beweisenden Aussage
($\triangleq$ Folgerung) gelangen. Jeder Zwischenschritt ist dabei
entweder unmittelbar klar oder muss wieder durch einen weiteren
(kleinen) Beweis belegt werden. Dabei m�ssen nicht alle Schritt v�llig
formal beschrieben werden, sondern es kommt darauf an, dass sich dem
Leser die eigentliche Strategie erschlie�t.

\begin{theorem}
\label{ExpoGTSquare}
Sei $n \in \N$. Falls $n \ge 4$, dann ist $2^n \ge n^2$.
\end{theorem}

Wir m�ssen also, in Abh�ngigkeit des Parameters $n$, die Richtigkeit
dieser Aussage belegen. Einfaches Ausprobieren ergibt, dass $2^4 = 16
\ge 16 = 4^2$ und $2^5 = 32 \ge 25 = 5^2$, d.h.~intuitiv scheint die
Aussage richtig zu sein. Wir wollen die Richtigkeit der Aussage nun
durch eine Reihe von (kleinen) Schritten belegen:

\begin{proof}

Wir haben schon gesehen, dass die Aussage f�r $n = 4$ und $n = 5$
richtig ist. Erh�hen wir $n$ auf $n + 1$, so verdoppelt sich der Wert
der linken Seite der Ungleichung von $2^n$ auf $2 \cdot 2^n =
2^{n+1}$. F�r die rechte Seite ergibt sich ein Verh�ltnis von
$(\frac{n+1}{n})^2$. Je gr��er $n$ wird, desto kleiner wird der Wert
$\frac{n+1}{n}$, d.h.~der maximale Wert ist bei $n = 4$ mit $1.25$
erreicht. Wir wissen $1.25^2 = 1.5625$. D.h.~immer wenn wir $n$ um
eins erh�hen, verdoppelt sich der Wert der linken Seite, wogegen sich
der Wert der rechten Seite um maximal das $1.5625$ fache erh�ht. Damit
muss die linke Seite der Ungleichung immer gr��er als die rechte Seite
sein.\qed
\end{proof}

Dieser Beweis war nur wenig formal, aber sehr ausf�hrlich und wurde durch
das Symbol "`$\#$"' beendet. Im Laufe der Zeit hat es sich eingeb�rgert, 
das Ende eines Beweises mit einem besonderen Marker abzuschlie�en. 
Besonders bekannt ist hier "`$\mathrm{qed}$"'\index{$\mathrm{qed}$}, 
eine Abk�rzung f�r die lateinische Floskel "`quod erat demonstrandum"', 
die mit "`was zu beweisen war"' �bersetzt werden kann. In neuerer Zeit 
werden statt "`$\mathrm{qed}$"' mit der gleichen Bedeutung meist die 
Symbole "`$\Box$"' oder "`$\#$"' \index{$\#$}\index{$\Box$}\index{qed} 
verwendet.

Nun stellt sich die Frage: "`Wie formal und ausf�hrlich muss ein
Beweis sein?"'  Diese Frage kann so einfach nicht beantwortet werden,
denn das h�ngt u.a.~davon ab, welche Lesergruppe durch den Beweis von
der Richtigkeit einer Aussage �berzeugt werden soll und wer den Beweis
schreibt. Ein Beweis f�r ein �bungsblatt sollte auch auf Kleinigkeiten
R�cksicht nehmen, wogegen ein solcher Stil f�r eine wissenschaftliche
Zeitschrift vielleicht nicht angebracht w�re, da die die potentielle
Leserschaft �ber ganz andere Erfahrungen und viel mehr
Hintergrundwissen verf�gt. Nun noch eine Bemerkung zum Thema
"`Formalismus"': Die menschliche Sprache ist unpr�zise, mehrdeutig und
Aussagen k�nnen oft auf verschiedene Weise interpretiert werden. Diese
Defizite sollen Formalismen\footnote{In diesem Zusammenhang sind
Programmiersprachen auch Formalismen, die eine pr�zise Beschreibung
von Algorithmen erzwingen und die durch einen Compiler verarbeitet
werden k�nnen.}  ausgleichen, d.h.~die Antwort muss lauten: "`So viele
Formalismen wie notwendig und so wenige wie m�glich!"'. Durch �bung
und Praxis lernt man die Balance zwischen diesen Anforderungen zu
halten und es zeigt sich bald, dass "`Ge�bte"' die formale
Beschreibung sogar wesentlich leichter verstehen.

\bigskip

Oft kann man andere, schon bekannte, Aussagen dazu verwenden, die
Richtigkeit einer Aussage zu belegen.

\begin{theorem}
\label{ExpoGTSquare2}
Sei $n \in \N$ die Summe von $4$ Quadratzahlen, die gr��er als $0$
sind, dann ist $2^n \ge n^2$.
\end{theorem}

\begin{proof}
Die Menge der Quadratzahlen ist $Q = \set{0, 1, 4, 9, 16, 25, 36,
  \dots}$, d.h.~$1$ ist die kleinste Quadratzahl, die gr��er als $0$
ist. Damit muss unsere Summe von $4$ Quadratzahlen gr��er als $4$
sein. Die Aussage folgt direkt aus Satz \ref{ExpoGTSquare}.
\qed
\end{proof}

\special{pdf: out 4 << /Title 
(Die Kontraposition)
/Dest [ @thispage /FitH @ypos ] >>}
\subsubsection{Die Kontraposition}
\label{KontraPos}
Mit Hilfe von direkten Beweisen haben wir Zusammenh�nge der Form
"`Wenn Aussage $H$ richtig ist, dann folgt daraus die Aussage $C$"'
untersucht. Manchmal ist es schwierig einen Beweis f�r eine solchen
Zusammenhang zu finden. V�llig gleichwertig ist die Behauptung "`Wenn
die Aussage $C$ falsch ist, dann ist die Aussage $H$ falsch"' und oft
ist eine solche Aussage leichter zu zeigen.

Die Kontraposition von Satz \ref{ExpoGTSquare} ist also die folgende
Aussage: "`Wenn nicht $2^n \ge n^2$, dann gilt nicht $n \ge 4$."'. Das
entspricht der Aussage: "`Wenn $2^n < n^2$, dann gilt $n < 4$."', was
offensichtlich zu der urspr�nglichen Aussage von
Satz \ref{ExpoGTSquare} gleichwertig ist.

\special{pdf: out 3 << /Title 
(Widerspruchsbeweise)
/Dest [ @thispage /FitH @ypos ] >>}
\subsection{Widerspruchsbeweise}
\label{IndirektBeweis}
Obwohl die Technik der Widerspruchsbeweise auf den ersten Blick sehr
kompliziert erscheint, ist sie sehr m�chtig und liefert oft sehr kurze
Beweise. Angenommen wir sollen die Richtigkeit einer Aussage "`aus der
Hypothese $H$ folgt $C$"' zeigen. Dazu beweisen wir, dass sich ein
Widerspruch ergibt, wenn wir, von $H$ und der Annahme, dass $C$ falsch
ist, ausgehen. Also war die Annahme falsch, und die Aussage $C$ muss
richtig sein.

Anschaulicher wird diese Beweistechnik durch folgendes Beispiel:
Nehmen wir einmal an, dass Alice eine b�rgerliche Frau ist und deshalb
auch keine Krone tr�gt. Es ist klar, dass jede K�nigin eine Krone
tr�gt. Wir sollen nun beweisen, dass Alice keine K�nigin ist. Dazu
nehmen wir an, dass Alice eine K�nigin ist, d.h.~Alice tr�gt eine
Krone. Dies ist ein Widerspruch! Also war unsere Annahme falsch, und
wir haben gezeigt, dass Alice keine K�nigin sein kann.

\goodbreak
\noindent Der Beweis zu folgendem Satz verwendet diese Technik:
\begin{theorem}
Sei $S$ eine endliche Untermenge einer unendlichen Menge $U$. Sei $T$
das Komplement von $S$ bzgl.~$U$, dann ist $T$ eine unendliche Menge.
\end{theorem}

\begin{proof}
Hier ist unsere Hypothese "`$S$ endlich, $U$ unendlich und $T$
Komplement von $S$ bzgl.~$U$"' und unsere Folgerung ist "`$T$ ist
unendlich"'. Wir nehmen also an, dass $T$ eine endliche Menge ist. Da
$T$ das Komplement von $S$ ist, gilt $S \cap T = \emptyset$, also ist
$\cnt(S) + \cnt(T) = \cnt (S \cap T) + \cnt (S \cup T) = \cnt (S \cup
T) = n$, wobei $n$ eine Zahl aus $\N$ ist (siehe
Abschnitt \ref{cntSet}). Damit ist $S \cup T = U$ eine endliche
Menge. Dies ist ein Widerspruch zu unserer Hypothese! Also war die
Annahme "`$T$ ist endlich"' falsch. \qed
\end{proof}

\special{pdf: out 3 << /Title 
(Der Schubfachschluss)
/Dest [ @thispage /FitH @ypos ] >>}
\subsection{Der Schubfachschluss}
\label{Schubfachschluss}
Der Schubfachschluss ist auch als \dindex{Dirichlets
Taubenschlagprinzip}\index{Taubenschlagprinzip} bekannt. 
Werden $n > k$ Tauben auf $k$ Boxen
verteilt, so gibt es mindestens eine Box in der sich wenigstens zwei
Tauben aufhalten. Allgemeiner formuliert sagt das Taubenschlagprinzip,
dass wenn $n$ Objekte auf $k$ Beh�lter aufgeteilt werden, dann gibt es
mindestens eine Box die $\lceil \frac{n}{k} \rceil$ Objekte enth�lt.

\begin{example}
Auf einer Party unterhalten sich $8$ Personen ($\triangleq$ Objekte),
dann gibt es mindestens einen Wochentag ($\triangleq$ Box) an dem
$\lceil \frac{8}{7} \rceil =2$ Personen aus dieser Gruppe Geburtstag
haben.
\end{example}

\special{pdf: out 3 << /Title 
(Gegenbeispiele)
/Dest [ @thispage /FitH @ypos ] >>}
\subsection{Gegenbeispiele}
Im wirklichen Leben wissen wir nicht, ob eine Aussage richtig oder
falsch ist. Oft sind wir dann mit einer Aussage konfrontiert, die auf
den ersten Blick richtig ist und sollen dazu ein Programm
entwickeln. Wir m�ssen also entscheiden, ob diese Aussage wirklich
richtig ist, denn sonst ist evtl.~alle Arbeit umsonst und hat hohe
Kosten verursacht. In solchen F�llen kann man versuchen, ein einziges
Beispiel daf�r zu finden, dass die Aussage falsch ist, um so unn�tige
Arbeit zu sparen.

\bigskip

\noindent Wir zeigen, dass die folgenden Vermutungen falsch sind:
\begin{conjecture}
Wenn $p \in \N$ eine Primzahl ist, dann ist $p$ ungerade.
\end{conjecture}

\begin{counterexample}
Die nat�rliche Zahl 2 ist eine Primzahl und $2$ ist gerade. \qed
\end{counterexample}

\begin{conjecture}
Es gibt keine Zahlen $a,b \in \N$, sodass $a \textrm{ mod } b = b
\textrm{ mod } a$.
\end{conjecture}

\begin{counterexample}
F�r $a = b = 2$ gilt $a \textrm{ mod } b = b \textrm{ mod } a = 0$. \qed
\end{counterexample}

\special{pdf: out 3 << /Title 
(Induktionsbeweise und das Induktionsprinzip)
/Dest [ @thispage /FitH @ypos ] >>}
\subsection{Induktionsbeweise und das Induktionsprinzip}
Eine der wichtigsten Beweismethoden der Informatik ist das
Induktionsprinzip. Wir wollen jetzt nachweisen, dass f�r jedes $n \in
\N$ eine bestimmte Eigenschaft $E$ gilt. Wir schreiben kurz $E(n)$ f�r
die Aussage "`$n$ besitzt die Eigenschaft $E$"'. Mit der
Schreibweise $E(0)$ dr�cken\footnote{Mit $E$ wird also ein Pr�dikat bezeichnet (siehe
Abschnitt \ref{MengenDef})} wir also aus, dass die
erste nat�rliche Zahl $0$ die Eigenschaft $E$ besitzt.

\noindent\textbf{Induktionsprinzip:} Es gelten 
\indudef%
{$E(0)$}% 
{F�r $n \ge 0$ gilt, wenn $E(k)$ f�r $k \le n$ korrekt ist,
dann ist auch $E(n+1)$ richtig.}

Dabei ist \textbf{\textsf{IA}} die Abk�rzung f�r
\dindex{Induktionsanfang} und \textbf{\textsf{IS}} ist die Kurzform von
\dindex{Induktionsschritt}. Die Voraussetzung ($\triangleq$ Hypothese)
$E(k)$ ist korrekt f�r $k \le n$ und wird im Induktionsschritt
als \dindex{Induktionsvoraussetzung} benutzt
(kurz \textbf{\textsf{IV}}). Hat man also den Induktionsanfang und den
Induktionsschritt gezeigt, dann ist es anschaulich, dass jede nat�rliche Zahl die
Eigenschaft $E$ haben muss.

Es gibt verschiedene Versionen von Induktionsbeweisen. Die bekannteste
Version ist die vollst�ndige Induktion, bei der Aussagen �ber
nat�rliche Zahlen gezeigt werden.

\special{pdf: out 4 << /Title 
(Die vollst�ndige Induktion)
/Dest [ @thispage /FitH @ypos ] >>}
\subsubsection{Die vollst�ndige Induktion}

Wie in Piratenfilmen �blich, seien Kanonenkugeln in einer Pyramide mit
quadratischer Grundfl�che gestapelt. Wir stellen uns die Frage,
wieviele Kugeln (in Abh�ngigkeit von der H�he) in einer solchen
Pyramide gestapelt sind.

\begin{theorem}
\label{Pyramid}
Mit einer quadratische Pyramide aus Kanonenkugeln der H�he $n \ge 1$
als Munition, k�nnen wir $\frac{n(n+1)(2n+1)}{6}$ Sch�sse abgeben.
\end{theorem}

\goodbreak
\begin{proof}
Einfacher formuliert: wir sollen zeigen, dass $\sum\limits_{i=1}^n i^2 =
\frac{n(n+1)(2n+1)}{6}$.
\induproof%
{Eine Pyramide der H�he $n = 1$ enth�lt $\frac{1 \cdot 2 \cdot 3}{6} =
  1$ Kugel. D.h.~wir haben die Eigenschaft f�r $n = 1$ verifiziert.}%
{F�r $k \le n$ gilt $\sum\limits_{i=1}^k i^2 = \frac{k(k+1)(2k+1)}{6}$.}%
{%
Wir m�ssen nun zeigen, dass $\sum\limits_{i=1}^{n+1} i^2 =
\frac{(n+1)((n+1)+1)(2(n+1)+1)}{6}$ gilt und dabei muss die
Induktionsvoraussetzung $\sum\limits_{i=1}^n i^2 = \frac{n(n+1)(2n+1)}{6}$
benutzt werden.  

\begin{displaymath}
\begin{array}{rcl}
\sum\limits_{i=1}^{n+1}i^2 &=& \sum\limits_{i=1}^{n}i^2 + (n + 1)^2\\
&\stackrel{\text{\textbf{\textsf{IV}}}}{=}& \frac{n(n+1)(2n+1)}{6} +
(n^2 + 2n + 1)\\
&=& \frac{2n^3 + 3n^2+n}{6} + (n^2 + 2n + 1)\\
&=& \frac{2n^3 + 9n^2+13n + 6}{6}\\
&=& \frac{(n+1)(2n^2+7n+6)}{6} \quad (\star)\\
&=& \frac{(n+1)(n+2)(2n+3)}{6} \quad (\star\star)\\
&=& \frac{(n+1)((n+1) + 1)(2(n + 1) + 1)}{6}\\
\end{array}
\end{displaymath}
Die Zeile $\star$ (bzw.~$\star\star$) ergibt sich, indem man $2n^3 +
9n^2+13n + 6$ durch $n+1$ teilt (bzw.~$2n^2+7n+6$ durch $n+2$). \qed
}
\end{proof}

\noindent Solche Induktionsbeweise treten z.B.~bei der Analyse von Programmen
immer wieder auf.

\special{pdf: out 4 << /Title 
(Induktive Definitionen)
/Dest [ @thispage /FitH @ypos ] >>}
\subsubsection{Induktive Definitionen}

Das Induktionsprinzip kann aber auch dazu verwendet werden,
(Daten-)Strukturen formal zu spezifizieren. Dazu werden in einem
ersten Schritt ($\triangleq$ Induktionsanfang) die "`atomaren"'
Objekte definiert und dann in einem zweiten Schritt die
zusammengesetzten Objekte ($\triangleq$ Induktionsschritt). Diese
Technik ist als \dindex{induktive Definition} bekannt.

\begin{example}
\noindent Ein Baum ist wie folgt definiert:

\medskip

\indudef{Ein einzelner Knoten $w$ ist ein \emph{Baum} und $w$ ist die
  \emph{Wurzel} dieses Baums.}{Seien $T_1, T_2, \dots, T_n$ B�ume mit den
  Wurzeln $\enu{k}{1}{n}$ und $w$ ein einzelner neuer Knoten. Verbinden wir
  den Knoten $w$ mit allen Wurzeln $\enu{k}{1}{n}$, dann entsteht ein neuer Baum
  mit der Wurzel $w$.} 
\end{example}

\begin{example}
\label{induexp}
\noindent Ein arithmetischer Ausdruck ist wie folgt definiert:

\medskip

\indudef{Jeder Buchstabe und jede Zahl ist ein arithmetischer
Ausdruck.}{Seien $E$ und $F$ Ausdr�cke, so sind auch $E + F$, $E * F$
und $[E]$ Ausdr�cke.}

\medskip

\noindent D.h.~$x$, $x+y$, $[2*x + z]$ sind arithmetische Ausdr�cke,
aber beispielsweise sind $x + $, $yy$, $][x+y$ sowie $x +* z$ keine
Ausdr�cke im Sinn dieser Definition.
\end{example}

Bei diesem Beispiel ahnt man schon, dass solche Techniken zur pr�zisen
Definition von Programmiersprachen und Dateiformaten gute Dienste
leisten. Induktive Definitionen haben noch einen weiteren Vorteil,
denn man kann leicht Induktionsbeweise konstruieren, die Aussagen �ber
induktiv definierte Objekte belegen.

\special{pdf: out 4 << /Title 
(Die strukturelle Induktion)
/Dest [ @thispage /FitH @ypos ] >>}
\subsubsection{Die strukturelle Induktion}

\begin{theorem}
\label{CntBrack}
Die Anzahl der �ffnenden Klammern eines arithmetischen Ausdrucks stimmt
mit der Anzahl der schlie�enden Klammern �berein.
\end{theorem}

Es ist offensichtlich, dass diese Aussage richtig ist, denn in
Ausdr�cken wie $(x + y) / 2$ oder $x + ((y/2) * z)$ muss ja zu jeder
�ffnenden Klammer eine schlie�ende Klammer existieren. Der n�chste
Beweis verwendet diese Idee zum die Aussage von Satz \ref{CntBrack}
mit Hilfe einer strukturellen Induktion zu zeigen.

\begin{proof}
Wir bezeichnen die Anzahl der �ffnenden Klammern eines Ausdrucks $E$
mit $\cnt_[(E)$ und verwenden die analoge Notation $\cnt_](E)$ f�r die
Anzahl der schlie�enden Klammern.

\induproof%
{
Die einfachsten Ausdr�cke sind Buchstaben und Zahlen. Die Anzahl der
�ffnenden und schlie�enden Klammern ist in beiden F�llen gleich $0$.
}
{
Sei $E$ ein Ausdruck, dann gilt $\cnt_[(E) = \cnt_](E)$.
}
{
 F�r einen Ausdruck $E + F$ gilt $\cnt_[(E + F) = \cnt_[(E
) +
 \cnt_[(F) \stackrel{\text{\textbf{\textsf{IV}}}}{=} \cnt_](E) +
 \cnt_](F) = \cnt_](E + F)$. V�llig analog zeigt man dies f�r $E *
 F$. F�r den Ausdruck $[E]$ ergibt sich $\cnt_[([E]) = \cnt_[(E) + 1
 \stackrel{\text{\textbf{\textsf{IV}}}}{=} \cnt_](E) + 1 = \cnt_]([E])$.
 In jedem Fall ist die Anzahl der �ffnenden Klammern gleich der Anzahl
 der schlie�enden Klammern.\qed
}
\end{proof}

\bigskip

Mit Hilfe von Satz \ref{CntBrack} k�nnen wir nun leicht ein Programm
entwickeln, das einen Plausibilit�tscheck (z.B.~direkt in einem Editor)
durchf�hrt und die Klammern z�hlt, bevor die Syntax von arithmetischen
Ausdr�cken �berpr�ft wird. Definiert man eine vollst�ndige
Programmiersprache induktiv, dann werden ganz �hnliche
Induktionsbeweise m�glich, d.h.~man kann die Techniken aus diesem
Beispiel relativ leicht auf die Praxis der Informatik �bertragen.


\begin{center}
\mbox{}
\vfill
$\star \star \star$ \textsc{Ende} $\star \star \star$
\end{center}

% Appendix
\appendix

% Index 
\cleardoublepage
\ifpdf
\special{pdf: out 2 << /Title 
(Stichwortverzeichnis) 
/Dest [ @thispage /FitH @ypos ] >>}
\fi
\addcontentsline{toc}{section}{Stichwortverzeichnis}
\def\indexname{Stichwortverzeichnis}
\makeatletter
\printindex
\makeatother

\end{document}
 
