\documentclass[11pt, a4paper, twoside, bibliography=totoc]{scrartcl}

% 20140918001 Change

\usepackage{palatino,eulervm}

\usepackage{german}
\usepackage{fontspec}
\usepackage{polyglossia}
\setmainlanguage[babelshorthands=true]{german}

\usepackage[german]{algorithm2e}

\usepackage{amssymb}
\usepackage{amsfonts}
\usepackage{amsmath}
\usepackage{booktabs}
\usepackage{graphicx}
\usepackage[xetex]{geometry}
\usepackage{scrlayer-scrpage}
\usepackage{multicol}
\usepackage{color}
\usepackage{makeidx}
\usepackage{enumerate}
%\usepackage{lineno}
%\usepackage{ifpdf}
\usepackage[nofancy]{svninfo}
\usepackage{wrapfig}
\usepackage{subfig}
\usepackage{bibgerm}
\usepackage{textcomp}
\usepackage[xetex,
            bookmarksopen=true,
            hyperfootnotes=false,
            breaklinks=true,
            bookmarks=false,
            pdfpagemode=UseOutlines,
            pdftitle={Elementare mathematische Begriffe, Probleme und Schreibweisen},
            pdfauthor={Steffen Reith},
            pdfcreator={Steffen Reith}
           ]{hyperref}
\hypersetup{colorlinks=true,citecolor=blue,linkcolor=blue,urlcolor=blue}
\usepackage{hypcap}
\usepackage{script}

% Algorithmen in C-Style
%\SetKwIF{If}{ElseIf}{Else}{if}{\{}{\}\\else if }{\}\\else\{}{\}}%
%\SetKwSwitch{Switch}{Case}{Other}{switch}{\{}{case}{default:}{\}}%
%\SetKwRepeat{Repeat}{do \{}{\} while}%
%\SetKwBlock{Begin}{\{}{\}}
%\SetKwFor{For}{for}{\{}{\}}
%\SetKwFor{While}{while}{\{}{\}}
\SetKwInput{KwData}{Eingabe}
\SetKwInput{KwResult}{Ergebnis}
\renewcommand{\listalgorithmcfname}{Algorithmenverzeichnis}%

% Baue einen Index
\makeindex

% Include global settings
% Special settings for discrete mathematics
\newif\ifdiscretemath
\discretemathfalse

\discretemathfalse
\algorithmsfalse
\komplextrue

% Baue einen Index
\makeindex

% Setze bibliography style
\bibliographystyle{alpha}

% Seitenaufteilung
\geometry{top=2.5cm, left=3.25cm, right=3.25cm, bottom=3.0cm}

% Suchpfad fuer Graphiken
\graphicspath{{pics/}}

%scrlayer-scrpage
\automark[subsection]{section}
\lofoot[]{}
\cofoot[]{\pagemark}
\rofoot[]{}
\refoot[]{}
\cefoot[]{\pagemark}
\lefoot[]{}

% Definitionen für die Titelseite
\newcommand{\docutyp}{Skript}
\newcommand{\lecture}{Elementare mathematische\\ Begriffe, Probleme und\\[0.5\bigskipamount] Schreibweisen}
\newcommand{\docudate}{Wintersemester 2014/2015}

\newcommand{\institution}{{\Large Hochschule RheinMain}\\
                          Fachbereich Design Informatik Medien}
\newcommand{\lecturer}{Prof.~Dr.~Steffen Reith}
\newcommand{\lectureremail}{\href{mailto:Steffen.Reith@hs-rm.de}{Steffen.Reith@hs-rm.de}}

\svnInfo $Id$

\newcommand{\writer}{Steffen Reith}
\newcommand{\reviser}{Steffen Reith}
\newcommand{\email}{Steffen.Reith@hs-rm.de}
\newcommand{\writtendate}{August 2006}

\begin{document}

\pagestyle{scrplain}

\begin{titlepage}

        \special{pdf: out 1 << /Title 
        (Elementare mathematische Begriffe, Probleme und Schreibweisen)
        /Dest [ @thispage /FitH @ypos ] >>}
        \vspace{40pt}
	\begin{center}

                \marginpar{\includegraphics[scale=0.20]{lehre}}
		\vspace{20pt}
		\textbf{\Large {\docutyp}}
			
		\vspace{20pt}
		\textbf{\Huge \lecture}
			
		\vspace{20pt}
		\textbf{\docudate}

		\vspace{20pt}
		\textbf{\lecturer}\\
		\textbf{\lectureremail}
		
		\vspace{120pt}
		{\institution}\\
		
		\vfill			
		\vspace{20pt}
		\begin{tabular}[t]{rl}
			Erstellt von: & {\writer}\\
                        Zuletzt bearbeitet von: & {\reviser}\\
			Email: & {\email}\\
			Erste Version vollendet: & {\writtendate}\\
			Version: & {\svnInfoRevision}\\
			Letzte Änderung: & {\svnInfoDate}\\
		\end{tabular}
	\end{center}
	\newpage
\end{titlepage}

\cleardoublepage

\vspace{0.3\textheight} 
\begin{raggedleft}
Wenn Leute nicht glauben, dass Mathematik\\
einfach ist, dann nur deshalb, weil sie nicht begreifen,\\
wie kompliziert das Leben ist.\\[\smallskipamount]
\hfill \textsc{John von Neumann}%
\end{raggedleft}

\vspace*{1.5cm}

\begin{raggedleft}
Wenn es eine gute Idee ist, dann mach es\\ 
einfach. Es ist viel einfacher sich hinterher zu\\
entschuldigen, als vorher dafür eine\\
Genehmigung zu bekommen.\\[\smallskipamount]
\hfill \textsc{Grace Hopper}%
\end{raggedleft}

\vspace*{1.5cm}

\begin{raggedleft}
Wake up! Time to die!\\[\smallskipamount]
\hfill \textsc{Leon} in Blade Runner%
\end{raggedleft}

\vfill

Dieses Skript ist aus den Fragen und Bemerkungen von Studenten des
Diplom, Bachelor und Master-Studiengangs Informatik an der Hochschule
RheinMain (ehemals Fachhochschule Wiesbaden) hervorgegangen. Ich danke
allen meinen Studenten für konstruktive Anmerkungen und
Verbesserungen. Dabei möchte ich besonders Frau Carola Henzel nennen,
die sehr viele Tippfehler (Mengen, Summen und Beweistechniken)
berichtigte. Herr Kim Stebel hat die Bemerkungen zu bipartiten Graphen
verbessert. Herr Norbert Wesp berichtete über sprachliche Fehler in
Abschnitt \ref{sec:proof} und Tippfehler in der Einleitung. 

Naturgemäß ist ein Skript nie fehlerfrei (ganz im Gegenteil!) und es
ändert (mit Sicherheit!) sich im Laufe der Zeit. Deshalb bin ich auf
weitere Verbesserungsvorschläge angewiesen.

\cleardoublepage

\special{pdf: out 2 << /Title 
(Inhaltsverzeichnis) 
/Dest [ @thispage /FitH @ypos ] >>}
\tableofcontents

\cleardoublepage

% Seitenstil (Fuss- und Kopfzeilen)
\pagestyle{scrheadings}

\pagenumbering{arabic}
	
%%%%%%%%%%%%%%%%%%%%%%%%% Sektion 1 %%%%%%%%%%%%%%%%%%%%%%%%%%%%%%%

% Eine kurze Einleitung
\special{pdf: out 2 << /Title 
(Vorwort) 
/Dest [ @thispage /FitH @ypos ] >>}
\section{Vorwort}

Viele Studenten haben, besonders in den ersten Semestern, oft
erhebliche Probleme mit mathematischen Notationen und Denkweisen, die
gerade in der (praktischen) Informatik eine extrem große und wichtige
Rolle spielen. Diese Probleme lassen sich nur durch stetiges,
selbstständiges, regelmäßiges und ausdauerndes Üben, "`selber
machen"' und durch selbstständiges Anfertigen von
Vorlesungsmitschriften überwinden. 

Lassen Sie sich nicht entmutigen, wenn das Lösen einer einzigen
Aufgabe manchmal mehrere Stunden benötigt, denn das ist völlig normal!
Leider zeigt es sich oft, dass die Schule solche Kompetenzen nicht
(mehr) vermittelt, d.h.~Sie werden diese am Anfang Ihres Studiums
mühsam, mit Schweiß und oft auch mit Frust erlernen müssen. Das ist 
völlig normal und Sie können sich sicher sein, dass Ihre Dozenten
diese Phase in ihrem Studium genauso wie Sie erlebt (und gemeistert) 
haben! Ein Geheimnis für den Studienerfolg ist die Einstellung "`ich werde nicht
aufgeben"'. Eine gewisse Leidenschaft für \emph{Ihr} Fach ist für die
notwendige Motivation natürlich unabdingbar. 

Sie erwerben sich durch die Übungsaufgaben u.a.~eine
Problemlösungskompetenz, die für spätere Projekte (z.B. Bachelor- und
Masterarbeit) extrem hilfreich ist. In der beruflichen Praxis werden
Sie sich auch regelmäßig mit sehr komplexen Aufgabenstellungen
beschäftigen, d.h.~diese Problemlösungsstrategien sind eine
Grundvoraussetzung für das Berufsleben eines erfolgreichen
Informatikers. Auch aus diesen Gründen ist es also sehr sinnvoll, die
(wöchentlichen) Übungsblätter oder Praktikumsaufgaben der jeweiligen
Veranstaltung selbstständig und regelmäßig zu bearbeiten! Es reicht
nicht, die in den Übungen präsentierten Lösungen zu konsumieren, denn
ein Lerneffekt stellt sich nur ein, wenn Sie die Aufgaben
selbstständig lösen. In Ihrer beruflichen Praxis werden Sie später
für das (selbstständige und vollständige) Lösen von Problemen bezahlt, deshalb 
sollte klar sein, dass sich Passivität (im Studium)  nicht auszahlt. 
 
Dabei ist es auch hilfreich, wenn Sie eine Aufgabe einmal nicht lösen können,
denn auch bei einer ausdauernden aber nicht erfolgreichen Bearbeitung
einer Aufgabe kommt es zu einem Lerneffekt! Werden die Übungen nicht
selbstständig erarbeitet, so bleibt der Lernerfolg mit an Sicherheit
grenzender Wahrscheinlichkeit aus, was sich sicherlich in der Klausur
zum ersten Mal (aber nicht zum letzten Mal) rächen wird. Leider zeigt
sich, dass diese Bemerkungen durch die Hörer regelmäßig ignoriert
werden, deshalb

\begin{center}
\large Warnung:  diese Schwierigkeiten lassen sich auf gar keinen Fall durch 
"`drei Tage Lernen vor der Klausur"' lösen!
\end{center}

Auch eine Woche wird dazu erfahrungsgemäß nicht ausreichen! Dieser
Hinweis lässt sich sinngemäß natürlich auch auf (fast) alle anderen
Fächer Ihres Studiums übertragen. Deshalb nochmal ein ernst gemeinter Ratschlag:
\begin{center}
	\large Nehmen Sie Ihr Studium selbst in den Hand, denn es ist \emph{Ihres}!
\end{center}
Auf jeden Fall ist es nicht das Studium Ihres Dozenten! Beachten Sie auch, dass es 
\emph{nicht} die "`Schuld"' Ihres Dozenten ist, wenn Ihr Studium nicht so erfolgreich
verläuft, wie Sie es sich erhofft haben. Sollten Sie mit einem Dozenten fachlich (oder 
auch menschlich) nicht zurecht kommen, so liegt es in \emph{Ihrer} Verantwortung einen Weg
zu finden ohne diesen "`lästigen"' Dozenten erfolgreich zu werden. Auch solche 
Probleme müssen Sie in Ihrem weiteren beruflichen Leben immer wieder meistern, d.h.~es nutzt
in einem Studium nichts mit seinem Schicksal zu hadern! In Ihrer schulischen Laufbahn
haben Sie solche Probleme in diesem Ausmaß sicherlich nicht meistern müssen ("`heile Welt"'), 
deshalb ist es möglicherweise sinnvoll sich helfen zu lassen (z.B.~durch (ältere) Mitstudenten, 
Dozenten oder die Fachschaft).

\goodbreak
\bigskip

Als Lehrender bekommt man am Anfang einer (eigentlich jeder) Vorlesung \emph{fast immer} die 
Frage "`Gibt es ein Skript oder müssen wir mitschreiben?"' gestellt. Einerseits ist diese Frage
wichtig, aber auf der anderen Seite ist sie falsch gestellt, denn Sie brauchen ein Skript
\emph{und} eine Mitschrift, um eine Vorlesung wirklich erfolgreich zu meistern. Ein Skript 
bringt Ihnen in der Regel wenig für den Studienerfolg, auch wenn alle Lerninhalte drin ganz genau 
beschrieben sind. Ähnlich verhält es sich mit einer Vorlesung. Für das tiefgründige Verständnis des 
Lerninhalts ist die Vor- und Nachbereitung einer Vorlesung essentiell. Dabei ist es entscheidend die
Nachbereitung zeitnah durchzuführen, d.h.~möglichst am gleichen Tag oder notfalls
zumindest in der gleichen Woche. Sie werden schnell lernen bei welchen Fächern Sie eine
unmittelbare Nachbereitung brauchen und bei welchen Fächern Sie die Zügel 
\emph{etwas} schleifen lassen können.

Für die Nacharbeit ist ein Skript hilfreich, da Sie die Lerninhalte dort (hoffentlich) zuverlässig nachschlagen können. Noch hilfreicher sind aber entsprechende Lehrbücher, die Sie in der Hochschulbibliothek finden können. Überhaupt sollte die Bibliothek einer Ihrer Lieblingsaufenthaltsorte werden, denn die Lehr- und Fachbücher enthalten das für Ihr Studium notwendige Wissen in ordentlich aufbereiteter Form und machen es dadurch leichter zugänglich. Auf keinen Fall reicht es,  Suchmaschinen wie "`Google"' oder "`Bing"' oder Quellen wie "`Wikipedia"' zu verwenden! Sie  verschwenden \emph{massiv} Zeit, wenn Sie sich die benötigten Informationen im Netz zusammen  suchen und Sie bleiben bezüglich deren Qualität weiterhin unsicher. Bitte bedenken Sie, dass eine große Portion von Fachwissen benötigt wird, um gute und schlechte Informationsquellen zu unterscheiden. Bei guten Fachbüchern können Sie recht sicher sein, dass die Autoren ausgewiesene Experten auf ihrem Gebiet sind. Trotzdem sollten Sie verschiedene Autoren "`ausprobieren"', um das
Buch zu finden, das Ihnen am besten weiterhilft, denn jeder Autor hat einen unterschiedlichen
Schreib- und Erklärstil. 

Das effektivste und wichtigste Mittel für die Nachbereitung ist eine persönlich erstellte Mitschrift, die Sie während der Veranstaltung selbst erstellen. Oft wird gerade von Anfängern bezweifelt, dass eine Mitschrift nützlich ist. Dazu werden allerlei Ausreden hervorgebracht. Für manche ist es "`zu viel Arbeit"' oder sie haben eine "`schlechte Handschrift"'. Auch die Argumente "`ich kann nicht 
mit der Hand schreiben und zuhören"' oder "`ich kann nicht schnell genug schreiben"' finden sich jedes Semester immer wieder. Damit stellt sich die Frage, warum eine handschriftliche  Mitschrift so wichtig ist. Der Grund ist einfach: Eine Mitschrift enthält die von Ihnen individualisierte Version des (unvertrauten) Lernstoffs und ist  ganz auf Ihr Denken und Ihre Anschauung zugeschnitten. Sie enthält Ihre Gedankenbilder, Fragen und die von Ihnen entdecken Zusammenhänge und vielleicht sogar kleine Beispiele. Weiterhin gibt es wissenschaftliche Hinweise die nahelegen, dass das Schreiben mit der Hand den Wissenserwerb unterstützt (z.B.~\cite{MuBe14}). 

Damit ergibt sich die Frage wie eine "`gute"' Mitschrift aussehen sollte. Die äußere Form ist
auf jeden Fall nicht sonderlich wichtig. Lassen Sie sich nicht von schön gesetzten und formatieren
Texten\footnote{Wie dieser Text zeigt ist dies recht einfach, wenn man das Textsatzsystem
\LaTeX\ verwendet.} blenden! Besser ist \emph{immer} eine Menge von selbst beschriebenen Papier 
mit Kaffeeflecken, Eselsohren, das durch den häufigen Gebrauch\footnote{Der Autor dieses Dokuments 
verwendet Teile seiner alten Vorlesungsmitschriften immer noch. Diese sind zu einer wichtigen Quelle 
geworden, da sich dort Hinweise über damalige Verständnisprobleme finden. Tipp: Heben Sie sich Ihre 
eigenen Ausarbeitungen für später auf, es lohnt sich!} schon ganz speckig geworden ist, als der neuste 
Hochglanzausdruck eines aus dem Internet gesaugten PDF-Files! In Ihrer Mitschrift sollte Wichtiges 
von Unwichtigem klar unterschieden werden. Das benötigt Konzentration, denn Sie müssen dies beim 
Hören der  Vorlesung für sich entscheiden. Mit ein wenig Übung werden Sie hier schnell einen Weg 
finden. Dabei muss der Tafelanschrieb nicht wörtlich übernommen werden, sondern Sie sollten ihn 
geeignet  verkürzen und umformen. Durch diese Denkleistung behält Ihr Gehirn schon einen Teil der 
Informationen ohne das Sie es merken. Aus diesem Grund reicht es nicht, wenn Sie 
die Mitschrift eines Kommilitonen kopieren. Verwenden Sie, wenn möglich, \emph{eigene}
Formulierungen, kurze Sätze, einzelne Wörter, Symbole und Abkürzungen. Die Gliederung
und die Überschriften übernehmen Sie einfach aus der Vorlesung. Sehr wichtige zentrale
Punkte kann man durch \emph{sparsames} (farbiges) Unterstreichen hervorheben. Schreiben
Sie ein Blatt nicht ganz voll. So bleibt Platz für eigene Anmerkungen, Ideen, Querverweise,
Fragen und Kommentare. Gerade diese sind extrem wertvoll für Ihren Lernprozess.  Ob Sie
die Fragen gleich oder am Anfang der nächsten Vorlesung stellen ist eigentlich egal,
allerdings werden Sie bemerken, dass sich viele Fragen von selbst lösen, wenn Sie sie einfach
einmal richtig aufgeschrieben haben.

\bigskip

In Ihrem Studium und in jeder Wissenschaft stellt sich sofort die Frage, wie man (neue)
Erkenntnisse gewinnen kann.  Da Sie sich in der Schule mit solchen Fragestellungen
noch nicht beschäftigt haben, sind diese Vorgehensweisen an Hochschulen gerade in der 
ersten Zeit ungewohnt. 

Eine mögliche Methode zur Erzeugung neuen Wissens funktioniert wie folgt: Hat man 
eine Kette von Aussagen, die \emph{zwingend} und
Schritt für Schritt auseinander hervorgehen, so nennt man das eine
\dindex{Deduktion}. Das Wort Deduktion kommt aus dem Lateinischen und
bedeutet "`ableiten"' (von Wasser) oder "`fortführen"'. Die initiale Aussage
dieser Kette wird \dindex{Hypothese} und die letzte Aussage wird
\dindex{Konklusion} genannt. Ist die Hypothese richtig, so muss auch
die Konklusion richtig sein. Dieses Vorgehen ist als \dindex{deduktive Methode}
\index{Methode!deduktive} bekannt und wird in der Mathematik rigoros 
angewendet, d.h.~\emph{niemals} werden Aussagen oder Definitionen verwendet, die
vorher nicht klar belegt wurden und jeder Schritt in der Kette muss \emph{genauestens}
begründet werden. Dieses Vorgehen ist am Anfang ungewohnt, da 
viele Menschen gewohnt sind einfach (unbelegte) Behauptungen aufzustellen und
daraus dann (die gewünschten) Schlüsse mit Hilfe von unpräzisen Argumenten zu ziehen.  
Im starken Kontrast dazu, stellt die deduktive Methode sicher, dass keine falschen 
Aussagen gemacht werden können (zumindest so lange die Deduktion fehlerfrei ist). 
Folgendes Beispiel gibt einen erste Idee für das Vorgehen der Mathematik:

Wir wissen, dass "`Sokrates ist ein Mensch"' eine wahre Aussage
ist. Dies stellt unsere Hypothese dar. Nun wissen wir auch "`alle
Menschen sind sterblich"'. Wir können daraus unsere Konklusion
folgern, dass "`Sokrates sterblich ist"', wodurch wir eine neue
Erkenntnis gewonnen haben.

Das Vorgehen keine mehrdeutigen Definitionen zu verwenden und jeden
Argumentationsschritt klar und logisch zu begründen wird als
(mathematische) \dindex{Strenge} bezeichnet und wird neben der
Mathematik auch in der Informatik und anderen auf der Mathematik
basierenden Wissenschaften verwendet. So kommentiert der berühmte
Mathematiker David Hilbert im Jahr $1900$ auf einem Kongress in Paris in
seinem Vortrag "`Mathematische Probleme"' das mathematische Vorgehen
wie folgt:

\begin{quote}
"`$\ldots$ welche berechtigten allgemeinen Forderungen an die Lösung
eines mathematischen Problems zu stellen sind: ich meine vor Allem
die, daß es gelingt, die Richtigkeit der Antwort durch eine endliche
Anzahl von Schlüssen darzuthun und zwar auf Grund einer endlichen
Anzahl von Voraussetzungen, welche in der Problemstellung liegen und
die jedesmal genau zu formuliren sind. Diese Forderung der logischen
Deduktion mittelst einer endlichen Anzahl von Schlüssen ist nichts
anderes als die Forderung der Strenge in der Beweisführung. In der
That die Forderung der Strenge, die in der Mathematik bekanntlich von
sprichwörtlicher Bedeutung geworden ist, entspricht einem allgemeinen
philosophischen Bedürfnis unseres Verstandes und andererseits kommt
durch ihre Erfüllung allein erst der gedankliche Inhalt und die
Fruchtbarkeit des Problems zur vollen Geltung. $\dots$"'
\end{quote}

In der \dindex{Erkenntnistheorie} sind auch andere Methoden des
Erkenntnisgewinns bekannt, wie z.B.~die Induktion. Hier versucht man
bestehende Tatsachen zu verallgemeinern, was (auch) zu nicht korrekten
Aussagen führen \emph{kann}, die man dann genauer untersuchen
muss. Dies könnte beispielsweise so aussehen: "`Alle Fahrzeuge auf dem
Parkplatz sind Autos"' und "`Alle Fahrzeuge sind rot"'. Daraus folgern
wir "`Alle Autos sind rot"'. Diese Vorgehensweise spielt hier (erst
einmal) keine oder eine ehr untergeordnete Rolle.

\bigskip

Weiterhin zeigt sich, dass die mathematischen Beweismethodiken
(Deduktionen) oft eng mit Konzepten aus der Softwareerstellung
verknüpft sind, d.h.~mathematisches Denken hat in der Praxis für
praktische Informatiker einen grossen Wert. Ein schönes Beispiel
hierfür ist die starke Ähnlichkeit von Induktionsbeweisen und
rekursiven Algorithmen. Dies zeigt, dass sich eine Einarbeitung in
mathematische Denkweisen auch für Informatiker lohnt, auch wenn ein
direkter Zusammenhang vielleicht nicht sofort ersichtlich ist.

\bigskip

Das hier vorliegende Dokument ist ein (sehr) kurzer Crashkurs mit einigen
Beispielen, der evtl.~Defizite aus der Schule ausgleichen helfen
soll. Um Hinweise zur Ergänzung dieses Skriptes wird dringend gebeten,
denn Vorlesungsskripte werden für Sie (und nicht für den/die Dozenten)
erstellt, d.h.~es ist in Ihrem eigenen Interesse, dass Verbesserungen und
Erweiterungen eingebaut werden. Beachten Sie auch, dass die Inhalte
der Vorlesungen von den Inhalten der jeweiligen Skripten abweichen
können! Es kann insbesondere Inhalte in einem Skript geben, die nicht
prüfungsrelevant sind, und prüfungsrelevante Inhalte, die nicht im
Skript zur Vorlesung enthalten sind.


\cleardoublepage

% Einige grundlegende Notationen
% Mengen, Relationen und Funktionen
\section{Grundlagen und Schreibweisen}

\subsection{Mengen}Es ist sehr schwer den fundamentalen Begriff der Menge mathematisch exakt
zu definieren. Aus diesem Grund soll uns hier die von Cantor im Jahr $1895$ gegebene
Erklärung genügen, da sie für unsere Zwecke völlig ausreichend ist:

\begin{definition}[Georg Cantor (\cite{Ca85})]
Unter einer ,\dindex{Menge}' verstehen wir jede Zusammenfassung $M$ von 
bestimmten wohlunterschiedenen Objecten $m$ unsrer Anschauung oder 
unseres Denkens (welche die ,\dindex{Elemente}' von $M$ genannt werden) zu 
einem Ganzen\footnote{Diese Zitat entspricht der originalen
Schreibweise von Cantor.}. 
\end{definition}

\noindent Für die Formulierung "`genau dann wenn"' verwenden wir im Folgenden
die Abkürzung \gdw\index{gdw=gdw.} um Schreibarbeit zu sparen.

\subsubsection{Die Elementbeziehung und die Enthaltenseinsrelation}
Sehr oft werden einfache große lateinische Buchstaben wie $N$, $M$, $A$, $B$ oder $C$ 
als Symbole für Mengen verwendet und kleine Buchstaben für die Elemente einer Menge.
Mengen von Mengen notiert man gerne mit kalligraphischen Buchstaben wie $\mathcal{A}$, 
$\mathcal{B}$ oder $\mathcal{M}$.
\begin{definition}
\label{InclSet}
Sei $M$ eine beliebige Menge, dann ist
\begin{itemize}
%
\item $a \in M$ \gdw\ $a$ ist ein Element der Menge
$M$\index{$\in$},
%
\item $a \not\in M$ \gdw\ $a$ ist kein Element der Menge $M$\index{$\not\in$},
%
\item $M \subseteq N$ \gdw\ aus $a \in M$ folgt $a \in N$ ($M$ ist
\dindex{Teilmenge}\index{Menge!Teil-} von $N$)\index{$\subseteq$},
%
\item $M \not\subseteq N$ \gdw\ es gilt nicht $M \subseteq
N$. Gleichwertig: es gibt ein $a \in M$ mit $a \not\in N$ ($M$ ist
keine Teilmenge von $N$)\index{$\not\subseteq$} und
%
\item $M \subset N$ \gdw\ es gilt $M \subseteq N$ und $M \not= N$ ($M$ ist
echte Teilmenge von $N$)\index{$\subset$}.
%
\end{itemize}
Statt $a \in M$ schreibt man auch $M \ni a$\index{$\ni$}, was in
einigen Fällen zu einer deutlichen Vereinfachung der Notation führt.
\end{definition}

\subsubsection{Definition spezieller Mengen}
\label{MengenDef}
Spezielle Mengen können auf verschiedene Art und Weise definiert
werden, wie z.B.
\begin{itemize}
%
\item durch Angabe von Elementen:\ So ist $\set{\enu{a}{1}{n}}$ die Menge,
die aus den Elementen $\enu{a}{1}{n}$ besteht, oder
%
\item durch eine Eigenschaft $E$:\ Dabei ist $\set{a \mid E(a)}$ die Menge
aller Elemente $a$, die die Eigenschaft\footnote{Die Eigenschaft $E$
kann man dann auch als \dindex{Prädikat} bezeichnen.} $E$ besitzen.
%
\end{itemize}
Alternativ zu der Schreibweise $\set{a \mid E(a)}$ wird auch oft 
$\set{a \colon E(a)}$ verwendet.

\goodbreak
\begin{example}
\label{ex:numbers}
Mengen, die durch die Angabe von Elementen definiert sind:
\begin{itemize}
%
\item $\mathbb{B} \eqd \set{0,1}$
%
\item $\N \eqd \set{0, 1, 2, 3, 4, 5, 6, 7, 8, \dots}$ (Menge der \dindex{natürlichen Zahlen}\index{Zahlen!natürlich})\index{$\N$}
%
\item $\Z \eqd \set{\dots, -4, -3, -2, -1, 0, 1, 2, 3, 4, \dots}$ (Menge der \dindex{ganzen Zahlen}\index{Zahlen!ganz})\index{$\Z$}
%
\item $2\Z \eqd \set{0, \pm 2, \pm 4, \pm 6, \pm 8, \dots}$ (Menge der geraden ganzen Zahlen)\index{$2\Z$}
%
\item $\PRIM \eqd \set{2, 3, 5, 7, 11, 13, 17, 19, \dots}$ (Menge der \dindex{Primzahlen})\index{$\PRIM$}
%
\end{itemize}
\end{example}

\begin{example}
Mengen, die durch eine Eigenschaft $E$ definiert sind:
\begin{itemize}
%
\item $\set{n \mid n \in \N \text{ und $n$ ist durch $3$ teilbar}}$
%
\item $\set{n \mid n \in \N \text{ und $n$ ist Primzahl und $n \le 40$}}$
%
\item $\emptyset \eqd \set{a \mid a \not= a}$ (die leere Menge)\index{$\emptyset$}
%
\end{itemize}
\end{example}

Aus Definition \ref{InclSet} ergibt sich, dass die leere Menge (Schreibweise: $\emptyset$) Teilmenge jeder Menge ist. Dabei ist zu beachten, dass 
$\set{\emptyset} \not= \emptyset$\index{$\set{\emptyset}$} gilt, denn $\set{\emptyset}$ 
enthält \emph{ein} Element (die leere Menge) und $\emptyset$ enthält \emph{kein} Element.

\subsubsection{Operationen auf Mengen}
\label{OpSetSect}

\begin{definition}
\label{OpSet}
Seien $A$ und $B$ beliebige Mengen, dann ist
\begin{itemize}
%
\item $A \cap B \eqd \set{a \mid a \in A \text{ und } a \in B}$
(\dindex{Schnitt}\index{Menge!Schnitt-} von $A$ und $B$)\index{$\cap$},
%
\item $A \cup B \eqd \set{a \mid a \in A \text{ oder } a \in B}$
(\dindex{Vereinigung}\index{Menge!Vereinigung-} von $A$ und $B$)\index{$\cup$},
%
\item $A \setminus B \eqd \set{a \mid a \in A \text{ und } a
\not\in B}$ (\dindex{Differenz}\index{Menge!Differenz-} von $A$ und $B$)\index{$\setminus$},
%
\item $\overline{A} \eqd M \setminus A$ (\dindex{Komplement}\index{Menge!Komplement-} von $A$ 
bezüglich einer festen Grundmenge $M$)\index{$\overline{A}$} und
%
\item $\PowerSet{A} \eqd \set{B \mid B \subseteq A}$ 
(\dindex{Potenzmenge}\index{Menge!Potenz-} von $A$)\index{$\PowerSet{A}$}.
%
\end{itemize}
Zwei Mengen $A$ und $B$ mit $A \cap B = \emptyset$ nennt
man \dindex{disjunkt}.

\begin{example}
Sei $A = \set{2, 3, 5, 7}$ und $B = \set{1, 2, 4 , 6}$, dann ist $A
\cap B = \set{2}$, $A \cup B = \set{1, 2, 3, 4, 5, 6, \allowbreak 7}$ und $A
\setminus B = \set{3, 5, 7}$. Wählen wir als Grundmenge die
natürlichen Zahlen, also $M = \N$, dann ist $\overline{A} = \set{n \in \N
\mid n \not= 2 \text{ und } n \not= 3 \text{ und } n
\not= 5 \text{ und } n \not= 7} = \set{1, 4, 6, 8, 9, 10, 11,
\dots}$. 

Als Potenzmenge der Menge $A$ ergibt sich die folgende Menge von Mengen
von natürlichen Zahlen $\PowerSet{A}
= \set{\emptyset,\allowbreak \set{2},\allowbreak \set{3},
\allowbreak \set{5},\allowbreak \set{7}, \allowbreak \set{2,3},
\set{2,5}, \allowbreak \set{2,7}, \allowbreak \set{3,5}, \allowbreak\set{3,7},
\allowbreak \set{5,7},\allowbreak \set{2,\allowbreak 3,5},\allowbreak
\set{2,3,7},\allowbreak \set{2,5,\allowbreak 7}, \allowbreak \set{3,\allowbreak 5,\allowbreak 7},\set{2,3,5,7}}$. 

Offensichtlich ist die Menge $\set{0,2,4,6,8, \dots }$ der geraden
natürlichen Zahlen und die Menge $\set{1,3,5,7,9, \dots }$ der
ungeraden natürlichen Zahlen disjunkt.
\end{example}
\end{definition}

\subsubsection{Gesetze für Mengenoperationen}
\label{SetOpSect}
Für die klassischen Mengenoperationen gelten die folgenden Beziehungen:
\begin{displaymath}
\begin{array}{rcll}
A \cap B &=& B \cap A & \text{Kommutativgesetz für den Schnitt}\\
A \cup B &=& B \cup A & \text{Kommutativgesetz für die Vereinigung}\\
A \cap (B \cap C) &=& (A \cap B) \cap C & \text{Assoziativgesetz für
den Schnitt}\\
A \cup (B \cup C) &=& (A \cup B) \cup C & \text{Assoziativgesetz für
die Vereinigung}\\
A \cap (B \cup C) &=& (A \cap B) \cup (A \cap C) & \text{Distributivgesetz}\\
A \cup (B \cap C) &=& (A \cup B) \cap (A \cup C) & \text{Distributivgesetz}\\
A \cap A &=& A & \text{Duplizitätsgesetz für den Schnitt}\\
A \cup A &=& A & \text{Duplizitätsgesetz für die Vereinigung}\\
A \cap (A \cup B) &=& A & \text{Absorptionsgesetz}\\
A \cup (A \cap B) &=& A & \text{Absorptionsgesetz}\\
\overline{A \cap B} &=& (\overline{A} \cup \overline{B}) &
\text{de-Morgansche Regel}\\
\overline{A \cup B} &=& (\overline{A} \cap \overline{B}) &
\text{de-Morgansche Regel}\\
\overline{\overline{A}} &=& A & \text{Gesetz des doppelten Komplements}
\end{array}
\end{displaymath}
Die "`de-Morganschen Regeln"' wurden nach dem englischen
Mathematiker \textsc{Augustus De Morgan}\footnote{\textborn $1806$ in
Madurai, Tamil Nadu, Indien - \textdied $1871$ in London, England}
benannt.

Als Abkürzung schreibt man statt $X_1 \cup X_2 \cup \dots \cup X_n$
(bzw.~$X_1 \cap X_2 \cap \dots \cap X_n$) einfach $\bigcup\limits_{i=1}^n X_i$
(bzw.~$\bigcap\limits_{i=1}^n X_i$). Möchte man alle Mengen $X_i$ mit
$i \in \N$ schneiden (bzw.~vereinigen), so schreibt man kurz
$\bigcap\limits_{i \in \N} X_i$ (bzw.~$\bigcup\limits_{i \in \N} X_i$).

\goodbreak

Oft benötigt man eine Verknüpfung von zwei Mengen, eine solche
Verknüpfung wird allgemein wie folgt definiert:

\begin{definition}["`Verknüpfung von Mengen"']
Seien $A$ und $B$ zwei Mengen und "`$\odot$"' eine beliebige
Verknüpfung zwischen den Elementen dieser Mengen, dann definieren wir
\begin{displaymath}
A \odot B \eqd \set{a \odot b \mid a \in A \text{ und } b \in B}.
\end{displaymath}
\end{definition}

\begin{example}
Die Menge $3\Z = \set{0, \pm 3, \pm 6, \pm 9, \dots}$ enthält alle
Vielfachen\footnote{Eigentlich müsste man statt $3\Z$ die Notation
$\set{3}\Z$ verwenden. Dies ist allerdings unüblich.} von $3$, damit
ist $3\Z + \set{1} = \set{1,\allowbreak 4,\allowbreak -2,\allowbreak
7,\allowbreak -5, 10, -8, \dots}$. Die Menge $3\Z + \set{1}$ schreibt
man kurz oft auch als $3\Z + 1$, wenn klar ist, was mit dieser
Abkürzung gemeint ist.
\end{example}

\subsubsection{Tupel (Vektoren) und das Kreuzprodukt}
Seien $A, A_1, \dots , A_n$ im folgenden Mengen, dann bezeichnet

\begin{itemize}
  % 
  \item $(\enu{a}{1}{n}) \eqd$ die Elemente $\enu{a}{1}{n}$ in genau dieser
  festgelegten \emph{Reihenfolge} und z.B.~$(3,2) \not= (2,3)$. Wir
  sprechen von einem $n$-Tupel\index{Tupel}\index{Tupel=$n$-Tupel}.
  % 
  \item $A_1 \times A_2 \times \dots \times
  A_n \eqd \set{(\enu{a}{1}{n}) \mid a_1 \in A_1, a_2 \in A_2, \dots
  ,a_n \in A_n }$ (Kreuzprodukt der Mengen $A_1, A_2, \dots ,
  A_n$)\index{Kreuzprodukt}\index{$\times$},
  %
  \item $A^n \eqd \underbrace{A \times A \times \dots \times
  A}_{n\text{-mal}}$ ($n$-faches Kreuzprodukt der Menge $A$)\index{$A^n$} und
  %
  \item speziell gilt $A^1 = \set{(a) \mid a \in A}$.
  %
\end{itemize}
Wir nennen $2$-Tupel auch \emph{Paare}\index{Paar}, $3$-Tupel
auch \dindex{Tripel}, $4$-Tupel auch \dindex{Quadrupel} und $5$-Tupel 
\dindex{Quintupel}. Bei $n$-Tupeln ist, im Gegensatz zu Mengen, eine 
Reihenfolge vorgegeben, d.h.~es gilt z.B.~immer $\set{a,b} = \set{b,a}$, aber 
im Allgemeinen $(a,b) \not= (b,a)$.

\begin{example}
Sei $A = \set{1, 2, 3}$ und $B = \set{a, b, c}$, dann bezeichnet das
Kreuzprodukt von $A$ und $B$ die Menge von Paaren $A \times B =
\set{(1,a), (1,b), (1,c), (2,a), (2,b), (2,c),\allowbreak (3,\allowbreak a), (3,b), (3,c)}$.
\end{example}

\subsubsection{Die Anzahl von Elementen in Mengen}
\label{cntSet}
Sei $A$ eine Menge, die endlich viele Elemente\footnote{Solche Mengen
werden als \dindex{endliche Mengen}\index{Menge!endliche} bezeichnet.}
enthält, dann ist
\begin{displaymath}
\cnt A \eqd \text{Anzahl der Elemente in der Menge $A$}.
\end{displaymath}
\noindent Beispielsweise ist $\cnt \set{4,7,9} = 3$. Mit dieser Definition gilt

\begin{itemize}
%
\item $\cnt(A^n) = (\cnt A)^n$\index{$\cnt$},
%
\item $\cnt \PowerSet{A} = 2^{\cnt A}$,
%
\item $\cnt A + \cnt B = \cnt(A \cup B) + \cnt (A \cap B)$ und
%
\item $\cnt A = \cnt (A \setminus B) + \cnt(A \cap B)$.
%
\end{itemize}

\subsection{Relationen und Funktionen}

\subsubsection{Eigenschaften von Relationen}
\label{PropRel}

Seien $\enu{A}{1}{n}$ beliebige Mengen, dann ist $R$ eine
\emph{$n$-stellige Relation}\index{Relation} \gdw 
\ $R \subseteq A_1 \times A_2 \times \dots \times A_n$. Eine
zweistellige Relation nennt man auch \dindex{binäre
Relation}\index{Relation!binär}. Oft werden auch Relationen
$R \subseteq A^n$ betrachtet, diese bezeichnet man dann als
$n$-stellige Relation über der Menge $A$.

\begin{definition}
Sei $R$ eine zweistellige Relation über $A$, dann ist $R$
\begin{itemize}
%
\item \dindex{reflexiv} \gdw \ $(a,a) \in R$ für alle $a \in A$,
%
\item \dindex{symmetrisch} \gdw \ aus $(a,b) \in R$ folgt $(b,a) \in R$,
%
\item \dindex{antisymmetrisch} \gdw \ aus $(a,b) \in R$ und $(b,a) \in R$ folgt $a =
b$,
%
\item \dindex{transitiv} \gdw \ aus $(a,b) \in R$ und $(b,c) \in R$ folgt $(a,c)
\in R$ und 
%
\item \dindex{linear} \gdw \ es gilt immer $(a,b) \in R$ oder $(b, a) \in R$.
%
\item Wir nennen $R$ eine \dindex{Halbordnung} \gdw $R$ ist reflexiv,
antisymmetrisch und transitiv,
%
\item eine \dindex{Ordnung} \gdw $R$ ist eine lineare Halbordnung und
%
\item eine \emph{Äquivalenzrelation}\index{Aquivalenzrelation=Äquivalenzrelation} 
\gdw $R$ reflexiv, transitiv und symmetrisch ist.
%
\end{itemize}
\end{definition}

\begin{example}
Die Teilmengenrelation "`$\subseteq$"' auf allen Teilmengen von $\Z$ ist
eine Halbordnung, aber keine Ordnung. 
\end{example}
\goodbreak

\begin{example}
Wir schreiben $a \equiv b \mod
n$, falls es eine ganze Zahl $q$ gibt, für die $a - b = q n$ gilt. Für $n \ge 2$
ist die Relation $R_n(a,b) \eqd \set{(a,b) \mid a \equiv b \mod n} \subseteq
\Z^2$ eine Äquivalenzrelation.
\end{example}

\subsubsection{Eigenschaften von Funktionen}
\label{PropFunc}
Seien $A$ und $B$ beliebige Mengen. $f$ ist eine \dindex{Funktion} von $A$ nach
$B$ (Schreibweise: $f \colon A \rightarrow B$) \gdw \ $f \subseteq A \times
B$ und für jedes $a \in A$ gibt es \emph{höchstens} ein $b \in B$ mit
$(a, b) \in f$. Ist also $(a,b) \in f$, so schreibt man $f(a) =
b$. Ebenfalls gebrächlich ist die Notation $a \mapsto b$.

\begin{remark}
Unsere Definition von Funktion umfasst auch mehrstellige
Funktionen. Seien $C$ und $B$ Mengen und $A = C^n$ das $n$-fache
Kreuzprodukt von $C$. Die Funktion $f \colon A \rightarrow B$ ist dann
eine $n$-stellige Funktion, denn sie bildet $n$-Tupel aus $C^n$ auf Elemente
aus $B$ ab.
\end{remark}

\begin{definition}
Sei $f$ eine $n$-stellige Funktion. Möchte man die Funktion $f$
benutzen, aber keine Namen für die Argumente vergeben, so
schreibt man auch 
\begin{displaymath}
f(\underbrace{\cdot, \cdot, \ldots , \cdot}_{\text{$n$-mal}})
\end{displaymath}
Ist also der Namen des Arguments einer einstelligen Funktion $g(x)$
für eine Betrachtung unwichtig, so kann man
$g(\cdot)$ \index{$f(\cdot)$} schreiben, um anzudeuten, dass $g$
einstellig ist, ohne dies weiter zu erwähnen.
\end{definition}

Sei nun $R \subseteq A_1 \times A_2 \times \dots \times A_n$ eine
$n$-stellige Relation, dann definieren wir $P^n_R \colon A_1 \times
A_2 \times \dots \times A_n \rightarrow \set{0,1}$ wie folgt:

\begin{displaymath}
P^n_R(\enu{x}{1}{n}) \eqd 
\left\{
\begin{array}{rl}
1,& \text{ falls $(\enu{x}{1}{n}) \in R$}\\
0,& \text{ sonst} 
\end{array}
\right.
\end{displaymath}
Eine solche ($n$-stellige) Funktion, die "`anzeigt"', ob ein Element 
aus $A_1 \times A_2 \times \dots \times A_n$ entweder zu $R$ gehört 
oder nicht, nennt man ($n$-stelliges) \dindex{Prädikat}.

\begin{example}
Sei $\mathbb{P} \eqd \set{n \in \N \mid \text{$n$ ist Primzahl}}$, dann
ist $\mathbb{P}$ eine $1$-stellige Relation über den natürlichen Zahlen. 
Das Prädikat $P^1_{\mathbb{P}}(n)$ liefert für eine natürliche Zahl
$n$ genau dann $1$, wenn $n$ eine Primzahl ist.
\end{example}

Ist für ein Prädikat $P^n_R$ sowohl die Relation $R$ als auch die
Stelligkeit $n$ aus dem Kontext klar, dann schreibt man auch kurz $P$
oder verwendet das Relationensymbol $R$ als Notation für das Prädikat
$P^n_R$. 

\bigskip

\noindent Nun legen wir zwei spezielle Funktionen fest, die oft sehr
hilfreich sind:
\begin{definition}
\label{floorceil}
Sei $\alpha \in \R$ eine beliebige reelle Zahl, dann gilt
\begin{itemize}
%
\item $\lceil \alpha \rceil \eqd \text{die kleinste ganze Zahl, die größer
oder gleich $\alpha$ ist}$ ($\triangleq$ "`Aufrunden"') \index{$\lceil \cdot \rceil$}
%
\item $\lfloor \alpha \rfloor \eqd \text{die größte ganze Zahl, die kleiner
oder gleich $\alpha$ ist}$ ($\triangleq$ "`Abrunden"') \index{$\lfloor \cdot \rfloor$}
%
\end{itemize}
\end{definition}

\begin{definition}
Für eine beliebige Funktion $f$ legen wir fest:
\begin{itemize}
%
\item Der \dindex{Definitionsbereich} von $f$ ist $D_f \eqd
\set{a \mid \text{es gibt ein $b$ mit $f(a) = b$}}$.
%
\item Der \dindex{Wertebereich} von $f$ ist $W_f \eqd
\set{b \mid \text{es gibt ein $a$ mit $f(a) = b$}}$.
%
\item Die Funktion $f \colon A \rightarrow B$ ist \dindex{total} \gdw $D_f
= A$.
% 
\item Die Funktion $f \colon A \rightarrow B$ heißt \dindex{surjektiv} \gdw $W_f = B$.
%
\item Die Funktion $f$ heißt \dindex{injektiv} (oder
eineindeutig\footnote{Achtung: Dieser Begriff wird manchmal
unterschiedlich, je nach Autor, in den Bedeutungen "`bijektiv"' oder
"`injektiv"' verwendet.}) \gdw\ immer wenn $f(a_1)\allowbreak =
f(a_2)$ gilt auch $a_1 = a_2$.
%
\item Die Funktion $f$ heißt \dindex{bijektiv} \gdw $f$ ist injektiv und surjektiv.
\end{itemize}
\end{definition}
Mit Hilfe der Kontraposition (siehe Abschnitt \ref{KontraPos}) kann
man für die Injektivität alternativ auch zeigen, dass immer wenn
$a_1 \not= a_2$, dann muss auch $f(a_1) \not= f(a_2)$ gelten.

\begin{example}
Sei die Funktion $f \colon \N \rightarrow \Z$ durch $f(n) = (-1)^n
\lceil \frac{n}{2} \rceil$ gegeben. Die Funktion $f$ ist surjektiv,
denn $f(0) = 0, f(1) = -1, f(2) = 1, f(3) = -2, f(4) = 2, \dots$, d.h.~die 
ungeraden natürlichen Zahlen werden auf die negativen ganzen Zahlen 
abgebildet, die geraden Zahlen aus $\N$ werden auf die positiven
ganzen Zahlen abgebildet und deshalb ist $W_f = \Z$.

Weiterhin ist $f$ auch injektiv, denn aus\footnote{Für die Definition
der Funktion $\lceil \cdot \rceil$ siehe Definition \ref{floorceil}.}
$(-1)^{a_1} \lceil \frac{a_1}{2} \rceil = (-1)^{a_2}
\lceil \frac{a_2}{2} \rceil$ folgt, dass entweder $a_1$ und $a_2$
gerade oder $a_1$ und $a_2$ ungerade, denn sonst würden auf der linken
und rechten Seite der Gleichung unterschiedliche Vorzeichen
auftreten. Ist $a_1$ gerade und $a_2$ gerade, dann gilt
$\lceil \frac{a_1}{2} \rceil = \lceil \frac{a_2}{2} \rceil$ und auch
$a_1 = a_2$. Sind $a_1$ und $a_2$ ungerade, dann gilt
$-\lceil \frac{a_1}{2} \rceil = -\lceil \frac{a_2}{2} \rceil$, woraus
auch folgt, dass $a_1 = a_2$.
%
Damit ist die Funktion $f$ bijektiv. Weiterhin ist $f$ auch total,
d.h.~$D_f = \N$.
\end{example}

\begin{definition}
Unter einem $n$-stelligen \dindex{Operator} $f$ (auf der Menge $Y$) versteht man in der Mathematik eine Funktion der Form $f \colon Y^n \rightarrow Y$.  Einfache Beispiele für zweistellige Operatoren sind der Additions- oder Multiplikationsoperator.
\end{definition}

\subsubsection{Hüllenoperatoren}

\begin{definition}
Sei $X$ eine Menge. Ein einstelliger Operator $\Psi \colon \PowerSet{X} \rightarrow \PowerSet{X}$ heißt \dindex{Hüllenoperator}\index{Operator!Hüllen}, wenn er die folgenden drei Eigenschaften erfüllt:

\begin{description}
%
\item[Einbettung:] für alle $A \in \PowerSet{X}$ gilt $A \subseteq \Psi(A))$
%
\item[Monotonie:] für alle $A,B \in \PowerSet{X}$ mit $A \subseteq B$ folgt $\Psi(A) \subseteq \Psi(B)$
%
\item[Abgeschlossenheit:] für alle $A \in \PowerSet{X}$ gilt $\Psi(\Psi(A)) = \Psi(A)$
%
\end{description}
\end{definition}

Aufgrund der Monotonieeigenschaft eines Hüllenoperators kann man bei der Abgeschlossenheit die Eigenschaft $\Psi(\Psi(A)) = \Psi(A)$ auch durch $\Psi(\Psi(A)) \subseteq \Psi(A)$ ersetzen. In der Informatik spielen Hüllenoperatoren eine große Rolle. Gute Beispiele hierfür sind z.B.~die \dindex{transitive Hülle} (vgl.~Computergraphik) oder die Kleene-Hülle (vgl.~Formale Sprachen). 

\ifdiscretemath
%
% Remove the subsubsection
%
\else

\subsubsection{Permutationen}
\label{Permutationen}
Sei $S$ eine beliebige endliche Menge, dann heißt eine bijektive Funktion $\pi$ der Form 
$\pi \colon S \rightarrow S$ \dindex{Permutation}\index{$\pi$}. Das bedeutet, dass die
Funktion $\pi$ Elemente aus $S$ wieder auf Elemente aus $S$ abbildet,
wobei für jedes $b \in S$ ein $a \in S$ mit $f(a) = b$ existiert
(Surjektivität) und falls $f(a_1) = f(a_2)$ gilt, dann ist $a_1 = a_2$
(Injektivität).

\begin{remark}
 Man kann den Permutationsbegriff auch auf unendliche Mengen erweitern, aber besonders häufig werden in der Informatik \emph{Permutationen von endlichen Mengen} benötigt. Aus diesem Grund sollen hier nur endliche Mengen $S$ betrachtet werden.
\end{remark}

Sei nun $S = \set{\range{1}{n}}$ (eine endliche Menge) und
$\pi \colon \set{\range{1}{n}} \rightarrow \set{\range{1}{n}}$ eine
Permutation. Permutationen dieser Art kann man sehr anschaulich mit
Hilfe einer Matrix aufschreiben:

\begin{displaymath}
\pi = \left( 
\begin{array}{cccc}
1 & 2 & \dots & n\\
\pi(1) & \pi(2) & \dots & \pi(n)
\end{array}
\right)
\end{displaymath}
Durch diese Notation wird klar, dass das Element $1$ der Menge $S$
durch das Element $\pi(1)$ ersetzt wird, das Element $2$ wird mit
$\pi(2)$ vertauscht und allgemein das Element $i$ durch $\pi(i)$ für
$1 \le i \le n$. In der zweiten Zeile dieser Matrixnotation findet
sich also \emph{jedes} (Surjektivität) Element der Menge $S$
genau \emph{einmal} (Injektivität).

\begin{example}
Sei $S = \set{\range{1}{3}}$ eine Menge mit drei Elementen. Dann gibt
es, wie man ausprobieren kann, genau $6$ Permutationen von $S$:

\begin{displaymath}
\begin{array}{rlrlrl}
\pi_1 &= \left( 
\begin{array}{ccc}
1 & 2 & 3\\
1 & 2 & 3
\end{array}
\right)
&
\pi_2 &= \left( 
\begin{array}{ccc}
1 & 2 & 3\\
1 & 3 & 2
\end{array}
\right)
&
\pi_3 &= \left( 
\begin{array}{ccc}
1 & 2 & 3\\
2 & 1 & 3
\end{array}
\right)\\[\bigskipamount]
%
\pi_4 &= \left( 
\begin{array}{ccc}
1 & 2 & 3\\
2 & 3 & 1
\end{array}
\right)
&
\pi_5 &= \left( 
\begin{array}{ccc}
1 & 2 & 3\\
3 & 1 & 2
\end{array}
\right)
&
\pi_6 &= \left( 
\begin{array}{ccc}
1 & 2 & 3\\
3 & 2 & 1
\end{array}
\right)\\
\end{array}
\end{displaymath}
\end{example}

\begin{theorem}
Sei $S$ eine endliche Menge mit $n = |S|$, dann gibt es genau $n!$
(Fakultät) verschiedene Permutationen von $S$.
\end{theorem}

\begin{proof}
Jede Permutation $\pi$ der Menge $S$ von $n$ Elementen kann als Matrix
der Form
\begin{displaymath}
\pi = \left( 
\begin{array}{cccc}
1 & 2 & \dots & n\\
\pi(1) & \pi(2) & \dots & \pi(n)
\end{array}
\right)
\end{displaymath}
aufgeschrieben werden. Damit ergibt sich die Anzahl der Permutationen
von $S$ durch die Anzahl der verschiedenen zweiten Zeilen solcher
Matrizen. In jeder solchen Zeile muss jedes der $n$ Elemente von $S$
genau einmal vorkommen, da $\pi$ eine bijektive Abbildung ist,
d.h.~wir haben für die erste Position der zweiten Zeile der
Matrixdarstellung genau $n$ verschiedene Möglichkeiten, für die zweite
Position noch $n - 1$ und für die dritte noch $n-2$. Für die $n$-te
Position bleibt nur noch $1$ mögliches Element aus $S$
übrig\footnote{Dies kann man sich auch als die Anzahl der
verschiedenen Möglichkeiten vorstellen, die bestehen, wenn man aus
einer Urne mit $n$ numerierten Kugeln alle Kugeln \emph{ohne}
Zurücklegen nacheinander zieht.}. Zusammengenommen haben wir also
$n \cdot (n - 1) \cdot (n - 2) \cdot (n - 3) \multdots 2 \cdot
1 = n!$ verschiedene mögliche Permutationen der Menge $S$.
\qed
\end{proof}

\fi


% Einige Grundlagen ueber Summen und Produkte
\special{pdf: out 3 << /Title 
(Summen und Produkte) 
/Dest [ @thispage /FitH @ypos ] >>}
\subsection{Summen und Produkte}

\special{pdf: out 4 << /Title 
(Summen) 
/Dest [ @thispage /FitH @ypos ] >>}
\subsubsection{Summen}
Zur abk�rzenden Schreibweise verwendet man f�r Summen das
Summenzeichen $\sum$\index{$\sum$}. Dabei ist
\begin{displaymath}
\sum_{i=1}^n a_i \eqd a_1 + a_2 + \dots + a_n.
\end{displaymath}
Mit Hilfe dieser Definition ergeben sich auf elementare Weise die
folgenden Rechenregeln:
\begin{itemize}
%
\item Sei $a_i = a$ f�r $1 \le i \le n$, dann gilt $\sum\limits_{i=1}^n a_i =
  n \cdot a$ (Summe gleicher Summanden).
%
\item $\sum\limits_{i=1}^n a_i = \sum\limits_{i=1}^m a_i +
  \sum\limits_{i = m + 1}^n a_i$, wenn $1 < m < n$ (Aufspalten einer Summe).
%
\item $\sum\limits_{i=1}^n (a_i + b_i + c_i + \dots) =
  \sum\limits_{i=1}^n a_i + \sum\limits_{i=1}^n b_i +
  \sum\limits_{i=1}^n c_i + \dots$ (Addition von Summen).
%
\item $\sum\limits_{i=1}^n a_i = \sum\limits_{i=l}^{n + l - 1} a_{i-l+1}$
  und $\sum\limits_{i=l}^n a_i = \sum\limits_{i=1}^{n - l + 1} a_{i + l
  - 1}$
  (Umnumerierung von Summen).
%
\item $\sum\limits_{i=1}^n \sum\limits_{j=1}^m a_{i,j} =
  \sum\limits_{j=1}^m \sum\limits_{i=1}^n a_{i,j}$ (Vertauschen der Summationsfolge).
%
\end{itemize}

\special{pdf: out 4 << /Title 
(Produkte) 
/Dest [ @thispage /FitH @ypos ] >>}
\subsubsection{Produkte}

Zur abk�rzenden Schreibweise verwendet man f�r Produkte das
Produktzeichen $\prod$\index{$\prod$}. Dabei ist
\begin{displaymath}
\prod_{i=1}^n a_i \eqd a_1 \cdot a_2 \multdots a_n.
\end{displaymath}
Mit Hilfe dieser Definition ergeben sich auf elementare Weise die
folgenden Rechenregeln:
\begin{itemize}
%
\item Sei $a_i = a$ f�r $1 \le i \le n$, dann gilt $\prod\limits_{i=1}^n a_i =
  a^n$ (Produkt gleicher Faktoren).
%
\item  $\prod\limits_{i=1}^n (c a_i) = c^n \prod\limits_{i=1}^n a_i$
  (Vorziehen von konstanten Faktoren)
%
\item $\prod\limits_{i=1}^n a_i = \prod\limits_{i=1}^m a_i \cdot
  \prod\limits_{i = m + 1}^n a_i$ , wenn $1 < m < n$ (Aufspalten in Teilprodukte).
%
\item $\prod\limits_{i=1}^n (a_i \cdot b_i \cdot c_i \cdot \ldots) =
  \prod\limits_{i=1}^n a_i \cdot \prod\limits_{i=1}^n b_i \cdot
  \prod\limits_{i=1}^n c_i \cdot \ldots$ (Das Produkt von Produkten).
%
\item $\prod\limits_{i=1}^n a_i = \prod\limits_{i=l}^{n + l - 1} a_{i-l+1}$
  und $\prod\limits_{i=l}^n a_i = \prod\limits_{i=1}^{n - l + 1} a_{i + l
  - 1}$
  (Umnumerierung von Produkten).
%
\item $\prod\limits_{i=1}^n \prod\limits_{j=1}^m a_{i,j} =
  \prod\limits_{j=1}^m \prod\limits_{i=1}^n a_{i,j}$ (Vertauschen der
  Reihenfolge bei Doppelprodukten).
%
\end{itemize}


% Logarithmen- und Potenzgesetze
\goodbreak
\ifpdf
\special{pdf: out 2 << /Title 
(Logarithmieren, Potenzieren und Radizieren) 
/Dest [ @thispage /FitH @ypos ] >>}
\fi
\subsection{Logarithmieren, Potenzieren und Radizieren}
Die Schreibweise $a^b$ ist eine Abk�rzung f�r 
\begin{displaymath}
a^b \eqd \underbrace{a \cdot a
\cdot \dots \cdot a}_{b-\text{mal}} 
\end{displaymath}
und wird als \dindex{Potenzierung} bezeichnet. Dabei wird $a$ als
\dindex{Basis}, $b$ als \dindex{Exponent} und $a^b$ als $b$-te
\dindex{Potenz} von $a$ bezeichnet.  Seien nun $r,s,t \in \R$ und $r,t
\ge 0$ durch die folgende Gleichung verbunden:
\begin{displaymath}
r^s = t.
\end{displaymath}
Dann l��t sich diese Gleichung wie folgt umstellen und es gelten die
folgenden Rechenregeln\index{Logarithmus}\index{Wurzel}\index{Radizieren}:

\begin{center}
\begin{tabular}{c|c|c}
Logarithmieren & Potenzieren & Radizieren\\
\hline
$\mathbf{s = \log_r t}$ & $\mathbf{t = r^s}$ &
\phantom{$\left(\frac{\frac{c}{d}a}{b}\right)$} $\mathbf{r = \sqrt[\mathbf{s}]{\mathbf{t}}}$\\
\hline
\begin{minipage}[t]{0.38\textwidth}
\begin{enumerate}[i)]
%
\item $\log_r (\frac{t_1}{t_2}) = \log_r t_1 - \log_r t_2$
%
\item $\log_r ({t_1} \cdot {t_2}) = \log_r t_1 + \log_r t_2$
%
\item $\log_r (t^u) = u \cdot \log_r t$
%
\item $\log_r (\sqrt[u]{t}) = \frac{1}{u} \cdot \log_r t$
%
\item $\frac{\log_r t}{\log_r u} = \log_u t$ (Basiswechsel)
%
\end{enumerate}
\end{minipage}
&
\begin{minipage}[t]{0.25\textwidth}
\begin{enumerate}[i)]
%
\item $r^{s_1} \cdot r^{s_2} = r^{s_1 + s_2}$
%
\item $\frac{r^{s_1}}{r^{s_2}} = r^{s_1 - s_2}$
%
\item $r_1^{s} \cdot r_2^{s} = (r_1 \cdot r_2)^{s}$
%
\item $\frac{r_1^{s}}{r_2^{s}} = \left(\frac{r_1}{r_2}\right)^{s}$
%
\item $(r^{s_1})^{s_2} = r^{s_1 \cdot s_2}$ 
%
\end{enumerate}
\end{minipage}
&
\begin{minipage}[t]{0.28\textwidth}
%
\begin{enumerate}[i)]
%
\item $\sqrt[\leftroot{1} s]{t_1} \cdot \sqrt[s]{t_2} = \sqrt[s]{t_1 \cdot t_2}$
%
\item $\frac{\sqrt[s]{t_1}}{\sqrt[s]{t_2}} = \sqrt[s]{\left(\frac{t_1}{t_2}\right)}$
%
\item $\sqrt[\uproot{3} s_1]{\sqrt[\uproot{3} s_2]{t}} =
  \sqrt[\uproot{3} s_1 \cdot s_2]{t}$
%
\end{enumerate}
\end{minipage}\\
\end{tabular}
\end{center}
Zus�tzlich gilt: Wenn $r > 1$, dann ist $s_1 < s_2$ \gdw $r^{s_1} <
r^{s_2}$ (Monotonie).

Da $\sqrt[s]{t} = t^{\left(\frac{1}{s}\right)}$ gilt, k�nnen die
Gesetze f�r das Radizieren leicht aus den Potenzierungsgesetzen
abgeleitet werden.  Weiterhin legen wir spezielle Schreibweisen f�r
die Logarithmen zur Basis $10$, $e$ (Eulersche Zahl) und $2$ fest:
$\lg t \eqd \log_{10} t$, $\ln t \eqd \log_{e} t$ und $\mathrm{lb}\, t \eqd \log_{2} t$. 
 



% Griechische Buchstaben
\ifpdf
\special{pdf: out 2 << /Title 
(Gebr�uchliche griechische Buchstaben) 
/Dest [ @thispage /FitH @ypos ] >>}
\fi
\subsection{Gebr�uchliche griechische Buchstaben}
In er Informatik ist es �blich, griechische Buchstaben zu
verwenden. Ein Grund hierf�r ist, dass es so m�glich wird mit einer
gr��eren Anzahl von Unbekannten arbeiten zu k�nnen, ohne
un�bersichtliche und oft unhandliche Indices benutzen zu m�ssen.
\index{griechische Buchstaben}\index{Buchstaben!griechische}

\bigskip

\noindent Kleinbuchstaben:
\begin{displaymath}
\begin{array}{c|c||c|c||c|c}
\text{Symbol} & \text{Bezeichnung} & \text{Symbol}
& \text{Bezeichnung} & \text{Symbol} & \text{Bezeichnung}\\
\hline
\alpha & \text{Alpha} & \beta   & \text{Beta}   & \gamma   & \text{Gamma}\\
\hline
\delta & \text{Delta} & \phi    & \text{Phi}    & \varphi  & \text{Phi}\\
\hline
\xi    & \text{Xi}    & \zeta   & \text{Zeta}   & \epsilon & \text{Epsilon}\\         
\hline
\theta & \text{Theta} & \lambda & \text{Lambda} & \pi      & \text{Pi}\\
\hline
\sigma & \text{Sigma} & \eta    & \text{Eta}    & \mu      & \text{Mu}
\end{array}
\end{displaymath}

\bigskip

\noindent Grossbuchstaben:
\begin{displaymath}
\begin{array}{c|c||c|c||c|c}
\text{Symbol} & \text{Bezeichnung} & \text{Symbol}
& \text{Bezeichnung} & \text{Symbol} & \text{Bezeichnung}\\
\hline
\Gamma & \text{Gamma} & \Delta & \text{Delta} & \Phi    & \text{Phi}\\
\hline
\Xi    & \text{Xi}    & \Theta & \text{Theta} & \Lambda & \text{Lambda}\\
\hline
\Pi    & \text{Pi}    & \Sigma & \text{Sigma} & \Psi    & \text{Psi}\\
\hline
\Omega & \text{Omega} & &
\end{array}
\end{displaymath}





\cleardoublepage

% Grundlagen der Logik
\special{pdf: out 2 << /Title 
(Einige (wenige) Grundlagen der elementaren Logik) 
/Dest [ @thispage /FitH @ypos ] >>}
\section{Einige (wenige) Grundlagen der elementaren Logik}
\label{BasisLogik}
Aussagen sind entweder \dindex{wahr} ($\triangleq 1$) oder
\dindex{falsch} ($\triangleq 0$). So sind die Aussagen 
\begin{center}
"`Wiesbaden liegt am Mittelmeer"' und "`$1 = 7$"'
\end{center}
sicherlich falsch, wogegen die Aussagen
\begin{center}
"`Wiesbaden liegt in Hessen"' und "`$11 = 11$"'
\end{center}
sicherlich wahr sind. Aussagen werden meist durch
\dindex{Aussagenvariablen} formalisiert, die nur die Werte $0$ oder $1$
annehmen k�nnen. Oft verwendet man auch eine oder mehrere Unbekannte,
  um eine Aussage zu parametrisieren. So k�nnte "`$P(x)$"' etwa f�r
"`Wiesbaden liegt im Bundesland $x$"' stehen,
d.h.~"`$P(\text{Hessen})$"' w�re wahr, wogegen "`$P(\text{Bayern})$"'
eine falsche Aussage ist. Solche Aussagen mit Parameter nennt man auch
\dindex{Pr�dikat}.

Um die Verkn�pfung von Aussagen auch formal aufschreiben zu k�nnen,
werden die folgenden logischen
Operatoren\index{Operator!logisch}\index{logischer Operator} 
verwendet

\begin{center}
\begin{tabular}{c|l|l}
Symbol & umgangssprachlicher Name & Name in der Logik\\
\hline
\dindex{$\sand$} & und & Konjunktion\\
\dindex{$\sor$} & oder & Disjunktion / Alternative\\
\dindex{$\sneg$} & nicht & Negation \\
\dindex{$\simpl$} & folgt & Implikation\\
\dindex{$\sequi$} & genau dann wenn (\emph{\gdw}\index{gdw=\gdw}) & �quivalenz\\
\end{tabular}
\end{center}
Zus�tzlich werden noch die Quantoren \dindex{$\exists$} ("`es existiert"') und
\dindex{$\forall$} ("`f�r alle"') verwendet, die z.B.~wie folgt gebraucht
werden k�nnen
\begin{description}
%
\item $\forall x \colon P(x)$ bedeutet "`F�r alle $x$ gilt die Aussage $P(x)$. 
%
\item $\exists x \colon P(x)$ bedeutet "`Es existiert ein $x$, f�r das die Aussage
  $P(x)$ gilt.
%
\end{description}
Oft l��t man sogar den Doppelpunkt weg und schreibt statt $\forall
x \colon P(x)$ vereinfachend $\forall x P(x)$.

\begin{example}
Die Aussage "`Jede gerade nat�rliche Zahl kann als Produkt von $2$ und einer
anderen nat�rlichen Zahl geschrieben werden"' l�sst sich dann wie
folgt schreiben
\begin{displaymath}
\forall n \in \N \colon ((n \text{ ist gerade}) \simpl (\exists m
\in \N \colon n = 2 \cdot m)) 
\end{displaymath}
Die folgende logische Formel wird wahr \gdw $n$ eine ungerade
nat�rliche Zahl ist.
\begin{displaymath}
\exists m \in \N \colon (n = 2 \cdot m + 1)
\end{displaymath}
\end{example}
F�r die logischen Konnektoren sind die folgenden Wahrheitswertetafeln
festgelegt:

\begin{center}
\begin{tabular}{c||c}
$p$ & $\neg p$\\
\hline
$0$ & $1$\\
$1$ & $0$
\end{tabular}
\hspace*{5em}
und
\hspace*{5em}
\begin{tabular}{c|c||c|c|c|c}
$p$ & $q$ & $p \wedge q$ & $p \vee q$ & $p \simpl q$ & $p \sequi q$\\
\hline
0 & 0 & 0 & 0 & 1 & 1\\   
0 & 1 & 0 & 1 & 1 & 0\\
1 & 0 & 0 & 1 & 0 & 0\\ 
1 & 1 & 1 & 1 & 1 & 1
\end{tabular}
\end{center}
Jetzt kann man Aussagen auch etwas komplexer verkn�pfen:
\begin{example}
Nun wird der $\sand$-Operator verwendet werden. Dazu soll die Aussage 
"`F�r alle nat�rlichen Zahlen $n$ und $m$ gilt, wenn $n$ kleiner
gleich $m$ und $m$ kleiner gleich $n$ gilt, dann ist $m$ gleich $n$"'
\begin{displaymath}
\forall n,m \in \N \colon (((n \le m) \sand (m \le n)) \simpl (n = m))
\end{displaymath}
\end{example}
Oft benutzt man noch den negierten Quantor \dindex{$\nexists$} ("`es existiert kein"'):
\setlength{\marginparwidth}{2.5cm}
\marginpar{
\flushleft\sffamily\tiny
Cubum autem in duos cubos, aut quadrato-quadratum in duos
quadrato-quadratos, et generaliter nullam in infinitum ultra quadratum
potestatem in duos eiusdem nominis fas est dividere cuius rei
demonstrationem mirabilem sane detexi. Hanc marginis exiguitas non
caperet.}
\begin{example}["`Gro�er Satz von Fermat"']
Die Richtigkeit dieser Aussage konnte erst $1994$ nach mehr als $350$
Jahren von Andrew Wiles und Richard Taylor gezeigt werden:
\begin{displaymath}
\forall n\in\N\, \nexists a,b,c \in \N \colon (((n > 2) \wedge
(a \cdot b \cdot c \not= 0)) \simpl a^n + b^n
= c^n)  
\end{displaymath}
F�r den Fall $n=2$ hat die Gleichung $a^n + b^n
= c^n$ ganzzahlige L�sungen (so genannte Pythagor�ische Zahlentripel) wie z.B.~$3^2+4^2=5^2$.
\end{example}



\cleardoublepage

% Einige grundlegende Beweistechniken
\special{pdf: out 2 << /Title 
(Einige formale Grundlagen von Beweistechniken) 
/Dest [ @thispage /FitH @ypos ] >>}
\section{Einige formale Grundlagen von Beweistechniken}
Praktisch arbeitende Informatiker glauben oft v�llig ohne (formale)
Beweistechniken auskommen zu k�nnen. Dabei meinen sie sogar, dass
formale Beweise keinerlei Berechtigung in der Praxis der Informatik
haben und bezeichnen solches Wissen als "`in der Praxis irrelevantes
Zeug, das nur von und f�r seltsame Wissenschaftler erfunden
wurde"'. Oft stellen sie sich auf den Standpunkt, dass die Korrektheit
von Programmen und Algorithmen durch "`Lassen wir es doch mal laufen
und probieren es aus!"' ($\triangleq$ Testen) belegt werden
k�nne. Diese Einstellung zeigt sich oft auch darin, dass Programme mit
Hilfe einer IDE schnell "`testweise"' �bersetzt werden, in der
Hoffnung oder (schlimmer) in der �berzeugung, dass ein �bersetzbares
Programm immer auch semantisch korrekt sei.

Theoretiker, die sich mit den Grundlagen der Informatik besch�ftigen,
vertreten oft den Standpunkt, dass die Korrektheit \emph{jedes}
Programms rigoros \emph{bewiesen} werden muss. Wahrscheinlich ist die
Position zwischen diesen beiden Extremen richtig, denn zum einen ist
der formale Beweis von (gro�en) Programmen oft nicht praktikabel (oder
m�glich) und zum anderen kann das Testen mit einer (relativ kleinen)
Menge von Eingaben sicherlich nicht belegen, dass ein Programm
vollst�ndig den Spezifikationen entspricht. Im praktischen Einsatz ist
es dann oft mit Eingaben konfrontiert, die zu einer fehlerhaften
Reaktion f�hren oder es sogar abst�rzen\footnote{Dies wird
eindrucksvoll durch viele Softwarepakete und verbreitete
Betriebssysteme im PC-Umfeld belegt.} lassen. Bei einfacher
Anwendersoftware sind solche Fehler �rgerlich, aber oft zu
verschmerzen. Bei sicherheitskritischer Software (z.B.~bei der
Regelung von Atomkraftwerken, Airbags und Bremssystemen in Autos, in
der Medizintechnik oder bei der Steuerung von Raumsonden) gef�hrden
solche Fehler menschliches Leben oder f�hren zu extrem hohen
finanziellen Verlusten und m�ssen deswegen unbedingt vermieden werden.

F�r den Praktiker bringen Kenntnisse �ber formale Beweise aber noch
andere Vorteile. Viele Beweise beschreiben direkt den zur L�sung
ben�tigten Algorithmus, d.h.~eigentlich wird die Richtigkeit einer
Aussage durch die (implizite) Angabe eines Algorithmus gezeigt. Aber
es gibt noch einen anderen Vorteil. Ist der umzusetzende Algorithmus
komplex (z.B.~aufgrund einer komplizierten Schleifenstruktur oder
einer verschachtelten Rekursion), so ist es unwahrscheinlich, eine
korrekte Implementation an den Kunden liefern zu k�nnen, ohne die
Hintergr�nde ($\triangleq$ Beweis) verstanden zu haben. All dies
zeigt, dass auch ein praktischer Informatiker Einblicke in
Beweistechniken haben solle. Interessanterweise zeigt die Erfahrung im
praktischen Umfeld auch, dass solches (theoretisches) Wissen �ber die
Hintergr�nde oft zu klarer strukturierten und effizienteren Programmen
f�hrt.

Aus diesen Gr�nden sollen in diesem Abschnitt einige grundlegende
Beweistechniken mit Hilfe von Beispielen vorgestellt werden.

\special{pdf: out 3 << /Title 
(Direkte Beweise)
/Dest [ @thispage /FitH @ypos ] >>}
\subsection{Direkte Beweise}
Um einen direkten Beweis zu f�hren, m�ssen wir, beginnend von einer
initialen Aussage ($\triangleq$ Hypothese), durch Angabe einer Folge
von (richtigen) Zwischenschritten zu der zu beweisenden Aussage
($\triangleq$ Folgerung) gelangen. Jeder Zwischenschritt ist dabei
entweder unmittelbar klar oder muss wieder durch einen weiteren
(kleinen) Beweis belegt werden. Dabei m�ssen nicht alle Schritt v�llig
formal beschrieben werden, sondern es kommt darauf an, dass sich dem
Leser die eigentliche Strategie erschlie�t.

\begin{theorem}
\label{ExpoGTSquare}
Sei $n \in \N$. Falls $n \ge 4$, dann ist $2^n \ge n^2$.
\end{theorem}

Wir m�ssen also, in Abh�ngigkeit des Parameters $n$, die Richtigkeit
dieser Aussage belegen. Einfaches Ausprobieren ergibt, dass $2^4 = 16
\ge 16 = 4^2$ und $2^5 = 32 \ge 25 = 5^2$, d.h.~intuitiv scheint die
Aussage richtig zu sein. Wir wollen die Richtigkeit der Aussage nun
durch eine Reihe von (kleinen) Schritten belegen:

\begin{proof}

Wir haben schon gesehen, dass die Aussage f�r $n = 4$ und $n = 5$
richtig ist. Erh�hen wir $n$ auf $n + 1$, so verdoppelt sich der Wert
der linken Seite der Ungleichung von $2^n$ auf $2 \cdot 2^n =
2^{n+1}$. F�r die rechte Seite ergibt sich ein Verh�ltnis von
$(\frac{n+1}{n})^2$. Je gr��er $n$ wird, desto kleiner wird der Wert
$\frac{n+1}{n}$, d.h.~der maximale Wert ist bei $n = 4$ mit $1.25$
erreicht. Wir wissen $1.25^2 = 1.5625$. D.h.~immer wenn wir $n$ um
eins erh�hen, verdoppelt sich der Wert der linken Seite, wogegen sich
der Wert der rechten Seite um maximal das $1.5625$ fache erh�ht. Damit
muss die linke Seite der Ungleichung immer gr��er als die rechte Seite
sein.\qed
\end{proof}

Dieser Beweis war nur wenig formal, aber sehr ausf�hrlich und wurde durch
das Symbol "`$\#$"' beendet. Im Laufe der Zeit hat es sich eingeb�rgert, 
das Ende eines Beweises mit einem besonderen Marker abzuschlie�en. 
Besonders bekannt ist hier "`$\mathrm{qed}$"'\index{$\mathrm{qed}$}, 
eine Abk�rzung f�r die lateinische Floskel "`quod erat demonstrandum"', 
die mit "`was zu beweisen war"' �bersetzt werden kann. In neuerer Zeit 
werden statt "`$\mathrm{qed}$"' mit der gleichen Bedeutung meist die 
Symbole "`$\Box$"' oder "`$\#$"' \index{$\#$}\index{$\Box$}\index{qed} 
verwendet.

Nun stellt sich die Frage: "`Wie formal und ausf�hrlich muss ein
Beweis sein?"'  Diese Frage kann so einfach nicht beantwortet werden,
denn das h�ngt u.a.~davon ab, welche Lesergruppe durch den Beweis von
der Richtigkeit einer Aussage �berzeugt werden soll und wer den Beweis
schreibt. Ein Beweis f�r ein �bungsblatt sollte auch auf Kleinigkeiten
R�cksicht nehmen, wogegen ein solcher Stil f�r eine wissenschaftliche
Zeitschrift vielleicht nicht angebracht w�re, da die die potentielle
Leserschaft �ber ganz andere Erfahrungen und viel mehr
Hintergrundwissen verf�gt. Nun noch eine Bemerkung zum Thema
"`Formalismus"': Die menschliche Sprache ist unpr�zise, mehrdeutig und
Aussagen k�nnen oft auf verschiedene Weise interpretiert werden. Diese
Defizite sollen Formalismen\footnote{In diesem Zusammenhang sind
Programmiersprachen auch Formalismen, die eine pr�zise Beschreibung
von Algorithmen erzwingen und die durch einen Compiler verarbeitet
werden k�nnen.}  ausgleichen, d.h.~die Antwort muss lauten: "`So viele
Formalismen wie notwendig und so wenige wie m�glich!"'. Durch �bung
und Praxis lernt man die Balance zwischen diesen Anforderungen zu
halten und es zeigt sich bald, dass "`Ge�bte"' die formale
Beschreibung sogar wesentlich leichter verstehen.

\bigskip

Oft kann man andere, schon bekannte, Aussagen dazu verwenden, die
Richtigkeit einer Aussage zu belegen.

\begin{theorem}
\label{ExpoGTSquare2}
Sei $n \in \N$ die Summe von $4$ Quadratzahlen, die gr��er als $0$
sind, dann ist $2^n \ge n^2$.
\end{theorem}

\begin{proof}
Die Menge der Quadratzahlen ist $Q = \set{0, 1, 4, 9, 16, 25, 36,
  \dots}$, d.h.~$1$ ist die kleinste Quadratzahl, die gr��er als $0$
ist. Damit muss unsere Summe von $4$ Quadratzahlen gr��er als $4$
sein. Die Aussage folgt direkt aus Satz \ref{ExpoGTSquare}.
\qed
\end{proof}

\special{pdf: out 4 << /Title 
(Die Kontraposition)
/Dest [ @thispage /FitH @ypos ] >>}
\subsubsection{Die Kontraposition}
\label{KontraPos}
Mit Hilfe von direkten Beweisen haben wir Zusammenh�nge der Form
"`Wenn Aussage $H$ richtig ist, dann folgt daraus die Aussage $C$"'
untersucht. Manchmal ist es schwierig einen Beweis f�r eine solchen
Zusammenhang zu finden. V�llig gleichwertig ist die Behauptung "`Wenn
die Aussage $C$ falsch ist, dann ist die Aussage $H$ falsch"' und oft
ist eine solche Aussage leichter zu zeigen.

Die Kontraposition von Satz \ref{ExpoGTSquare} ist also die folgende
Aussage: "`Wenn nicht $2^n \ge n^2$, dann gilt nicht $n \ge 4$."'. Das
entspricht der Aussage: "`Wenn $2^n < n^2$, dann gilt $n < 4$."', was
offensichtlich zu der urspr�nglichen Aussage von
Satz \ref{ExpoGTSquare} gleichwertig ist.

\special{pdf: out 3 << /Title 
(Widerspruchsbeweise)
/Dest [ @thispage /FitH @ypos ] >>}
\subsection{Widerspruchsbeweise}
\label{IndirektBeweis}
Obwohl die Technik der Widerspruchsbeweise auf den ersten Blick sehr
kompliziert erscheint, ist sie sehr m�chtig und liefert oft sehr kurze
Beweise. Angenommen wir sollen die Richtigkeit einer Aussage "`aus der
Hypothese $H$ folgt $C$"' zeigen. Dazu beweisen wir, dass sich ein
Widerspruch ergibt, wenn wir, von $H$ und der Annahme, dass $C$ falsch
ist, ausgehen. Also war die Annahme falsch, und die Aussage $C$ muss
richtig sein.

Anschaulicher wird diese Beweistechnik durch folgendes Beispiel:
Nehmen wir einmal an, dass Alice eine b�rgerliche Frau ist und deshalb
auch keine Krone tr�gt. Es ist klar, dass jede K�nigin eine Krone
tr�gt. Wir sollen nun beweisen, dass Alice keine K�nigin ist. Dazu
nehmen wir an, dass Alice eine K�nigin ist, d.h.~Alice tr�gt eine
Krone. Dies ist ein Widerspruch! Also war unsere Annahme falsch, und
wir haben gezeigt, dass Alice keine K�nigin sein kann.

\goodbreak
\noindent Der Beweis zu folgendem Satz verwendet diese Technik:
\begin{theorem}
Sei $S$ eine endliche Untermenge einer unendlichen Menge $U$. Sei $T$
das Komplement von $S$ bzgl.~$U$, dann ist $T$ eine unendliche Menge.
\end{theorem}

\begin{proof}
Hier ist unsere Hypothese "`$S$ endlich, $U$ unendlich und $T$
Komplement von $S$ bzgl.~$U$"' und unsere Folgerung ist "`$T$ ist
unendlich"'. Wir nehmen also an, dass $T$ eine endliche Menge ist. Da
$T$ das Komplement von $S$ ist, gilt $S \cap T = \emptyset$, also ist
$\cnt(S) + \cnt(T) = \cnt (S \cap T) + \cnt (S \cup T) = \cnt (S \cup
T) = n$, wobei $n$ eine Zahl aus $\N$ ist (siehe
Abschnitt \ref{cntSet}). Damit ist $S \cup T = U$ eine endliche
Menge. Dies ist ein Widerspruch zu unserer Hypothese! Also war die
Annahme "`$T$ ist endlich"' falsch. \qed
\end{proof}

\special{pdf: out 3 << /Title 
(Der Schubfachschluss)
/Dest [ @thispage /FitH @ypos ] >>}
\subsection{Der Schubfachschluss}
\label{Schubfachschluss}
Der Schubfachschluss ist auch als \dindex{Dirichlets
Taubenschlagprinzip}\index{Taubenschlagprinzip} bekannt. 
Werden $n > k$ Tauben auf $k$ Boxen
verteilt, so gibt es mindestens eine Box in der sich wenigstens zwei
Tauben aufhalten. Allgemeiner formuliert sagt das Taubenschlagprinzip,
dass wenn $n$ Objekte auf $k$ Beh�lter aufgeteilt werden, dann gibt es
mindestens eine Box die $\lceil \frac{n}{k} \rceil$ Objekte enth�lt.

\begin{example}
Auf einer Party unterhalten sich $8$ Personen ($\triangleq$ Objekte),
dann gibt es mindestens einen Wochentag ($\triangleq$ Box) an dem
$\lceil \frac{8}{7} \rceil =2$ Personen aus dieser Gruppe Geburtstag
haben.
\end{example}

\special{pdf: out 3 << /Title 
(Gegenbeispiele)
/Dest [ @thispage /FitH @ypos ] >>}
\subsection{Gegenbeispiele}
Im wirklichen Leben wissen wir nicht, ob eine Aussage richtig oder
falsch ist. Oft sind wir dann mit einer Aussage konfrontiert, die auf
den ersten Blick richtig ist und sollen dazu ein Programm
entwickeln. Wir m�ssen also entscheiden, ob diese Aussage wirklich
richtig ist, denn sonst ist evtl.~alle Arbeit umsonst und hat hohe
Kosten verursacht. In solchen F�llen kann man versuchen, ein einziges
Beispiel daf�r zu finden, dass die Aussage falsch ist, um so unn�tige
Arbeit zu sparen.

\bigskip

\noindent Wir zeigen, dass die folgenden Vermutungen falsch sind:
\begin{conjecture}
Wenn $p \in \N$ eine Primzahl ist, dann ist $p$ ungerade.
\end{conjecture}

\begin{counterexample}
Die nat�rliche Zahl 2 ist eine Primzahl und $2$ ist gerade. \qed
\end{counterexample}

\begin{conjecture}
Es gibt keine Zahlen $a,b \in \N$, sodass $a \textrm{ mod } b = b
\textrm{ mod } a$.
\end{conjecture}

\begin{counterexample}
F�r $a = b = 2$ gilt $a \textrm{ mod } b = b \textrm{ mod } a = 0$. \qed
\end{counterexample}

\special{pdf: out 3 << /Title 
(Induktionsbeweise und das Induktionsprinzip)
/Dest [ @thispage /FitH @ypos ] >>}
\subsection{Induktionsbeweise und das Induktionsprinzip}
Eine der wichtigsten Beweismethoden der Informatik ist das
Induktionsprinzip. Wir wollen jetzt nachweisen, dass f�r jedes $n \in
\N$ eine bestimmte Eigenschaft $E$ gilt. Wir schreiben kurz $E(n)$ f�r
die Aussage "`$n$ besitzt die Eigenschaft $E$"'. Mit der
Schreibweise $E(0)$ dr�cken\footnote{Mit $E$ wird also ein Pr�dikat bezeichnet (siehe
Abschnitt \ref{MengenDef})} wir also aus, dass die
erste nat�rliche Zahl $0$ die Eigenschaft $E$ besitzt.

\noindent\textbf{Induktionsprinzip:} Es gelten 
\indudef%
{$E(0)$}% 
{F�r $n \ge 0$ gilt, wenn $E(k)$ f�r $k \le n$ korrekt ist,
dann ist auch $E(n+1)$ richtig.}

Dabei ist \textbf{\textsf{IA}} die Abk�rzung f�r
\dindex{Induktionsanfang} und \textbf{\textsf{IS}} ist die Kurzform von
\dindex{Induktionsschritt}. Die Voraussetzung ($\triangleq$ Hypothese)
$E(k)$ ist korrekt f�r $k \le n$ und wird im Induktionsschritt
als \dindex{Induktionsvoraussetzung} benutzt
(kurz \textbf{\textsf{IV}}). Hat man also den Induktionsanfang und den
Induktionsschritt gezeigt, dann ist es anschaulich, dass jede nat�rliche Zahl die
Eigenschaft $E$ haben muss.

Es gibt verschiedene Versionen von Induktionsbeweisen. Die bekannteste
Version ist die vollst�ndige Induktion, bei der Aussagen �ber
nat�rliche Zahlen gezeigt werden.

\special{pdf: out 4 << /Title 
(Die vollst�ndige Induktion)
/Dest [ @thispage /FitH @ypos ] >>}
\subsubsection{Die vollst�ndige Induktion}

Wie in Piratenfilmen �blich, seien Kanonenkugeln in einer Pyramide mit
quadratischer Grundfl�che gestapelt. Wir stellen uns die Frage,
wieviele Kugeln (in Abh�ngigkeit von der H�he) in einer solchen
Pyramide gestapelt sind.

\begin{theorem}
\label{Pyramid}
Mit einer quadratische Pyramide aus Kanonenkugeln der H�he $n \ge 1$
als Munition, k�nnen wir $\frac{n(n+1)(2n+1)}{6}$ Sch�sse abgeben.
\end{theorem}

\goodbreak
\begin{proof}
Einfacher formuliert: wir sollen zeigen, dass $\sum\limits_{i=1}^n i^2 =
\frac{n(n+1)(2n+1)}{6}$.
\induproof%
{Eine Pyramide der H�he $n = 1$ enth�lt $\frac{1 \cdot 2 \cdot 3}{6} =
  1$ Kugel. D.h.~wir haben die Eigenschaft f�r $n = 1$ verifiziert.}%
{F�r $k \le n$ gilt $\sum\limits_{i=1}^k i^2 = \frac{k(k+1)(2k+1)}{6}$.}%
{%
Wir m�ssen nun zeigen, dass $\sum\limits_{i=1}^{n+1} i^2 =
\frac{(n+1)((n+1)+1)(2(n+1)+1)}{6}$ gilt und dabei muss die
Induktionsvoraussetzung $\sum\limits_{i=1}^n i^2 = \frac{n(n+1)(2n+1)}{6}$
benutzt werden.  

\begin{displaymath}
\begin{array}{rcl}
\sum\limits_{i=1}^{n+1}i^2 &=& \sum\limits_{i=1}^{n}i^2 + (n + 1)^2\\
&\stackrel{\text{\textbf{\textsf{IV}}}}{=}& \frac{n(n+1)(2n+1)}{6} +
(n^2 + 2n + 1)\\
&=& \frac{2n^3 + 3n^2+n}{6} + (n^2 + 2n + 1)\\
&=& \frac{2n^3 + 9n^2+13n + 6}{6}\\
&=& \frac{(n+1)(2n^2+7n+6)}{6} \quad (\star)\\
&=& \frac{(n+1)(n+2)(2n+3)}{6} \quad (\star\star)\\
&=& \frac{(n+1)((n+1) + 1)(2(n + 1) + 1)}{6}\\
\end{array}
\end{displaymath}
Die Zeile $\star$ (bzw.~$\star\star$) ergibt sich, indem man $2n^3 +
9n^2+13n + 6$ durch $n+1$ teilt (bzw.~$2n^2+7n+6$ durch $n+2$). \qed
}
\end{proof}

\noindent Solche Induktionsbeweise treten z.B.~bei der Analyse von Programmen
immer wieder auf.

\special{pdf: out 4 << /Title 
(Induktive Definitionen)
/Dest [ @thispage /FitH @ypos ] >>}
\subsubsection{Induktive Definitionen}

Das Induktionsprinzip kann aber auch dazu verwendet werden,
(Daten-)Strukturen formal zu spezifizieren. Dazu werden in einem
ersten Schritt ($\triangleq$ Induktionsanfang) die "`atomaren"'
Objekte definiert und dann in einem zweiten Schritt die
zusammengesetzten Objekte ($\triangleq$ Induktionsschritt). Diese
Technik ist als \dindex{induktive Definition} bekannt.

\begin{example}
\noindent Ein Baum ist wie folgt definiert:

\medskip

\indudef{Ein einzelner Knoten $w$ ist ein \emph{Baum} und $w$ ist die
  \emph{Wurzel} dieses Baums.}{Seien $T_1, T_2, \dots, T_n$ B�ume mit den
  Wurzeln $\enu{k}{1}{n}$ und $w$ ein einzelner neuer Knoten. Verbinden wir
  den Knoten $w$ mit allen Wurzeln $\enu{k}{1}{n}$, dann entsteht ein neuer Baum
  mit der Wurzel $w$.} 
\end{example}

\begin{example}
\label{induexp}
\noindent Ein arithmetischer Ausdruck ist wie folgt definiert:

\medskip

\indudef{Jeder Buchstabe und jede Zahl ist ein arithmetischer
Ausdruck.}{Seien $E$ und $F$ Ausdr�cke, so sind auch $E + F$, $E * F$
und $[E]$ Ausdr�cke.}

\medskip

\noindent D.h.~$x$, $x+y$, $[2*x + z]$ sind arithmetische Ausdr�cke,
aber beispielsweise sind $x + $, $yy$, $][x+y$ sowie $x +* z$ keine
Ausdr�cke im Sinn dieser Definition.
\end{example}

Bei diesem Beispiel ahnt man schon, dass solche Techniken zur pr�zisen
Definition von Programmiersprachen und Dateiformaten gute Dienste
leisten. Induktive Definitionen haben noch einen weiteren Vorteil,
denn man kann leicht Induktionsbeweise konstruieren, die Aussagen �ber
induktiv definierte Objekte belegen.

\special{pdf: out 4 << /Title 
(Die strukturelle Induktion)
/Dest [ @thispage /FitH @ypos ] >>}
\subsubsection{Die strukturelle Induktion}

\begin{theorem}
\label{CntBrack}
Die Anzahl der �ffnenden Klammern eines arithmetischen Ausdrucks stimmt
mit der Anzahl der schlie�enden Klammern �berein.
\end{theorem}

Es ist offensichtlich, dass diese Aussage richtig ist, denn in
Ausdr�cken wie $(x + y) / 2$ oder $x + ((y/2) * z)$ muss ja zu jeder
�ffnenden Klammer eine schlie�ende Klammer existieren. Der n�chste
Beweis verwendet diese Idee zum die Aussage von Satz \ref{CntBrack}
mit Hilfe einer strukturellen Induktion zu zeigen.

\begin{proof}
Wir bezeichnen die Anzahl der �ffnenden Klammern eines Ausdrucks $E$
mit $\cnt_[(E)$ und verwenden die analoge Notation $\cnt_](E)$ f�r die
Anzahl der schlie�enden Klammern.

\induproof%
{
Die einfachsten Ausdr�cke sind Buchstaben und Zahlen. Die Anzahl der
�ffnenden und schlie�enden Klammern ist in beiden F�llen gleich $0$.
}
{
Sei $E$ ein Ausdruck, dann gilt $\cnt_[(E) = \cnt_](E)$.
}
{
 F�r einen Ausdruck $E + F$ gilt $\cnt_[(E + F) = \cnt_[(E
) +
 \cnt_[(F) \stackrel{\text{\textbf{\textsf{IV}}}}{=} \cnt_](E) +
 \cnt_](F) = \cnt_](E + F)$. V�llig analog zeigt man dies f�r $E *
 F$. F�r den Ausdruck $[E]$ ergibt sich $\cnt_[([E]) = \cnt_[(E) + 1
 \stackrel{\text{\textbf{\textsf{IV}}}}{=} \cnt_](E) + 1 = \cnt_]([E])$.
 In jedem Fall ist die Anzahl der �ffnenden Klammern gleich der Anzahl
 der schlie�enden Klammern.\qed
}
\end{proof}

\bigskip

Mit Hilfe von Satz \ref{CntBrack} k�nnen wir nun leicht ein Programm
entwickeln, das einen Plausibilit�tscheck (z.B.~direkt in einem Editor)
durchf�hrt und die Klammern z�hlt, bevor die Syntax von arithmetischen
Ausdr�cken �berpr�ft wird. Definiert man eine vollst�ndige
Programmiersprache induktiv, dann werden ganz �hnliche
Induktionsbeweise m�glich, d.h.~man kann die Techniken aus diesem
Beispiel relativ leicht auf die Praxis der Informatik �bertragen.


\cleardoublepage

% Graphentheorie
\ifpdf
\special{pdf: out 2 << /Title 
(Graphen und Graphenalgorithmen) 
/Dest [ @thispage /FitH @ypos ] >>}
\fi
\section{Graphen und Graphenalgorithmen}
\ifpdf
\special{pdf: out 3 << /Title 
(Grundlagen) 
/Dest [ @thispage /FitH @ypos ] >>}
\fi
\subsection{Grundlagen}
Die Theorie der Graphen ist heute zu einem unverzichtbaren Bestandteil
der Informatik geworden. Viele Probleme wie z.B.~das Verlegen von
Leiterbahnen auf einer Platine, die Modellierung von Netzwerken oder
die L�sung von Routingproblemen in Verkehrsnetzen benutzen Graphen
oder Algorithmen, die Graphen als Datenstruktur verwenden. Auch schon
bekannte Datenstrukturen wie Listen und B�umen k�nnen als Graphen
aufgefasst werden. All dies gibt einen Anhaltspunkt, dass die
Graphentheorie eine sehr zentrale Rolle f�r die Informatik spielt und
vielf�ltige Anwendungen hat. In diesem Kontext ist es wichtig zu
bemerken, dass der Begriff des Graphen in der Informatik \emph{nicht}
im Sinne von Graph einer Funktion gebraucht wird, sondern wie folgt
definiert ist:

\begin{definition}
Ein \dindex{gerichteter Graph}\index{Graph!gerichtet} $G = (V,E)$ ist
ein Paar, das aus einer Menge von \dindex{Knoten} $V$ und einer Menge
von \dindex{Kanten} $E \subseteq V \times V$
(\dindex{Kantenrelation}\index{Relation!Kante}) besteht. Eine Kante
$k = (u,v)$ aus $E$ kann als Verbindung zwischen den Knoten $u,v \in
V$ aufgefasst werden. Aus diesem Grund nennt man $u$
auch \dindex{Startknoten}\index{Knoten!Start} und
$v$ \dindex{Endknoten}\index{Knoten!End}. Zwei Knoten, die durch eine
Kante verbunden sind, hei�en auch \dindex{benachbart}
oder \dindex{adjazent}.

Ein Graph $H = (V', E')$ mit $V' \subseteq V$ und $E' \subset E$ hei�t
\dindex{Untergraph} von $G$.
\end{definition}

Ein Graph $(V,E)$ hei�t \dindex{endlich} \gdw die Menge der Knoten $V$
endlich ist. Obwohl man nat�rlich auch unendliche Graphen betrachten
kann, werden wir uns in diesem Abschnitt nur mit endlichen Graphen
besch�ftigen, da diese f�r den Informatiker von gro�em Nutzen sind.

\bigskip

Da wir eine Kante $(u,v)$ als Verbindung zwischen den Knoten $u$ und
$v$ interpretieren k�nnen, bietet es sich an Graphen durch Diagramme
darzustellen. Dabei wird die Kante $(u,v)$ durch einen Pfeil von $u$
nach $v$ dargestellt. Drei Beispiele f�r eine bildliche Darstellung
von Graphen finden sich in Abbildung \ref{gGraphen}.

\ifpdf
\special{pdf: out 3 << /Title 
(Einige Eigenschaften von Graphen) 
/Dest [ @thispage /FitH @ypos ] >>}
\fi
\subsection{Einige Eigenschaften von Graphen}
Der Graph in Abbildung \ref{gGraphen}(c) hat eine besondere
Eigenschaft, denn offensichtlich kann man die Knotenmenge $V_{1c}
= \set{0,1,2,3,4,5,6,7,8}$ in zwei disjunkte Teilmengen $V_{1c}^l
= \set{0,1,2,3}$ und $V_{1c}^r = \set{4,5,6,7,8}$ so aufteilen, dass
keine Kante zwischen zwei Knoten aus $V_{1c}^l$ oder $V_{1c}^r$
verl�uft.

\begin{definition}
Ein Graph $G = (V,E)$ hei�t \dindex{bipartit}, wenn gilt:
\begin{enumerate}
%
\item Es gibt zwei Teilmengen $V^l$ und $V^r$ von $V$ mit $V =
V^l \cup V^r$ und $V^l \cap V^r = \emptyset$ und 
%
\item f�r jede Kante $(u,v) \in E$ gilt $u \in V^l$ und $v \in V^r$.
%
\end{enumerate}
\end{definition}

Bipartite Graphen haben viele Anwendungen, weil man jede bin�re
Relation $R \subseteq A \times B$ nat�rlich als bipartiten Graph
auffassen kann, dessen Kanten von Knoten aus $A$ zu Knoten aus $B$
laufen.

\begin{example}
Gegeben sei ein bipartiter Graph $G = (V,E)$ mit $V = V^F \cup V^M$
und $V^F \cap V^M = \emptyset$. Die Knoten aus $V^F$ symbolisieren
Frauen und $V^M$ symbolisiert eine Menge von M�nnern. Kann sich eine
Frau vorstellen einen Mann zu heiraten, so wird der entsprechende
Knoten aus $V^F$ mit dem passenden Knoten aus $V^M$ durch eine Kante
verbunden.  Eine \dindex{Heirat} ist nun eine Kantenmenge $H \subseteq
E$, so dass keine zwei Kanten aus $H$ einen gemeinsamen Knoten
besitzen. Das \dindex{Heiratsproblem} ist nun die Aufgabe f�r $G$ eine
Heirat $H$ zu finden, so dass alle Frauen heiraten k�nnen, d.h.~es ist
das folgende Problem zu l�sen:

\goodbreak
\prob{MARRIAGE}{%
Bipartiter Graph $G = (V,E)$ mit $V = V^F \cup V^M$ und $V^F \cap V^M
= \emptyset$}{%
Eine Heirat $H$ mit $\cnt H = \cnt V^F$
}

Im Beispielgraphen \ref{gGraphen}(c) gibt es keine L�sung f�r das
Heiratsproblem, denn f�r die Knoten ($\triangleq$ Kandidatinnen) $2$ und
$3$ existieren nicht ausreichend viele Partner, d.h.~keine Heirat in
diesem Graphen enth�lt zwei Kanten die sowohl $2$ als auch $3$ als
Startknoten haben.

\medskip

Obwohl dieses Beispiel auf den ersten Blick nur von untergeordneter
Bedeutung erscheint, kann man es auf eine Vielfalt von Anwendungen
�bertragen. Immer wenn die Elemente zweier disjunkter Mengen durch
eine Beziehung verbunden sind, kann man dies als bipartiten Graphen
auffassen. Sollen nun die Bed�rfnisse der einen Menge v�llig
befriedigt werden, so ist dies wieder ein Heiratsproblem. Beispiele
mit mehr praktischem Bezug finden sich u.a.~bei Beziehungen zwischen
K�ufern und Anbietern.
\end{example}

\begin{figure}
\centering
\subfigure[Ein gerichteter Graph mit $5$
Knoten]{\includegraphics[scale=1.2]{graphex1.eps}}
\hspace*{2em}
\subfigure[Ein planarer gerichteter
Graph mit $5$ Knoten]{\includegraphics[scale=1.2]{graphex3.eps}}
\hspace*{2em}
\subfigure[Ein gerichteter bipartiter 
Graph]{\includegraphics[scale=1.2]{graphex2.eps}}
\caption{Beispiele f�r gerichtete Graphen}
\label{gGraphen}
\end{figure}

Oft beschr�nken wir uns auch auf eine Unterklasse von Graphen, bei
denen die Kanten keine "`Richtung"' haben (siehe
Abbildung \ref{ugGraphen}) und einfach durch eine Verbindungslinie
symbolisiert werden k�nnen:

\begin{definition}
Sei $G=(V,E)$ ein Graph. Ist die Kantenrelation
$E$ \dindex{symmetrisch}, d.h.~gibt es zu jeder Kante $(u,v) \in E$
auch eine Kante $(v,u) \in E$ (siehe auch Abschnitt \ref{PropRel}),
dann bezeichnen wir $G$ als \dindex{ungerichteten
Graphen}\index{Graph!ungerichtet} oder kurz als \dindex{Graph}.
\end{definition}

Es ist praktisch die Kanten $(u,v)$ und $(v,u)$ eines ungerichteten
Graphen als Menge $\set{u,v}$ mit zwei Elementen aufzufassen. Diese
Vorgehensweise f�hrt zu einem kleinen technischen Problem. Eine Kante
$(u,u)$ mit gleichem Start- und Endknoten nennen wir, entsprechend der
intuitiven Darstellung eines Graphens als Diagramm, \dindex{Schleife}.
Wandelt man nun solch eine Kante in eine Menge um, so w�rde nur eine
einelementige Menge entstehen. Aus diesem Grund legen wir fest, dass
ungerichtete Graphen \dindex{schleifenfrei} sind.

\begin{definition}
Der (ungerichtete) Graph $K = (V,E)$ hei�t \dindex{vollst�ndig}, wenn
f�r alle $u,v \in V$ mit $u \neq v$ auch $(u,v) \in E$ gilt,
d.h.~jeder Knoten des Graphen ist mit allen anderen Knoten
verbunden. Ein Graph $O=(V,\emptyset)$ ohne Kanten wird
als \dindex{Nullgraph}\index{Graph!Null} bezeichnet.
\end{definition}
Mit dieser Definition ergibt sich, dass die Graphen in
Abbildung \ref{ugGraphen}(a) und Abbildung \ref{ugGraphen}(b)
vollst�ndig sind. Der Nullgraph $(V,\emptyset)$ ist Untergraph jedes
beliebigen Graphen $(V,E)$. Diese Definitionen lassen sich nat�rlich
auch analog auf gerichtete Graphen �bertragen.

\begin{figure}
\centering
\subfigure[Vollst�ndiger ungerichteter Graph 
$K_{16}$]{\includegraphics[scale=0.7]{kclique.eps}}
\hfill
\subfigure[Vollst�ndiger ungerichteter Graph
$K_{20}$]{\includegraphics[scale=0.7]{kclique3.eps}}
\subfigure[Zuf�lliger Graph mit $32$ Knoten]{\includegraphics[scale=0.7]{random.eps}}
\hfill
\subfigure[Regul�rer Graph mit Grad $3$]{\includegraphics[scale=0.7]{moebius.eps}}
\caption{Beispiele f�r ungerichtete Graphen}
\label{ugGraphen}
\end{figure}

\begin{definition}
Sei $G = (V,E)$ ein gerichteter Graph und $v \in V$ ein beliebiger
Knoten. Der \dindex{Ausgrad} von $v$ (kurz:
$\mathrm{outdeg}(v)$\index{$\mathrm{outdeg}(v)$}) ist dann die Anzahl
der Kanten in $G$, die $v$ als als Startknoten haben. Analog ist
der \dindex{Ingrad} von $v$ 
(kurz: $\mathrm{indeg}(v)$\index{$\mathrm{indeg}(v)$}) die Anzahl der Kanten
in $G$, die $v$ als Endknoten haben. 

Bei ungerichteten Graphen gilt f�r jeden Knoten
$\mathrm{outdeg}(v)=\mathrm{indeg}(v)$. Aus diesem Grund schreiben wir
kurz $\mathrm{deg}(v)$ und bezeichnen dies als \emph{Grad von
$v$}\index{Grad}. 
Ein Graph $G$ hei�t \dindex{regul�r} \gdw alle
Knoten von $G$ den gleichen Grad haben.
\end{definition}
Die Diagramme der Graphen in den Abbildungen \ref{gGraphen}
und \ref{ugGraphen} haben Eigenschaft, dass sich einige Kanten
schneiden. Es stellt sich die Frage, ob man diese Diagramme auch so
zeichnen kann, dass keine �berschneidungen auftreten. Diese
Eigenschaft von Graphen wollen wir durch die folgende Definition
festhalten:
\begin{definition}
Ein Graph $G$ hei�t \dindex{planar}, wenn sich sein Diagramm ohne
�berschneidungen zeichnen l��t.
\end{definition}

\begin{example}
Der Graph in Abbildung \ref{gGraphen}(a) ist, wie man leicht
nachpr�fen kann, planar, da die Diagramme aus
Abbildung \ref{gGraphen}(a) und \ref{gGraphen}(b) den gleichen Graphen
repr�sentieren.
\end{example}
Auch planare Graphen haben eine anschauliche Bedeutung. Der Schaltplan
einer elektronischen Schaltung kann als Graph aufgefasst werden. Die
Knoten entsprechen den Stellen an denen die Bauteile aufgel�tet werden
m�ssen, und die Kanten entsprechen den Leiterbahnen auf der
Platine. In diesem Zusammenhang bedeutet planar nun, ob man die
Leiterbahnen kreuzungsfrei verlegen kann, d.h.~ob es m�glich ist, eine
Platine zu fertigen, die mit einer Kupferschicht auskommt. In der
Praxis kommen oft Platinen mit mehreren Schichten zum Einsatz
("`Multilayer-Platine"'). Ein Grund daf�r kann sein, dass der
"`Schaltungsgraph"' nicht planar war und deshalb mehrere Schichten
ben�tigt werden. Da Platinen mit mehreren Schichten in der Fertigung
deutlich teurer sind als solche mit einer Schicht, hat sich
Planarit�tseigenschaft von Graphen also auch unmittelbare finanzielle
Auswirkungen.

\ifpdf
\special{pdf: out 2 << /Title 
(Wege, Kreise, W�lder und B�ume) 
/Dest [ @thispage /FitH @ypos ] >>}
\fi
\subsection{Wege, Kreise, W�lder und B�ume}

\begin{definition}
Sei $G=(V,E)$ ein Graph und $u,v \in V$. Eine Folge von Knoten
$\enu{u}{0}{l} \in V$ mit $u = u_0$, $v = u_l$ und $(u_i,u_{i+1}) \in
E$ f�r $0 \le i \le l - 1$ hei�t \emph{Weg von $u$ nach $v$ der L�nge
$l$}\index{Weg}. Der Knoten $u$
wird \dindex{Startknoten}\index{Knoten!Start} und $v$
wird \dindex{Endknoten}\index{Knoten!End} des Wegs genannt.

Ein Weg bei dem Start- und Endknoten gleich sind,
hei�t \dindex{geschlossener Weg}\index{Weg!geschlossen}. Ein
geschlossener Weg, bei dem kein Knoten au�er dem Startknoten mehrfach
enthalten ist, wird \dindex{Kreis} genannt.
\end{definition}
Mit dieser Definition wird klar, dass der Graph in
Abbildung \ref{gGraphen}(a) den Kreis $1,2,3,\dots,5,1$ mit
Startknoten $1$ hat.

\begin{definition}
Sei $G=(V,E)$ ein Graph. Zwei Knoten $u,v \in V$
hei�en \dindex{zusammenh�ngend}, wenn es einen Weg von $u$ nach $v$
gibt. Der Graph $G$ hei�t \dindex{zusammenh�ngend}, wenn jeder Knoten
von $G$ mit jedem anderen Knoten von $G$ zusammenh�ngt. 

Sei $G'$ ein zusammenh�ngender Untergraph von $G$ mit einer besonderen
Eigenschaft: Nimmt man einen weiteren Knoten von $G$ zu $G'$ hinzu,
dann ist der neu entstandene Graph nicht mehr zusammenh�ngend, d.h.~es
gibt keinen Weg zu diesem neu hinzugekommenen Knoten. Solch einen
Untergraph nennt man \dindex{Zusammenhangskomponente}.
\end{definition}
Offensichtlich sind die Graphen in den
Abbildungen \ref{gGraphen}(a), \ref{ugGraphen}(a), \ref{ugGraphen}(b)
und \ref{ugGraphen}(d) zusammenh�ngend und sie haben genau eine
Zusammenhangskomponente. Man kann sich sogar leicht �berlegen, dass
die Eigenschaft $u$ h�ngt mit $v$ zusammen
eine \emph{�quivalenzrelation} (siehe Abschnitt \ref{PropRel})
darstellt.

Mit Hilfe der Definition des geschlossenen Wegs l��t sich nun der
Begriff der B�ume definieren, die eine sehr wichtige Unterklasse der
Graphen darstellen.
\goodbreak
\begin{definition}
Ein Graph $G$ hei�t
\begin{itemize}
%
\item \dindex{Wald}, wenn es keinen geschlossenen Weg mit L�nge $\ge
1$ in $G$ gibt und 
%
\item \dindex{Baum}, wenn $G$ ein zusammenh�ngender Wald ist,
d.h.~wenn er nur genau eine Zusammenhangskomponente hat.
%
\end{itemize}
\end{definition}

\ifpdf
\special{pdf: out 2 << /Title 
(Die Repr�sentation von Graphen und einige Algorithmen) 
/Dest [ @thispage /FitH @ypos ] >>}
\fi
\subsection{Die Repr�sentation von Graphen und einige Algorithmen}
Nachdem Graphen eine gro�e Bedeutung sowohl in der praktischen als
auch in der theoretischen Informatik erlangt haben, stellt sich noch
die Frage, wie man Graphen effizient als Datenstruktur in einem
Computer ablegt. Dabei soll es m�glich sein Graphen effizient zu
speichern und zu manipulieren. 

Die erste Idee, Graphen als dynamische Datenstrukturen zu
repr�sentieren, scheitert an dem relativ ineffizienten Zugriff auf die
Knoten und Kanten bei dieser Art der Darstellung. Sie ist nur von
Vorteil, wenn ein Graph nur sehr wenige Kanten enth�lt. Die folgende
Methode der Speicherung von Graphen hat sich als effizient erwiesen
und erm�glicht auch die leichte Manipulation des Graphens:
\begin{definition}
Sei $G=(V,E)$ ein gerichteter Graph mit $V = \set{\enu{v}{1}{n}}$. Wir
definieren eine $n \times n$ Matrix $A_G =(a_{i,j})_{1 \le i,j, \le
n}$ durch
\begin{displaymath}
a_{i,j} =
\left\{
\begin{array}{ll}
1,& \text{ falls $(v_i, v_j) \in E$}\\
0,& \text{ sonst}
\end{array}
\right.
\end{displaymath}
Die so definierte Matrix $A_G$ mit Eintr�gen aus der Menge $\set{0,1}$
hei�t \dindex{Adjazenzmatrix} von $G$.
\end{definition}

\begin{example}
F�r den gerichteten Graphen aus Abbildung \ref{gGraphen}(a) ergibt sich die
folgende Adjazenzmatrix:
\begin{displaymath}
A_{G_{1a}} =
\left(
\begin{array}{ccccc}
0 & 1 & 0 & 0 & 0\\ 
0 & 0 & 1 & 0 & 0\\
0 & 0 & 0 & 1 & 0\\
0 & 0 & 0 & 0 & 1\\
1 & 0 & 0 & 0 & 0
\end{array}
\right)
\end{displaymath}
Die Adjazenzmatrix eines ungerichteten Graphen erkennt man daran, dass
sie spiegelsymmetrisch zu Diagonale von links oben nach rechts unten
ist (die Kantenrelation\index{Kantenrelation} ist symmetrisch) und
dass die Diagonale aus $0$-Eintr�gen besteht (der Graphen hat keine
Schleifen). F�r den vollst�ndigen Graphen $K_{16}$ aus
Abbildung \ref{ugGraphen}(a) ergibt sich offensichtlich die folgende Adjazenzmatrix:
\begin{displaymath}
A_{G_{1a}} =
\left(
\begin{array}{cccccccccccccccc}
0 & 1 & 1 & 1 & 1 & 1 & 1 & 1 & 1 & 1 & 1 & 1 & 1 & 1 & 1 & 1\\ 
1 & 0 & 1 & 1 & 1 & 1 & 1 & 1 & 1 & 1 & 1 & 1 & 1 & 1 & 1 & 1\\ 
1 & 1 & 0 & 1 & 1 & 1 & 1 & 1 & 1 & 1 & 1 & 1 & 1 & 1 & 1 & 1\\ 
1 & 1 & 1 & 0 & 1 & 1 & 1 & 1 & 1 & 1 & 1 & 1 & 1 & 1 & 1 & 1\\ 
1 & 1 & 1 & 1 & 0 & 1 & 1 & 1 & 1 & 1 & 1 & 1 & 1 & 1 & 1 & 1\\ 
1 & 1 & 1 & 1 & 1 & 0 & 1 & 1 & 1 & 1 & 1 & 1 & 1 & 1 & 1 & 1\\ 
1 & 1 & 1 & 1 & 1 & 1 & 0 & 1 & 1 & 1 & 1 & 1 & 1 & 1 & 1 & 1\\ 
1 & 1 & 1 & 1 & 1 & 1 & 1 & 0 & 1 & 1 & 1 & 1 & 1 & 1 & 1 & 1\\ 
1 & 1 & 1 & 1 & 1 & 1 & 1 & 1 & 0 & 1 & 1 & 1 & 1 & 1 & 1 & 1\\ 
1 & 1 & 1 & 1 & 1 & 1 & 1 & 1 & 1 & 0 & 1 & 1 & 1 & 1 & 1 & 1\\ 
1 & 1 & 1 & 1 & 1 & 1 & 1 & 1 & 1 & 1 & 0 & 1 & 1 & 1 & 1 & 1\\ 
1 & 1 & 1 & 1 & 1 & 1 & 1 & 1 & 1 & 1 & 1 & 0 & 1 & 1 & 1 & 1\\ 
1 & 1 & 1 & 1 & 1 & 1 & 1 & 1 & 1 & 1 & 1 & 1 & 0 & 1 & 1 & 1\\ 
1 & 1 & 1 & 1 & 1 & 1 & 1 & 1 & 1 & 1 & 1 & 1 & 1 & 0 & 1 & 1\\ 
1 & 1 & 1 & 1 & 1 & 1 & 1 & 1 & 1 & 1 & 1 & 1 & 1 & 1 & 0 & 1\\ 
1 & 1 & 1 & 1 & 1 & 1 & 1 & 1 & 1 & 1 & 1 & 1 & 1 & 1 & 1 & 0
\end{array}
\right)
\end{displaymath}
\end{example}
Mit Hilfe der Adjazenzmatrix und Algorithmus \ref{Reach} kann man
leicht berechnen, ob ein Weg von einem Knoten $u$ zu einem Knoten $v$
existiert. Mit einer ganz �hnlichen Idee kann man auch leicht die
Anzahl der Zusammenhangskomponenten berechnen (siehe
Algorithmus \ref{Kompo}). Dieser Algorithmus markiert die Knoten der
einzelnen Zusammenhangskomponenten auch mit unterschiedlichen
"`Farben"', die hier durch Zahlen repr�sentiert werden.

\restylealgo{ruled}
\begin{algorithm}
\caption{Erreichbarkeit in Graphen}
\label{Reach}
\KwData{Ein Graph $G=(V,E)$ und zwei Knoten $u,v \in V$}
\KwResult{\texttt{true} wenn es einen Weg von $u$ nach $v$
gibt, \texttt{false} sonst}
\BlankLine
\dontprintsemicolon

markiert = \texttt{true}\;
markiere Startknoten $u \in V$\;

\BlankLine
\dontprintsemicolon

\While{(markiert)}{

markiert = \texttt{false}\;

\For{(alle markierten Knoten $w \in V$)}{

\If{($w \in V$ ist adjazent zu einem unmarkierten Knoten $w' \in V$)}{
markiere Knoten $w'$\;
markiert = \texttt{true}\;
}

}

}

\eIf{($v$ ist markiert)}{
\Return \texttt{true}\;
}{
\Return \texttt{false}\;
}

\printsemicolon
\end{algorithm}


\restylealgo{ruled}
\begin{algorithm}
\caption{Zusammenhangskomponenten}
\label{Kompo}
\KwData{Ein Graph $G=(V,E)$}
\KwResult{Anzahl der Zusammenhangskomponenten von $G$}
\BlankLine
\dontprintsemicolon

kFarb = 0\;

\BlankLine
\dontprintsemicolon

\While{(es gibt einen unmarkierten Knoten $u \in V$)}{

kFarb++\;
markiere $u \in V$ mit kFarb\;
\BlankLine
\dontprintsemicolon

markiert=\texttt{true}\;

\While{(markiert)}{

markiert=\texttt{false}\;
\BlankLine
\dontprintsemicolon

\For{(alle mit kFarb markierten Knoten $v \in V$)}{

\If{($v \in V$ ist adjazent zu einem unmarkierten Knoten $v' \in V$)}{
markiere Knoten $v' \in V$ mit kFarb\;
markiert=\texttt{true}\;
}

}

}

}

\Return kFarb\;
\printsemicolon
\end{algorithm}

\begin{definition}
Sei $G = (V,E)$ ein ungerichteter Graph. Eine Funktion $f \colon
V \rightarrow \set{\range{1}{k}}$
hei�t \dindex{$k$-F�rbung}\index{F�rbung} des Graphen $G$. Anschaulich
ordnet die Funktion $f$ jedem Knoten eine von $k$ verschiedenen Farben
zu, die hier durch die Zahlen $\range{1}{k}$ symbolisiert werden. Eine
F�rbung hei�t \emph{vertr�glich}\index{F�rbung!vertr�glich}, wenn f�r
alle Kanten $(u,v) \in E$ gilt, dass $f(u) \neq f(v)$, d.h.~zwei
adjazente Knoten werden nie mit der gleichen Farbe markiert.
\end{definition}

Auch das F�rbbarkeitsproblem spielt in der Praxis der Informatik eine
wichtige Rolle. Ein Beispiel daf�r ist die Planung eines
Mobilfunknetzes. Dabei werden die Basisstationen eines Mobilfunknetzes
als Knoten eines Graphen repr�sentiert. Zwei Knoten werden mit einer
Kante verbunden, wenn Sie geographisch so verteilt sind, dass sie sich
beim Senden auf der gleichen Frequenz gegenseitig st�ren
k�nnen. Existiert eine vertr�gliche $k$-F�rbung f�r diesen Graphen, so
ist es m�glich, ein st�rungsfreies Mobilfunktnetz mit $k$
verschiedenen Funkfrequenzen aufzubauen. Dabei entsprechen die Farben
den verf�gbaren Frequenzen. Bei der Planung eines solchen
Mobilfunknetzes ist also das folgende Problem zu l�sen:
\dprob{COLORABILIY}{
Ein ungerichteter Graph $G$ und eine Zahl $k \in \N$.
}
{
Gibt es eine vertr�gliche F�rbung von $G$ mit $k$ Farben?
}
Dieses Problem geh�rt zu einer (sehr gro�en) Klasse von (praktisch
relevanten) Problemen, f�r die bis heute keine effizienten Algorithmen
bekannt sind (Stichwort: \NP-Vollst�ndigkeit). Vielf�ltige Ergebnisse
der Theoretischen Informatik zeigen sogar, dass man nicht hoffen darf,
dass ein schneller Algorithmus zur L�sung des F�rbbarkeitsproblems
existiert.


\cleardoublepage

% Grundlagen der Komplexittstheorie
\ifpdf
\special{pdf: out 2 << /Title 
(Komplexit�t von Algorithmen) 
/Dest [ @thispage /FitH @ypos ] >>}
\fi

\section{Komplexit�t von Algorithmen}
\label{KomplexSect}

F�r viele st�ndig auftretende Berechnungsprobleme, wie Sortieren, die
arithmetischen Operationen (Addition, Multiplikation, Division),
Fourier-Transformation etc., sind sehr effiziente Algorithmen
konstruiert worden. F�r wenige andere praktische Probleme wei� man,
dass sie nicht oder nicht effizient algorithmisch l�sbar sind. Im
Gegensatz dazu ist f�r einen sehr gro�en Teil von Fragestellungen aus
den verschiedensten Anwendungsbereichen wie Operations Research,
Netzwerkdesign, Programmoptimierung, Datenbanken,
Betriebssystem-Entwicklung und vielen mehr jedoch nicht bekannt, ob
sie effiziente Algorithmen besitzen (vgl.~die Abbildungen
\ref{NPVollProbs1} und \ref{NPVollProbs2}). Diese Problematik h�ngt
mit der seit �ber drei�ig Jahren offenen $\P\stackrel{?}{=}\NP$-Frage
zusammen, wahrscheinlich gegenw�rtig das wichtigste ungel�ste Problem
der theoretischen Informatik. Es wurde sogar k�rzlich auf Platz 1 der
Liste der so genannten \emph{Millenium Prize Problems} des Clay
Mathematics Institute gesetzt\footnote{siehe http://www.claymath.org/millennium}.  
Diese Liste umfasst sieben offene Probleme aus der gesamten
Mathematik. Das Clay Institute zahlt jedem, der eines dieser Probleme 
l�st, eine Million US-Dollar.

In diesem Abschnitt werden die wesentlichen Begriffe aus dem Kontext
des $\P\stackrel{?}{=}\NP$-Problems und des Begriffes der
$\NP$-Vollst�ndigkeit erl�utert, um so die Grundlagen f�r das
Verst�ndnis derartiger Probleme zu schaffen und deren Beurteilung zu
erm�glichen.

\ifpdf
\special{pdf: out 3 << /Title 
(Effizient l�sbare Probleme: die Klasse P) 
/Dest [ @thispage /FitH @ypos ] >>}
\fi
\subsection{Effizient l�sbare Probleme: die Klasse $\P$}

Jeder, der schon einmal mit der Aufgabe konfrontiert wurde, einen
Algorithmus f�r ein gegebenes Problem zu entwickeln, kennt die
Hauptschwierigkeit dabei: Wie kann ein effizienter Algorithmus
gefunden werden, der das Problem mit m�glichst wenigen Rechenschritten
l�st? Um diese Frage beantworten zu k�nnen, muss man sich zun�chst
einige Gedanken �ber die verwendeten Begriffe, n�mlich "`Problem"',
"`Algorithmus"', "`Zeitbedarf"' und "`effizient"', machen.

Was ist ein "`Problem?"' Jedem Programmierer ist diese Frage intuitiv
klar: Man bekommt geeignete Eingaben, und das Programm soll die
gew�nschten Ausgaben ermitteln. Ein einfaches Beispiel ist das Problem
MULT. (Jedes Problem soll mit einem eindeutigen Namen versehen und
dieser in Gro�buchstaben geschrieben werden.) Hier bekommt man zwei
ganze Zahlen als Eingabe und soll das Produkt beider Zahlen berechnen,
d.h.~das Programm berechnet einfach eine zweistellige Funktion. Es hat
sich gezeigt, dass man sich bei der Untersuchung von Effizienzfragen
auf eine abgeschw�chte Form von Problemen beschr�nken kann, n�mlich
sogenannte \dindex{Entscheidungsprobleme}. Hier ist die Aufgabe, eine
gew�nschte Eigenschaft der Eingaben zu testen. Hat die aktuelle
Eingabe die gew�nschte Eigenschaft, dann gibt man den Wert $1$
($\triangleq$ \texttt{true}) zur�ck
(man spricht dann auch von einer 
\dindex{positiven Instanz}\index{Instanz!positiv} des Problems), hat 
die Eingabe die Eigenschaft nicht, dann gibt man den Wert $0$
($\triangleq$ \texttt{false}) zur�ck. Oder anders formuliert: Das
Programm berechnet eine Funktion, die den Wert 0 oder 1 zur�ck gibt
und partitioniert damit die Menge der m�glichen Eingaben in zwei
Teile: die Menge der Eingaben mit der gew�nschten Eigenschaft und die
Menge der Eingaben, die die gew�nschte Eigenschaft nicht
besitzen. Folgendes Beispiel soll das Konzept verdeutlichen:

\prob{PARITY}
{Positive Integerzahl $x$}
{Ist die Anzahl der Ziffern 1 in der Bin�rdarstellung von $x$ ungerade?}

Es soll also ein Programm entwickelt werden, das die Parit�t einer
Integerzahl $x$ berechnet. Eine m�gliche Entscheidungsproblem-Variante
des Problems MULT ist die folgende:

\goodbreak
\prob{$\mathrm{MULT}_\mathrm{D}$}
{Integerzahlen $x,y$, positive Integerzahl $i$}
{Ist das $i$-te Bit in $x\cdot y$ gleich $1$?}
Offensichtlich sind die Probleme MULT und MULT$\rm_D$ gleich schwierig (oder
leicht) zu l�sen.

\begin{wrapfigure}[12]{r}{5cm}
\centerline{\includegraphics[scale=0.6]{niko.eps}}
\caption{Der Graph $G_N$}
\label{nikolaus} 
\end{wrapfigure}
Im Weiteren wollen wir uns haupts�chlich mit Problemen besch�ftigen,
die aus dem Gebiet der Graphentheorie stammen.  Das hat zwei
Gr�nde. Zum einen k�nnen, wie sich noch zeigen wird, viele praktisch
relevante Probleme mit Hilfe von Graphen modelliert werden, und zum
anderen sind sie anschaulich und oft relativ leicht zu verstehen.  Ein
(ungerichteter) Graph $G$ besteht aus einer Menge von Knoten $V$ und
einer Menge von Kanten $E$, die diese Knoten verbinden. Man schreibt:
$G=(V,E)$. Ein wohlbekanntes Beispiel ist der Nikolausgraph:
$G_N=\bigl(\{1,2,3,4,5\},\allowbreak\{(1,2),\allowbreak(1,3),\allowbreak
(1,5),\allowbreak(2,3),\allowbreak(2,5),\allowbreak(3,4),\allowbreak
(3,5),\allowbreak(4,5)\}\bigr)$.  Es gibt also f�nf Knoten
$V=\{1,2,3,4,5\}$, die durch die Kanten in
$E=\{(1,2),\allowbreak(1,3),\allowbreak(1,5),\allowbreak(2,3),\allowbreak
(2,5),\allowbreak(3,4),\allowbreak(3,5),\allowbreak(4,5)\}$ verbunden
werden (siehe Abbildung \ref{nikolaus}).


Ein prominentes Problem in der Graphentheorie ist es, eine sogenannte
Knotenf�rbung zu finden. Dabei wird jedem Knoten eine Farbe
zugeordnet, und man verbietet, dass zwei Knoten, die durch eine Kante
verbunden sind, die gleiche Farbe zugeordnet wird. Nat�rlich ist die
Anzahl der Farben, die verwendet werden d�rfen, durch eine feste
nat�rliche Zahl $k$ beschr�nkt. Genau wird das Problem, ob ein Graph
$k$-f�rbbar ist, wie folgt beschrieben:

\prob{$k\mathrm{COL}$}
{Ein Graph $G=(V,E)$}
{Hat $G$ eine Knotenf�rbung mit h�chstens $k$ Farben?}

Offensichtlich ist der Beispielgraph $G_N$ nicht mit drei Farben f�rbbar
(aber mit $4$ Farben, wie man leicht ausprobieren kann), und jedes
Programm f�r das Problem $3\mathrm{COL}$ m�sste ermitteln, dass $G_N$ die
gew�nschte Eigenschaft ($3$-F�rbbarkeit) nicht hat.

Man kann sich nat�rlich fragen, was das k�nstlich erscheinende
Problem $3\mathrm{COL}$ mit der Praxis zu tun hat. Das folgende einfache
Beispiel soll das verdeutlichen. Man nehme das Szenario an, dass
ein gro�er Telefonprovider in einer Ausschreibung drei
Funkfrequenzen f�r einen neuen Mobilfunkstandard erworben hat. Da er
schon �ber ein Mobilfunknetz verf�gt, sind die Sendemasten schon
gebaut. Aus technischen Gr�nden d�rfen Sendemasten, die zu eng stehen,
nicht mit der gleichen Frequenz funken, da sie sich sonst st�ren
w�rden. In der graphentheoretischen Welt modelliert man die
Sendestationen mit Knoten eines Graphen, und "`nahe"' zusammenstehende
Sendestationen symbolisiert man mit einer Kante zwischen den Knoten,
f�r die sie stehen. Die Aufgabe des Mobilfunkplaners ist es nun, eine
$3$-F�rbung f�r den entstehenden Graphen zu finden. Offensichtlich
kann das Problem verallgemeinert werden, wenn man sich nicht auf drei
Farben/Frequenzen festlegt; dann aber ergeben sich genau die oben
definierten Probleme $k\mathrm{COL}$ f�r beliebige Zahlen $k$.


Als n�chstes ist zu kl�ren, was unter einem "`Algorithmus"' zu
verstehen ist. Ein Algorithmus ist eine endliche, formale Beschreibung
einer Methode, die ein Problem l�st (z.B.~ein Programm in einer
beliebigen Programmiersprache). Diese Methode muss also alle Eingaben
mit der gesuchten Eigenschaft von den Eingaben, die diese Eigenschaft
nicht haben, unterscheiden k�nnen. Man legt fest, dass der Algorithmus
f�r erstere den Wert $1$ und f�r letztere den Wert $0$ ausgeben
soll. Wie soll die "`Laufzeit"' eines Algorithmus gemessen werden? Um
dies festlegen zu k�nnen, muss man sich zun�chst auf ein sogenanntes
\dindex{Berechnungsmodell} festlegen. Das kann man damit vergleichen,
welche Hardware f�r die Implementation des Algorithmus verwendet
werden soll. F�r die weiteren Analysen soll das folgende einfache
C-artige Modell verwendet werden: Es wird (grob!) die Syntax von C
verwendet und festgelegt, dass jede Anweisung in einem Schritt
abgearbeitet werden kann. Gleichzeitig beschr�nkt man sich auf zwei
Datentypen: einen Integer-Typ und zus�tzlich Arrays dieses
Integer-Typs (wobei Array-Grenzen nicht deklariert werden m�ssen,
sondern sich aus dem Gebrauch ergeben). Dieses primitive
Maschinenmodell ist deshalb geeignet, weil man zeigen kann, dass jeder
so formulierte Algorithmus auf realen Computern implementiert werden
kann, ohne eine substantielle Verlangsamung zu erfahren.  
%
(Dies gilt zumindest, wenn die verwendeten Zahlen nicht
�berm��ig wachsen, d.h., wenn alle verwendeten Variablen nicht zu viel
Speicher belegen. Genaueres zu dieser Problematik -- man spricht von
der Unterscheidung zwischen \dindex{uniformem Komplexit�tsma�} und
\dindex{Bitkomplexit�t} -- findet sich in \cite[S.~62f]{Sch01}.)

Umgekehrt kann man ebenfalls sagen, dass dieses einfache Modell die
Realit�t genau genug widerspiegelt, da auch reale Programme ohne allzu
gro�en Zeitverlust auf diesem Berechnungsmodell simuliert werden
k�nnen. Offensichtlich ist die \emph{Eingabe} der Parameter, von dem
die Rechenzeit f�r einen festen Algorithmus abh�ngt. In den
vergangenen Jahrzehnten, in denen das Gebiet der Analyse von
Algorithmen entstand, hat die Erfahrung gezeigt, dass die L�nge der
Eingabe, also die Anzahl der Bits die ben�tigt werden, um die Eingabe
zu speichern, ein geeignetes und robustes Ma� ist, in der die
Rechenzeit gemessen werden kann. Auch der Aufwand, die Eingabe selbst
festzulegen (zu konstruieren), h�ngt schlie�lich von ihrer L�nge ab,
nicht davon, ob sich irgendwo in der Eingabe eine $0$ oder $1$
befindet.

\ifpdf
\special{pdf: out 4 << /Title 
(Das Problem der 2-F�rbbarkeit) 
/Dest [ @thispage /FitH @ypos ] >>}
\fi
\subsubsection{Das Problem der 2-F�rbbarkeit}

\prob{$2\mathrm{COL}$}
{Ein Graph $G=(V,E)$}
{Hat $G$ eine Knotenf�rbung mit h�chstens $2$ Farben?}

Es ist bekannt, dass dieses Problem mit einem sogenannten
\dindex{Greedy-Algorithmus} gel�st werden kann: Beginne mit einem
beliebigen Knoten in $G$ (z.B.~$v_1$) und f�rbe ihn mit Farbe 1. F�rbe
dann die Nachbarn dieses Knoten mit 2, die Nachbarn dieser Nachbarn
wieder mit 1, usw.  Falls $G$ aus mehreren Komponenten
(d.h.~zusammenh�ngenden Teilgraphen) besteht, muss dieses Verfahren
f�r jede Komponente wiederholt werden. $G$ ist schlie�lich 2-f�rbbar,
wenn bei obiger Prozedur keine inkorrekte F�rbung entsteht. Diese Idee
f�hrt zu Algorithmus \ref{TColAlgo}.

\begin{figure}
\begin{center}
\restylealgo{ruled}
\begin{algorithm}[H]
\caption{Ein Algorithmus zur Berechnung einer $2$-F�rbung eines Graphen}
\label{TColAlgo}
%
\KwData{Graph $G = (\{v_1, \dots v_n\}, E)$;}
\KwResult{$1$ wenn es eine $2$-F�rbung f�r $G$ gibt, $0$ sonst}
\BlankLine
\Begin{
\tcc*[f]{Array f�r die F�rbung initialisieren}

\For{$($$i = 1$ to $n$$)$}{
$\mathrm{Farbe}[i] = 0$\;
}
\BlankLine
$\mathrm{Farbe}[1] = 1$\;
\Repeat{$(\mathrm{aktKompoBearbeiten})$}{
  aktKompoBearbeiten = \texttt{false}\;
  \For{$($$i = 1$ $\mathrm{to}$ $n$$)$}{
    \For{$($$j = 1$ $\mathrm{to}$ $n $$)$}{
      \tcc*[f]{Kante bei der ein Knoten noch nicht gef�rbt?}

      \If{$(($$(v_i,v_j) \in E$$)$ und $(\mathrm{Farbe}[i] \not= 0$$)$ und 
          $(\mathrm{Farbe}[j] = 0$$))$}{
        \tcc*[f]{$v_j$ bekommt eine andere Farbe als $v_i$}
      
        $\mathrm{Farbe}[j] = 3 - \mathrm{Farbe}[i]$\;
        aktKompoBearbeiten = \texttt{true}\;
        \tcc*[f]{Alle direkten Nachbarn von $v_j$ pr�fen}

        \For{$($$k = 1$ $\mathrm{to}$ $n $$)$}{
          \tcc*[f]{Kollision beim F�rben aufgetreten?}

          \If{$(($$(v_j,v_k) \in E$$)$ und 
              $(\mathrm{Farbe}[i] = \mathrm{Farbe}[k]$$))$}{
            \tcc*[f]{Kollision! Graph nicht $2$-f�rbbar}

            \Return $0$\;
          }
        } 
      }
    }
  }
  \tcc*[f]{Ist die aktuelle Zusammenhangkomponente v�llig gef�rbt?}

  \If{$(\mathrm{not(aktKompoBearbeiten)})$}{
    $i = 1$\;
    \tcc*[f]{Suche nach einer weiteren Zusammenhangskomponente von $G$}

    \Repeat{$(\mathrm{not(weiterSuchen)}$ $\mathrm{und}$ $(i \le n))$}{
      \tcc*[f]{Liegt $v_i$ in einer neuen Zusammenhangskomponente von $G$?}

      \If{$(\mathrm{Farbe}[i] = 0)$}{
        $\mathrm{Farbe}[i] = 1$\;
        \tcc*[f]{Neue Zusammenhangskomponente bearbeiten}
    
        aktKompoBearbeiten = \texttt{true}\;
        \tcc*[f]{Suche nach neuer Zusammenhangskomponente abbrechen}

        weiterSuchen = \texttt{false}\;

      }
      $i = i + 1$\;
    }
  }
}
\tcc*[f]{$2$-F�rbung gefunden}

\Return $1$\;
}

\end{algorithm}
\end{center}
\end{figure}

Die Laufzeit von Algorithmus \ref{TColAlgo} kann wie folgt abgesch�tzt
werden: Die erste \textbf{for}-Schleife ben�tigt $n$ Schritte.  In der
\textbf{while}-Schleife wird entweder mindestens ein Knoten gef�rbt
und die Schleife dann erneut ausgef�hrt, oder es wird kein Knoten
gef�rbt und die Schleife dann verlassen; also wird diese Schleife
h�chstens $n$-mal ausgef�hrt. Innerhalb der \textbf{while}-Schleife
finden sich drei ineinander verschachtelte \textbf{for}-Schleifen, die
alle jeweils $n$-mal durchlaufen werden, und eine
\textbf{while}-Schleife, die maximal $n$-mal durchlaufen wird.

Damit ergibt sich also eine Gesamtlaufzeit der Gr��enordnung $n^4$,
wobei $n$ die Anzahl der Knoten des Eingabe-Graphen $G$ ist. Wie gro�
ist nun die Eingabel�nge, also die Anzahl der ben�tigten Bits zur
Speicherung von $G$?  Sicherlich muss jeder Knoten in dieser
Speicherung vertreten sein, d.h.~also, dass mindestens $n$ Bits zur
Speicherung von $G$ ben�tigt werden. Die Eingabel�nge ist also
mindestens $n$. Daraus folgt, dass die Laufzeit des Algorithmus also
h�chstens von der Gr��enordnung $N^4$ ist, wenn $N$ die Eingabel�nge
bezeichnet.

Tats�chlich sind (bei Verwendung geeigneter Datenstrukturen wie Listen
oder Queues) wesentlich effizientere Verfahren f�r $2\mathrm{COL}$ m�glich.
Aber auch schon das obige einfache Verfahren zeigt: $2\mathrm{COL}$ hat einen
Polynomialzeitalgorithmus, also $2\mathrm{COL} \in\P$.

Alle Probleme f�r die Algorithmen existieren, die eine Anzahl von
Rechenschritten ben�tigen, die durch ein beliebiges Polynom beschr�nkt
ist, bezeichnet man mit $\P$ ("`$\P$"' steht dabei f�r
"`\textbf{P}olyno\-mi\-al\-zeit"'). Auch dabei wird die Rechenzeit in
der L�nge der Eingabe gemessen, d.h. in der Anzahl der Bits, die
ben�tigt werden, um die Eingabe zu speichern (zu kodieren). Die Klasse
$\P$\index{$\P$} wird auch als Klasse der \emph{effizient l�sbaren
Probleme} bezeichnet. Dies ist nat�rlich wieder eine idealisierte
Auffassung: Einen Algorithmus mit einer Laufzeit $n^{57}$, wobei $n$
die L�nge der Eingabe bezeichnet, kann man schwer als effizient
bezeichnen. Allerdings hat es sich in der Praxis gezeigt, dass f�r
fast alle bekannten Probleme in $\P$ auch Algorithmen existieren,
deren Laufzeit durch ein Polynom
kleinen\footnote{Aktuell \emph{scheint} es kein praktisch relevantes
Problem aus $\P$ zu geben, f�r das es keinen Algorithmus mit einer
Laufzeit von weniger als $n^{12}$ gibt.} Grades beschr�nkt ist.

In diesem Licht ist die Definition der Klasse $\P$ auch f�r praktische
Belange von Relevanz. Dass eine polynomielle Laufzeit etwas
substanziell Besseres darstellt als exponentielle Laufzeit (hier
betr�gt die ben�tigte Rechenzeit $2^{c \cdot n}$ f�r eine Konstante
$c$, wobei $n$ wieder die L�nge der Eingabe bezeichnet), zeigt die
Tabelle "`Rechenzeitbedarf von Algorithmen"'. Zu beachten ist, dass
bei einem Exponentialzeit-Algorithmus mit $c=1$ eine Verdoppelung der
"`Geschwindigkeit"' der verwendeten Maschine (also Taktzahl pro
Sekunde) es nur erlaubt, eine um h�chstens $1$ Bit l�ngere Eingabe in
einer bestimmten Zeit zu bearbeiten. Bei einem Linearzeit-Algorithmus
hingegen verdoppelt sich auch die m�gliche Eingabel�nge; bei einer
Laufzeit von $n^k$ vergr��ert sich die m�gliche Eingabel�nge immerhin
noch um den Faktor $\sqrt[k]{2}$. Deswegen sind Probleme, f�r die nur
Exponentialzeit-Algorithmen existieren, praktisch nicht l�sbar; daran
�ndert sich auch nichts Wesentliches durch die Einf�hrung von immer
schnelleren Rechnern.

\ifalgorithms
\else
\begin{figure}
{\renewcommand{\arraystretch}{1.2}\doublerulesep.2pt\arrayrulewidth.5pt
\footnotesize
\begin{center}
\begin{tabular}{|c|c|c|c|c|c|c|} \hline\hline
\multicolumn{1}{|c}{Anzahl ben"otigter} 
& \multicolumn{6}{c|}{Eingabel"ange $n$}\\ \cline{2-7}
Instruktionen & 10 & 20 & 30 & 40 & 50 & 60 \\ 
\hline\hline
 & 0,00001 & 0,00002 & 0,00003 & 0,00004 & 0,00005 & 0,00006 \\
\raisebox{1.5ex}[-1.5ex]{$n$} 
 & Sekunden & Sekunden & Sekunden & Sekunden & Sekunden & Sekunden \\ \hline
 & 0,0001 & 0,0004 & 0,0009 & 0,0016 & 0,0025 & 0,0036 \\
\raisebox{1.5ex}[-1.5ex]{$n^2$} 
 & Sekunden & Sekunden & Sekunden & Sekunden & Sekunden & Sekunden \\ \hline
 & 0,001 & 0,008 & 0,027 & 0,064 & 0,125 & 0,216 \\
\raisebox{1.5ex}[-1.5ex]{$n^3$} 
 & Sekunden & Sekunden & Sekunden & Sekunden & Sekunden & Sekunden \\ \hline
 & 0,1 & 3,2 & 24,3 & 1,7 & 5,2 & 13,0 \\
\raisebox{1.5ex}[-1.5ex]{$n^5$} 
 & Sekunden & Sekunden & Sekunden & Minuten & Minuten & Minuten \\ 
\hline\hline
 & 0,001 & 1 & 17,9 & 12,7 & 35,7 & 366 \\
\raisebox{1.5ex}[-1.5ex]{$2^n$} 
 & Sekunden & Sekunde & Minuten & Tage & Jahre & Jahrhunderte \\ \hline
 & 0,059 & 58 & 6,5 & 3855 & $2 \cdot 10^8$ & $1,3 \cdot 10^{13}$ \\
\raisebox{1.5ex}[-1.5ex]{$3^n$} 
 & Sekunden & Minuten & Jahre & Jahrhunderte & Jahrhunderte & Jahrhunderte \\ \hline\hline
\end{tabular}
\end{center}
}
\centerline{Rechenzeitbedarf von Algorithmen auf einem "`$1$-MIPS"'-Rechner}
\end{figure}
\fi

Nun stellt sich nat�rlich sofort die Frage: Gibt es f�r jedes Problem
einen effizienten Algorithmus?  Man kann relativ leicht zeigen, dass
die Antwort auf diese Frage "`Nein"' ist. Die Schwierigkeit bei dieser
Fragestellung liegt aber darin, dass man von vielen Problemen
\emph{nicht wei�}, ob sie effizient l�sbar sind.  Ganz konkret: Ein
effizienter Algorithmus f�r das Problem $2\mathrm{COL}$ ist
Algorithmus
\ref{TColAlgo}. Ist es m�glich, ebenfalls einen
Polynomialzeitalgorithmus f�r $3\mathrm{COL}$ zu finden? Viele Informatiker
besch�ftigen sich seit den 60er Jahren des letzten Jahrhunderts
intensiv mit dieser Frage. Dabei kristallisierte sich heraus, dass
viele praktisch relevante Probleme, f�r die kein effizienter
Algorithmus bekannt ist, eine gemeinsame Eigenschaft besitzen, n�mlich
die der \emph{effizienten �berpr�fbarkeit} von geeigneten
L�sungen. Auch $3\mathrm{COL}$ geh�rt zu dieser Klasse von Problemen, wie sich
in K�rze zeigen wird. Aber wie soll man zeigen, dass f�r ein Problem
kein effizienter Algorithmus existiert?  Nur weil kein Algorithmus
bekannt ist, bedeutet das noch nicht, dass keiner existiert.

Es ist bekannt, dass die oberen Schranken (also die Laufzeit von
bekannten Algorithmen) und die unteren Schranken (mindestens ben�tigte
Laufzeit) f�r das Problem PARITY (und einige wenige weitere, ebenfalls
sehr einfach-geartete Probleme) sehr nahe zusammen liegen. Das
bedeutet also, dass nur noch unwesentliche Verbesserungen der
Algorithmen f�r des PARITY-Problem erwartet werden k�nnen. Beim
Problem $3\mathrm{COL}$ ist das ganz anderes: Die bekannten unteren
und oberen Schranken liegen extrem weit auseinander. Deshalb ist nicht
klar, ob nicht doch (extrem) bessere Algorithmen als heute bekannt im
Bereich des M�glichen liegen. Aber wie untersucht man solch eine
Problematik?  Man m�sste ja �ber unendlich viele Algorithmen f�r das
Problem $3\mathrm{COL}$ Untersuchungen anstellen. Dies ist �u�erst
schwer zu handhaben und deshalb ist der einzige bekannte Ausweg, das
Problem mit einer Reihe von weiteren (aus bestimmten Gr�nden)
interessierenden Problemen zu vergleichen und zu zeigen, dass unser zu
untersuchendes Problem nicht leichter zu l�sen ist als diese
anderen. Hat man das geschafft, ist eine untere Schranke einer
speziellen Art gefunden: Unser Problem ist nicht leichter l�sbar, als
alle Probleme der Klasse von Problemen, die f�r den Vergleich
herangezogen wurden. Nun ist aus der Beschreibung der Aufgabe aber
schon klar, dass auch diese Aufgabe schwierig zu l�sen ist, weil ein
Problem nun mit unendlich vielen anderen Problemen zu vergleichen
ist. Es zeigt sich aber, dass diese Aufgabe nicht aussichtslos
ist. Bevor diese Idee weiter ausgef�hrt wird, soll zun�chst die Klasse
von Problemen untersucht werden, die f�r diesen Vergleich herangezogen
werden sollen, n�mlich die Klasse $\NP$\index{$\NP$}.

\ifpdf
\special{pdf: out 3 << /Title 
(Effizient �berpr�fbare Probleme: die Klasse NP) 
/Dest [ @thispage /FitH @ypos ] >>}
\fi
\subsection{Effizient �berpr�fbare Probleme: die Klasse $\NP$}

Wie schon erw�hnt, gibt es eine gro�e Anzahl verschiedener Probleme,
f�r die kein effizienter Algorithmus bekannt ist, die aber eine
gemeinsame Eigenschaft haben: die \emph{effiziente �berpr�fbarkeit von
L�sungen} f�r dieses Problem. Diese Eigenschaft soll an dem schon
bekannten Problem $3\mathrm{COL}$ veranschaulicht werden: Angenommen,
man hat einen beliebigen Graphen $G$ gegeben; wie bereits erw�hnt ist
kein effizienter Algorithmus bekannt, der entscheiden kann, ob der
Graph $G$ eine $3$-F�rbung hat (d.h., ob der fiktive Mobilfunkprovider
mit $3$ Funkfrequenzen auskommt). Hat man aber aus irgendwelchen
Gr�nden eine \emph{potenzielle} Knotenf�rbung vorliegen, dann ist es
leicht, diese potenzielle Knotenf�rbung zu �berpr�fen und
festzustellen, ob sie eine
\emph{korrekte} F�rbung des Graphen ist, wie Algorithmus \ref{TestCOL} zeigt.

\begin{figure}
\begin{center}
\restylealgo{ruled}
\begin{algorithm}[H]
\caption{Ein Algorithmus zur �berpr�fung einer potentiellen F�rbung}
\label{TestCOL}
\KwData{Graph $G = (\set{v_1, \dots v_n}, E)$ und eine potenzielle Knotenf�rbung}
\KwResult{$1$ wenn die F�rbung korrekt, $0$ sonst}
\BlankLine
\Begin{
    \For{$(i = 1$ $\mathrm{to}$ $n)$}{
      \For{$(j = 1$ $\mathrm{to}$ $n)$}{
        \If{$(((v_i,v_j) \in E)$ $\mathrm{und}$ $(v_i \text{ und }
            v_j \text{ sind gleich gef�rbt}))$}{
          \Return $1$\;
        }
      }
    }
  }
  \Return $0$\;
\end{algorithm}
\end{center}
\end{figure}

Das Problem $3\mathrm{COL}$ hat also die Eigenschaft, dass eine potenzielle
L�sung leicht daraufhin �berpr�ft werden kann, ob sie eine
tats�chliche, d.h.~korrekte, L�sung ist.  Viele andere praktisch
relevante Probleme, f�r die kein effizienter Algorithmus bekannt ist,
besitzen ebenfalls diese Eigenschaft. Dies soll noch an einem weiteren
Beispiel verdeutlicht werden, dem so genannten
\dindex{Hamiltonkreis-Problem}\index{Problem!Hamiltonkreis}.

Sei wieder ein Graph $G = \bigl(\{v_1, \dots v_n \},E\bigr)$
gegeben. Diesmal ist eine Rundreise entlang der Kanten von $G$
gesucht, die bei einem Knoten $v_{i_1}$ aus $G$ startet, wieder bei
$v_{i_1}$ endet und jeden Knoten genau einmal besucht. Genauer wird
diese Rundreise im Graphen $G$ durch eine Folge von $n$ Knoten
$(v_{i_1}, v_{i_2}, v_{i_3}, \dots ,v_{i_{n-1}}, v_{i_{n}},)$
beschrieben, wobei gelten soll, dass alle Knoten
$v_{i_1},\dots,v_{i_n}$ verschieden sind und die Kanten $(v_{i_1},
v_{i_2}),\allowbreak (v_{i_2}, v_{i_3}),\allowbreak \dots
,(v_{i_{n-1}}, v_{i_{n}})$ und $(v_{i_{n}}, v_{i_1})$ in $G$
vorkommen.  Eine solche Folge von Kanten wird als \dindex{Hamiltonscher
Kreis} bezeichnet. Ein Hamiltonscher Kreis in einem Graphen $G$ ist
also ein Kreis, der jeden Knoten des Graphen genau einmal besucht.
Das Problem, einen Hamiltonschen Kreis in einem Graphen zu finden,
bezeichnet man mit \textsf{HAMILTON}:

\dprob{HAMILTON}
{Ein Graph $G$}
{Hat $G$ einen Hamiltonschen Kreis?}

Auch f�r dieses Problem ist kein effizienter Algorithmus bekannt. Aber
auch hier ist offensichtlich: Bekommt man einen Graphen gegeben und
eine Folge von Knoten, dann kann man sehr leicht �berpr�fen, ob sie
ein Hamiltonscher Kreis ist -- dazu ist lediglich zu testen, ob alle
Knoten genau einmal besucht werden und auch alle Kanten im gegebenen
Graphen vorhanden sind.

Hat man erst einmal die Beobachtung gemacht, dass viele Probleme die
Eigenschaft der effizienten �berpr�fbarkeit haben, ist es naheliegend,
sie in einer Klasse zusammenzufassen und gemeinsam zu untersuchen. Die
Hoffnung dabei ist, dass sich alle Aussagen, die man �ber diese Klasse
herausfindet, sofort auf alle Probleme anwenden lassen. Solche
�berlegungen f�hrten zur Geburt der Klasse $\NP$, in der man alle
effizient �berpr�fbaren Probleme zusammenfasst. Aber wie kann man
solch eine Klasse untersuchen? Man hat ja noch nicht einmal ein
Maschinenmodell (oder eine Programmiersprache) zur Verf�gung, um solch
eine Eigenschaft zu modellieren. Um ein Programm f�r effizient
�berpr�fbare Probleme zu schreiben, braucht man erst eine M�glichkeit,
die zu �berpr�fenden m�glichen L�sungen zu ermitteln und sie dann zu
testen, d.h.~man muss die Programmiersprache f�r $\NP$ in einer
geeigneten Weise mit mehr "`Berechnungskraft"' ausstatten.

Die erste L�sungsidee f�r $\NP$-Probleme, n�mlich alle in Frage
kommenden L�sungen in einer $\mathbf{for}$-Schleife aufzuz�hlen, f�hrt
zu Exponentialzeit-L�sungsalgorithmen, denn es gibt im Allgemeinen
einfach so viele potenzielle L�sungen. Um erneut auf das Problem
$3\mathrm{COL}$ zur�ckzukommen: Angenommen, $G$ ist ein Graph mit $n$
Knoten. Dann gibt es $3^n$ potenzielle F�rbungen, die �berpr�ft werden
m�ssen, denn es gibt $3$ M�glichkeiten den ersten Knoten zu f�rben,
$3$ M�glichkeiten den zweiten Knoten zu f�rben, usw., und damit $3^n$
viele zu �berpr�fende potenzielle F�rbungen. W�rde man diese in einer
$\mathbf{for}$-Schleife aufz�hlen und auf Korrektheit testen, so
f�hrte das also zu einem Exponentialzeit-Algorithmus.  Auf der anderen
Seite gibt es aber Probleme, die in Exponentialzeit gel�st werden
k�nnen, aber nicht zu der Intuition der effizienten �berpr�fbarkeit
der Klasse $\NP$ passen.  Das Berechnungsmodell f�r $\NP$ kann also
nicht einfach so gewonnen werden, dass exponentielle Laufzeit
zugelassen wird, denn damit w�re man �ber das Ziel hinausgeschossen.

H�tte man einen Parallelrechner zur Verf�gung mit so vielen
Prozessoren wie es potenzielle L�sungen gibt, dann k�nnte man das
Problem schnell l�sen, denn jeder Prozessor kann unabh�ngig von allen
anderen Prozessoren eine potenzielle F�rbung �berpr�fen. Es zeigt sich
aber, dass auch dieses Berechnungsmodell zu m�chtig w�re. Es gibt
Probleme, die wahrscheinlich nicht im obigen Sinne effizient
�berpr�fbar sind, aber mit solch einem Parallelrechner trotzdem
(effizient) gel�st werden k�nnten. In der Praxis w�rde uns ein
derartiger paralleler Algorithmus auch nichts n�tzen, da man einen
Rechner mit exponentiell vielen Prozessoren, also enormem
Hardwareaufwand, zu konstruieren h�tten. Also muss auch dieses
Berechnungsmodell wieder etwas schw�cher gemacht werden.

Eine Abschw�chung der gerade untersuchten Idee des Parallelrechners
f�hrt zu folgendem Vorgehen: Man "`r�t"' f�r den ersten Knoten eine
beliebige Farbe, dann f�r den zweiten Knoten auch wieder eine
beliebige Farbe, solange bis f�r den letzten Knoten eine Farbe gew�hlt
wurde. Danach �berpr�ft man die geratene F�rbung und akzeptiert die
Eingabe, wenn die geratene F�rbung eine korrekte Knotenf�rbung
ist. Die Eingabe ist eine positive Eingabeinstanz des $\NP$-Problems
$3\mathrm{COL}$, falls es eine potenzielle L�sung (F�rbung) gibt, die
sich bei der �berpr�fung als korrekt herausstellt, d.h.~im
beschriebenen Rechnermodell: falls es eine M�glichkeit zu raten gibt,
sodass am Ende akzeptiert (Wert $1$ ausgegeben) wird. Man kann also
die Berechnung durch einen Baum mit $3$-fachen Verzweigungen
darstellen (vgl.~Abbildung \ref{Tree3COL}).
%
\begin{figure}
\centerline{\includegraphics[scale=0.6]{tree2.eps}}
\caption{Ein Berechnungsbaum f�r das $3\mathrm{COL}$-Problem}
\label{Tree3COL}
\end{figure}
%
An den Kanten des Baumes findet sich das Resultat der Rateanweisung
der dar�berliegenden Verzweigung. Jeder Pfad in diesem sogenannten
\dindex{Berechnungsbaum} entspricht daher einer Folge von
Farbzuordnungen an die Knoten, d.h.~einer potenziellen F�rbung. Der
Graph ist $3$-f�rbbar, falls sich auf mindestens einem Pfad eine
korrekte F�rbung ergibt, falls also auf mindestens einem Pfad die
�berpr�fungsphase erfolgreich ist; der Beispielgraph besitzt 6
korrekte $3$-F�rbungen, ist also eine positive Instanz des
$3\mathrm{COL}$-Problems.

Eine weitere, vielleicht intuitivere Vorstellung f�r die Arbeitsweise
dieser $\NP$-Maschine ist die, dass bei jedem Ratevorgang $3$
verschiedene unabh�ngige Prozesse gestartet werden, die aber nicht
miteinander kommunizieren d�rfen. In diesem Sinne hat man es hier mit
einem eingeschr�nkten Parallelrechner zu tun: Beliebige Aufspaltung
(fork) ist erlaubt, aber keine Kommunikation zwischen den Prozessen
ist m�glich. W�rde man Kommunikation zulassen, h�tte man erneut den
allgemeinen Parallelrechner mit exponentiell vielen Prozessoren von
oben, der sich ja als zu m�chtig f�r $\NP$ herausgestellt hat.

Es hat sich also gezeigt, dass eine Art "`Rateanweisung"' ben�tigt
wird. In der Programmiersprache f�r $\NP$ verwendet man dazu das neue
Schl�sselwort $\mathbf{guess}(m)$, wobei $m$ die Anzahl von
M�glichkeiten ist, aus denen eine geraten wird, und legt fest, dass
auch die Anweisung $\mathbf{guess}(m)$ nur einen Takt Zeit f�r ihre
Abarbeitung ben�tigt.  Berechnungen, die, wie soeben beschrieben,
verschiedene M�glichkeiten raten k�nnen, hei�en
\dindex{nichtdeterministisch}. Es sei wiederholt, dass \emph{festgelegt}
(definiert) wird, dass ein nichtdeterministischer Algorithmus bei
einer Eingabe den Wert $1$ berechnet, falls \emph{eine
M�glichkeit} geraten werden kann, sodass der Algorithmus auf die
Anweisung "`\textbf{return} $1$"' st�sst. Die Klasse $\NP$
umfasst nun genau die Probleme, die von nichtdeterministischen
Algorithmen mit polynomieller Laufzeit gel�st werden k�nnen.
"`$\NP$"' steht dabei f�r
"`\textbf{n}ichtdeterministi\-sche \textbf{P}olynomialzeit"', nicht
etwa, wie mitunter zu lesen, f�r "`Nicht-Polynomialzeit"'.  (Eine
formale Pr�sentation der �quivalenz zwischen effizienter
�berpr�fbarkeit und Polynomialzeit in der $\NP$-Pro\-gram\-miersprache
findet sich z.B.~in \cite[Kapitel 2.3]{GaJo79}.)

Mit Hilfe eines nichtdeterministischen Algorithmus kann das
$3\mathrm{COL}$-Problem in Polynomialzeit gel�st werden (siehe
Algorithmus \ref{3COLAlgo}). Die zweite Phase von
Algorithmus \ref{3COLAlgo}, die
\dindex{�berpr�fungsphase}, entspricht dabei genau dem oben
angegebenen Algorithmus zum effizienten �berpr�fen von m�glichen
L�sungen des $3\mathrm{COL}$-Problems
(vgl.~Algorithmus \ref{TestCOL}).

\begin{figure}
\begin{center}
\restylealgo{ruled}
\begin{algorithm}[H]
\caption{Ein nichtdeterministischer Algorithmus f�r $3\mathrm{COL}$}
\label{3COLAlgo}
\KwData{Graph $G = (\{v_1, \dots v_n\}, E)$}
\KwResult{$1$ wenn eine F�rbung existiert, $0$ sonst}
\BlankLine
\Begin{
    \tcc*[f]{Ratephase}

    \For{$(i = 1$ $\mathrm{to}$ $n)$}{
        $\mathrm{Farbe}[i] = \mathrm{guess}(3)$\;
    }
    \tcc*[f]{�berpr�fungsphase}

    \For{$(i = 1$ $\mathrm{to}$ $n)$}{
      \For{$(j = 1$ $\mathrm{to}$ $n)$}{
        \If{$(((v_i,v_j) \in E)$ $\mathrm{und}$ $(v_i \text{ und }
            v_j \text{ sind gleich gef�rbt}))$}{
          \Return $0$\;
        }
      }
    }
  }
  \Return $1$\;
\end{algorithm}
\end{center}
\end{figure}

Dieser nichtdeterministische Algorithmus l�uft in Polynomialzeit, denn
man ben�tigt f�r einen Graphen mit $n$ Knoten mindestens $n$ Bits um
ihn zu speichern (kodieren), und der Algorithmus braucht im
schlechtesten Fall $O(n)$ (Ratephase) und $O(n^2)$
(�berpr�fungsphase), also insgesamt $O(n^2)$ Takte Zeit.  Damit ist
gezeigt, dass $3\mathrm{COL}$ in der Klasse $\NP$ enthalten ist, denn
es wurde ein nichtdeterministischer Polynomialzeitalgorithmus
gefunden, der $3\mathrm{COL}$ l�st. Ebenso einfach k�nnte man nun
einen nichtdeterministischen Polynomialzeitalgorithmus entwickeln, der
das Problem \textsf{HAMILTON} l�st: Der Algorithmus wird in einer
ersten Phase eine Knotenfolge raten und dann in einer zweiten Phase
�berpr�fen, dass die Bedingungen, die an einen Hamiltonschen Kreis
gestellt werden, bei der geratenen Folge erf�llt sind. Dies zeigt,
dass auch \textsf{HAMILTON} in der Klasse $\NP$ liegt.

Dass eine nichtdeterministische Maschine nicht gebaut werden kann,
spielt hier keine Rolle. Nichtdeterministische Berechnungen sollen
hier lediglich als Gedankenmodell f�r unsere Untersuchungen
herangezogen werden, um Aussagen �ber die (Nicht-) Existenz von
effizienten Algorithmen machen zu k�nnen.

\ifpdf
\special{pdf: out 3 << /Title 
(Schwierigste Probleme in NP: der Begriff der NP-Vollst�ndigkeit) 
/Dest [ @thispage /FitH @ypos ] >>}
\fi
\subsection{Schwierigste Probleme in $\NP$: der Begriff der 
$\NP$-Vollst�ndigkeit}

Es ist nun klar, was es bedeutet, dass ein Problem in $\NP$ liegt. Es
liegt aber auch auf der Hand, dass alle Probleme aus $\P$ auch in
$\NP$ liegen, da bei der Einf�hrung von $\NP$ ja nicht verlangt wurde,
dass die $\mathbf{guess}$-Anweisung verwendet werden muss. Damit ist
jeder deterministische Algorithmus automatisch auch ein
(eingeschr�nkter) nichtdeterministischer Algorithmus. Nun ist aber
auch schon bekannt, dass es Probleme in $\NP$ gibt, z.B.~$3\mathrm{COL}$ und
weitere Probleme, von denen nicht bekannt ist, ob sie in $\P$
liegen. Das f�hrt zu der Vermutung, dass $\P\neq\NP$.

Es gibt also in $\NP$ anscheinend unterschiedlich schwierige Probleme:
einerseits die $\P$-Probleme (also die leichten Probleme), und
andererseits die Probleme, von denen man nicht wei�, ob sie in $\P$
liegen (die schweren Probleme).  Es liegt also nahe, eine allgemeine
M�glichkeit zu suchen, Probleme in $\NP$ bez�glich ihrer Schwierigkeit
zu vergleichen. Ziel ist, wie oben erl�utert, eine Art von unterer
Schranke f�r Probleme wie $3\mathrm{COL}$: Es soll gezeigt werden,
dass $3\mathrm{COL}$ mindestens so schwierig ist, wie jedes andere
Problem in $\NP$, also in gewissem Sinne ein \emph{schwierigstes
Problem in $\NP$} ist.

F�r diesen Vergleich der Schwierigkeit ist die erste Idee nat�rlich,
einfach die Laufzeit von (bekannten) Algorithmen f�r das Problem
heranzuziehen. Dies ist jedoch nicht erfolgversprechend, denn was soll
eine "`gr��te"' Laufzeit sein, die Programme f�r "`schwierigste"'
Probleme in $\NP$ ja haben m�ssten?  Au�erdem h�ngt die Laufzeit eines
Algorithmus vom verwendeten Berechnungsmodell ab. So kennen
Turingmaschinen keine Arrays im Gegensatz zu der hier verwendeten
C-Variante. Also w�rde jeder Algorithmus, der Arrays verwendet,
auf einer Turingmaschine m�hsam simuliert werden m�ssen und damit
langsamer abgearbeitet werden, als bei einer Hochsprache, die Arrays
enth�lt. Obwohl sich die Komplexit�t eines Problems nicht �ndert,
w�rde man sie verschieden messen, je nachdem welches Berechnungsmodell
verwendet w�rde. Ein weiterer Nachteil dieses Definitionsversuchs w�re
es, dass die Komplexit�t (Schwierigkeit) eines Problems mit bekannten
Algorithmen gemessen w�rde. Das w�rde aber bedeuten, dass jeder neue
und schnellere Algorithmus Einfluss auf die Komplexit�t h�tte, was
offensichtlich so keinen Sinn macht.  Aus diesen und anderen Gr�nden
f�hrt die erste Idee nicht zum Ziel.

Eine zweite, erfolgversprechendere Idee ist die folgende: Ein Problem
$A$ ist nicht (wesentlich) schwieriger als ein Problem $B$, wenn man
$A$ mit der Hilfe von $B$ (als Unterprogramm) effizient l�sen
kann. Ein einfaches Beispiel ist die Multiplikation von $n$ Zahlen.
Angenommen, man hat schon ein Programm, dass $2$ Zahlen multiplizieren
kann; dann ist es nicht wesentlich schwieriger, auch $n$ Zahlen zu
multiplizieren, wenn die Routine f�r die Multiplikation von $2$ Zahlen
verwendet wird. Dieser Ansatz ist unter dem Namen \emph{relative
  Berechenbarkeit} bekannt, der genau den oben beschriebenen
Sachverhalt widerspiegelt: Multiplikation von $n$ Zahlen (so genannte
\emph{iterierte Multiplikation}) ist relativ zur Multiplikation zweier
Zahlen (leicht) berechenbar.

Da das Prinzip der relativen Berechenbarkeit so allgemein gehalten
ist, gibt es innerhalb der theoretischen Informatik sehr viele
verschiedene Auspr�gungen dieses Konzepts. F�r die
$\P$-$\NP$-Problematik ist folgende Version der relativen
Berechenbarkeit, d.h.~die folgende Art von erlaubten
"`Unterprogrammaufrufen"', geeignet:\\
%
Seien zwei Probleme $A$ und $B$ gegeben. Das Problem $A$ ist nicht
schwerer als $B$, falls es eine effizient zu berechnende
Transformation $T$ gibt, die Folgendes leistet: Wenn $x$ eine
Eingabeinstanz von Problem $A$ ist, dann ist $T(x)$ eine
Eingabeinstanz f�r $B$. Weiterhin gilt: $x$ ist \emph{genau dann} eine
positive Instanz von $A$ (d.h.~ein Entscheidungsalgorithmus f�r $A$
muss den Wert $1$ f�r Eingabe $x$ liefern), wenn $T(x)$ eine positive
Instanz von Problem $B$ ist. Erneut soll "`effizient berechenbar"'
hier bedeuten: in Polynomialzeit berechenbar. Es muss also einen
Polynomialzeitalgorithmus geben, der die Transformation $T$ ausf�hrt.
Das Entscheidungsproblem $A$ ist damit effizient transformierbar in
das Problem $B$. Man sagt auch: $A$ ist reduzierbar auf $B$; oder
intuitiver: $A$ ist nicht schwieriger als $B$, oder $B$ ist mindestens
so schwierig wie $A$. Formal schreibt man dann $A\leq B$.

Um f�r dieses Konzept ein wenig mehr Intuition zu gewinnen, sei
erw�hnt, dass man sich eine solche Transformation auch wie folgt
vorstellen kann: $A$ l�sst sich auf $B$ reduzieren, wenn ein
Algorithmus f�r $A$ angegeben werden kann, der ein Unterprogramm $U_B$
f�r $B$ genau so verwendet wie in Algorithmus \ref{RedAlgo} gezeigt.

Dabei ist zu beachten, dass das Unterprogramm f�r $B$ nur genau einmal
und zwar am Ende aufgerufen werden darf.  Das Ergebnis des Algorithmus
f�r $A$ ist genau das Ergebnis, das dieser Unterprogrammaufruf
liefert.  Es gibt zwar, wie oben erw�hnt, auch allgemeinere
Auspr�gungen der relativen Berechenbarkeit, die diese Einschr�nkung
nicht haben, diese sind aber f�r die folgenden Untersuchungen nicht
relevant.

Nachdem nun ein Vergleichsbegriff f�r die Schwierigkeit von Problemen
aus $\NP$ gefunden wurde, kann auch definiert werden, was unter einem
"`schwierigsten"' Problem in $\NP$ zu verstehen ist. Ein Problem $C$
ist ein schwierigstes Problem $\NP$, wenn alle anderen Probleme in
$\NP$ h�chstens so schwer wie $C$ sind. Formaler ausgedr�ckt sind dazu
zwei Eigenschaften von $C$ nachzuweisen:

\begin{enumerate}[{\sffamily(1)}]
\item $C$ ist ein Problem aus $\NP$.
\item $C$ ist mindestens so schwierig wie jedes andere $\NP$-Problem
$A$; d.h.: f�r alle Probleme $A$ aus $\NP$ gilt: $A \le C$.
\end{enumerate}

\begin{figure}
\begin{center}
\restylealgo{ruled}
\begin{algorithm}[H]
\caption{Algorithmische Darstellung der Benutzung einer Reduktionsfunktion}
\label{RedAlgo}
%
\KwData{Instanz $x$ f�r das Problem $A$}
\KwResult{$1$ wenn $x \in A$ und $0$ sonst}
\BlankLine
\Begin{
\tcc*[f]{$T$ ist die Reduktionsfunktion (polynomialzeitberechenbar)}

berechne $y = T(x)$\;
\tcc*[f]{$y$ ist Instanz des Problems $B$}

$z = U_B(y)$\;

\tcc*[f]{$z$ ist $1$ genau dann, wenn $x \in A$ gilt}

\Return $z$\;
}
\end{algorithm}
\end{center}
\end{figure}

Solche schwierigsten Probleme in $\NP$ sind unter der Bezeichnung
\emph{$\NP$-vollst�ndige Probleme}\index{$\NP$-vollst�ndig}
bekannt. Nun sieht die Aufgabe, von einem Problem zu zeigen, dass es
$\NP$-vollst�ndig ist, ziemlich hoffnungslos aus. Immerhin ist zu
zeigen, dass f�r alle Probleme aus $\NP$ -- und damit unendlich viele
-- gilt, dass sie h�chstens so schwer sind wie das zu untersuchende
Problem, und damit scheint man der Schwierigkeit beim Nachweis unterer
Schranken nicht entgangen zu sein. Dennoch konnten der russische
Mathematiker Leonid Levin und der amerikanische Mathematiker Stephan
Cook Anfang der siebziger Jahre des letzten Jahrhunderts unabh�ngig
voneinander die Existenz von solchen $\NP$-vollst�ndigen Probleme
zeigen. Hat man nun erst einmal \emph{ein} solches Problem
identifiziert, ist die Aufgabe, \emph{weitere} $\NP$-vollst�ndige
Probleme zu finden, wesentlich leichter. Dies ist sehr leicht
einzusehen: Ein $\NP$-Problem $C$ ist ein schwierigstes Problem in
$\NP$, wenn es ein anderes schwierigstes Problem $B$ gibt, sodass $C$
nicht leichter als $B$ ist.  Das f�hrt zu folgendem "`Kochrezept"':
\medskip

\noindent{\sffamily\bfseries Nachweis der $\NP$-Vollst�ndigkeit eines
Problems $C$:}
\begin{enumerate}[{\sffamily(1)}]
\item Zeige, dass $C$ in $\NP$ enthalten ist, indem daf�r ein
geeigneter nichtdeterministischer Polynomialzeitalgorithmus
konstruiert wird.
\item Suche ein geeignetes "`�hnliches"' schwierigstes Problem $B$ in
$\NP$ und zeige, dass $C$ nicht leichter als $B$ ist. Formal: Finde ein
$\NP$-vollst�ndiges Problem $B$ und zeige $B \le C$ mit Hilfe einer
geeigneten Transformation $T$.
\end{enumerate}

Den zweiten Schritt kann man oft relativ leicht mit Hilfe von
bekannten Sammlungen $\NP$-vollst�ndiger Problemen erledigen. Das Buch
von Garey und Johnson \cite{GaJo79} ist eine solche Sammlung (siehe
auch die Abbildungen \ref{NPVollProbs1} und \ref{NPVollProbs2}), die
mehr als $300$ $\NP$-vollst�ndige Probleme enth�lt. Dazu w�hlt man ein
m�glichst �hnliches Problem aus und versucht dann eine geeignete
Reduktionsfunktion f�r das zu untersuchende Problem zu finden.

\ifpdf
\special{pdf: out 4 << /Title 
(Traveling Salesperson ist NP-vollst�ndig) 
/Dest [ @thispage /FitH @ypos ] >>}
\fi
\subsubsection{Traveling Salesperson ist $\NP$-vollst�ndig}

Wie kann man zeigen, dass Traveling Salesperson $\NP$-vollst�ndig ist?
Dazu wird zuerst die genaue Definition dieses Problems ben�tigt:

\dprob{TRAVELING SALESPERSON (TSP)}%
{Eine Menge von St�dten $C = \{c_1, \dots ,c_n \}$ und eine $n \times
  n$ Entfernungsmatrix $D$, wobei das Element $D[i,j]$ der Matrix $D$
  die Entfernung zwischen Stadt $c_i$ und $c_j$ angibt. Weiterhin eine
  Obergrenze $k \ge 0$ f�r die maximal erlaubte L�nge der Tour}%
{Gibt es eine Rundreise, die einerseits alle St�dte besucht, aber
  andererseits eine Gesamtl�nge von h�chstens $k$ hat?}

Nun zum ersten Schritt des Nachweises der $\NP$-Vollst�ndigkeit von
\textsf{TSP}: Offensichtlich geh�rt auch das Traveling Salesperson Problem zur
Klasse $\NP$, denn man kann nichtdeterministisch eine Folge von $n$
St�dten raten (eine potenzielle Rundreise) und dann leicht �berpr�fen,
ob diese potenzielle Tour durch alle St�dte verl�uft und ob die
zur�ckzulegende Entfernung maximal $k$ betr�gt.  Damit ist der erste
Schritt zum Nachweis der $\NP$-Vollst�ndigkeit von \textsf{TSP} getan und Punkt
(1) des "`Kochrezepts"' abgehandelt.

Als n�chstes (Punkt (2)) soll von einem anderen $\NP$-vollst�ndigen
Problem gezeigt werden, dass es effizient in \textsf{TSP}
transformiert werden kann. Geeignet dazu ist das im Text betrachtete
Hamitonkreis-Problem, das bekannterma�en $\NP$-vollst�ndig ist.  Es
ist also zu zeigen: \textsf{HAMILTON} $\le$ \textsf{TSP}.

Folgende Idee f�hrt zum Ziel: Gegeben ist eine Instanz $G=(V,E)$ von
\textsf{HAMILTON}. Transformiere $G$ in folgende Instanz von \textsf{TSP}: Als
St�dtemenge $C$ w�hlen wir die Knoten $V$ des Graphen $G$. Die
Entfernungen zwischen den St�dten sind definiert wie folgt:
$D[i,j]=1$, falls es in $E$ eine Kante von Knoten $i$ zu Knoten $j$
gibt, ansonsten setzt man $D[i,j]$ auf einen sehr gro�en Wert, also
z.B. $n+1$, wenn $n$ die Anzahl der Knoten von $G$ ist. Dann gilt
klarerweise: Wenn $G$ einen Hamiltonschen Kreis besitzt, dann ist der
gleiche Kreis eine Rundreise in $C$ mit Gesamtl�nge $n$.  Wenn $G$
keinen Hamiltonschen Kreis besitzt, dann kann es keine Rundreise durch
die St�dte $C$ mit L�nge h�chstens $n$ geben, denn jede Rundreise muss
mindestens eine Strecke von einer Stadt $i$ nach einer Stadt $j$
zur�cklegen, die keiner Kante in $G$ entspricht (denn ansonsten h�tte
$G$ ja einen Hamiltonschen Kreis). Diese einzelne Strecke von $i$ nach
$j$ hat dann aber schon L�nge $n+1$ und damit ist eine Gesamtl�nge von
$n$ oder weniger nicht mehr erreichbar. In
Abbildung \ref{TSPRedExample} finden sich zwei Beispiele f�r die
Wirkungsweise der Transformation, die durch Algorithmus \ref{TSPRed}
berechnet wird.

\begin{figure}
\begin{center}
\fbox{
\begin{minipage}{0.95\textwidth}
\begin{minipage}{0.46\textwidth}
Aus dem Graphen $G$ links berechnet die Transformation die rechte
Eingabe f�r das \textsf{TSP}. Die dick gezeichneten Verbindungen deuten
eine Entfernung von $1$ an, wogegen d�nne Linien eine Entfernung
von $6$ symbolisieren. Weil $G$ den Hamiltonkreis $1,2,3,4,5,1$ hat,
gibt es rechts eine Rundreise $1,2,3,4,5,1$ mit Gesamtl�nge $5$.
\end{minipage}
\begin{minipage}{0.46\textwidth}
\centerline{\includegraphics[scale=0.40]{CG1.eps}}
\end{minipage}

\bigskip

\begin{minipage}{0.46\textwidth}
Im Gegensatz dazu berechnet die Transformation hier aus dem Graphen
$G'$ auf der linken eine Eingabe f�r das \textsf{TSP} auf der rechten
Seite, die, wie man sich leicht �berzeugt, keine Rundreise mit einer
maximalen Gesamtl�nge von $5$ hat. Dies liegt daran, dass der
urspr�ngliche Graph $G'$ keinen Hamiltonschen Kreis hatte.
\end{minipage}
\begin{minipage}{0.46\textwidth}
\centerline{\includegraphics[scale=0.40]{CG2.eps}}
\end{minipage}
\end{minipage}
}
\end{center}
\caption{Beispiele f�r die Wirkungsweise von Algorithmus \ref{TSPRed}}
\label{TSPRedExample}
\end{figure}


\begin{figure}[t]
\begin{center}
\restylealgo{ruled}
\begin{algorithm}[H]
\caption{Ein Algorithmus f�r die Reduktion von \textsf{HAMILTON} auf \textsf{TSP}}
\label{TSPRed}
\KwData{Graph $G=(V, E)$, wobei $V = \set{\range{1}{n}}$}
\KwResult{Eine Instanz $(C, D, k)$ f�r \textsf{TSP}}
\BlankLine
\Begin{
    \tcc*[f]{Die Knoten entsprechen den St�dten}

    $C = V$\;       

    \tcc*[f]{�berpr�fe alle potentiell existierenden Kanten}
    
    \For{$(i = 1$ $\mathrm{to}$ $n)$}{
      \For{$(j = 1$ $\mathrm{to}$ $n)$}{
        \uIf{$((v_i,v_j) \in E)$}{%
            \tcc*[f]{Kanten entsprechen kleinen Entfernungen}

            $D[i][j] = 1$\;       
         }
         \Else{%
            \tcc*[f]{nicht existierende Kante, dann sehr gro�e Entfernung}

            $D[i][j] = n + 1$\;       
         }
        }
      }
      \tcc*[f]{Gesamtl�nge $k$ der Rundreise ist Anzahl der St�dte $n$}

      $k = n$\;
      \tcc*[f]{Gebe die berechnete \textsf{TSP}-Instanz zur�ck}

      \Return $(C,D,k)$\;
}
\end{algorithm}
\end{center}
\end{figure}

\ifpdf
\special{pdf: out 3 << /Title 
(Die Auswirkungen der NP-Vollst�ndigkeit) 
/Dest [ @thispage /FitH @ypos ] >>}
\fi
\subsection{Die Auswirkungen der $\NP$-Vollst�ndigkeit}
Welche Bedeutung haben nun die $\NP$-vollst�ndigen Probleme f�r die
Klasse $\NP$? K�nnte jemand einen deterministischen
Polynomialzeitalgorithmus $\mathcal{A}_C$ f�r ein
$\NP$-voll\-st�ndiges Problem $C$ angeben, dann h�tte man f�r jedes
$\NP$-Problem einen Polynomialzeitalgorithmus gefunden
(d.h.~$\P=\NP$). Diese �berraschende Tatsache l�sst sich leicht
einsehen, denn f�r jedes Problem $A$ aus $\NP$ gibt es eine
Transformation $T$ mit der Eigenschaft, dass $x$ genau dann eine
positive Eingabeinstanz von $A$ ist, wenn $T(x)$ eine positive Instanz
von $C$ ist. Damit l�st Algorithmus \ref{PolyAlgoNP} das Problem $A$
in Polynomialzeit. Es gilt also: Ist irgendein $\NP$-vollst�ndiges
Problem effizient l�sbar, dann ist $\P=\NP$.


\begin{figure}
\begin{center}
\restylealgo{ruled}
\begin{algorithm}[H]
\caption{Ein fiktiver Algorithmus f�r Problem $A$}
\label{PolyAlgoNP}
\KwData{Instanz $x$ f�r das Problem $A$}
\KwResult{\texttt{true}, wenn $x \in A$, \texttt{false} sonst}
\BlankLine
\Begin{
   \tcc*[f]{$T$ ist die postulierte Reduktionsfunktion}

   $y = T(x)$\;
   $z =\mathcal{A}_C(y)$\;
   \BlankLine
   \Return $z$\;
}
\end{algorithm}
\end{center}
\end{figure}

Sei nun angenommen, dass jemand $\P \not= \NP$ gezeigt hat. In diesem
Fall ist aber auch klar, dass dann f�r kein $\NP$-vollst�ndiges
Problem ein Polynomialzeitalgorithmus existieren kann, denn sonst
w�rde sich ja der Widerspruch $\P = \NP$ ergeben.  Ist das Problem $C$
also $\NP$-vollst�ndig, so gilt: $C$ hat genau dann einen effizienten
Algorithmus, wenn $\P=\NP$, also wenn jedes Problem in $\NP$ einen
effizienten Algorithmus besitzt. Diese Eigenschaft macht die
$\NP$-vollst�ndigen Probleme f�r die Theoretiker so interessant, denn
eine Klasse von unendlich vielen Problemen kann untersucht werden,
indem man nur ein einziges Problem betrachtet. Man kann sich das auch
wie folgt vorstellen: Alle relevanten Eigenschaften aller Probleme aus
$\NP$ wurden in ein einziges Problem "`destilliert"'. Die
$\NP$-vollst�ndigen Probleme sind also in diesem Sinn
\emph{prototypische} $\NP$-Probleme.

Trotz intensiver Bem�hungen in den letzten 30 Jahren konnte bisher
niemand einen Polynomialzeitalgorithmus f�r ein $\NP$-vollst�ndiges
Problem finden. Dies ist ein Grund daf�r, dass man heute
$\P \not= \NP$ annimmt. Leider konnte auch dies bisher nicht gezeigt
werden, aber in der theoretischen Informatik gibt es starke Indizien
f�r die Richtigkeit dieser Annahme, sodass heute die gro�e Mehrheit
der Forscher von $\P
\not= \NP$ ausgeht.

F�r die Praxis bedeutet dies Folgendes: Hat man von einem in der
Realit�t auftretenden Problem gezeigt, dass es $\NP$-vollst�ndig ist,
dann kann man getrost aufh�ren, einen effizienten Algorithmus zu
suchen. Wie wir ja gesehen haben, kann ein solcher n�mlich (zumindest
unter der gut begr�ndbaren Annahme $\P\neq\NP$) nicht existieren.

Nun ist auch eine Antwort f�r das $3\mathrm{COL}$-Problem gefunden. Es wurde
gezeigt \cite{GaJo79}, dass $k\mathrm{COL}$ f�r $k \ge 3$ $\NP$-vollst�ndig
ist. Der fiktive Mobilfunkplaner hat also Pech gehabt: Es ist
unwahrscheinlich, dass er jemals ein korrektes effizientes
Planungsverfahren finden wird.

Ein $\NP$-Vollst�ndigkeitsnachweis eines Problems ist also ein starkes
Indiz f�r seine praktische Nicht-Handhabbarkeit. Auch die
$\NP$-Vollst�ndigkeit eines Problems, das mit dem Spiel
\emph{Minesweeper} zu tun hat, bedeutet demnach
lediglich, dass dieses Problem h�chstwahrscheinlich nicht effizient
l�sbar sein wird. Ein solcher Vollst�ndigkeitsbeweis hat nichts mit
einem Schritt in Richtung auf eine L�sung des
$\P\stackrel{?}{=}\NP$-Problems zu tun, wie irref�hrenderweise
gelegentlich zu lesen ist. �brigens ist auch f�r eine Reihe weiterer
Spiele ihre $\NP$-Vollst�ndigkeit bekannt. Dazu geh�ren {u.a.}
bestimmte Puzzle- und Kreuzwortspiele. Typische Brettspiele, wie Dame,
Schach oder GO, sind hingegen (verallgemeinert auf Spielbretter der
Gr��e $n\times n$) \textbf{PSPACE}-vollst�ndig. Die Klasse
\textbf{PSPACE} ist eine noch deutlich m�chtigere Klasse als
$\NP$. Damit sind also diese Spiele noch viel komplexer als
Minesweeper und andere $\NP$-vollst�ndige Probleme.


\begin{figure}
\begin{center}
\footnotesize
Problemnummern in "`[\dots]"' beziehen sich auf die Sammlung von
Garey und Johnson \cite{GaJo79}.

\columnsep1em
\begin{multicols}{2}
\dprob{CLUSTER \textrm{[GT19]}}
{Netzwerk $G=(V,E)$, positive Integerzahl $K$}
{Gibt es eine Menge von mindestens $K$ Knoten, die paarweise miteinander
verbunden sind?}
%
\dprob{NETZWERK-AUFTEILUNG \textrm{[ND16]}}
{Netzwerk $G=(V,E)$, Kapazit�t f�r jede Kante in $E$, positive
Integerzahl $K$}
{Kann man das Netzwerk so in zwei Teile zerlegen, dass die Gesamtkapazit�t
aller Verbindungen zwischen den beiden Teilen mindestens $K$ betr�gt?}
%
\dprob{NETZWERK-REDUNDANZ \textrm{[ND18]}}
{Netzwerk $G=(V,E)$, Kosten f�r Verbindungen zwischen je zwei Knoten aus 
$V$, Budget $B$}
{Kann $G$ so um Verbindungen erweitert werden, dass zwischen je zwei Knoten
mindestens zwei Pfade existieren und die Gesamtkosten f�r die Erweiterung
h�chstens $B$ betragen?}
%
\dprob{OBJEKTE SPEICHERN \textrm{[SR1]}}
{eine Menge $U$ von Objekten mit Speicherbedarf $s(u)$ f�r jedes $u \in U$;
Kachelgr��e $S$, positive Integerzahl $K$}
{K�nnen die Objekte in $U$ auf $K$ Kacheln verteilt werden?}
%
\dprob{DATENKOMPRESSION \textrm{[SR8]}}
{endliche Menge $R$ von Strings �ber festgelegtem Alphabet, positive
Integerzahl $K$}
{Gibt es einen String $S$ der L�nge h�chstens $K$, sodass jeder String aus $R$
als Teilfolge von $S$ vorkommt?}
%
\dprob{KLEINSTER SCHL�SSEL \textrm{[SR26]}}
{Relationales Datenbankschema, gegeben durch Attributmenge $A$ und
funktionale Abh�ngigkeiten auf $A$, positive Integerzahl $K$}
{Gibt es einen Schl�ssel mit h�chstens $K$ Attributen?}
%
\goodbreak
%
\dprob{VERLETZUNG DER BCNF \textrm{[SR29]}}
{Relationales Datenbankschema, gegeben durch Attributmenge $A$ und
funktionale Abh�ngigkeiten auf $A$, Teilmenge $A' \subseteq A$}
{Verletzt die Menge $A'$ die Boyce-Codd-Normalform?}
%
\dprob{MULTIPROZ-SCHEDULE \textrm{[SS8]}}
{Menge $T$ von Tasks, L�nge f�r jede Task, Anzahl $m$ von Prozessoren, 
Positive Integerzahl $D$ ("`Deadline"')}
{Gibt es ein $m$-Prozessor-Schedule f�r $T$ mit Ausf�hrungszeit h�chstens
$D$?}
%
\dprob{PREEMPT-SCHEDULE \textrm{[SS12]}}
{Menge $T$ von Tasks, L�nge f�r jede Task, Pr�zedenzrelation auf den
Tasks, Anzahl $m$ von Prozessoren, Positive Integerzahl $D$
("`Deadline"')}
{Gibt es ein $m$-Prozessor-Schedule f�r $T$, das die
Pr�zedenzrelationen ber�cksichtigt und Ausf�hrungszeit h�chstens $D$
hat?}
%
\dprob{DEADLOCK-VERMEIDUNG \textrm{[SS22]}}
{Menge von Prozessen, Menge von Ressourcen, aktuelle Zust�nde der 
Prozesse und aktuell allokierte Ressourcen}
{Gibt es einen Kontrollfluss, der zum Deadlock f�hrt?}
%
\dprob{SCHLEIFENVARIABLEN \textrm{[PO3]}}
{Menge $V$ von Variablen, die in einer Schleife benutzt werden, f�r jede
Variable einen G�ltigkeitsbereich, positive Integerzahl $K$}
{K�nnen die Schleifenvariablen mit h�ch\-stens $K$ Registern gespeichert werden?}
%
\dprob{FORMALE REKURSION \textrm{[PO20]}}
{Menge $A$ von Prozedur-Identifiern, Pascal-Programmfragment mit
Deklarationen und Aufrufen der Prozeduren aus $A$}
{Ist eine der Prozeduren aus $A$ formal rekursiv?}
\end{multicols}
\end{center}
\caption{Eine kleine Sammlung $\NP$-vollst�ndiger Probleme (Teil 1)}
\label{NPVollProbs1}
\end{figure}

\begin{figure}
\begin{center}
\footnotesize
Problemnummern in "`[\dots]"' beziehen sich auf die Sammlung von
Garey und Johnson \cite{GaJo79}.

\columnsep1em
\begin{multicols}{2}
\dprob{LR($K$)-GRAMMATIK \textrm{[AL15]}}
{Kontextfreie Grammatik $G$, positive Integerzahl $K$ (un�r)}
{Ist $G$ keine LR($K$)-Grammatik?}
%
\dprob{Zwangsbedingungen \textrm{[LO5]}}
{Menge von Booleschen Constraints, positive Integerzahl $K$}
{K�nnen mindestens $K$ der Constraints gleichzeitig erf�llt werden?}
%
\dprob{INTEGER PROGRAMMING  \textrm{[MP1]}}
{Lineares Programm}
{Hat das Programm eine L�sung, die nur ganzzahlige Werte enth�lt?}
%
\dprob{KREUZWORTR�TSEL \textrm{[GP15]}}%
{Menge $W$ von W�rtern, Gitter mit schwarzen und wei�en Feldern}
{K�nnen die wei�en Felder des Gitters mit W�rtern aus $W$? gef�llt werden}%
\end{multicols}
\caption{Eine kleine Sammlung $\NP$-vollst�ndiger Probleme (Teil 2)}
\label{NPVollProbs2}
\end{center}
\end{figure}

\ifpdf
\special{pdf: out 3 << /Title 
(Der Umgang mit NP-vollst�ndigen Problemen in der Praxis) 
/Dest [ @thispage /FitH @ypos ] >>}
\fi
\subsection{Der Umgang mit $\NP$-vollst�ndigen Problemen in der Praxis}

Viele in der Praxis bedeutsame Probleme sind $\NP$-vollst�ndig
(vgl.~die Abbildungen \ref{NPVollProbs1} und \ref{NPVollProbs2}). Ein
Anwendungsentwickler wird es aber sicher schwer haben, seinem
Management mitteilen zu m�ssen, dass ein aktuelles Projekt nicht
durchgef�hrt werden kann, weil keine geeigneten Algorithmen zur
Verf�gung stehen (Wahrscheinlich w�rden in diesem Fall einfach
"`geeignetere"' Entwickler eingestellt werden!). Es stellt sich daher
also die Frage, wie man mit solchen $\NP$-vollst�ndigen Problemen in
der Praxis umgeht. Zu dieser Fragestellung hat die theoretische
Informatik ein ausgefeiltes Instrumentarium entwickelt.

Eine erste Idee w�re es, sich mit Algorithmen zufrieden zu geben, die
mit Zufallszahlen arbeiten und die nur mit sehr gro�er
Wahrscheinlichkeit die richtige L�sung berechnen, aber sich auch mit
kleiner (vernachl�ssigbarer) Wahrscheinlichkeit irren d�rfen. Solche
Algorithmen sind als \emph{probabilistische} oder \emph{randomisierte
Algorithmen}\index{Algorithmus!probabilistisch}\index{Algorithmus!randomisiert}
bekannt \cite{mora95} und werden beispielsweise in der Kryptographie
mit sehr gro�em Erfolg angewendet.  Das prominenteste Beispiel hierf�r
sind Algorithmen, die testen, ob eine gegebene Zahl eine Primzahl ist
und sich dabei fast nie irren. Primzahlen spielen bekannterma�en im
RSA-Verfahren und damit bei PGP und �hnlichen Verschl�sselungen eine
zentrale Rolle.  Es konnte aber gezeigt werden, dass probabilistische
Algorithmen uns bei den $\NP$-vollst�ndigen Problemen wohl nicht
weiterhelfen. So wei� man heute, dass die Klasse der Probleme, die
sich mit probabilistischen Algorithmen effizient l�sen l�sst,
h�chstwahrscheinlich nicht die Klasse $\NP$ umfasst. Deshalb liegen
(h�chstwahrscheinlich) insbesondere alle $\NP$-vollst�ndigen Probleme
au�erhalb der M�glichkeiten von effizienten probabilistischen
Algorithmen.

Nun k�nnte man auch versuchen, "`exotischere"' Computer zu bauen.  In
der letzten Zeit sind zwei potenzielle Auswege bekannt geworden:
DNA-Computer und Quantencomputer.

Es konnte gezeigt werden, dass DNA-Computer (siehe \cite{pau98}) jedes
$\NP$-vollst�ndige Problem in Polynomialzeit l�sen k�nnen. F�r diese
Berechnungsst�rke hat man aber einen Preis zu zahlen: Die Anzahl und
damit die Masse der DNA-Molek�le, die f�r die Berechnung ben�tigt
werden, w�chst exponentiell in der Eingabel�nge. Das bedeutet, dass
schon bei recht kleinen Eingaben mehr Masse f�r eine Berechnung
gebraucht w�rde, als im ganzen Universum vorhanden ist. Bisher ist
kein Verfahren bekannt, wie dieses Masseproblem gel�st werden kann,
und es sieht auch nicht so aus, als ob es gel�st werden kann, wenn $\P
\not= \NP$ gilt. Dieses Problem erinnert an das oben im Kontext von
Parallelrechnern schon erw�hnte Ph�nomen: Mit exponentiell vielen
Prozessoren lassen sich $\NP$-vollst�ndige Probleme l�sen, aber solche
Parallelrechner haben nat�rlich explodierende Hardware-Kosten.
 
Der anderer Ausweg k�nnten Quantencomputer sein
(siehe \cite{gru99}). Hier scheint die Situation zun�chst g�nstiger zu
sein: Die Fortschritte bei der Quantencomputer-Forschung verlaufen
immens schnell, und es besteht die berechtigte Hoffnung, dass
Quantencomputer mittelfristig verf�gbar sein werden. Aber auch hier
sagen theoretische Ergebnisse voraus, dass Quantencomputer
(h�chstwahrscheinlich) keine $\NP$-vollst�ndigen Probleme l�sen
k�nnen. Trotzdem sind Quantencomputer interessant, denn es ist
bekannt, dass wichtige Probleme existieren, f�r die kein
Polynomialzeitalgorithmus bekannt ist und die wahrscheinlich nicht
$\NP$-vollst�ndig sind, die aber auf Quantencomputern effizient gel�st
werden k�nnen. Das prominenteste Beispiel hierf�r ist die Aufgabe,
eine ganze Zahl in ihre Primfaktoren zu zerlegen.

\goodbreak
Die bisher angesprochenen Ideen lassen also die Frage, wie man mit
$\NP$-vollst�ndigen Problemen umgeht, unbeantwortet. In der Praxis
gibt es im Moment zwei Hauptansatzpunkte: Die erste M�glichkeit ist
die, die Allgemeinheit des untersuchten Problems zu beschr�nken und
eine spezielle Version zu betrachten, die immer noch f�r die geplante
Anwendung ausreicht. Zum Beispiel sind Graphenprobleme oft einfacher,
wenn man zus�tzlich fordert, dass die Knoten des Graphen in der
(Euklidischen) Ebene lokalisiert sind. Deshalb sollte die erste Idee
bei der Behandlung von $\NP$-vollst�ndigen Problemen immer sein, zu
untersuchen, welche Einschr�nkungen man an das Problem machen kann,
ohne die praktische Aufgabenstellung zu verf�lschen. Gerade diese
Einschr�nkungen k�nnen dann effiziente Algorithmen erm�glichen.

Die zweite M�glichkeit sind sogenannte
\dindex{Approximationsalgorithmen} (siehe \cite{ACGJNO99}). Die Idee
hier ist es, nicht die optimalen L�sungen zu suchen, sondern sich mit
einem kleinen garantierten Fehler zufrieden zu geben.  Dazu folgendes
Beispiel. Es ist bekannt, dass das \textsf{TSP} auch dann noch
$\NP$-vollst�ndig ist, wenn man annimmt, dass die St�dte in der
Euklidischen Ebene lokalisiert sind, d.h.~man kann die St�dte in einer
fiktiven Landkarte einzeichnen, sodass die Entfernungen zwischen den
St�dten proportional zu den Abst�nden auf der Landkarte sind.  Das ist
sicherlich in der Praxis keine einschr�nkende Abschw�chung des
Problems und zeigt, dass die oben erw�hnte Methode nicht immer zum
Erfolg f�hren muss: Hier bleibt auch das eingeschr�nkte Problem
$\NP$-vollst�ndig. Aber f�r diese eingeschr�nkte \textsf{TSP}-Variante
ist ein Polynomialzeitalgorithmus bekannt, der immer eine Rundreise
berechnet, die h�chstens um einen beliebig w�hlbaren Faktor schlechter
ist, als die optimale L�sung. Ein Chip-Hersteller, der bei der
Best�ckung seiner Platinen die Wege der Roboterk�pfe minimieren
m�chte, kann also beschlie�en, sich mit einer Tour zufrieden zu geben,
die um 5 \% schlechter ist als die optimale. F�r dieses Problem
existiert ein effizienter Algorithmus!  Dieser ist f�r die Praxis
v�llig ausreichend.


\begin{center}
\mbox{}
\vfill
$\star \star \star$ \textsc{Ende} $\star \star \star$
\end{center}

% Appendix
\cleardoublepage
\appendix

% Index 
\cleardoublepage
\special{pdf: out 2 << /Title 
(Stichwortverzeichnis) 
/Dest [ @thispage /FitH @ypos ] >>}
\addcontentsline{toc}{section}{Stichwortverzeichnis}
\def\indexname{Stichwortverzeichnis}
\makeatletter
\printindex
\makeatother

\cleardoublepage
\bibliography{mbasic}

\end{document}
