% !TeX spellcheck = de_DE_frami
\documentclass[11pt, a4paper, twoside, bibliography=totoc]{scrartcl}

\usepackage{german}
\usepackage{fontspec}
\usepackage{polyglossia}
\setmainlanguage[babelshorthands=true]{german}

\usepackage[german]{algorithm2e}

\usepackage{amssymb}
\usepackage{amsfonts}
\usepackage{amsmath}
\usepackage{booktabs}
\usepackage{graphicx}
\usepackage[xetex]{geometry}
\usepackage{scrlayer-scrpage}
\usepackage{multicol}
\usepackage{color}
\usepackage{makeidx}
\usepackage{enumerate}
%\usepackage{lineno}
\usepackage{gitinfo}
\usepackage{wrapfig}
\usepackage{subfig}
\usepackage{bibgerm}
\usepackage{textcomp}
\usepackage[xetex,
            bookmarksopen=true,
            hyperfootnotes=false,
            breaklinks=true,
            bookmarks=false,
            pdfpagemode=UseOutlines,
            pdftitle={Elementare mathematische Begriffe, Probleme und Schreibweisen, Grundlagen der Logik, Einführung in die Komplexitätstheorie}, % TODO: 
            pdfauthor={Tanja Almerothh},
            pdfcreator={Steffen Reith}
           ]{hyperref}
\hypersetup{colorlinks=true,citecolor=blue,linkcolor=blue,urlcolor=blue}
\usepackage{hypcap}
\usepackage[numbered]{bookmark}
\usepackage{script}

% Algorithmen in C-Style
%\SetKwIF{If}{ElseIf}{Else}{if}{\{}{\}\\else if }{\}\\else\{}{\}}%
%\SetKwSwitch{Switch}{Case}{Other}{switch}{\{}{case}{default:}{\}}%
%\SetKwRepeat{Repeat}{do \{}{\} while}%
%\SetKwBlock{Begin}{\{}{\}}
%\SetKwFor{For}{for}{\{}{\}}
%\SetKwFor{While}{while}{\{}{\}}
\SetKwInput{KwData}{Eingabe}
\SetKwInput{KwResult}{Ergebnis}
\renewcommand{\listalgorithmcfname}{Algorithmenverzeichnis}%

% Baue einen Index
\makeindex

% Include global settings
% Special settings for discrete mathematics
\newif\ifdiscretemath
\newif\ifalgorithms
\newif\ifkomplex

\discretemathfalse
\algorithmsfalse
\komplextrue

% Baue einen Index
\makeindex

% Setze bibliography style
\bibliographystyle{alpha}

% Seitenaufteilung
\geometry{top=2.5cm, left=3.25cm, right=3.25cm, bottom=3.0cm}

% Suchpfad fuer Graphiken
\graphicspath{{pics/}}

%scrlayer-scrpage
\automark[subsection]{section}
\lofoot[]{}
\cofoot[]{\pagemark}
\rofoot[]{}
\refoot[]{}
\cefoot[]{\pagemark}
\lefoot[]{}

% Definitionen für die Titelseite
\newcommand{\docutyp}{Skript}
\newcommand{\lecture}{Ein\\ sehr langer \\[0.5\bigskipamount] Titel}
%\newcommand{\docudate}{Wintersemester 2018/2019}
\newcommand{\docudate}{Sommersemester 20XX}

\newcommand{\institution}{{\Large Hochschule RheinMain}\\
                          Fachbereich Design Informatik Medien}
\newcommand{\lecturer}{Somebody }
\newcommand{\lectureremail}{\href{mailto:Somebody.Anybody@hs-rm.de}{somebody.Anybody@hs-rm.de}}

\newcommand{\writer}{Tanja Almeroth}
\newcommand{\reviser}{xxx xxx}
\newcommand{\email}{Tanja.Almeroth@hs-rm.de}
\newcommand{\writtendate}{Mai 2019}

\begin{document}

\pagestyle{scrplain}

\begin{titlepage}

        \vspace{40pt}
	\begin{center}

                
		\vspace{20pt}
		\textbf{\Large {\docutyp}}
			
		\vspace{20pt}
		\textbf{\Huge \lecture}
			
		\vspace{20pt}
		\textbf{\docudate}

		\vspace{20pt}
		\textbf{\lecturer}\\
		\textbf{\lectureremail}
		
		\vspace{120pt}
		{\institution}\\
		
		\vfill			
		\vspace{20pt}
		\begin{tabular}[t]{rl}
			Erstellt von: & {\gitAuthorName}\\
                        Zuletzt bearbeitet von: & {\gitCommitterName}\\
			Email: & {\gitAuthorEmail}\\
			Erste Version vollendet: & {\writtendate}\\
			Version: & {\gitAbbrevHash}\\
			Letzte Änderung: & {\gitAuthorIsoDate}\\
		\end{tabular}
	\end{center}
	\newpage
\end{titlepage}

\cleardoublepage

\vspace{0.3\textheight} 
%\begin{raggedleft}
%Wenn Leute nicht glauben, dass Mathematik\\
%einfach ist, dann nur deshalb, weil sie nicht begreifen,\\
%wie kompliziert das Leben ist.\\[\smallskipamount]
%\hfill \textsc{John von Neumann}%
%\end{raggedleft}

\vspace*{1.5cm}

%\begin{raggedleft}
%Wenn es eine gute Idee ist, dann mach es\\ 
%einfach. Es ist viel einfacher sich hinterher zu\\
%entschuldigen, als vorher dafür eine\\
%Genehmigung zu bekommen.\\[\smallskipamount]
%\hfill \textsc{Grace Hopper}%
%\end{raggedleft}

\vspace*{1.5cm}
%
%\begin{raggedleft}
%Wake up! Time to die!\\[\smallskipamount]
%\hfill \textsc{Leon} in Blade Runner%
%\end{raggedleft}

\vfill

%Dieses Skript ist aus den Fragen und Bemerkungen von Studenten des
%Diplom, Bachelor und Master-Studiengangs Informatik an der Hochschule
%RheinMain (ehemals Fachhochschule Wiesbaden) hervorgegangen. Ich danke
%allen meinen Studenten für konstruktive Anmerkungen und
%Verbesserungen. Dabei möchte ich besonders Frau Carola Henzel nennen,
%die sehr viele Tippfehler (Mengen, Summen und Beweistechniken)
%berichtigte. Herr Kim Stebel hat die Bemerkungen zu bipartiten Graphen
%verbessert. Herr Norbert Wesp berichtete über sprachliche Fehler in
%Abschnitt \ref{sec:proof} und Tippfehler in der Einleitung. Eine größere 
%Anzahl von Tipp- und Flüchtigkeitsfehlern wurden von Herrn Marcell Dietl
%entdeckt und gemeldet. Danke!
%
%Naturgemäß ist ein Skript nie fehlerfrei (ganz im Gegenteil!) und es
%ändert (mit Sicherheit!) sich im Laufe der Zeit. Es sollte Ihnen klar 
%sein, dass Fehler, Ungenauigkeiten und Unklarheiten natürlich nur 
%aus didaktischen Gründen und zu Ihrer  Belustigung eingebaut 
%wurden. Finden Sie diese Fehler und verbessern Sie mich!

\cleardoublepage

\tableofcontents

\cleardoublepage

% Seitenstil (Fuss- und Kopfzeilen)
\pagestyle{scrheadings}

\pagenumbering{arabic}
	
%%%%%%%%%%%%%%%%%%%%%%%%% Sektion 1 %%%%%%%%%%%%%%%%%%%%%%%%%%%%%%%

% Eine kurze Einleitung

\section{Introduction}

Lore ipsum \cite{PACGGHSY}


\subsection{}


\paragraph{Proof Genral}
-explain what this is


\paragraph{Coq IDE}

\paragraph{Coq in the command line}





\cleardoublepage

% Kapitel mit Inhalt
%\include{}

\cleardoublepage



\begin{center}
\mbox{}
\vfill
$\star \star \star$ \textsc{Ende} $\star \star \star$
\end{center}

% Appendix
\cleardoublepage
\appendix

% Index 
\cleardoublepage
\addcontentsline{toc}{section}{Stichwortverzeichnis}
\def\indexname{Stichwortverzeichnis}
\makeatletter
\printindex
\makeatother

\cleardoublepage
\bibliography{mbasic}

\end{document}
