\documentclass[11pt, a4paper, twoside]{scrartcl}

\usepackage{german}
\usepackage{ae}
\usepackage{aecompl}
\usepackage[latin1]{inputenc}
\usepackage{amssymb}
\usepackage{amsfonts}
\usepackage{amsmath}
\usepackage{graphicx}
\usepackage{geometry}
\usepackage{fancyhdr}
\usepackage{script}
\usepackage{multicol}
\usepackage{color}
\usepackage{makeidx}
\usepackage{enumerate}
\usepackage[dvipdfm,
            bookmarksopen=true,
            bookmarks=false,
	    pdfpagemode=UseOutlines,
            pdftitle={Einige elementare Begriffe und Schreibweisen der Mathematik},
            pdfauthor={Steffen Reith},
            pdfcreator={Steffen Reith}
           ]{hyperref}
\usepackage[nofancy]{svninfo}
\usepackage{nassi}
\usepackage{wrapfig}
\usepackage{subfigure}

% Setze bibliography style
\bibliographystyle{alpha}

% Seitenaufteilung
\geometry{top=3.0cm, left=2.5cm, right=2.5cm, bottom=3.0cm}

% Seitenstil (Fu�- und Kopfzeilen)
\pagestyle{headings}
\pagestyle{fancy}

% Suchpfad f�r Graphiken
\graphicspath{{pics/}}

\renewcommand{\sectionmark}[1]{\markboth{\bf\boldmath\thesection\ #1}{}}
\renewcommand{\subsectionmark}[1]{\markright{\thesubsection\ #1}}
\lhead[\fancyplain{}{\thepage}]{\fancyplain{}{\rightmark}}
\chead[\fancyplain{}{}]{\fancyplain{}{}}
\rhead[\fancyplain{}{\leftmark}]{\fancyplain{}{\thepage}}
\lfoot[\fancyplain{}{}]{\fancyplain{}{}}
\cfoot[\fancyplain{}{}]{\fancyplain{}{}}
\rfoot[\fancyplain{}{}]{\fancyplain{}{}}

\fancypagestyle{plain}{
    \fancyhead{}
    \renewcommand{\headrulewidth}{0pt}
}

% Aufz�hlung mit (i), (ii), (iii), (iv) usw.
%\renewcommand{\theenumi}{\roman{enumi}}
%\renewcommand{\labelenumi}{(\theenumi)}

% Definitionen f�r Titelseite
\newcommand{\quotename}{\textit}
\newcommand{\emphasize}{\emph}

\newcommand{\docutyp}{Skript}
\newcommand{\lecture}{Einige elementare Begriffe und
  Schreibweisen der Mathematik}
\newcommand{\docudate}{Sommersemester 2006}

\newcommand{\institution}{{\Large Fachhochschule Wiesbaden}\\
                          Fachbereich Design Informatik Medien}
\newcommand{\lecturer}{Prof.~Dr.~Steffen Reith}
\newcommand{\lectureremail}{reith@informatik.fh-wiesbaden.de}

\svnInfo $Id$

\newcommand{\writer}{Steffen Reith}
\newcommand{\reviser}{Steffen Reith}
\newcommand{\email}{reith@informatik.fh-wiesbaden.de}
\newcommand{\writtendate}{Juli 2006}

\begin{document}

\begin{titlepage}
        \ifpdf
        \special{pdf: out 1 << /Title 
        (Einige elementare Begriffe und Schreibweisen der Mathematik) 
        /Dest [ @thispage /FitH @ypos ] >>}
        \fi	
        \vspace{60pt}
	\begin{center}
		\vspace{20pt}
		\textbf{\huge {\docutyp}} \\
			
		\vspace{30pt}
		\textbf{\huge {\lecture}} \\
			
		\vspace{20pt}
		\textbf{\docudate} \\

		\vspace{20pt}
		\textbf{\lecturer} \\
		\textbf{\lectureremail}
		\\
		\vspace{120pt}
		{\institution} \\
		
		\vfill			
		\vspace{20pt}
		\begin{tabular}[t]{rl}
			Erstellt von: & {\writer} \\
                        Zuletzt �berarbeitet von: & {\reviser} \\
			Email: & {\email}\\
			Erste Version vollendet: & {\writtendate} \\
			Version: & {\svnInfoRevision} \\
			Date: & {\svnInfoDate} \\
		\end{tabular}
	\end{center}
	\newpage
\end{titlepage}

\pagenumbering{roman}

\special{pdf: out 2 << /Title 
(Inhaltsverzeichnis) 
/Dest [ @thispage /FitH @ypos ] >>}
\tableofcontents

\newpage
	
% Haupt-Textteil
\pagenumbering{arabic}

%%%%%%%%%%%%%%%%%%%%%%%%% Sektion 1 %%%%%%%%%%%%%%%%%%%%%%%%%%%%%%%

% Einige Informationen �ber Mengen
\ifpdf
\special{pdf: out 2 << /Title 
(Elementare Begriffe und Schreibweisen) 
/Dest [ @thispage /FitH @ypos ] >>}
\fi
\section{Elementare Begriffe und Schreibweisen}

\ifpdf
\special{pdf: out 3 << /Title 
(Elementbeziehung und Enthaltenseinsrelation) 
/Dest [ @thispage /FitH @ypos ] >>}
\fi
\subsection{Elementbeziehung und Enthaltenseinsrelation}


\begin{center}
\mbox{}
\vfill
$\star \star \star$ \textsc{Ende} $\star \star \star$
\end{center}

% Literatur
\cleardoublepage
\ifpdf
\special{pdf: out 2 << /Title 
(Literatur) 
/Dest [ @thispage /FitH @ypos ] >>}
\fi
\def\bibname{Literatur}
\addcontentsline{toc}{section}{Literatur}
\bibliography{algo}

\end{document}
 
