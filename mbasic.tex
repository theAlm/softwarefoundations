\documentclass[11pt, a4paper, twoside]{scrartcl}

\usepackage{german}
\usepackage{ae}
\usepackage{aecompl}
\usepackage[german]{algorithm2e}
\usepackage[latin1]{inputenc}
\usepackage{amssymb}
\usepackage{amsfonts}
\usepackage{amsmath}
\usepackage{booktabs}
\usepackage{graphicx}
\usepackage[dvipdfm]{geometry}
\usepackage{fancyhdr}
\usepackage{script}
\usepackage{multicol}
\usepackage{color}
\usepackage{makeidx}
\usepackage{enumerate}
%\usepackage{lineno}
%\usepackage{ifpdf}
\usepackage[nofancy]{svninfo}
\usepackage{wrapfig}
\usepackage{subfig}
\usepackage{bibgerm}
\usepackage{textcomp}
\usepackage[dvipdfmx,
            bookmarksopen=true,
            breaklinks=true,
            bookmarks=false,
            pdfpagemode=UseOutlines,
            pdftitle={Elementare mathematische Begriffe, Probleme und Schreibweisen},
            pdfauthor={Steffen Reith},
            pdfcreator={Steffen Reith}
           ]{hyperref}

% Algorithmen in C-Style
%\SetKwIF{If}{ElseIf}{Else}{if}{\{}{\}\\else if }{\}\\else\{}{\}}%
%\SetKwSwitch{Switch}{Case}{Other}{switch}{\{}{case}{default:}{\}}%
%\SetKwRepeat{Repeat}{do \{}{\} while}%
%\SetKwBlock{Begin}{\{}{\}}
%\SetKwFor{For}{for}{\{}{\}}
%\SetKwFor{While}{while}{\{}{\}}
\SetKwInput{KwData}{Eingabe}
\SetKwInput{KwResult}{Ergebnis}
\renewcommand{\listalgorithmcfname}{Algorithmenverzeichnis}%

% Baue einen Index
\makeindex

% Include global settings
% Special settings for discrete mathematics
\newif\ifdiscretemath
\newif\ifalgorithms
\newif\ifkomplex

\discretemathfalse
\algorithmsfalse
\komplextrue

% Baue einen Index
\makeindex

% Setze bibliography style
\bibliographystyle{own}

% Seitenaufteilung
\geometry{top=2.5cm, left=3.25cm, right=3.25cm, bottom=3.0cm}

% Suchpfad fr Graphiken
\graphicspath{{pics/}}

\renewcommand{\sectionmark}[1]{\markboth{\bf\boldmath\thesection\ #1}{}}
\renewcommand{\subsectionmark}[1]{\markright{\thesubsection\ #1}}
\lhead[\fancyplain{}{\thepage}]{\fancyplain{}{\rightmark}}
\chead[\fancyplain{}{}]{\fancyplain{}{}}
\rhead[\fancyplain{}{\leftmark}]{\fancyplain{}{\thepage}}
\lfoot[\fancyplain{}{}]{\fancyplain{}{}}
\cfoot[\fancyplain{}{}]{\fancyplain{}{}}
\rfoot[\fancyplain{}{}]{\fancyplain{}{}}

\fancypagestyle{plain}{
    \fancyhead{}
    \renewcommand{\headrulewidth}{0pt}
}

% Aufzhlung mit (i), (ii), (iii), (iv) usw.
%\renewcommand{\theenumi}{\roman{enumi}}
%\renewcommand{\labelenumi}{(\theenumi)}

% Definitionen f�r die Titelseite
\newcommand{\docutyp}{Skript}
\newcommand{\lecture}{Elementare mathematische\\ Begriffe, Probleme und\\[0.5\bigskipamount] Schreibweisen}
\newcommand{\docudate}{Wintersemester 2010/2011}

\newcommand{\institution}{{\Large Hochschule RheinMain}\\
                          Fachbereich Design Informatik Medien}
\newcommand{\lecturer}{Prof.~Dr.~Steffen Reith}
\newcommand{\lectureremail}{reith@informatik.fh-wiesbaden.de}

\svnInfo $Id$

\newcommand{\writer}{Steffen Reith}
\newcommand{\reviser}{Steffen Reith}
\newcommand{\email}{reith@informatik.fh-wiesbaden.de}
\newcommand{\writtendate}{August 2006}

%\linenumbers

%

\begin{document}

\pagestyle{empty}

\begin{titlepage}

        \special{pdf: out 1 << /Title 
        (Elementare mathematische Begriffe, Probleme und Schreibweisen)
        /Dest [ @thispage /FitH @ypos ] >>}

        \vspace{60pt}
	\begin{center}
		\vspace{20pt}
		\textbf{\Large {\docutyp}}
			
		\vspace{20pt}
		\textbf{\Huge \lecture}
			
		\vspace{20pt}
		\textbf{\docudate}

		\vspace{20pt}
		\textbf{\lecturer}\\
		\textbf{\lectureremail}
		
		\vspace{120pt}
		{\institution}\\
		
		\vfill			
		\vspace{20pt}
		\begin{tabular}[t]{rl}
			Erstellt von: & {\writer}\\
                        Zuletzt bearbeitet von: & {\reviser}\\
			Email: & {\email}\\
			Erste Version vollendet: & {\writtendate}\\
			Version: & {\svnInfoRevision}\\
			Letzte �nderung: & {\svnInfoDate}\\
		\end{tabular}
	\end{center}
	\newpage
\end{titlepage}

\cleardoublepage

\thispagestyle{empty}
\vspace{0.3\textheight} 
\begin{raggedleft}
Wenn Leute nicht glauben, dass Mathematik\\
einfach ist, dann nur deshalb, weil sie nicht begreifen,\\
wie kompliziert das Leben ist.\\[\medskipamount]
\hfill \textsc{John von Neumann}%
\end{raggedleft}

\vspace*{1cm}

\begin{raggedleft}
Wenn es eine gute Idee ist, dann mach es\\ 
einfach. Es ist viel einfacher sich hinterher zu\\
entschuldigen, als vorher daf�r eine\\
Genehmigung zu bekommen.\\[\medskipamount]
\hfill \textsc{Grace Hopper}%
\end{raggedleft}

\vfill

Dieses Skript ist aus den Fragen und Bemerkungen von Studenten des
Diplom, Bachelor und Master-Studiengangs Informatik an der Hochschule
RheinMain (ehemals Fachhochschule Wiesbaden) hervorgegangen. Ich danke
allen meinen Studenten f�r konstruktive Anmerkungen und
Verbesserungen. Dabei m�chte ich besonders Frau Carola Henzel nennen,
die sehr viele Tippfehler (Mengen, Summen und Beweistechniken)
berichtigte. Herr Kim Stebel hat die Bemerkungen zu bipartiten Graphen
verbessert.

Naturgem�� ist ein Skript nie fehlerfrei (ganz im Gegenteil!) und es
�ndert (mit Sicherheit!) sich im Laufe der Zeit. Deshalb bin ich auf
weitere Verbesserungsvorschl�ge angewiesen.

\cleardoublepage

\special{pdf: out 2 << /Title 
(Inhaltsverzeichnis) 
/Dest [ @thispage /FitH @ypos ] >>}
\tableofcontents

\cleardoublepage

% Seitenstil (Fuss- und Kopfzeilen)
\pagestyle{headings}
\pagestyle{fancy}
\pagenumbering{arabic}
	
%%%%%%%%%%%%%%%%%%%%%%%%% Sektion 1 %%%%%%%%%%%%%%%%%%%%%%%%%%%%%%%

% Einige grundlegende Notationen
% Mengen, Relationen und Funktionen
\ifpdf
\special{pdf: out 2 << /Title 
(Elementare Begriffe und Schreibweisen) 
/Dest [ @thispage /FitH @ypos ] >>}
\fi
\section{Elementare Begriffe und Schreibweisen}

\ifpdf
\special{pdf: out 3 << /Title 
(Elementbeziehung und Enthaltenseinsrelation) 
/Dest [ @thispage /FitH @ypos ] >>}
\fi
\subsection{Elementbeziehung und Enthaltenseinsrelation}


% Einige Grundlagen ueber Summen und Produkte
\subsection{Summen und Produkte}

\subsubsection{Summen}
Zur abkürzenden Schreibweise verwendet man für Summen das
Summenzeichen $\sum$\index{$\sum$}. Dabei ist
\begin{displaymath}
\sum_{i=1}^n a_i \eqd a_1 + a_2 + \dots + a_n.
\end{displaymath}
Mit Hilfe dieser Definition ergeben sich auf elementare Weise die
folgenden Rechenregeln:
\begin{itemize}
%
\item Sei $a_i = a$ für $1 \le i \le n$, dann gilt $\sum\limits_{i=1}^n a_i =
  n \cdot a$ (Summe gleicher Summanden).
%
\item $\sum\limits_{i=1}^n a_i = \sum\limits_{i=1}^m a_i +
  \sum\limits_{i = m + 1}^n a_i$, wenn $1 < m < n$ (Aufspalten einer Summe).
%
\item $\sum\limits_{i=1}^n (a_i + b_i + c_i + \dots) =
  \sum\limits_{i=1}^n a_i + \sum\limits_{i=1}^n b_i +
  \sum\limits_{i=1}^n c_i + \dots$ (Addition von Summen).
%
\item $\sum\limits_{i=1}^n a_i = \sum\limits_{i=l}^{n + l - 1} a_{i-l+1}$
  und $\sum\limits_{i=l}^n a_i = \sum\limits_{i=1}^{n - l + 1} a_{i + l
  - 1}$
  (Umnumerierung von Summen).
%
\item $\sum\limits_{i=1}^n \sum\limits_{j=1}^m a_{i,j} =
  \sum\limits_{j=1}^m \sum\limits_{i=1}^n a_{i,j}$ (Vertauschen der Summationsfolge).
%
\end{itemize}
Manchmal verwendet man keine Laufindizes an ein Summenzeichen, sondern man beschreibt 
(durch ein Prädikat) welche Zahlen aufsummiert werden sollen. So kann man eine Funktion definieren,
die die Summe aller Teiler einer natürlichen Zahl liefert:
\begin{displaymath}
\sigma(n) \eqd \sum\limits_{\begin{array}{c}\scriptstyle t\, \le\, n\\\scriptstyle t \text{ teilt } n\end{array}} t
\end{displaymath}

\subsubsection{Produkte}

Zur abkürzenden Schreibweise verwendet man für Produkte das
Produktzeichen $\prod$\index{$\prod$}. Dabei ist
\begin{displaymath}
\prod_{i=1}^n a_i \eqd a_1 \cdot a_2 \multdots a_n.
\end{displaymath}
Mit Hilfe dieser Definition ergeben sich auf elementare Weise die
folgenden Rechenregeln:
\begin{itemize}
%
\item Sei $a_i = a$ für $1 \le i \le n$, dann gilt $\prod\limits_{i=1}^n a_i =
  a^n$ (Produkt gleicher Faktoren).
%
\item  $\prod\limits_{i=1}^n (c a_i) = c^n \prod\limits_{i=1}^n a_i$
  (Vorziehen von konstanten Faktoren)
%
\item $\prod\limits_{i=1}^n a_i = \prod\limits_{i=1}^m a_i \cdot
  \prod\limits_{i = m + 1}^n a_i$ , wenn $1 < m < n$ (Aufspalten in Teilprodukte).
%
\item $\prod\limits_{i=1}^n (a_i \cdot b_i \cdot c_i \cdot \ldots) =
  \prod\limits_{i=1}^n a_i \cdot \prod\limits_{i=1}^n b_i \cdot
  \prod\limits_{i=1}^n c_i \cdot \ldots$ (Das Produkt von Produkten).
%
\item $\prod\limits_{i=1}^n a_i = \prod\limits_{i=l}^{n + l - 1} a_{i-l+1}$
  und $\prod\limits_{i=l}^n a_i = \prod\limits_{i=1}^{n - l + 1} a_{i + l
  - 1}$
  (Umnumerierung von Produkten).
%
\item $\prod\limits_{i=1}^n \prod\limits_{j=1}^m a_{i,j} =
  \prod\limits_{j=1}^m \prod\limits_{i=1}^n a_{i,j}$ (Vertauschen der
  Reihenfolge bei Doppelprodukten).
%
\end{itemize}
Ähnlich wie bei Summen kann man bei Produkten auch ohne Laufindex arbeiten. So ist z.B.~die
\dindex{Eulersche $\phi$-Funktion} wie folgt definiert:
\begin{displaymath}
\phi(n) \eqd n \prod_{p \mid n} (1 - \frac{1}{p})
\end{displaymath}
In das Produkt gehen also alle Primteiler $p$ von $n$ mit dem Faktor $1 - 1/p$ ein.

Oft werden Summen- oder Produktsymbole verwendet bei denen der Startindex 
größer als der Stopindex ist. Solche Summen bzw.~Produkte sind "`leer"', 
d.h.~es wird nichts summiert bzw.~multipliziert. Sind dagegen Start- und
Endindex gleich, so tritt nur genau ein Wert auf, d.h.~das Summen- 
bzw.~Produktsymbol hat in diesem Fall keine Auswirkung. Es ergeben sich
also die folgenden Rechenregeln:
\begin{itemize}
%
\item Seien $n, m \in \Z$ und $n < m$, dann 
\begin{displaymath}
\sum_{i=m}^n a_i = 0  \text{ und } \prod_{i = m}^n a_i = 1
\end{displaymath}
%
\item Sei $n \in \Z$, dann 
\begin{displaymath}
\sum_{i=n}^n a_i = a_n = \prod_{i = n}^n a_i
\end{displaymath}
\end{itemize}

\begin{example}
Die folgende Identität wird Euler zugeschrieben. Erstaunlicherweise kann man damit eine Summe von Kehrwerten von Potenzen mit einem Produkt von Primzahlen in Verbindung bringen.
\begin{displaymath}
\sum_{n=1}^{\infty}{\frac{1}{n^s}} = \prod_{p} {\frac{1}{1 - \frac{1}{p^s}}}
\end{displaymath}
Diese Identiät kann man 
Die Summe auf der linken Seite ist auch als \dindex{Riemannsche Zetafuntion $\zeta(s)$} bekannt. Erstaunlicherweise gilt dann der folgende Zusammenhang
\begin{displaymath}
\zeta(2) = \sum_{n=1}^{\infty}{\frac{1}{n^2}} = \prod_{p} {\frac{1}{1 - \frac{1}{p^2}}} = \frac{\pi^2}{6},
\end{displaymath}
denn durch diese Gleichung wird die Menge der Primzahlen in Beziehung zur Kreiszahl $\pi$ gebracht.
\end{example}


% Logarithmen- und Potenzgesetze
\goodbreak
\subsection{Logarithmieren, Potenzieren und Radizieren}
Die Schreibweise $a^b$ ist eine Abkürzung für 
\begin{displaymath}
a^b \eqd \underbrace{a \cdot a
\cdot \ldots \cdot a}_{b-\text{mal}} 
\end{displaymath}
und wird als \dindex{Potenzierung} bezeichnet. Dabei ist $a$ die \dindex{Basis}, $b$ der \dindex{Exponent} und $a^b$ die $b$-te \dindex{Potenz} von $a$. Seien nun  $r,s,t \in \R$ und $r,t \ge 0$ durch die folgende Gleichung verbunden:
\begin{displaymath}
r^s = t.
\end{displaymath}
Dann lässt sich diese Gleichung wie folgt umstellen und es gelten die
folgenden Rechenregeln\index{Logarithmus}\index{Wurzel}\index{Radizieren}:

\begin{center}
\begin{tabular}{c|c|c}
Logarithmieren & Potenzieren & Radizieren\\
\hline
$\mathbf{s = \log_r t}$ & $\mathbf{t = r^s}$ &
\phantom{$\left(\frac{\frac{c}{d}a}{b}\right)$} $\mathbf{r = \sqrt[\mathbf{s}]{\mathbf{t}}}$\\
\hline
\begin{minipage}[t]{0.38\textwidth}
\begin{enumerate}[i)]
%
\item $\log_r (\frac{u}{v}) = \log_r u - \log_r v$
%
\item $\log_r ({u} \cdot {v}) = \log_r u + \log_r v$
%
\item $\log_r (t^u) = u \cdot \log_r t$
%
\item $\log_r (\sqrt[u]{t}) = \frac{1}{u} \cdot \log_r t$
%
\item $\frac{\log_r t}{\log_r u} = \log_u t$ (Basiswechsel)
%
\end{enumerate}
\end{minipage}
&
\begin{minipage}[t]{0.25\textwidth}
\begin{enumerate}[i)]
%
\item $r^{u} \cdot r^{v} = r^{u + v}$
%
\item $\frac{r^{u}}{r^{v}} = r^{u - v}$
%
\item $u^{s} \cdot v^{s} = (u \cdot v)^{s}$
%
\item $\frac{u^{s}}{v^{s}} = \left(\frac{u}{v}\right)^{s}$
%
\item $(r^{u})^{v} = r^{u \cdot v}$ 
%
\end{enumerate}
\end{minipage}
&
\begin{minipage}[t]{0.28\textwidth}
%
\begin{enumerate}[i)]
%
\item $\sqrt[\leftroot{1} s]{u} \cdot \sqrt[s]{v} = \sqrt[s]{u \cdot v}$
%
\item $\frac{\sqrt[s]{u}}{\sqrt[s]{v}} = \sqrt[s]{\left(\frac{u}{v}\right)}$
%
\item $\sqrt[u]{\sqrt[v]{t}} =
  \sqrt[u \cdot v]{t}$
%
\end{enumerate}
\end{minipage}\\
\end{tabular}
\end{center}
Zusätzlich gilt: Wenn $r > 1$, dann ist $s_1 < s_2$ \gdw $r^{s_1} <
r^{s_2}$ (Monotonie).

Da $\sqrt[s]{t} = t^{\left(\frac{1}{s}\right)}$ gilt, können die Gesetze für das Radizieren leicht aus den Potenzierungsgesetzen abgeleitet werden.  Weiterhin legen wir spezielle Schreibweisen für die Logarithmen\index{Logarithmus} zur Basis $10$, $e$ (Eulersche Zahl) und $2$ fest: $\lg t \eqd \log_{10} t$, $\ln t \eqd \log_{e} t$ und $\mathrm{lb}\, t \eqd \log_{2} t$.

\begin{example}
Die folgende Identität wird Euler zugeschrieben. Erstaunlicherweise kann man damit eine Summe von Kehrwerten von Potenzen mit einem Produkt von Primzahlen in Verbindung bringen.
\begin{displaymath}
\sum_{n=1}^{\infty}{\frac{1}{n^s}} = \prod_{p} {\frac{1}{1 - \frac{1}{p^s}}}
\end{displaymath}
Die Summe auf der linken Seite ist auch als \dindex{Riemannsche Zetafuntion $\zeta(s)$} bekannt.
\end{example}

% Griechische Buchstaben
\special{pdf: out 2 << /Title 
(Gebräuchliche griechische Buchstaben) 
/Dest [ @thispage /FitH @ypos ] >>}
\subsection{Gebräuchliche griechische Buchstaben}
In der Informatik, Mathematik und Physik ist es üblich, griechische 
Buchstaben zu verwenden. Ein Grund hierfür ist, dass es so möglich 
wird mit einer größeren Anzahl von Unbekannten arbeiten zu können, ohne
unübersichtliche und oft unhandliche Indizes benutzen zu müssen.
\index{griechische Buchstaben}\index{Buchstaben!griechische}

\bigskip

\noindent Kleinbuchstaben:
\begin{displaymath}
\begin{array}{c|c||c|c||c|c}
\text{Symbol} & \text{Bezeichnung} & \text{Symbol}
& \text{Bezeichnung} & \text{Symbol} & \text{Bezeichnung}\\
\hline
\alpha & \text{Alpha} & \beta   & \text{Beta}   & \gamma   & \text{Gamma}\\
\hline
\delta & \text{Delta} & \phi    & \text{Phi}    & \varphi  & \text{Phi}\\
\hline
\xi    & \text{Xi}    & \zeta   & \text{Zeta}   & \epsilon & \text{Epsilon}\\         
\hline
\theta & \text{Theta} & \lambda & \text{Lambda} & \pi      & \text{Pi}\\
\hline
\sigma & \text{Sigma} & \eta    & \text{Eta}    & \mu      & \text{Mu}
\end{array}
\end{displaymath}

\bigskip

\noindent Grossbuchstaben:
\begin{displaymath}
\begin{array}{c|c||c|c||c|c}
\text{Symbol} & \text{Bezeichnung} & \text{Symbol}
& \text{Bezeichnung} & \text{Symbol} & \text{Bezeichnung}\\
\hline
\Gamma & \text{Gamma} & \Delta & \text{Delta} & \Phi    & \text{Phi}\\
\hline
\Xi    & \text{Xi}    & \Theta & \text{Theta} & \Lambda & \text{Lambda}\\
\hline
\Pi    & \text{Pi}    & \Sigma & \text{Sigma} & \Psi    & \text{Psi}\\
\hline
\Omega & \text{Omega} & &
\end{array}
\end{displaymath}





% Grundlagen der Logik
\ifpdf
\special{pdf: out 2 << /Title 
(Einige Grundlagen der Logik) 
/Dest [ @thispage /FitH @ypos ] >>}
\fi
\section{Einige Grundlagen der Logik}
Aussagen sind entweder \dindex{wahr} ($\triangleq 1$) oder
\dindex{falsch}($\triangleq 0$). So ist die Aussage 
\begin{center}
"`Wiesbaden liegt am Mittelmeer"'
\end{center}
sicherlich falsch, wogegen die Aussage
\begin{center}
"`Wiesbaden liegt in Hessen"'
\end{center}
sicherlich wahr ist. Aussagen werden meist durch
\dindex{Aussagenvariablen} ausgedr�ckt, die nur die Werte $0$ oder $1$
annehmen k�nnen. Um die Verkn�pfung von Aussagen auch formal
aufschreiben zu k�nnen, werden die folgenden logischen Operatoren
verwendet

\begin{center}
\begin{tabular}{l|l|l}
Symbol & umgangssprachlicher Name & Name in der Logik\\
\hline
\dindex{$\sand$} & und & Konjunktion\\
\dindex{$\sor$} & oder & Disjunktion / Alternative\\
\dindex{$\sneg$} & nicht & Negation \\
\dindex{$\simpl$} & folgt & Implikation\\
\dindex{$\sequi$} & genau dann wenn (gdw.) & �quivalenz\\
\end{tabular}
\end{center}

Zus�tzlich werden noch die Quantoren $\exists$ ("`es existiert"') und
$\forall$ ("`f�r alle"') verwendet, die z.B.~wie folgt gebraucht
werden k�nnen
\begin{description}
%
\item $\forall x \colon p(x)$ "`F�r alle $x$ gilt die Aussage $p(x)$. 
%
\item $\exists x \colon p(x)$ "`Es existiert ein $x$, f�r das die Aussage
  $p(x) gilt$.
%
\end{description}

\begin{example}
Die Aussage "`Jede gerade nat�rliche Zahl kann als Produkt von $2$ und einer
anderen nat�rlichen Zahl geschrieben werden"' l�sst sich dann wie
folgt schreiben
\begin{displaymath}
(\forall n \in \N \colon n \text{ ist gerade }) \simpl (\exists m
\in \N \colon n = 2 \cdot m) 
\end{displaymath}

F�r die logischen Konnektoren sind die folgenden Wahrheitswertetafeln
festgelegt:

\begin{center}
\begin{tabular}{c|c}
$p$ & $\neg p$\\
\hline
$0$ & $1$\\
$1$ & $0$
\end{tabular}
und
\begin{tabular}{c|c|c|c|c|c}
$p$ & $q$ & $p \wedge q$ & $p \vee q$ & $p \simpl q$ & $p \sequi q$\\
\hline
0 & 0 & 0 & 0 & 1 & 1\\   
0 & 1 & 0 & 1 & 1 & 0\\
1 & 0 & 0 & 1 & 0 & 0\\ 
1 & 1 & 0 & 1 & 1 & 1
\end{tabular}
\end{center}

\end{example}


% Graphentheorie
\special{pdf: out 2 << /Title 
(Graphen und Graphenalgorithmen) 
/Dest [ @thispage /FitH @ypos ] >>}
\section{Graphen und Graphenalgorithmen}


\ifpdf
\special{pdf: out 3 << /Title 
(Einf�hrung) 
/Dest [ @thispage /FitH @ypos ] >>}
\fi
\subsection{Einf�hrung}
Sehr viele Probleme lassen sich durch Objekte und Verbindungen oder
Beziehungen zwischen diesen Objekten beschreiben. Ein sch�nes Beispiel
hierf�r ist das 
\dindex{K�nigsberger Br�ckenproblem}\index{Br�ckenproblem}, das $1736$ 
von Leonhard Euler\footnote{Der Schweizer Mathematiker Leonhard Euler 
wurde $1707$ in Basel geboren und starb $1783$ in St.~Petersburg.}  
formuliert und gel�st wurde. Zu dieser Zeit hatte 
K�nigsberg\footnote{K�nigsberg hei�t heute Kaliningrad.} genau sieben 
Br�cken wie die folgende sehr grobe Karte zeigt:

\begin{figure}[h]
\begin{center}
\includegraphics[scale=0.65]{BrueckenProb.eps}
\end{center}
\end{figure}

\begin{wrapfigure}[15]{r}{8cm}
\begin{center}
\includegraphics[scale=1.0]{BrueckenProb2.eps}
\end{center}
\caption{Der formalisierte Stadtplan.}
\label{BProb2}
\end{wrapfigure}
Die verschiedenen Stadtteile sind dabei mit A-D bezeichnet. Euler
stellte sich nun die Frage, ob es m�glich ist, einen Spaziergang in
einem beliebigen Stadtteil zu beginnen, jede Br�cke \emph{genau
einmal} zu �berqueren und den Spaziergang am Startpunkt zu
beenden. Ein solcher Weg soll \emph{Euler-Spaziergang} hei�en. Die
Frage l�sst sich leicht beantworten, wenn der Stadtplan wie nebenstehend
formalisiert wird.

Die Stadtteile sind bei der Formalisierung zu Knoten geworden und die
Br�cken werden durch Kanten zwischen den Knoten
symbolisiert\footnote{Abbildung \ref{BProb2} nennt
man \dindex{Multigraph}, denn hier starten mehrere Kanten von
\emph{einem} Knoten und enden in \emph{einem} anderen Knoten.}.

Angenommen es g�be in K�nigsberg einen Euler-Spazier\-gang, dann m�sste f�r jeden
Knoten in Abbildung \ref{BProb2} die folgende Eigenschaft erf�llt
sein: die Anzahl der Kanten die mit einem Knoten verbunden sind ist
gerade, weil f�r jede Ankunft (�ber eine Br�cke) in einem Stadtteil
ein Verlassen eines Stadtteil (�ber eine Br�cke) notwendig ist.

\ifpdf
\special{pdf: out 3 << /Title 
(Grundlagen) 
/Dest [ @thispage /FitH @ypos ] >>}
\fi
\subsection{Grundlagen}
Die Theorie der Graphen ist heute zu einem unverzichtbaren Bestandteil
der Informatik geworden. Viele Probleme, wie z.B.~das Verlegen von
Leiterbahnen auf einer Platine, die Modellierung von Netzwerken oder
die L�sung von Routingproblemen in Verkehrsnetzen benutzen Graphen
oder Algorithmen, die Graphen als Datenstruktur verwenden. Auch schon
bekannte Datenstrukturen wie Listen und B�ume k�nnen als Graphen
aufgefasst werden. All dies gibt einen Anhaltspunkt, dass die
Graphentheorie eine sehr zentrale Rolle f�r die Informatik spielt und
vielf�ltige Anwendungen hat. In diesem Kontext ist es wichtig zu
bemerken, dass der Begriff des Graphen in der Informatik \emph{nicht}
im Sinne von Graph einer Funktion gebraucht wird, sondern wie folgt
definiert ist:

\begin{definition}
Ein \dindex{gerichteter Graph}\index{Graph!gerichtet} $G = (V,E)$ ist
ein Paar, das aus einer Menge von \dindex{Knoten} $V$ und einer Menge
von \dindex{Kanten} $E \subseteq V \times V$
(\dindex{Kantenrelation}\index{Relation!Kante}) besteht. Eine Kante
$k = (u,v)$ aus $E$ kann als Verbindung zwischen den Knoten $u,v \in
V$ aufgefasst werden. Aus diesem Grund nennt man $u$
auch \dindex{Startknoten}\index{Knoten!Start} und
$v$ \dindex{Endknoten}\index{Knoten!End}. Zwei Knoten, die durch eine
Kante verbunden sind, hei�en auch \dindex{benachbart}
oder \dindex{adjazent}.

Ein Graph $H = (V', E')$ mit $V' \subseteq V$ und $E' \subset E$ hei�t
\dindex{Untergraph} von $G$.
\end{definition}

Ein Graph $(V,E)$ hei�t \dindex{endlich} \gdw die Menge der Knoten $V$
endlich ist. Obwohl man nat�rlich auch unendliche Graphen betrachten
kann, werden wir uns in diesem Abschnitt nur mit endlichen Graphen
besch�ftigen, da diese f�r den Informatiker von gro�em Nutzen sind.

\bigskip

Da wir eine Kante $(u,v)$ als Verbindung zwischen den Knoten $u$ und
$v$ interpretieren k�nnen, bietet es sich an, Graphen durch Diagramme
darzustellen. Dabei wird die Kante $(u,v)$ durch einen Pfeil von $u$
nach $v$ dargestellt. Drei Beispiele f�r eine bildliche Darstellung
von gerichteten Graphen finden sich in Abbildung \ref{gGraphen}.

\ifpdf
\special{pdf: out 3 << /Title 
(Einige Eigenschaften von Graphen) 
/Dest [ @thispage /FitH @ypos ] >>}
\fi
\subsection{Einige Eigenschaften von Graphen}
Der Graph in Abbildung \ref{gGraphen}(c) hat eine besondere
Eigenschaft, denn offensichtlich kann man die Knotenmenge $V_{1c}
= \set{0,1,2,3,4,5,6,7,8}$ in zwei disjunkte Teilmengen $V_{1c}^l
= \set{0,1,2,3}$ und $V_{1c}^r = \set{4,5,6,7,8}$ so aufteilen, dass
keine Kante zwischen zwei Knoten aus $V_{1c}^l$ oder $V_{1c}^r$
verl�uft.

\begin{definition}
Ein Graph $G = (V,E)$ hei�t \dindex{bipartit}, wenn gilt:
\begin{enumerate}
%
\item Es gibt zwei Teilmengen $V^l$ und $V^r$ von $V$ mit $V =
V^l \cup V^r$, $V^l \cap V^r = \emptyset$ und 
%
\item f�r jede Kante $(u,v) \in E$ gilt $u \in V^l$ und $v \in V^r$.
%
\end{enumerate}
\end{definition}

Bipartite Graphen haben viele Anwendungen, weil man jede bin�re
Relation $R \subseteq A \times B$ ganz nat�rlich als bipartiten Graph
auffassen kann, dessen Kanten von Knoten aus $A$ zu Knoten aus $B$
laufen.

\begin{example}
Gegeben sei ein bipartiter Graph $G = (V,E)$ mit $V = V^F \cup V^M$
und $V^F \cap V^M = \emptyset$. Die Knoten aus $V^F$ symbolisieren
Frauen und $V^M$ symbolisiert eine Menge von M�nnern. Kann sich eine
Frau vorstellen einen Mann zu heiraten, so wird der entsprechende
Knoten aus $V^F$ mit dem passenden Knoten aus $V^M$ durch eine Kante
verbunden.  Eine \dindex{Heirat} ist nun eine Kantenmenge $H \subseteq
E$, so dass keine zwei Kanten aus $H$ einen gemeinsamen Knoten
besitzen. Das \dindex{Heiratsproblem} ist nun die Aufgabe f�r $G$ eine
Heirat $H$ zu finden, so dass alle Frauen heiraten k�nnen, d.h.~es ist
das folgende Problem zu l�sen:

\goodbreak
\prob{MARRIAGE}{%
Bipartiter Graph $G = (V,E)$ mit $V = V^F \cup V^M$ und $V^F \cap V^M
= \emptyset$}{%
Eine Heirat $H$ mit $\cnt H = \cnt V^F$
}

Im Beispielgraphen \ref{gGraphen}(c) gibt es keine L�sung f�r das
Heiratsproblem, denn f�r die Knoten ($\triangleq$ Kandidatinnen) $2$ und
$3$ existieren nicht ausreichend viele Partner, d.h.~keine Heirat in
diesem Graphen enth�lt zwei Kanten die sowohl $2$ als auch $3$ als
Startknoten haben.

\medskip

Obwohl dieses Beispiel auf den ersten Blick nur von untergeordneter
Bedeutung erscheint, kann man es auf eine Vielfalt von Anwendungen
�bertragen. Immer wenn die Elemente zweier disjunkter Mengen durch
eine Beziehung verbunden sind, kann man dies als bipartiten Graphen
auffassen. Sollen nun die Bed�rfnisse der einen Menge v�llig
befriedigt werden, so ist dies wieder ein Heiratsproblem. Beispiele
mit mehr praktischem Bezug finden sich u.a.~bei Beziehungen zwischen
K�ufern und Anbietern.
\end{example}

\begin{figure}
\centering
\subfigure[Ein gerichteter Graph mit $5$
Knoten]{\includegraphics[scale=1.2]{graphex1.eps}}
\hspace*{2em}
\subfigure[Ein planarer gerichteter
Graph mit $5$ Knoten]{\includegraphics[scale=1.2]{graphex3.eps}}
\hspace*{2em}
\subfigure[Ein gerichteter bipartiter 
Graph]{\includegraphics[scale=1.2]{graphex2.eps}}
\caption{Beispiele f�r gerichtete Graphen}
\label{gGraphen}
\end{figure}

Oft beschr�nken wir uns auch auf eine Unterklasse von Graphen, bei
denen die Kanten keine "`Richtung"' haben (siehe
Abbildung \ref{ugGraphen}) und einfach durch eine Verbindungslinie
symbolisiert werden k�nnen:

\begin{definition}
Sei $G=(V,E)$ ein Graph. Ist die Kantenrelation
$E$ \dindex{symmetrisch}, d.h.~gibt es zu jeder Kante $(u,v) \in E$
auch eine Kante $(v,u) \in E$ (siehe auch Abschnitt \ref{PropRel}),
dann bezeichnen wir $G$ als \dindex{ungerichteten
Graphen}\index{Graph!ungerichtet} oder kurz als \dindex{Graph}.
\end{definition}

Es ist praktisch, die Kanten $(u,v)$ und $(v,u)$ eines ungerichteten
Graphen als Menge $\set{u,v}$ mit zwei Elementen aufzufassen. Diese
Vorgehensweise f�hrt zu einem kleinen technischen Problem. Eine Kante
$(u,u)$ mit gleichem Start- und Endknoten nennen wir, entsprechend der
intuitiven Darstellung eines Graphens als Diagramm, \dindex{Schleife}.
Wandelt man nun solch eine Kante in eine Menge um, so w�rde nur eine
einelementige Menge entstehen. Aus diesem Grund legen wir fest, dass
ungerichtete Graphen \dindex{schleifenfrei} sind.

\begin{definition}
Der (ungerichtete) Graph $K = (V,E)$ hei�t \dindex{vollst�ndig}, wenn
f�r alle $u,v \in V$ mit $u \neq v$ auch $(u,v) \in E$ gilt,
d.h.~jeder Knoten des Graphen ist mit allen anderen Knoten
verbunden. Ein Graph $O=(V,\emptyset)$ ohne Kanten wird
als \dindex{Nullgraph}\index{Graph!Null} bezeichnet.
\end{definition}
Mit dieser Definition ergibt sich, dass die Graphen in
Abbildung \ref{ugGraphen}(a) und Abbildung \ref{ugGraphen}(b)
vollst�ndig sind. Der Nullgraph $(V,\emptyset)$ ist Untergraph jedes
beliebigen Graphen $(V,E)$. Diese Definitionen lassen sich nat�rlich
auch analog auf gerichtete Graphen �bertragen.

\begin{figure}
\centering
\subfigure[Vollst�ndiger ungerichteter Graph 
$K_{16}$]{\includegraphics[scale=0.7]{kclique.eps}}
\hfill
\subfigure[Vollst�ndiger ungerichteter Graph
$K_{20}$]{\includegraphics[scale=0.7]{kclique3.eps}}
\subfigure[Zuf�lliger Graph mit $32$ Knoten]{\includegraphics[scale=0.7]{random.eps}}
\hfill
\subfigure[Regul�rer Graph mit Grad $3$]{\includegraphics[scale=0.7]{moebius.eps}}
\caption{Beispiele f�r ungerichtete Graphen}
\label{ugGraphen}
\end{figure}

\begin{definition}
Sei $G = (V,E)$ ein gerichteter Graph und $v \in V$ ein beliebiger
Knoten. Der \dindex{Ausgrad} von $v$ (kurz:
$\mathrm{outdeg}(v)$\index{outdeg=$\mathrm{outdeg}(v)$}) ist dann 
die Anzahl der Kanten in $G$, die $v$ als Startknoten haben. Analog 
ist der \dindex{Ingrad} von $v$ 
(kurz: $\mathrm{indeg}(v)$\index{indeg=$\mathrm{indeg}(v)$}) die 
Anzahl der Kanten in $G$, die $v$ als Endknoten haben.

Bei ungerichteten Graphen gilt f�r jeden Knoten
$\mathrm{outdeg}(v)=\mathrm{indeg}(v)$. Aus diesem Grund schreiben wir
kurz $\mathrm{deg}(v)$ und bezeichnen dies als \emph{Grad von
$v$}\index{Grad}. 
Ein Graph $G$ hei�t \dindex{regul�r} \gdw alle
Knoten von $G$ den gleichen Grad haben.
\end{definition}
Die Diagramme der Graphen in den Abbildungen \ref{gGraphen}
und \ref{ugGraphen} haben die Eigenschaft, dass sich einige Kanten
schneiden. Es stellt sich die Frage, ob man diese Diagramme auch so
zeichnen kann, dass keine �berschneidungen auftreten. Diese
Eigenschaft von Graphen wollen wir durch die folgende Definition
festhalten:
\begin{definition}
Ein Graph $G$ hei�t \dindex{planar}, wenn sich sein Diagramm ohne
�berschneidungen zeichnen l��t.
\end{definition}

\begin{example}
Der Graph in Abbildung \ref{gGraphen}(a) ist, wie man leicht
nachpr�fen kann, planar, da die Diagramme aus
Abbildung \ref{gGraphen}(a) und \ref{gGraphen}(b) den gleichen Graphen
repr�sentieren.
\end{example}
Auch planare Graphen haben eine anschauliche Bedeutung. Der Schaltplan
einer elektronischen Schaltung kann als Graph aufgefasst werden. Die
Knoten entsprechen den Stellen an denen die Bauteile aufgel�tet werden
m�ssen, und die Kanten entsprechen den Leiterbahnen auf der
Platine. In diesem Zusammenhang bedeutet planar, ob man die
Leiterbahnen kreuzungsfrei verlegen kann, d.h.~ob es m�glich ist, eine
Platine zu fertigen, die mit einer Kupferschicht auskommt. In der
Praxis kommen oft Platinen mit mehreren Schichten zum Einsatz
("`Multilayer-Platine"'). Ein Grund daf�r kann sein, dass der
"`Schaltungsgraph"' nicht planar war und deshalb mehrere Schichten
ben�tigt werden. Da Platinen mit mehreren Schichten in der Fertigung
deutlich teurer sind als solche mit einer Schicht, hat die
Planarit�tseigenschaft von Graphen somit auch unmittelbare finanzielle
Auswirkungen.

\ifpdf
\special{pdf: out 3 << /Title 
(Wege, Kreise, W�lder und B�ume) 
/Dest [ @thispage /FitH @ypos ] >>}
\fi
\subsection{Wege, Kreise, W�lder und B�ume}

\begin{definition}
Sei $G=(V,E)$ ein Graph und $u,v \in V$. Eine Folge von Knoten
$\enu{u}{0}{l} \in V$ mit $u = u_0$, $v = u_l$ und $(u_i,u_{i+1}) \in
E$ f�r $0 \le i \le l - 1$ hei�t \emph{Weg von $u$ nach $v$ der L�nge
$l$}\index{Weg}. Der Knoten $u$
wird \dindex{Startknoten}\index{Knoten!Start} und $v$
wird \dindex{Endknoten}\index{Knoten!End} des Wegs genannt.

Ein Weg, bei dem Start- und Endknoten gleich sind,
hei�t \dindex{geschlossener Weg}\index{Weg!geschlossen}. Ein
geschlossener Weg, bei dem kein Knoten au�er dem Startknoten mehrfach
enthalten ist, wird \dindex{Kreis} genannt.
\end{definition}
Mit dieser Definition wird klar, dass der Graph in
Abbildung \ref{gGraphen}(a) den Kreis $1,2,3,\dots,5,1$ mit
Startknoten $1$ hat.

\begin{definition}
Sei $G=(V,E)$ ein Graph. Zwei Knoten $u,v \in V$
hei�en \dindex{zusammenh�ngend}, wenn es einen Weg von $u$ nach $v$
gibt. Der Graph $G$ hei�t \dindex{zusammenh�ngend}, wenn jeder Knoten
von $G$ mit jedem anderen Knoten von $G$ zusammenh�ngt. 

Sei $G'$ ein zusammenh�ngender Untergraph von $G$ mit einer besonderen
Eigenschaft: Nimmt man einen weiteren Knoten von $G$ zu $G'$ hinzu,
dann ist der neu entstandene Graph nicht mehr zusammenh�ngend, d.h.~es
gibt keinen Weg zu diesem neu hinzugekommenen Knoten. Solch einen
Untergraph nennt man \dindex{Zusammenhangskomponente}.
\end{definition}
Offensichtlich sind die Graphen in den
Abbildungen \ref{gGraphen}(a), \ref{ugGraphen}(a), \ref{ugGraphen}(b)
und \ref{ugGraphen}(d) zusammenh�ngend und haben genau eine
Zusammenhangskomponente. Man kann sich sogar leicht �berlegen, dass
die Eigenschaft "`$u$ h�ngt mit $v$"' zusammen
eine \emph{�quivalenzrelation} (siehe Abschnitt \ref{PropRel})
darstellt.

Mit Hilfe der Definition des geschlossenen Wegs l�sst sich nun der
Begriff der B�ume definieren, die eine sehr wichtige Unterklasse der
Graphen darstellen.
\goodbreak
\begin{definition}
Ein Graph $G$ hei�t
\begin{itemize}
%
\item \dindex{Wald}, wenn es keinen geschlossenen Weg mit L�nge $\ge
1$ in $G$ gibt und 
%
\item \dindex{Baum}, wenn $G$ ein zusammenh�ngender Wald ist,
d.h.~wenn er nur genau eine Zusammenhangskomponente hat.
%
\end{itemize}
\end{definition}

\begin{figure}
\begin{center}
\includegraphics[scale=0.65]{wald.eps}
\end{center}
\caption{Ein Wald mit zwei B�umen}
\label{wald}
\end{figure}

\ifpdf
\special{pdf: out 3 << /Title 
(Die Repr�sentation von Graphen und einige Algorithmen) 
/Dest [ @thispage /FitH @ypos ] >>}
\fi
\subsection{Die Repr�sentation von Graphen und einige Algorithmen}
Nachdem Graphen eine gro�e Bedeutung sowohl in der praktischen als
auch in der theoretischen Informatik erlangt haben, stellt sich noch
die Frage, wie man Graphen effizient als Datenstruktur in einem
Computer ablegt. Dabei soll es m�glich sein, Graphen effizient zu
speichern und zu manipulieren. 

Die erste Idee, Graphen als dynamische Datenstrukturen zu
repr�sentieren, scheitert an dem relativ ineffizienten Zugriff auf die
Knoten und Kanten bei dieser Art der Darstellung. Sie ist nur von
Vorteil, wenn ein Graph nur sehr wenige Kanten enth�lt. Die folgende
Methode der Speicherung von Graphen hat sich als effizient erwiesen
und erm�glicht auch die leichte Manipulation des Graphens:
\begin{definition}
Sei $G=(V,E)$ ein gerichteter Graph mit $V = \set{\enu{v}{1}{n}}$. Wir
definieren eine $n \times n$ Matrix $A_G =(a_{i,j})_{1 \le i,j, \le
n}$ durch
\begin{displaymath}
a_{i,j} =
\left\{
\begin{array}{ll}
1,& \text{ falls $(v_i, v_j) \in E$}\\
0,& \text{ sonst}
\end{array}
\right.
\end{displaymath}
Die so definierte Matrix $A_G$ mit Eintr�gen aus der Menge $\set{0,1}$
hei�t \dindex{Adjazenzmatrix} von $G$.
\end{definition}

\begin{example}
F�r den gerichteten Graphen aus Abbildung \ref{gGraphen}(a) ergibt sich die
folgende Adjazenzmatrix:
\begin{displaymath}
A_{G_5} =
\left(
\begin{array}{ccccc}
0 & 1 & 0 & 0 & 0\\ 
0 & 0 & 1 & 0 & 0\\
0 & 0 & 0 & 1 & 0\\
0 & 0 & 0 & 0 & 1\\
1 & 0 & 0 & 0 & 0
\end{array}
\right)
\end{displaymath}
Die Adjazenzmatrix eines ungerichteten Graphen erkennt man daran, dass
sie spiegelsymmetrisch zu Diagonale von links oben nach rechts unten
ist (die Kantenrelation\index{Kantenrelation} ist symmetrisch) und
dass die Diagonale aus $0$-Eintr�gen besteht (der Graphen hat keine
Schleifen). F�r den vollst�ndigen Graphen $K_{16}$ aus
Abbildung \ref{ugGraphen}(a) ergibt sich offensichtlich die folgende
Adjazenzmatrix:
\begin{displaymath}
A_{K_{16}} =
\left(
\begin{array}{cccccccccccccccc}
0 & 1 & 1 & 1 & 1 & 1 & 1 & 1 & 1 & 1 & 1 & 1 & 1 & 1 & 1 & 1\\ 
1 & 0 & 1 & 1 & 1 & 1 & 1 & 1 & 1 & 1 & 1 & 1 & 1 & 1 & 1 & 1\\ 
1 & 1 & 0 & 1 & 1 & 1 & 1 & 1 & 1 & 1 & 1 & 1 & 1 & 1 & 1 & 1\\ 
1 & 1 & 1 & 0 & 1 & 1 & 1 & 1 & 1 & 1 & 1 & 1 & 1 & 1 & 1 & 1\\ 
1 & 1 & 1 & 1 & 0 & 1 & 1 & 1 & 1 & 1 & 1 & 1 & 1 & 1 & 1 & 1\\ 
1 & 1 & 1 & 1 & 1 & 0 & 1 & 1 & 1 & 1 & 1 & 1 & 1 & 1 & 1 & 1\\ 
1 & 1 & 1 & 1 & 1 & 1 & 0 & 1 & 1 & 1 & 1 & 1 & 1 & 1 & 1 & 1\\ 
1 & 1 & 1 & 1 & 1 & 1 & 1 & 0 & 1 & 1 & 1 & 1 & 1 & 1 & 1 & 1\\ 
1 & 1 & 1 & 1 & 1 & 1 & 1 & 1 & 0 & 1 & 1 & 1 & 1 & 1 & 1 & 1\\ 
1 & 1 & 1 & 1 & 1 & 1 & 1 & 1 & 1 & 0 & 1 & 1 & 1 & 1 & 1 & 1\\ 
1 & 1 & 1 & 1 & 1 & 1 & 1 & 1 & 1 & 1 & 0 & 1 & 1 & 1 & 1 & 1\\ 
1 & 1 & 1 & 1 & 1 & 1 & 1 & 1 & 1 & 1 & 1 & 0 & 1 & 1 & 1 & 1\\ 
1 & 1 & 1 & 1 & 1 & 1 & 1 & 1 & 1 & 1 & 1 & 1 & 0 & 1 & 1 & 1\\ 
1 & 1 & 1 & 1 & 1 & 1 & 1 & 1 & 1 & 1 & 1 & 1 & 1 & 0 & 1 & 1\\ 
1 & 1 & 1 & 1 & 1 & 1 & 1 & 1 & 1 & 1 & 1 & 1 & 1 & 1 & 0 & 1\\ 
1 & 1 & 1 & 1 & 1 & 1 & 1 & 1 & 1 & 1 & 1 & 1 & 1 & 1 & 1 & 0
\end{array}
\right)
\end{displaymath}
\end{example}
Mit Hilfe der Adjazenzmatrix und Algorithmus \ref{Reach} kann man
leicht berechnen, ob ein Weg von einem Knoten $u$ zu einem Knoten $v$
existiert. Mit einer ganz �hnlichen Idee kann man auch leicht die
Anzahl der Zusammenhangskomponenten berechnen (siehe
Algorithmus \ref{Kompo}). Dieser Algorithmus markiert die Knoten der
einzelnen Zusammenhangskomponenten auch mit unterschiedlichen
"`Farben"', die hier durch Zahlen repr�sentiert werden.

\restylealgo{ruled}
\begin{algorithm}
\caption{Erreichbarkeit in Graphen}
\label{Reach}
\KwData{Ein Graph $G=(V,E)$ und zwei Knoten $u,v \in V$}
\KwResult{\texttt{true} wenn es einen Weg von $u$ nach $v$
gibt, \texttt{false} sonst}
\BlankLine

markiert = \texttt{true}\;
markiere Startknoten $u \in V$\;

\BlankLine

\While{(markiert)}{

markiert = \texttt{false}\;

\For{(alle markierten Knoten $w \in V$)}{

\If{($w \in V$ ist adjazent zu einem unmarkierten Knoten $w' \in V$)}{
markiere Knoten $w'$\;
markiert = \texttt{true}\;
}

}

}

\eIf{($v$ ist markiert)}{
\Return \texttt{true}\;
}{
\Return \texttt{false}\;
}

\printsemicolon
\end{algorithm}

\restylealgo{ruled}
\begin{algorithm}
\caption{Zusammenhangskomponenten}
\label{Kompo}
\KwData{Ein Graph $G=(V,E)$}
\KwResult{Anzahl der Zusammenhangskomponenten von $G$}
\BlankLine
\printsemicolon

kFarb = 0\;

\BlankLine

\While{(es gibt einen unmarkierten Knoten $u \in V$)}{

kFarb++\;
markiere $u \in V$ mit kFarb\;
\BlankLine
\printsemicolon

markiert=\texttt{true}\;

\While{(markiert)}{

markiert=\texttt{false}\;
\BlankLine
\printsemicolon

\For{(alle mit kFarb markierten Knoten $v \in V$)}{
\If{($v \in V$ ist adjazent zu einem unmarkierten Knoten $v' \in V$)}{
markiere Knoten $v' \in V$ mit kFarb\;
markiert=\texttt{true}\;
}
}
}
}
\Return kFarb\;
\end{algorithm}

\begin{definition}
Sei $G = (V,E)$ ein ungerichteter Graph. Eine Funktion $f \colon
V \rightarrow \set{\range{1}{k}}$
hei�t \dindex{$k$-F�rbung}\index{F�rbung} des Graphen $G$. Anschaulich
ordnet die Funktion $f$ jedem Knoten eine von $k$ verschiedenen Farben
zu, die hier durch die Zahlen $\range{1}{k}$ symbolisiert werden. Eine
F�rbung hei�t \emph{vertr�glich}\index{F�rbung!vertr�glich}, wenn f�r
alle Kanten $(u,v) \in E$ gilt, dass $f(u) \neq f(v)$, d.h.~zwei
adjazente Knoten werden nie mit der gleichen Farbe markiert.
\end{definition}

\ifkomplex
F�r viele Probleme der Graphentheorie gibt es mit hoher
Wahrscheinlichkeit keinen effizienten Algorithmus. Mehr Informationen
zu diesem Thema finden sich in Abschnitt \ref{KomplexSect}.
\else
Auch das F�rbbarkeitsproblem spielt in der Praxis der Informatik eine
wichtige Rolle. Ein Beispiel daf�r ist die Planung eines
Mobilfunknetzes. Dabei werden die Basisstationen eines Mobilfunknetzes
als Knoten eines Graphen repr�sentiert. Zwei Knoten werden mit einer
Kante verbunden, wenn Sie geographisch so verteilt sind, dass sie sich
beim Senden auf der gleichen Frequenz gegenseitig st�ren
k�nnen. Existiert eine vertr�gliche $k$-F�rbung f�r diesen Graphen, so
ist es m�glich, ein st�rungsfreies Mobilfunktnetz mit $k$
verschiedenen Funkfrequenzen aufzubauen. Dabei entsprechen die Farben
den verf�gbaren Frequenzen. Bei der Planung eines solchen
Mobilfunknetzes ist also das folgende Problem zu l�sen:

\goodbreak
\dprob{$k\mathrm{COL}$}{
Ein ungerichteter Graph $G$ und eine Zahl $k \in \N$.
}
{
Gibt es eine vertr�gliche F�rbung von $G$ mit $k$ Farben?
}
Dieses Problem geh�rt zu einer (sehr gro�en) Klasse von (praktisch
relevanten) Problemen, f�r die bis heute keine effizienten Algorithmen
bekannt sind (Stichwort: \NP-Vollst�ndigkeit). Vielf�ltige Ergebnisse 
der Theoretischen Informatik zeigen sogar, dass man nicht hoffen darf, 
dass ein schneller Algorithmus zur L�sung des F�rbbarkeitsproblems existiert.
\fi



% Grundlagen der Komplexittstheorie
\special{pdf: out 2 << /Title 
(Komplexität) 
/Dest [ @thispage /FitH @ypos ] >>}
\section{Komplexität}
\label{KomplexSect}

Für viele ständig auftretende Berechnungsprobleme, wie Sortieren, die
arithmetischen Operationen (Addition, Multiplikation, Division),
Fourier-Transformation etc., sind sehr effiziente Algorithmen
konstruiert worden. Für wenige andere praktische Probleme weiß man,
dass sie nicht oder nicht effizient algorithmisch lösbar sind. Im
Gegensatz dazu ist für einen sehr großen Teil von Fragestellungen aus
den verschiedensten Anwendungsbereichen wie Operations Research,
Netzwerkdesign, Programmoptimierung, Datenbanken,
Betriebssystem-Entwicklung und vielen mehr jedoch nicht bekannt, ob
sie effiziente Algorithmen besitzen (vgl.~die Abbildungen
\ref{NPVollProbs1} und \ref{NPVollProbs2}). Diese Problematik hängt
mit der seit über dreißig Jahren offenen $\P\stackrel{?}{=}\NP$-Frage
zusammen, wahrscheinlich gegenwärtig das wichtigste ungelöste Problem
der theoretischen Informatik. Es wurde sogar kürzlich auf Platz 1 der
Liste der so genannten \emph{Millennium Prize Problems} des Clay
Mathematics Institute gesetzt\footnote{siehe \url{http://www.claymath.org/millennium-problems}}.  
Diese Liste umfasst sieben offene Probleme aus der gesamten
Mathematik. Das Clay Institute zahlt jedem, der eines dieser Probleme 
löst, eine Million US-Dollar.

In diesem Abschnitt werden die wesentlichen Begriffe aus dem Kontext
des $\P\stackrel{?}{=}\NP$-Problems und des Begriffes der
$\NP$-Vollständigkeit erläutert, um so die Grundlagen für das
Verständnis derartiger Probleme zu schaffen und deren Beurteilung zu
ermöglichen.

\special{pdf: out 3 << /Title 
(Effizient lösbare Probleme: die Klasse P) 
/Dest [ @thispage /FitH @ypos ] >>}
\subsection{Effizient lösbare Probleme: die Klasse $\P$}

Jeder, der schon einmal mit der Aufgabe konfrontiert wurde, einen
Algorithmus für ein gegebenes Problem zu entwickeln, kennt die
Hauptschwierigkeit dabei: Wie kann ein effizienter Algorithmus
gefunden werden, der das Problem mit möglichst wenigen Rechenschritten
löst? Um diese Frage beantworten zu können, muss man sich zunächst
einige Gedanken über die verwendeten Begriffe, nämlich "`Problem"',
"`Algorithmus"', "`Zeitbedarf"' und "`effizient"', machen.

Was ist ein "`Problem"'? Jedem Programmierer ist diese Frage intuitiv
klar: Man bekommt geeignete Eingaben, und das Programm soll die
gewünschten Ausgaben ermitteln. Ein einfaches Beispiel ist das Problem
MULT. (Jedes Problem soll mit einem eindeutigen Namen versehen und
dieser in Großbuchstaben geschrieben werden.) Hier bekommt man zwei
ganze Zahlen als Eingabe und soll das Produkt beider Zahlen berechnen,
d.h.~das Programm berechnet einfach eine zweistellige Funktion. Es hat
sich gezeigt, dass man sich bei der Untersuchung von Effizienzfragen
auf eine abgeschwächte Form von Problemen beschränken kann, nämlich
sogenannte \dindex{Entscheidungsprobleme}. Hier ist die Aufgabe, eine
gewünschte Eigenschaft der Eingaben zu testen. Hat die aktuelle
Eingabe die gewünschte Eigenschaft, dann gibt man den Wert $1$
($\triangleq$ \texttt{true}) zurück
(man spricht dann auch von einer 
\dindex{positiven Instanz}\index{Instanz!positiv} des Problems), hat 
die Eingabe die Eigenschaft nicht, dann gibt man den Wert $0$
($\triangleq$ \texttt{false}) zurück. Oder anders formuliert: Das
Programm berechnet eine Funktion, die den Wert 0 oder 1 zurück gibt
und partitioniert damit die Menge der möglichen Eingaben in zwei
Teile: die Menge der Eingaben mit der gewünschten Eigenschaft und die
Menge der Eingaben, die die gewünschte Eigenschaft nicht
besitzen. Folgendes Beispiel soll das Konzept verdeutlichen:

\prob{PARITY}
{Positive Integerzahl $x$}
{Ist die Anzahl der Ziffern 1 in der Binärdarstellung von $x$ ungerade?}

Es soll also ein Programm entwickelt werden, das die Parität einer
Integerzahl $x$ berechnet. Eine mögliche Entscheidungsproblem-Variante
des Problems MULT ist die folgende:

\goodbreak
\prob{$\mathrm{MULT}_\mathrm{D}$}
{Integerzahlen $x,y$, positive Integerzahl $i$}
{Ist das $i$-te Bit in $x\cdot y$ gleich $1$?}
Offensichtlich sind die Probleme MULT und MULT$_\mathrm{D}$ gleich schwierig (oder
leicht) zu lösen.

\begin{wrapfigure}[14]{r}{5cm}
\centerline{\includegraphics[scale=0.7]{niko.eps}}
\caption{Der Graph $G_N$}
\label{nikolaus} 
\end{wrapfigure}
Im Weiteren wollen wir uns hauptsächlich mit Problemen beschäftigen,
die aus dem Gebiet der Graphentheorie stammen.  Das hat zwei
Gründe. Zum einen können, wie sich noch zeigen wird, viele praktisch
relevante Probleme mit Hilfe von Graphen modelliert werden, und zum
anderen sind sie anschaulich und oft relativ leicht zu verstehen.  Ein
(ungerichteter) Graph $G$ besteht aus einer Menge von Knoten $V$ und
einer Menge von Kanten $E$, die diese Knoten verbinden. Man schreibt:
$G=(V,E)$. Ein wohlbekanntes Beispiel ist der Nikolausgraph:
$G_N=\bigl(\{1,2,3,4,5\},\allowbreak\{(1,2),\allowbreak(1,3),\allowbreak
(1,5),\allowbreak(2,3),\allowbreak(2,5),\allowbreak(3,4),\allowbreak
(3,5),\allowbreak(4,5)\}\bigr)$.  Es gibt also fünf Knoten
$V=\{1,2,3,4,5\}$, die durch die Kanten in
$E=\{(1,2),\allowbreak(1,3),\allowbreak(1,5),\allowbreak(2,3),\allowbreak
(2,5),\allowbreak(3,4),\allowbreak(3,5),\allowbreak(4,5)\}$ verbunden
werden (siehe Abbildung \ref{nikolaus}).


Ein prominentes Problem in der Graphentheorie ist es, eine sogenannte
Knotenfärbung zu finden. Dabei wird jedem Knoten eine Farbe
zugeordnet, und man verbietet, dass zwei Knoten, die durch eine Kante
verbunden sind, die gleiche Farbe zugeordnet wird. Natürlich ist die
Anzahl der Farben, die verwendet werden dürfen, durch eine feste
natürliche Zahl $k$ beschränkt. Genau wird das Problem, ob ein Graph
$k$-färbbar ist, wie folgt beschrieben:

\prob{$k\mathrm{COL}$}
{Ein Graph $G=(V,E)$}
{Hat $G$ eine Knotenfärbung mit höchstens $k$ Farben?}

Offensichtlich ist der Beispielgraph $G_N$ nicht mit drei Farben färbbar
(aber mit $4$ Farben, wie man leicht ausprobieren kann), und jedes
Programm für das Problem $3\mathrm{COL}$ müsste ermitteln, dass $G_N$ die
gewünschte Eigenschaft ($3$-Färbbarkeit) nicht hat.

Man kann sich natürlich fragen, was das künstlich erscheinende
Problem $3\mathrm{COL}$ mit der Praxis zu tun hat. Das folgende einfache
Beispiel soll das verdeutlichen. Man nehme das Szenario an, dass
ein großer Telefonprovider in einer Ausschreibung drei
Funkfrequenzen für einen neuen Mobilfunkstandard erworben hat. Da er
schon über ein Mobilfunknetz verfügt, sind die Sendemasten schon
gebaut. Aus technischen Gründen dürfen Sendemasten, die zu eng stehen,
nicht mit der gleichen Frequenz funken, da sie sich sonst stören
würden. In der graphentheoretischen Welt modelliert man die
Sendestationen mit Knoten eines Graphen, und "`nahe"' zusammenstehende
Sendestationen symbolisiert man mit einer Kante zwischen den Knoten,
für die sie stehen. Die Aufgabe des Mobilfunkplaners ist es nun, eine
$3$-Färbung für den entstehenden Graphen zu finden. Offensichtlich
kann das Problem verallgemeinert werden, wenn man sich nicht auf drei
Farben/Frequenzen festlegt; dann aber ergeben sich genau die oben
definierten Probleme $k\mathrm{COL}$ für beliebige Zahlen $k$.

Als nächstes ist zu klären, was unter einem "`Algorithmus"' zu
verstehen ist. Ein Algorithmus ist eine endliche, formale Beschreibung
einer Methode, die ein Problem löst (z.B.~ein Programm in einer
beliebigen Programmiersprache). Diese Methode muss also alle Eingaben
mit der gesuchten Eigenschaft von den Eingaben, die diese Eigenschaft
nicht haben, unterscheiden können. Man legt fest, dass der Algorithmus
für erstere den Wert $1$ und für letztere den Wert $0$ ausgeben
soll. Wie soll die "`Laufzeit"' eines Algorithmus gemessen werden? Um
dies festlegen zu können, muss man sich zunächst auf ein sogenanntes
\dindex{Berechnungsmodell} festlegen. Das kann man damit vergleichen,
welche Hardware für die Implementation des Algorithmus verwendet
werden soll. Für die weiteren Analysen soll das folgende einfache
C-artige Modell verwendet werden: Es wird (grob!) die Syntax von C
verwendet und festgelegt, dass jede Anweisung in einem Schritt
abgearbeitet werden kann. Gleichzeitig beschränkt man sich auf zwei
Datentypen: einen Integer-Typ und zusätzlich Arrays dieses
Integer-Typs (wobei Array-Grenzen nicht deklariert werden müssen,
sondern sich aus dem Gebrauch ergeben). Dieses primitive
Maschinenmodell ist deshalb geeignet, weil man zeigen kann, dass jeder
so formulierte Algorithmus auf realen Computern implementiert werden
kann, ohne eine substantielle Verlangsamung zu erfahren.  
%
(Dies gilt zumindest, wenn die verwendeten Zahlen nicht
übermäßig wachsen, d.h., wenn alle verwendeten Variablen nicht zu viel
Speicher belegen. Genaueres zu dieser Problematik -- man spricht von
der Unterscheidung zwischen \dindex{uniformem Komplexitätsmaß} und
\dindex{Bitkomplexität} -- findet sich in \cite[S.~62f]{Sch01}.)

Umgekehrt kann man ebenfalls sagen, dass dieses einfache Modell die
Realität genau genug widerspiegelt, da auch reale Programme ohne allzu
großen Zeitverlust auf diesem Berechnungsmodell simuliert werden
können. Offensichtlich ist die \emph{Eingabe} der Parameter, von dem
die Rechenzeit für einen festen Algorithmus abhängt. In den
vergangenen Jahrzehnten, in denen das Gebiet der Analyse von
Algorithmen entstand, hat die Erfahrung gezeigt, dass die Länge der
Eingabe, also die Anzahl der Bits, die benötigt werden um die Eingabe
zu speichern, ein geeignetes und robustes Maß ist, in der die
Rechenzeit gemessen werden kann. Auch der Aufwand, die Eingabe selbst
festzulegen (zu konstruieren), hängt schließlich von ihrer Länge ab,
nicht davon, ob sich irgendwo in der Eingabe eine $0$ oder $1$
befindet.

\special{pdf: out 4 << /Title 
(Das Problem der 2-Färbbarkeit) 
/Dest [ @thispage /FitH @ypos ] >>}
\subsubsection{Das Problem der 2-Färbbarkeit}
Das Problem der 2-Färbbarkeit ist wie folgt definiert:

\prob{$2\mathrm{COL}$}
{Ein Graph $G=(V,E)$}
{Hat $G$ eine Knotenfärbung mit höchstens $2$ Farben?}
Es ist bekannt, dass dieses Problem mit einem sogenannten
\dindex{Greedy-Algorithmus} gelöst werden kann: Beginne mit einem
beliebigen Knoten in $G$ (z.B.~$v_1$) und färbe ihn mit Farbe 1. Färbe
dann die Nachbarn dieses Knoten mit 2, die Nachbarn dieser Nachbarn
wieder mit 1, usw.  Falls $G$ aus mehreren Komponenten
(d.h.~zusammenhängenden Teilgraphen) besteht, muss dieses Verfahren
für jede Komponente wiederholt werden. $G$ ist schließlich 2-färbbar,
wenn bei obiger Prozedur keine inkorrekte Färbung entsteht. Diese Idee
führt zu Algorithmus \ref{TColAlgo}.

Die Laufzeit von Algorithmus \ref{TColAlgo} kann wie folgt abgeschätzt
werden: Die erste \textbf{for}-Schleife benötigt $n$ Schritte.  In der
\textbf{while}-Schleife wird entweder mindestens ein Knoten gefärbt
und die Schleife dann erneut ausgeführt, oder es wird kein Knoten
gefärbt und die Schleife dann verlassen; also wird diese Schleife
höchstens $n$-mal ausgeführt. Innerhalb der \textbf{while}-Schleife
finden sich drei ineinander verschachtelte \textbf{for}-Schleifen, die
alle jeweils $n$-mal durchlaufen werden, und eine
\textbf{while}-Schleife, die maximal $n$-mal durchlaufen wird.

Damit ergibt sich also eine Gesamtlaufzeit der Größenordnung $n^4$,
wobei $n$ die Anzahl der Knoten des Eingabe-Graphen $G$ ist. Wie groß
ist nun die Eingabelänge, also die Anzahl der benötigten Bits zur
Speicherung von $G$?  Sicherlich muss jeder Knoten in dieser
Speicherung vertreten sein, d.h.~also, dass mindestens $n$ Bits zur
Speicherung von $G$ benötigt werden. Die Eingabelänge ist also
mindestens $n$. Daraus folgt, dass die Laufzeit des Algorithmus also
höchstens von der Größenordnung $N^4$ ist, wenn $N$ die Eingabelänge
bezeichnet.

\goodbreak
\restylealgo{ruled}
\begin{algorithm}
\caption{Algorithmus zur Berechnung einer $2$-Färbung eines Graphen}
\label{TColAlgo}
%
\KwData{Graph $G = (\{v_1, \dots v_n\}, E)$;}
\KwResult{$1$ wenn es eine $2$-Färbung für $G$ gibt, $0$ sonst}
\BlankLine
\Begin{
\For{$($$i = 1$ to $n$$)$}{
$\mathrm{Farbe}[i] = 0$\;
}
\BlankLine
$\mathrm{Farbe}[1] = 1$\;
\Repeat{$(\mathrm{aktKompoBearbeiten})$}{
  aktKompoBearbeiten = \texttt{false}\;
  \For{$($$i = 1$ $\mathrm{to}$ $n$$)$}{
    \For{$($$j = 1$ $\mathrm{to}$ $n $$)$}{
      \tcc*[f]{Kante mit noch ungefärbten Knoten?}

      \If{$(($$(v_i,v_j) \in E$$)$ und $(\mathrm{Farbe}[i] \not= 0$$)$ und 
          $(\mathrm{Farbe}[j] = 0$$))$}{
        \tcc*[f]{$v_j$ bekommt eine andere Farbe als $v_i$}
      
        $\mathrm{Farbe}[j] = 3 - \mathrm{Farbe}[i]$\;
        aktKompoBearbeiten = \texttt{true}\;
        \tcc*[f]{Alle direkten Nachbarn von $v_j$ prüfen}

        \For{$($$k = 1$ $\mathrm{to}$ $n $$)$}{
          \tcc*[f]{Kollision beim Färben aufgetreten?}

          \If{$(($$(v_j,v_k) \in E$$)$ und 
              $(\mathrm{Farbe}[i] = \mathrm{Farbe}[k]$$))$}{
            \tcc*[f]{Kollision! Graph nicht $2$-färbbar}

            \Return $0$\;
          }
        } 
      }
    }
  }
  \tcc*[f]{Ist die aktuelle Zusammenhangskomponente völlig gefärbt?}

  \If{$(\mathrm{not(aktKompoBearbeiten)})$}{
    $i = 1$\;
    \tcc*[f]{Suche nach einer weiteren Zusammenhangskomponente von $G$}

    \Repeat{$(\mathrm{not(weiterSuchen)}$ $\mathrm{und}$ $(i \le n))$}{
      \tcc*[f]{Liegt $v_i$ in einer neuen Zusammenhangskomponente von $G$?}

      \If{$(\mathrm{Farbe}[i] = 0)$}{
        $\mathrm{Farbe}[i] = 1$\;
        \tcc*[f]{Neue Zusammenhangskomponente bearbeiten}
    
        aktKompoBearbeiten = \texttt{true}\;
        \tcc*[f]{Suche nach neuer Zusammenhangskomponente abbrechen}

        weiterSuchen = \texttt{false}\;

      }
      $i = i + 1$\;
    }
  }
}
\tcc*[f]{$2$-Färbung gefunden}

\Return $1$.
}
\end{algorithm}

Tatsächlich sind (bei Verwendung geeigneter Datenstrukturen wie Listen
oder Queues) wesentlich effizientere Verfahren für $2\mathrm{COL}$ möglich.
Aber auch schon das obige einfache Verfahren zeigt: $2\mathrm{COL}$ hat einen
Polynomialzeitalgorithmus, also $2\mathrm{COL} \in\P$.

Alle Probleme für die Algorithmen existieren, die eine Anzahl von
Rechenschritten benötigen, die durch ein beliebiges Polynom beschränkt
ist, bezeichnet man mit $\P$ ("`$\P$"' steht dabei für
"`\textbf{P}olyno\-mi\-al\-zeit"'). Auch dabei wird die Rechenzeit in
der Länge der Eingabe gemessen, d.h. in der Anzahl der Bits, die
benötigt werden, um die Eingabe zu speichern (zu kodieren). Die Klasse
$\P$\index{P=$\P$} wird auch als Klasse der \emph{effizient lösbaren
Probleme} bezeichnet. Dies ist natürlich wieder eine idealisierte
Auffassung: Einen Algorithmus mit einer Laufzeit $n^{57}$, wobei $n$
die Länge der Eingabe bezeichnet, kann man schwer als effizient
bezeichnen. Allerdings hat es sich in der Praxis gezeigt, dass für
fast alle bekannten Probleme in $\P$ auch Algorithmen existieren,
deren Laufzeit durch ein Polynom
kleinen\footnote{Aktuell \emph{scheint} es kein praktisch relevantes
Problem aus $\P$ zu geben, für das es keinen Algorithmus mit einer
Laufzeit von weniger als $n^{12}$ gibt.} Grades beschränkt ist.

In diesem Licht ist die Definition der Klasse $\P$ auch für praktische
Belange von Relevanz. Dass eine polynomielle Laufzeit etwas
substanziell Besseres darstellt als exponentielle Laufzeit (hier
beträgt die benötigte Rechenzeit $2^{c \cdot n}$ für eine Konstante
$c$, wobei $n$ wieder die Länge der Eingabe bezeichnet), zeigt die
Tabelle "`Rechenzeitbedarf von Algorithmen"'. Zu beachten ist, dass
bei einem Exponentialzeit-Algorithmus mit $c=1$ eine Verdoppelung der
"`Geschwindigkeit"' der verwendeten Maschine (also Taktzahl pro
Sekunde) es nur erlaubt, eine um höchstens $1$ Bit längere Eingabe in
einer bestimmten Zeit zu bearbeiten. Bei einem Linearzeit-Algorithmus
hingegen verdoppelt sich auch die mögliche Eingabelänge; bei einer
Laufzeit von $n^k$ vergrößert sich die mögliche Eingabelänge immerhin
noch um den Faktor $\sqrt[k]{2}$. Deswegen sind Probleme, für die nur
Exponentialzeit-Algorithmen existieren, praktisch nicht lösbar; daran
ändert sich auch nichts Wesentliches durch die Einführung von immer
schnelleren Rechnern.

\ifalgorithms
\else
\begin{figure}
{\renewcommand{\arraystretch}{1.2}\doublerulesep.2pt\arrayrulewidth.5pt
\footnotesize
\begin{center}
\begin{tabular}{|c|c|c|c|c|c|c|} \hline\hline
\multicolumn{1}{|c}{Anzahl der} 
& \multicolumn{6}{c|}{Eingabel"ange $n$}\\ \cline{2-7}
Takte & 10 & 20 & 30 & 40 & 50 & 60 \\ 
\hline\hline
 & 0,00001 & 0,00002 & 0,00003 & 0,00004 & 0,00005 & 0,00006 \\
\raisebox{1.5ex}[-1.5ex]{$n$} 
 & Sekunden & Sekunden & Sekunden & Sekunden & Sekunden & Sekunden \\ \hline
 & 0,0001 & 0,0004 & 0,0009 & 0,0016 & 0,0025 & 0,0036 \\
\raisebox{1.5ex}[-1.5ex]{$n^2$} 
 & Sekunden & Sekunden & Sekunden & Sekunden & Sekunden & Sekunden \\ \hline
 & 0,001 & 0,008 & 0,027 & 0,064 & 0,125 & 0,216 \\
\raisebox{1.5ex}[-1.5ex]{$n^3$} 
 & Sekunden & Sekunden & Sekunden & Sekunden & Sekunden & Sekunden \\ \hline
 & 0,1 & 3,2 & 24,3 & 1,7 & 5,2 & 13,0 \\
\raisebox{1.5ex}[-1.5ex]{$n^5$} 
 & Sekunden & Sekunden & Sekunden & Minuten & Minuten & Minuten \\ 
\hline\hline
 & 0,001 & 1 & 17,9 & 12,7 & 35,7 & 366 \\
\raisebox{1.5ex}[-1.5ex]{$2^n$} 
 & Sekunden & Sekunde & Minuten & Tage & Jahre & Jahrhunderte \\ \hline
 & 0,059 & 58 & 6,5 & 3855 & $2 \cdot 10^8$ & $1,3 \cdot 10^{13}$ \\
\raisebox{1.5ex}[-1.5ex]{$3^n$} 
 & Sekunden & Minuten & Jahre & Jahrhunderte & Jahrhunderte & Jahrhunderte \\ \hline\hline
\end{tabular}
\end{center}
}
\caption{Rechenzeitbedarf von Algorithmen auf einem "`$1$-MIPS"'-Rechner}
\label{runtimetab}
\end{figure}
\fi

Nun stellt sich natürlich sofort die Frage: Gibt es für jedes Problem
einen effizienten Algorithmus?  Man kann relativ leicht zeigen, dass
die Antwort auf diese Frage "`Nein"' ist. Die Schwierigkeit bei dieser
Fragestellung liegt aber darin, dass man von vielen Problemen
\emph{nicht weiß}, ob sie effizient lösbar sind.  Ganz konkret: Ein
effizienter Algorithmus für das Problem $2\mathrm{COL}$ ist
Algorithmus
\ref{TColAlgo}. Ist es möglich, ebenfalls einen
Polynomialzeitalgorithmus für $3\mathrm{COL}$ zu finden? Viele Informatiker
beschäftigen sich seit den 60er Jahren des letzten Jahrhunderts
intensiv mit dieser Frage. Dabei kristallisierte sich heraus, dass
viele praktisch relevante Probleme, für die kein effizienter
Algorithmus bekannt ist, eine gemeinsame Eigenschaft besitzen, nämlich
die der \emph{effizienten Überprüfbarkeit} von geeigneten
Lösungen. Auch $3\mathrm{COL}$ gehört zu dieser Klasse von Problemen, wie sich
in Kürze zeigen wird. Aber wie soll man zeigen, dass für ein Problem
kein effizienter Algorithmus existiert?  Nur weil kein Algorithmus
bekannt ist, bedeutet das noch nicht, dass keiner existiert.

Es ist bekannt, dass die oberen Schranken (also die Laufzeit von
bekannten Algorithmen) und die unteren Schranken (mindestens benötigte
Laufzeit) für das Problem PARITY (und einige wenige weitere, ebenfalls
sehr einfach-geartete Probleme) sehr nahe zusammen liegen. Das
bedeutet also, dass nur noch unwesentliche Verbesserungen der
Algorithmen für des PARITY-Problem erwartet werden können. Beim
Problem $3\mathrm{COL}$ ist das ganz anders: Die bekannten unteren
und oberen Schranken liegen extrem weit auseinander. Deshalb ist nicht
klar, ob nicht doch (extrem) bessere Algorithmen als heute bekannt im
Bereich des Möglichen liegen. Aber wie untersucht man solch eine
Problematik?  Man müsste ja über unendlich viele Algorithmen für das
Problem $3\mathrm{COL}$ Untersuchungen anstellen. Dies ist äußerst
schwer zu handhaben und deshalb ist der einzige bekannte Ausweg, das
Problem mit einer Reihe von weiteren (aus bestimmten Gründen)
interessierenden Problemen zu vergleichen und zu zeigen, dass unser zu
untersuchendes Problem nicht leichter zu lösen ist als diese
anderen. Hat man das geschafft, ist eine untere Schranke einer
speziellen Art gefunden: Unser Problem ist nicht leichter lösbar, als
alle Probleme der Klasse von Problemen, die für den Vergleich
herangezogen wurden. Nun ist aus der Beschreibung der Aufgabe aber
schon klar, dass auch diese Aufgabe schwierig zu lösen ist, weil ein
Problem nun mit unendlich vielen anderen Problemen zu vergleichen
ist. Es zeigt sich aber, dass diese Aufgabe nicht aussichtslos
ist. Bevor diese Idee weiter ausgeführt wird, soll zunächst die Klasse
von Problemen untersucht werden, die für diesen Vergleich herangezogen
werden sollen, nämlich die Klasse $\NP$\index{NP=$\NP$}.

\special{pdf: out 3 << /Title 
(Effizient überprüfbare Probleme: die Klasse NP) 
/Dest [ @thispage /FitH @ypos ] >>}
\subsection{Effizient überprüfbare Probleme: die Klasse $\NP$}

Wie schon erwähnt, gibt es eine große Anzahl verschiedener Probleme,
für die kein effizienter Algorithmus bekannt ist, die aber eine
gemeinsame Eigenschaft haben: die \emph{effiziente Überprüfbarkeit von
Lösungen} für dieses Problem. Diese Eigenschaft soll an dem schon
bekannten Problem $3\mathrm{COL}$ veranschaulicht werden: Angenommen,
man hat einen beliebigen Graphen $G$ gegeben; wie bereits erwähnt ist
kein effizienter Algorithmus bekannt, der entscheiden kann, ob der
Graph $G$ eine $3$-Färbung hat (d.h., ob der fiktive Mobilfunkprovider
mit $3$ Funkfrequenzen auskommt). Hat man aber aus irgendwelchen
Gründen eine \emph{potenzielle} Knotenfärbung vorliegen, dann ist es
leicht, diese potenzielle Knotenfärbung zu überprüfen und
festzustellen, ob sie eine \emph{korrekte} Färbung des Graphen ist, 
wie Algorithmus \ref{TestCOL} zeigt.

\restylealgo{ruled}
\begin{algorithm}
\caption{Ein Algorithmus zur Überprüfung einer potentiellen Färbung}
\label{TestCOL}
\KwData{Graph $G = (\set{v_1, \dots v_n}, E)$ und eine potenzielle Knotenfärbung}
\KwResult{$1$ wenn die Färbung korrekt, $0$ sonst}
\BlankLine
\Begin{
   \tcc*[f]{Teste systematisch alle Kanten}
    
    \For{$(i = 1$ $\mathrm{to}$ $n)$}{
      \For{$(j = 1$ $\mathrm{to}$ $n)$}{
       \tcc*[f]{Test auf Verletzung der Knotenfärbung}
    
        \If{$(((v_i,v_j) \in E)$ $\mathrm{und}$ $(v_i \text{ und }
            v_j \text{ sind gleich gefärbt}))$}{
          \Return $0$\;
        }
      }
    }
   \Return $1$\;
  }
\end{algorithm}

Das Problem $3\mathrm{COL}$ hat also die Eigenschaft, dass eine
potenzielle Lösung leicht daraufhin überprüft werden kann, ob sie eine
tatsächliche, d.h.~korrekte, Lösung ist.  Viele andere praktisch
relevante Probleme, für die kein effizienter Algorithmus bekannt ist,
besitzen ebenfalls diese Eigenschaft. Dies soll noch an einem weiteren
Beispiel verdeutlicht werden, dem so genannten
\dindex{Hamiltonkreis-Problem}\index{Problem!Hamiltonkreis}.

Sei wieder ein Graph $G = \bigl(\{v_1, \dots v_n \},E\bigr)$
gegeben. Diesmal ist eine Rundreise entlang der Kanten von $G$
gesucht, die bei einem Knoten $v_{i_1}$ aus $G$ startet, wieder bei
$v_{i_1}$ endet und jeden Knoten genau einmal besucht. Genauer wird
diese Rundreise im Graphen $G$ durch eine Folge von $n$ Knoten
$(v_{i_1}, v_{i_2}, v_{i_3}, \dots ,v_{i_{n-1}}, v_{i_{n}},)$
beschrieben, wobei gelten soll, dass alle Knoten
$v_{i_1},\dots,v_{i_n}$ verschieden sind und die Kanten $(v_{i_1},
v_{i_2}),\allowbreak (v_{i_2}, v_{i_3}),\allowbreak \dots
,(v_{i_{n-1}}, v_{i_{n}})$ und $(v_{i_{n}}, v_{i_1})$ in $G$
vorkommen. Eine solche Folge von Kanten wird als \dindex{Hamiltonscher
Kreis} bezeichnet. Ein Hamiltonscher Kreis in einem Graphen $G$ ist
also ein Kreis, der jeden Knoten des Graphen genau einmal besucht.
Das Problem, einen Hamiltonschen Kreis in einem Graphen zu finden,
bezeichnet man mit \textsf{HAMILTON}:

\dprob{HAMILTON}
{Ein Graph $G$}
{Hat $G$ einen Hamiltonschen Kreis?}

Auch für dieses Problem ist kein effizienter Algorithmus bekannt. Aber
auch hier ist offensichtlich: Bekommt man einen Graphen gegeben und
eine Folge von Knoten, dann kann man sehr leicht überprüfen, ob sie
ein Hamiltonscher Kreis ist -- dazu ist lediglich zu testen, ob alle
Knoten genau einmal besucht werden und auch alle Kanten im gegebenen
Graphen vorhanden sind.

Hat man erst einmal die Beobachtung gemacht, dass viele Probleme die
Eigenschaft der effizienten Überprüfbarkeit haben, ist es naheliegend,
sie in einer Klasse zusammenzufassen und gemeinsam zu untersuchen. Die
Hoffnung dabei ist, dass sich alle Aussagen, die man über diese Klasse
herausfindet, sofort auf alle Probleme anwenden lassen. Solche
Überlegungen führten zur Geburt der Klasse $\NP$, in der man alle
effizient überprüfbaren Probleme zusammenfasst. Aber wie kann man
solch eine Klasse untersuchen? Man hat ja noch nicht einmal ein
Maschinenmodell (oder eine Programmiersprache) zur Verfügung, um solch
eine Eigenschaft zu modellieren. Um ein Programm für effizient
überprüfbare Probleme zu schreiben, braucht man erst eine Möglichkeit,
die zu überprüfenden möglichen Lösungen zu ermitteln und sie dann zu
testen, d.h.~man muss die Programmiersprache für $\NP$ in einer
geeigneten Weise mit mehr "`Berechnungskraft"' ausstatten.

Die erste Lösungsidee für $\NP$-Probleme, nämlich alle in Frage
kommenden Lösungen in einer $\mathbf{for}$-Schleife aufzuzählen, führt
zu Exponentialzeit-Lösungsalgorithmen, denn es gibt im Allgemeinen
einfach so viele potenzielle Lösungen. Um erneut auf das Problem
$3\mathrm{COL}$ zurückzukommen: Angenommen, $G$ ist ein Graph mit $n$
Knoten. Dann gibt es $3^n$ potenzielle Färbungen, die überprüft werden
müssen, denn es gibt $3$ Möglichkeiten den ersten Knoten zu färben,
$3$ Möglichkeiten den zweiten Knoten zu färben, usw., und damit $3^n$
viele zu überprüfende potenzielle Färbungen. Würde man diese in einer
$\mathbf{for}$-Schleife aufzählen und auf Korrektheit testen, so
führte das also zu einem Exponentialzeit-Algorithmus.  Auf der anderen
Seite gibt es aber Probleme, die in Exponentialzeit gelöst werden
können, aber nicht zu der Intuition der effizienten Überprüfbarkeit
der Klasse $\NP$ passen.  Das Berechnungsmodell für $\NP$ kann also
nicht einfach so gewonnen werden, dass exponentielle Laufzeit
zugelassen wird, denn damit wäre man über das Ziel hinausgeschossen.

Hätte man einen Parallelrechner zur Verfügung mit so vielen
Prozessoren wie es potenzielle Lösungen gibt, dann könnte man das
Problem schnell lösen, denn jeder Prozessor kann unabhängig von allen
anderen Prozessoren eine potenzielle Färbung überprüfen. Es zeigt sich
aber, dass auch dieses Berechnungsmodell zu mächtig wäre. Es gibt
Probleme, die wahrscheinlich nicht im obigen Sinne effizient
überprüfbar sind, aber mit solch einem Parallelrechner trotzdem
(effizient) gelöst werden könnten. In der Praxis würde uns ein
derartiger paralleler Algorithmus auch nichts nützen, da man einen
Rechner mit exponentiell vielen Prozessoren, also enormem
Hardwareaufwand, zu konstruieren hätte. Also muss auch dieses
Berechnungsmodell wieder etwas schwächer gemacht werden.

Eine Abschwächung der gerade untersuchten Idee des Parallelrechners
führt zu folgendem Vorgehen: Man "`rät"' für den ersten Knoten eine
beliebige Farbe, dann für den zweiten Knoten auch wieder eine
beliebige Farbe, solange bis für den letzten Knoten eine Farbe gewählt
wurde. Danach überprüft man die geratene Färbung und akzeptiert die
Eingabe, wenn die geratene Färbung eine korrekte Knotenfärbung
ist. Die Eingabe ist eine positive Eingabeinstanz des $\NP$-Problems
$3\mathrm{COL}$, falls es eine potenzielle Lösung (Färbung) gibt, die
sich bei der Überprüfung als korrekt herausstellt, d.h.~im
beschriebenen Rechnermodell: falls es eine Möglichkeit zu raten gibt,
sodass am Ende akzeptiert (Wert $1$ ausgegeben) wird. Man kann also
die Berechnung durch einen Baum mit $3$-fachen Verzweigungen
darstellen (vgl.~Abbildung \ref{Tree3COL}).
%
\begin{figure}
\centerline{\includegraphics[scale=0.55]{tree2.eps}}
\caption{Ein Berechnungsbaum für das $3\mathrm{COL}$-Problem}
\label{Tree3COL}
\end{figure}
%
An den Kanten des Baumes findet sich das Resultat der Rateanweisung
der darüberliegenden Verzweigung. Jeder Pfad in diesem sogenannten
\dindex{Berechnungsbaum} entspricht daher einer Folge von
Farbzuordnungen an die Knoten, d.h.~einer potenziellen Färbung. Der
Graph ist $3$-färbbar, falls sich auf mindestens einem Pfad eine
korrekte Färbung ergibt, falls also auf mindestens einem Pfad die
Überprüfungsphase\index{Uberprüfungsphase=Überprüfungsphase} 
erfolgreich ist; der Beispielgraph besitzt sechs korrekte $3$-Färbungen, 
ist also eine positive Instanz des $3\mathrm{COL}$-Problems.

Eine weitere, vielleicht intuitivere Vorstellung für die Arbeitsweise
dieser $\NP$-Ma\-schi\-ne ist die, dass bei jedem Ratevorgang $3$
verschiedene unabhängige Prozesse gestartet werden, die aber nicht
miteinander kommunizieren dürfen. In diesem Sinne hat man es hier mit
einem eingeschränkten Parallelrechner zu tun: Beliebige Aufspaltung
(fork) ist erlaubt, aber keine Kommunikation zwischen den Prozessen
ist möglich. Würde man Kommunikation zulassen, hätte man erneut den
allgemeinen Parallelrechner mit exponentiell vielen Prozessoren von
oben, der sich ja als zu mächtig für $\NP$ herausgestellt hat.

Es hat sich also gezeigt, dass eine Art "`Rateanweisung"' benötigt
wird. In der Programmiersprache für $\NP$ verwendet man dazu das neue
Schlüsselwort $\mathbf{guess}(m)$, wobei $m$ die Anzahl von
Möglichkeiten ist, aus denen eine geraten wird, und legt fest, dass
auch die Anweisung $\mathbf{guess}(m)$ nur einen Takt Zeit für ihre
Abarbeitung benötigt.  Berechnungen, die, wie soeben beschrieben,
verschiedene Möglichkeiten raten können, heißen
\dindex{nichtdeterministisch}. Es sei wiederholt, dass \emph{festgelegt}
(definiert) wird, dass ein nichtdeterministischer Algorithmus bei
einer Eingabe den Wert $1$ berechnet, falls \emph{eine
Möglichkeit} geraten werden kann, sodass der Algorithmus auf die
Anweisung "`\textbf{return} $1$"' stößt. Die Klasse $\NP$
umfasst nun genau die Probleme, die von nichtdeterministischen
Algorithmen mit polynomieller Laufzeit gelöst werden können.
"`$\NP$"' steht dabei für
"`\textbf{n}ichtdeterministi\-sche \textbf{P}olynomialzeit"', nicht
etwa, wie mitunter zu lesen, für "`Nicht-Polynomialzeit"'.  (Eine
formale Präsentation der Äquivalenz zwischen effizienter
Überprüfbarkeit und Polynomialzeit in der $\NP$-Pro\-gram\-miersprache
findet sich z.B.~in \cite[Kapitel 2.3]{GaJo79}.)

Mit Hilfe eines nichtdeterministischen Algorithmus kann das
$3\mathrm{COL}$-Problem in Polynomialzeit gelöst werden (siehe
Algorithmus \ref{3COLAlgo}). Die zweite Phase von
Algorithmus \ref{3COLAlgo}, die
\emph{Überprüfungsphase}\index{Uberprüfungsphase=Überprüfungsphase}, 
entspricht dabei genau dem oben angegebenen Algorithmus zum
effizienten Überprüfen von möglichen Lösungen des
$3\mathrm{COL}$-Problems (vgl.~Algorithmus \ref{TestCOL}).

\restylealgo{ruled}
\begin{algorithm}
\caption{Ein nichtdeterministischer Algorithmus für $3\mathrm{COL}$}
\label{3COLAlgo}
\KwData{Graph $G = (\{v_1, \dots v_n\}, E)$}
\KwResult{$1$ wenn eine Färbung existiert, $0$ sonst}
\BlankLine
\Begin{
    \tcc*[f]{Ratephase}

    \For{$(i = 1$ $\mathrm{to}$ $n)$}{
        $\mathrm{Farbe}[i] = \mathrm{guess}(3)$\;
    }
    \tcc*[f]{Überprüfungsphase}

    \For{$(i = 1$ $\mathrm{to}$ $n)$}{
      \For{$(j = 1$ $\mathrm{to}$ $n)$}{
        \If{$(((v_i,v_j) \in E)$ $\mathrm{und}$ $(v_i \text{ und }
            v_j \text{ sind gleich gefärbt}))$}{
          \Return $0$\;
        }
      }
    }
  \Return $1$\;
  }
\end{algorithm}

Dieser nichtdeterministische Algorithmus läuft in Polynomialzeit, denn
man benötigt für einen Graphen mit $n$ Knoten mindestens $n$ Bits, um
ihn zu speichern (kodieren), und der Algorithmus braucht im
schlechtesten Fall $O(n)$ (Ratephase) und $O(n^2)$
(Überprüfungsphase), also insgesamt $O(n^2)$ Takte Zeit.  Damit ist
gezeigt, dass $3\mathrm{COL}$ in der Klasse $\NP$ enthalten ist, denn
es wurde ein nichtdeterministischer Polynomialzeitalgorithmus
gefunden, der $3\mathrm{COL}$ löst. Ebenso einfach könnte man nun
einen nichtdeterministischen Polynomialzeitalgorithmus entwickeln, der
das Problem \textsf{HAMILTON} löst: Der Algorithmus wird in einer
ersten Phase eine Knotenfolge raten und dann in einer zweiten Phase
überprüfen, dass die Bedingungen, die an einen Hamiltonschen Kreis
gestellt werden, bei der geratenen Folge erfüllt sind. Dies zeigt,
dass auch \textsf{HAMILTON} in der Klasse $\NP$ liegt.

Dass eine nichtdeterministische Maschine nicht gebaut werden kann,
spielt hier keine Rolle. Nichtdeterministische Berechnungen sollen
hier lediglich als Gedankenmodell für unsere Untersuchungen
herangezogen werden, um Aussagen über die (Nicht-) Existenz von
effizienten Algorithmen machen zu können.

\special{pdf: out 3 << /Title 
(Schwierigste Probleme in NP: der Begriff der NP-Vollständigkeit) 
/Dest [ @thispage /FitH @ypos ] >>}
\subsection{Schwierigste Probleme in $\NP$: der Begriff der 
$\NP$-Vollständigkeit}

Es ist nun klar, was es bedeutet, dass ein Problem in $\NP$ liegt. Es
liegt aber auch auf der Hand, dass alle Probleme aus $\P$ auch in
$\NP$ liegen, da bei der Einführung von $\NP$ ja nicht verlangt wurde,
dass die $\mathbf{guess}$-Anweisung verwendet werden muss. Damit ist
jeder deterministische Algorithmus automatisch auch ein
(eingeschränkter) nichtdeterministischer Algorithmus. Nun ist aber
auch schon bekannt, dass es Probleme in $\NP$ gibt, z.B.~$3\mathrm{COL}$ und
weitere Probleme, von denen nicht bekannt ist, ob sie in $\P$
liegen. Das führt zu der Vermutung, dass $\P\neq\NP$.

Es gibt also in $\NP$ anscheinend unterschiedlich schwierige Probleme:
einerseits die $\P$-Probleme (also die leichten Probleme), und
andererseits die Probleme, von denen man nicht weiß, ob sie in $\P$
liegen (die schweren Probleme).  Es liegt also nahe, eine allgemeine
Möglichkeit zu suchen, Probleme in $\NP$ bezüglich ihrer Schwierigkeit
zu vergleichen. Ziel ist, wie oben erläutert, eine Art von unterer
Schranke für Probleme wie $3\mathrm{COL}$: Es soll gezeigt werden,
dass $3\mathrm{COL}$ mindestens so schwierig ist, wie jedes andere
Problem in $\NP$, also in gewissem Sinne ein \emph{schwierigstes
Problem in $\NP$} ist.

Für diesen Vergleich der Schwierigkeit ist die erste Idee natürlich,
einfach die Laufzeit von (bekannten) Algorithmen für das Problem
heranzuziehen. Dies ist jedoch nicht erfolgversprechend, denn was soll
eine "`größte"' Laufzeit sein, die Programme für "`schwierigste"'
Probleme in $\NP$ ja haben müssten?  Außerdem hängt die Laufzeit eines
Algorithmus vom verwendeten Berechnungsmodell ab. So kennen
Turingmaschinen keine Arrays im Gegensatz zu der hier verwendeten
C-Variante. Also würde jeder Algorithmus, der Arrays verwendet,
auf einer Turingmaschine mühsam simuliert werden müssen und damit
langsamer abgearbeitet werden, als bei einer Hochsprache, die Arrays
enthält. Obwohl sich die Komplexität eines Problems nicht ändert,
würde man sie verschieden messen, je nachdem welches Berechnungsmodell
verwendet würde. Ein weiterer Nachteil dieses Definitionsversuchs wäre
es, dass die Komplexität (Schwierigkeit) eines Problems mit bekannten
Algorithmen gemessen würde. Das würde aber bedeuten, dass jeder neue
und schnellere Algorithmus Einfluss auf die Komplexität hätte, was
offensichtlich so keinen Sinn macht.  Aus diesen und anderen Gründen
führt die erste Idee nicht zum Ziel.

Eine zweite, erfolgversprechendere Idee ist die folgende: Ein Problem
$A$ ist nicht (wesentlich) schwieriger als ein Problem $B$, wenn man
$A$ mit der Hilfe von $B$ (als Unterprogramm) effizient lösen
kann. Ein einfaches Beispiel ist die Multiplikation von $n$ Zahlen.
Angenommen, man hat schon ein Programm, dass $2$ Zahlen multiplizieren
kann; dann ist es nicht wesentlich schwieriger, auch $n$ Zahlen zu
multiplizieren, wenn die Routine für die Multiplikation von $2$ Zahlen
verwendet wird. Dieser Ansatz ist unter dem Namen \emph{relative
  Berechenbarkeit} bekannt, der genau den oben beschriebenen
Sachverhalt widerspiegelt: Multiplikation von $n$ Zahlen (so genannte
\emph{iterierte Multiplikation}) ist relativ zur Multiplikation zweier
Zahlen (leicht) berechenbar.

Da das Prinzip der relativen Berechenbarkeit so allgemein gehalten
ist, gibt es innerhalb der theoretischen Informatik sehr viele
verschiedene Ausprägungen dieses Konzepts. Für die
$\P$-$\NP$-Problematik ist folgende Version der relativen
Berechenbarkeit, d.h.~die folgende Art von erlaubten
"`Unterprogrammaufrufen"', geeignet:\\
%
Seien zwei Probleme $A$ und $B$ gegeben. Das Problem $A$ ist nicht
schwerer als $B$, falls es eine effizient zu berechnende
Transformation $T$ gibt, die Folgendes leistet: Wenn $x$ eine
Eingabeinstanz von Problem $A$ ist, dann ist $T(x)$ eine
Eingabeinstanz für $B$. Weiterhin gilt: $x$ ist \emph{genau dann} eine
positive Instanz von $A$ (d.h.~ein Entscheidungsalgorithmus für $A$
muss den Wert $1$ für Eingabe $x$ liefern), wenn $T(x)$ eine positive
Instanz von Problem $B$ ist. Erneut soll "`effizient berechenbar"'
hier bedeuten: in Polynomialzeit berechenbar. Es muss also einen
Polynomialzeitalgorithmus geben, der die Transformation $T$ ausführt.
Das Entscheidungsproblem $A$ ist damit effizient transformierbar in
das Problem $B$. Man sagt auch: $A$ ist reduzierbar auf $B$; oder
intuitiver: $A$ ist nicht schwieriger als $B$, oder $B$ ist mindestens
so schwierig wie $A$. Formal schreibt man dann $A\leq B$.

Um für dieses Konzept ein wenig mehr Intuition zu gewinnen, sei
erwähnt, dass man sich eine solche Transformation auch wie folgt
vorstellen kann: $A$ lässt sich auf $B$ reduzieren, wenn ein
Algorithmus für $A$ angegeben werden kann, der ein Unterprogramm $U_B$
für $B$ genau so verwendet wie in Algorithmus \ref{RedAlgo} gezeigt.

Dabei ist zu beachten, dass das Unterprogramm für $B$ nur genau einmal
und zwar am Ende aufgerufen werden darf.  Das Ergebnis des Algorithmus
für $A$ ist genau das Ergebnis, das dieser Unterprogrammaufruf
liefert.  Es gibt zwar, wie oben erwähnt, auch allgemeinere
Ausprägungen der relativen Berechenbarkeit, die diese Einschränkung
nicht haben, diese sind aber für die folgenden Untersuchungen nicht
relevant.

Nachdem nun ein Vergleichsbegriff für die Schwierigkeit von Problemen
aus $\NP$ gefunden wurde, kann auch definiert werden, was unter einem
"`schwierigsten"' Problem in $\NP$ zu verstehen ist. Ein Problem $C$
ist ein schwierigstes Problem in $\NP$, wenn alle anderen Probleme in
$\NP$ höchstens so schwer wie $C$ sind. Formaler ausgedrückt sind dazu
zwei Eigenschaften von $C$ nachzuweisen:

\begin{enumerate}[{\sffamily(1)}]
\item $C$ ist ein Problem aus $\NP$.
\item $C$ ist mindestens so schwierig wie jedes andere $\NP$-Problem
$A$; d.h.: für alle Probleme $A$ aus $\NP$ gilt: $A \le C$.
\end{enumerate}

\restylealgo{ruled}
\begin{algorithm}
\caption{Algorithmische Darstellung der Benutzung einer Reduktionsfunktion}
\label{RedAlgo}
%
\KwData{Instanz $x$ für das Problem $A$}
\KwResult{$1$ wenn $x \in A$ und $0$ sonst}
\BlankLine
\Begin{
\tcc*[f]{$T$ ist die Reduktionsfunktion (polynomialzeitberechenbar)}

berechne $y = T(x)$\;
\tcc*[f]{$y$ ist Instanz des Problems $B$}

$z = U_B(y)$\;

\tcc*[f]{$z$ ist $1$ genau dann, wenn $x \in A$ gilt}

\Return $z$\;
}
\end{algorithm}

Solche schwierigsten Probleme in $\NP$ sind unter der Bezeichnung
\emph{$\NP$-vollständige Probleme}\index{vollständig}
bekannt. Nun sieht die Aufgabe, von einem Problem zu zeigen, dass es
$\NP$-vollständig ist, ziemlich hoffnungslos aus. Immerhin ist zu
zeigen, dass für alle Probleme aus $\NP$ -- und damit unendlich viele
-- gilt, dass sie höchstens so schwer sind wie das zu untersuchende
Problem, und damit scheint man der Schwierigkeit beim Nachweis unterer
Schranken nicht entgangen zu sein. Dennoch konnten der russische
Mathematiker Leonid Levin und der amerikanische Mathematiker Stephen
Cook Anfang der siebziger Jahre des letzten Jahrhunderts unabhängig
voneinander die Existenz von solchen $\NP$-vollständigen Problemen
zeigen. Hat man nun erst einmal \emph{ein} solches Problem
identifiziert, ist die Aufgabe, \emph{weitere} $\NP$-vollständige
Probleme zu finden, wesentlich leichter. Dies ist sehr leicht
einzusehen: Ein $\NP$-Problem $C$ ist ein schwierigstes Problem in
$\NP$, wenn es ein anderes schwierigstes Problem $B$ gibt, sodass $C$
nicht leichter als $B$ ist.  Das führt zu folgendem "`Kochrezept"':
\medskip

\noindent{\sffamily\bfseries Nachweis der $\NP$-Vollständigkeit 
eines Problems $C$:}
\begin{enumerate}[i)]
\item Zeige, dass $C$ in $\NP$ enthalten ist, indem dafür ein
geeigneter nichtdeterministischer Polynomialzeitalgorithmus
konstruiert wird.
\item Suche ein geeignetes "`ähnliches"' schwierigstes Problem $B$ in
$\NP$ und zeige, dass $C$ nicht leichter als $B$ ist. Formal: Finde ein
$\NP$-vollständiges Problem $B$ und zeige $B \le C$ mit Hilfe einer
geeigneten Transformation $T$.
\end{enumerate}

Den zweiten Schritt kann man oft relativ leicht mit Hilfe von
bekannten Sammlungen $\NP$-vollständiger Problemen erledigen. Das Buch
von Garey und Johnson \cite{GaJo79} ist eine solche Sammlung (siehe
auch die Abbildungen \ref{NPVollProbs1} und \ref{NPVollProbs2}), die
mehr als $300$ $\NP$-vollständige Probleme enthält. Dazu wählt man ein
möglichst ähnliches Problem aus und versucht dann eine geeignete
Reduktionsfunktion für das zu untersuchende Problem zu finden.

\goodbreak
\special{pdf: out 4 << /Title 
(Traveling Salesperson ist NP-vollständig) 
/Dest [ @thispage /FitH @ypos ] >>}
\subsubsection{Traveling Salesperson ist $\NP$-vollständig}
Wie kann man zeigen, dass Traveling Salesperson $\NP$-vollständig ist?
Dazu wird zuerst die genaue Definition dieses Problems benötigt:

\dprob{TRAVELING SALESPERSON (TSP)}%
{Eine Menge von Städten $C = \{c_1, \dots ,c_n \}$ und eine $n \times
  n$ Entfernungsmatrix $D$, wobei das Element $D[i,j]$ der Matrix $D$
  die Entfernung zwischen Stadt $c_i$ und $c_j$ angibt. Weiterhin eine
  Obergrenze $k \ge 0$ für die maximal erlaubte Länge der Tour}%
{Gibt es eine Rundreise, die einerseits alle Städte besucht, aber
  andererseits eine Gesamtlänge von höchstens $k$ hat?}

Nun zum ersten Schritt des Nachweises der $\NP$-Vollständigkeit von
\textsf{TSP}: Offensichtlich gehört auch das Traveling Salesperson Problem zur
Klasse $\NP$, denn man kann nichtdeterministisch eine Folge von $n$
Städten raten (eine potenzielle Rundreise) und dann leicht überprüfen,
ob diese potenzielle Tour durch alle Städte verläuft und ob die
zurückzulegende Entfernung maximal $k$ beträgt. Ein entsprechender
nichtdeterministischer Polynomialzeitalgorithmus ist leicht zu
erstellen. Damit ist der erste Schritt zum Nachweis der
$\NP$-Vollständigkeit von \textsf{TSP} getan und Punkt (1) des
"`Kochrezepts"' abgehandelt.

Als nächstes (Punkt (2)) soll von einem anderen $\NP$-vollständigen
Problem gezeigt werden, dass es effizient in \textsf{TSP}
transformiert werden kann. Geeignet dazu ist das im Text betrachtete
Hamitonkreis-Problem, das bekanntermaßen $\NP$-vollständig ist. Es
ist also zu zeigen: \textsf{HAMILTON} $\le$ \textsf{TSP}.

Folgende Idee führt zum Ziel: Gegeben ist eine Instanz $G=(V,E)$ von
\textsf{HAMILTON}. Transformiere $G$ in folgende Instanz von \textsf{TSP}: Als
Städtemenge $C$ wählen wir die Knoten $V$ des Graphen $G$. Die
Entfernungen zwischen den Städten sind definiert wie folgt:
$D[i,j]=1$, falls es in $E$ eine Kante von Knoten $i$ zu Knoten $j$
gibt, ansonsten setzt man $D[i,j]$ auf einen sehr großen Wert, also
z.B. $n+1$, wenn $n$ die Anzahl der Knoten von $G$ ist. Dann gilt
klarerweise: Wenn $G$ einen Hamiltonschen Kreis besitzt, dann ist der
gleiche Kreis eine Rundreise in $C$ mit Gesamtlänge $n$.  Wenn $G$
keinen Hamiltonschen Kreis besitzt, dann kann es keine Rundreise durch
die Städte $C$ mit Länge höchstens $n$ geben, denn jede Rundreise muss
mindestens eine Strecke von einer Stadt $i$ nach einer Stadt $j$
zurücklegen, die keiner Kante in $G$ entspricht (denn ansonsten hätte
$G$ ja einen Hamiltonschen Kreis). Diese einzelne Strecke von $i$ nach
$j$ hat dann aber schon Länge $n+1$ und damit ist eine Gesamtlänge von
$n$ oder weniger nicht mehr erreichbar. Die Abbildung \ref{TSPRedExample} 
zeigt zwei Beispiele für die Wirkungsweise der Transformation, die
durch Algorithmus \ref{TSPRed} in Polynomialzeit berechnet wird.

\begin{figure}
\begin{center}
\fbox{
\begin{minipage}{0.93\textwidth}
\begin{minipage}{0.46\textwidth}
Aus dem Graphen $G$ links berechnet die Transformation die rechte
Eingabe für das \textsf{TSP}. Die dick gezeichneten Verbindungen deuten
eine Entfernung von $1$ an, wogegen dünne Linien eine Entfernung
von $6$ symbolisieren. Weil $G$ den Hamiltonkreis $1,2,3,4,5,1$ hat,
gibt es rechts eine Rundreise $1,2,3,4,5,1$ mit Gesamtlänge $5$.
\end{minipage}
\begin{minipage}{0.46\textwidth}
\centerline{\hspace*{3em}\includegraphics[scale=0.40]{CG1.eps}}
\end{minipage}

\bigskip

\begin{minipage}{0.46\textwidth}
Im Gegensatz dazu berechnet die Transformation hier aus dem Graphen
$G'$ auf der linken eine Eingabe für das \textsf{TSP} auf der rechten
Seite, die, wie man sich leicht überzeugt, keine Rundreise mit einer
maximalen Gesamtlänge von $5$ hat. Dies liegt daran, dass der
ursprüngliche Graph $G'$ keinen Hamiltonschen Kreis hatte.
\end{minipage}
\begin{minipage}{0.46\textwidth}
\centerline{\hspace*{3em}\includegraphics[scale=0.40]{CG2.eps}}
\end{minipage}
\end{minipage}
}
\end{center}
\caption[Beispiele für die Wirkungsweise von Algorithmus \ref*{TSPRed}]{Beispiele für die Wirkungsweise von Algorithmus \ref{TSPRed}}
\label{TSPRedExample}
\end{figure}

\restylealgo{ruled}
\begin{algorithm}
\caption{Ein Algorithmus für die Reduktion von \textsf{HAMILTON} auf \textsf{TSP}}
\label{TSPRed}
\KwData{Graph $G=(V, E)$, wobei $V = \set{\range{1}{n}}$}
\KwResult{Eine Instanz $(C, D, k)$ für \textsf{TSP}}
\BlankLine
\Begin{
    \tcc*[f]{Die Knoten entsprechen den Städten}

    $C = V$\;       

    \tcc*[f]{Überprüfe alle potentiell existierenden Kanten}
    
    \For{$(i = 1$ $\mathrm{to}$ $n)$}{
      \For{$(j = 1$ $\mathrm{to}$ $n)$}{
        \uIf{$((v_i,v_j) \in E)$}{%
            \tcc*[f]{Kanten entsprechen kleinen Entfernungen}

            $D[i][j] = 1$\;       
         }
         \Else{%
            \tcc*[f]{nicht existierende Kante, dann sehr große Entfernung}

            $D[i][j] = n + 1$\;       
         }
        }
      }
      \tcc*[f]{Gesamtlänge $k$ der Rundreise ist Anzahl der Städte $n$}

      $k = n$\;
      \tcc*[f]{Gebe die berechnete \textsf{TSP}-Instanz zurück}

      \Return $(C,D,k)$\;
}
\end{algorithm}

\special{pdf: out 3 << /Title 
(Die Auswirkungen der NP-Vollständigkeit) 
/Dest [ @thispage /FitH @ypos ] >>}
\subsection{Die Auswirkungen der $\NP$-Vollständigkeit}
Welche Bedeutung haben nun die $\NP$-vollstän\-digen Probleme für die
Klasse $\NP$? Könnte jemand einen deterministischen
Polynomialzeitalgorithmus $\mathcal{A}_C$ für ein
$\NP$-voll\-stän\-diges Problem $C$ angeben, dann hätte man für jedes
$\NP$-Problem einen Polynomialzeitalgorithmus gefunden
(d.h.~$\P=\NP$). Diese überraschende Tatsache lässt sich leicht
einsehen, denn für jedes Problem $A$ aus $\NP$ gibt es eine
Transformation $T$ mit der Eigenschaft, dass $x$ genau dann eine
positive Eingabeinstanz von $A$ ist, wenn $T(x)$ eine positive Instanz
von $C$ ist. Damit löst Algorithmus \ref{PolyAlgoNP} das Problem $A$
in Polynomialzeit. Es gilt also: Ist irgendein $\NP$-vollstän\-diges
Problem effizient lösbar, dann ist $\P=\NP$.

\restylealgo{ruled}
\begin{algorithm}
\caption{Ein fiktiver Algorithmus für Problem $A$}
\label{PolyAlgoNP}
\KwData{Instanz $x$ für das Problem $A$}
\KwResult{\texttt{true}, wenn $x \in A$, \texttt{false} sonst}
\BlankLine
\Begin{
   \tcc*[f]{$T$ ist die postulierte Reduktionsfunktion}

   $y = T(x)$\;
   $z =\mathcal{A}_C(y)$\;
   \BlankLine
   \Return $z$\;
}
\end{algorithm}

Sei nun angenommen, dass jemand $\P \not= \NP$ gezeigt hat. In diesem
Fall ist aber auch klar, dass dann für kein $\NP$-vollständiges
Problem ein Polynomialzeitalgorithmus existieren kann, denn sonst
würde sich ja der Widerspruch $\P = \NP$ ergeben.  Ist das Problem $C$
also $\NP$-vollständig, so gilt: $C$ hat genau dann einen effizienten
Algorithmus, wenn $\P=\NP$, also wenn jedes Problem in $\NP$ einen
effizienten Algorithmus besitzt. Diese Eigenschaft macht die
$\NP$-vollständigen Probleme für die Theoretiker so interessant, denn
eine Klasse von unendlich vielen Problemen kann untersucht werden,
indem man nur ein einziges Problem betrachtet. Man kann sich das auch
wie folgt vorstellen: Alle relevanten Eigenschaften aller Probleme aus
$\NP$ wurden in ein einziges Problem "`destilliert"'. Die
$\NP$-vollständigen Probleme sind also in diesem Sinn
\emph{prototypische} $\NP$-Probleme.

Trotz intensiver Bemühungen in den letzten 30 Jahren konnte bisher
niemand einen Polynomialzeitalgorithmus für ein $\NP$-vollständiges
Problem finden. Dies ist ein Grund dafür, dass man heute
$\P \not= \NP$ annimmt. Leider konnte auch dies bisher nicht gezeigt
werden, aber in der theoretischen Informatik gibt es starke Indizien
für die Richtigkeit dieser Annahme, sodass heute die große Mehrheit
der Forscher von $\P
\not= \NP$ ausgeht.

Für die Praxis bedeutet dies Folgendes: Hat man von einem in der
Realität auftretenden Problem gezeigt, dass es $\NP$-vollständig ist,
dann kann man getrost aufhören, einen effizienten Algorithmus zu
suchen. Wie wir ja gesehen haben, kann ein solcher nämlich (zumindest
unter der gut begründbaren Annahme $\P\neq\NP$) nicht existieren.

Nun ist auch eine Antwort für das $3\mathrm{COL}$-Problem gefunden. Es wurde
gezeigt \cite{GaJo79}, dass $k\mathrm{COL}$ für $k \ge 3$ $\NP$-vollständig
ist. Der fiktive Mobilfunkplaner hat also Pech gehabt: Es ist
unwahrscheinlich, dass er jemals ein korrektes effizientes
Planungsverfahren finden wird.

Ein $\NP$-Vollständigkeitsnachweis eines Problems ist also ein starkes
Indiz für seine praktische Nicht-Handhabbarkeit. Auch die
$\NP$-Vollständigkeit eines Problems, das mit dem Spiel
\emph{Minesweeper} zu tun hat, bedeutet demnach
lediglich, dass dieses Problem höchstwahrscheinlich nicht effizient
lösbar sein wird. Ein solcher Vollständigkeitsbeweis hat nichts mit
einem Schritt in Richtung auf eine Lösung des
$\P\stackrel{?}{=}\NP$-Problems zu tun, wie irreführenderweise
gelegentlich zu lesen ist. Übrigens ist auch für eine Reihe weiterer
Spiele ihre $\NP$-Vollständigkeit bekannt. Dazu gehören {u.a.}
bestimmte Puzzle- und Kreuzwortspiele. Typische Brettspiele, wie Dame,
Schach oder GO, sind hingegen (verallgemeinert auf Spielbretter der
Größe $n\times n$) \textbf{PSPACE}-vollständig. Die Klasse
\textbf{PSPACE} ist eine noch deutlich mächtigere Klasse als
$\NP$. Damit sind also diese Spiele noch viel komplexer als
Minesweeper und andere $\NP$-vollständige Probleme.


\begin{figure}
\begin{center}
\footnotesize
Problemnummern in "`[\dots]"' beziehen sich auf die Sammlung von
Garey und Johnson \cite{GaJo79}.

\columnsep1em
\begin{multicols}{2}
\dprob{CLUSTER \textrm{[GT19]}}
{Netzwerk $G=(V,E)$, positive Integerzahl $K$}
{Gibt es eine Menge von mindestens $K$ Knoten, die paarweise miteinander
verbunden sind?}
%
\dprob{NETZ-AUFTEILUNG \textrm{[ND16]}}
{Netzwerk $G=(V,E)$, Kapazität für jede Kante in $E$, positive
Integerzahl $K$}
{Kann man das Netzwerk so in zwei Teile zerlegen, dass die Gesamtkapazität
aller Verbindungen zwischen den beiden Teilen mindestens $K$ beträgt?}
%
\dprob{NETZ-REDUNDANZ \textrm{[ND18]}}
{Netzwerk $G=(V,E)$, Kosten für Verbindungen zwischen je zwei Knoten aus 
$V$, Budget $B$}
{Kann $G$ so um Verbindungen erweitert werden, dass zwischen je zwei Knoten
mindestens zwei Pfade existieren und die Gesamtkosten für die Erweiterung
höchstens $B$ betragen?}
%
\dprob{OBJEKTE SPEICHERN \textrm{[SR1]}}
{Eine Menge $U$ von Objekten mit Speicherbedarf $s(u)$ für jedes $u \in U$;
Kachelgröße $S$, positive Integerzahl $K$}
{Können die Objekte in $U$ auf $K$ Kacheln verteilt werden?}
%
\dprob{DATENKOMPRESSION \textrm{[SR8]}}
{Endliche Menge $R$ von Strings über festgelegtem Alphabet, positive
Integerzahl $K$}
{Gibt es einen String $S$ der Länge höchstens $K$, sodass jeder String aus $R$
als Teilfolge von $S$ vorkommt?}
%
\dprob{$K$-SCHLÜSSEL \textrm{[SR26]}}
{Relationales Datenbankschema, gegeben durch Attributmenge $A$ und
funktionale Abhängigkeiten auf $A$, positive Integerzahl $K$}
{Gibt es einen Schlüssel mit höchstens $K$ Attributen?}
%
\goodbreak
%
\dprob{BCNF \textrm{[SR29]}}
{Relationales Datenbankschema, gegeben durch Attributmenge $A$ und
funktionale Abhängigkeiten auf $A$, Teilmenge $A' \subseteq A$}
{Verletzt die Menge $A'$ die Boyce-Codd-Normalform?}
%
\dprob{MP-SCHEDULE \textrm{[SS8]}}
{Menge $T$ von Tasks, Länge für jede Task, Anzahl $m$ von Prozessoren, 
positive Integerzahl $D$ ("`Deadline"')}
{Gibt es ein $m$-Prozessor-Schedule für $T$ mit Ausführungszeit höchstens
$D$?}
%
\dprob{PREEMPT-SCHEDULE \textrm{[SS12]}}
{Menge $T$ von Tasks, Länge für jede Task, Präzedenzrelation auf den
Tasks, Anzahl $m$ von Prozessoren, positive Integerzahl $D$
("`Deadline"')}
{Gibt es ein $m$-Prozessor-Schedule für $T$, das die
Präzedenzrelationen berücksichtigt und Ausführungszeit höchstens $D$
hat?}
%
\dprob{DEADLOCK \textrm{[SS22]}}
{Menge von Prozessen, Menge von Ressourcen, aktuelle Zustände der 
Prozesse und aktuell allokierte Ressourcen}
{Gibt es einen Kontrollfluss, der zum Deadlock führt?}
%
\dprob{$K$-REGISTER \textrm{[PO3]}}
{Menge $V$ von Variablen, die in einer Schleife benutzt werden, für jede
Variable einen Gültigkeitsbereich, positive Integerzahl $K$}
{Können die Schleifenvariablen mit höch\-stens $K$ Registern gespeichert werden?}
%
\dprob{REKURSION \textrm{[PO20]}}
{Menge $A$ von Prozedur-Identifiern, Pascal-Programm\-fragment mit
Deklarationen und Aufrufen der Prozeduren aus $A$}
{Ist eine der Prozeduren aus $A$ formal rekursiv?}
\end{multicols}
\end{center}
\caption{Eine kleine Sammlung $\NP$-vollständiger Probleme (Teil 1)}
\label{NPVollProbs1}
\end{figure}

\begin{figure}
\begin{center}
\footnotesize
Problemnummern in "`[\dots]"' beziehen sich auf die Sammlung von
Garey und Johnson \cite{GaJo79}.

\columnsep1em
\begin{multicols}{2}
\dprob{LR($K$)-GRAMMATIK \textrm{[AL15]}}
{Kontextfreie Grammatik $G$, positive Integerzahl $K$ (unär)}
{Ist die Grammatik $G$ nicht LR($K$)?}
%
\dprob{ZWANGSBEDINGUNG \textrm{[LO5]}}
{Menge von Booleschen Constraints, positive Integerzahl $K$}
{Können mindestens $K$ der Constraints gleichzeitig erfüllt werden?}
%
\dprob{INTEGER PROGRAM \textrm{[MP1]}}
{Lineares Programm}
{Hat das Programm eine Lösung, die nur ganzzahlige Werte enthält?}
%
\dprob{KREUZWORTRÄTSEL \textrm{[GP15]}}%
{Menge $W$ von Wörtern, Gitter mit schwarzen und weißen Feldern}
{Können die weißen Felder des Gitters mit Wörtern aus $W$ gefüllt werden?}%
\end{multicols}
\caption{Eine kleine Sammlung $\NP$-vollständiger Probleme (Teil 2)}
\label{NPVollProbs2}
\end{center}
\end{figure}

\special{pdf: out 3 << /Title 
(Der Umgang mit NP-vollständigen Problemen in der Praxis) 
/Dest [ @thispage /FitH @ypos ] >>}
\subsection{Der Umgang mit $\NP$-vollständigen Problemen in der Praxis}

Viele in der Praxis bedeutsame Probleme sind $\NP$-vollständig
(vgl.~die Abbildungen \ref{NPVollProbs1} und \ref{NPVollProbs2}). Ein
Anwendungsentwickler wird es aber sicher schwer haben, seinem
Management mitteilen zu müssen, dass ein aktuelles Projekt nicht
durchgeführt werden kann, weil keine geeigneten Algorithmen zur
Verfügung stehen (Wahrscheinlich würden in diesem Fall einfach
"`geeignetere"' Entwickler eingestellt werden!). Es stellt sich daher
also die Frage, wie man mit solchen $\NP$-vollständigen Problemen in
der Praxis umgeht. Zu dieser Fragestellung hat die theoretische
Informatik ein ausgefeiltes Instrumentarium entwickelt.

Eine erste Idee wäre es, sich mit Algorithmen zufrieden zu geben, die
mit Zufallszahlen arbeiten und die nur mit sehr großer
Wahrscheinlichkeit die richtige Lösung berechnen, aber sich auch mit
kleiner (vernachlässigbarer) Wahrscheinlichkeit irren dürfen. Solche
Algorithmen sind als \emph{probabilistische} oder \emph{randomisierte
Algorithmen}\index{Algorithmus!probabilistisch}\index{Algorithmus!randomisiert}
bekannt \cite{mora95} und werden beispielsweise in der Kryptographie
mit sehr großem Erfolg angewendet.  Das prominenteste Beispiel hierfür
sind Algorithmen, die testen, ob eine gegebene Zahl eine Primzahl ist
und sich dabei fast nie irren. Primzahlen spielen bekanntermaßen im
RSA-Verfahren und damit bei PGP und ähnlichen Verschlüsselungen eine
zentrale Rolle.  Es konnte aber gezeigt werden, dass probabilistische
Algorithmen uns bei den $\NP$-vollständigen Problemen wohl nicht
weiterhelfen. So weiß man heute, dass die Klasse der Probleme, die
sich mit probabilistischen Algorithmen effizient lösen lässt,
höchstwahrscheinlich nicht die Klasse $\NP$ umfasst. Deshalb liegen
(höchstwahrscheinlich) insbesondere alle $\NP$-vollständigen Probleme
außerhalb der Möglichkeiten von effizienten probabilistischen
Algorithmen.

Nun könnte man auch versuchen, "`exotischere"' Computer zu bauen.  In
der letzten Zeit sind zwei potenzielle Auswege bekannt geworden:
DNA-Computer und Quantencomputer.

Es konnte gezeigt werden, dass DNA-Computer (siehe \cite{pau98}) jedes
$\NP$-vollständige Problem in Polynomialzeit lösen können. Für diese
Berechnungsstärke hat man aber einen Preis zu zahlen: Die Anzahl und
damit die Masse der DNA-Moleküle, die für die Berechnung benötigt
werden, wächst exponentiell in der Eingabelänge. Das bedeutet, dass
schon bei recht kleinen Eingaben mehr Masse für eine Berechnung
gebraucht würde, als im ganzen Universum vorhanden ist. Bisher ist
kein Verfahren bekannt, wie dieses Masseproblem gelöst werden kann,
und es sieht auch nicht so aus, als ob es gelöst werden kann, wenn $\P
\not= \NP$ gilt. Dieses Problem erinnert an das oben im Kontext von
Parallelrechnern schon erwähnte Phänomen: Mit exponentiell vielen
Prozessoren lassen sich $\NP$-vollständige Probleme lösen, aber solche
Parallelrechner haben natürlich explodierende Hardware-Kosten.
 
Der anderer Ausweg könnten Quantencomputer sein
(siehe \cite{Ho08,gru99}). Hier scheint die Situation zunächst günstiger zu
sein: Die Fortschritte bei der Quanten\-computer-Forschung verlaufen
immens schnell, und es besteht die berechtigte Hoffnung, dass
Quantencomputer mittelfristig verfügbar sein werden. Aber auch hier
sagen theoretische Ergebnisse voraus, dass Quantencomputer
(höchstwahrscheinlich) keine $\NP$-vollständigen Probleme lösen
können. Trotzdem sind Quantencomputer interessant, denn es ist
bekannt, dass wichtige Probleme existieren, für die kein
Polynomialzeitalgorithmus bekannt ist und die wahrscheinlich nicht
$\NP$-vollständig sind, die aber auf Quantencomputern effizient gelöst
werden können. Das prominenteste Beispiel hierfür ist die Aufgabe,
eine ganze Zahl in ihre Primfaktoren zu zerlegen.

\goodbreak
Die bisher angesprochenen Ideen lassen also die Frage, wie man mit
$\NP$-vollständigen Problemen umgeht, unbeantwortet. In der Praxis
gibt es im Moment zwei Hauptansatzpunkte: Die erste Möglichkeit ist
die, die Allgemeinheit des untersuchten Problems zu beschränken und
eine spezielle Version zu betrachten, die immer noch für die geplante
Anwendung ausreicht. Zum Beispiel sind Graphenprobleme oft einfacher,
wenn man zusätzlich fordert, dass die Knoten des Graphen in der
(Euklidischen) Ebene lokalisiert sind. Deshalb sollte die erste Idee
bei der Behandlung von $\NP$-vollständigen Problemen immer sein, zu
untersuchen, welche Einschränkungen man an das Problem machen kann,
ohne die praktische Aufgabenstellung zu verfälschen. Gerade diese
Einschränkungen können dann effiziente Algorithmen ermöglichen.

Die zweite Möglichkeit sind sogenannte
\dindex{Approximationsalgorithmen} (vgl.~\cite{ACGJNO99}). Die Idee
hier ist es, nicht die optimalen Lösungen zu suchen, sondern sich mit
einem kleinen garantierten Fehler zufrieden zu geben.  Dazu folgendes
Beispiel. Es ist bekannt, dass das \textsf{TSP} auch dann noch
$\NP$-vollständig ist, wenn man annimmt, dass die Städte in der
Euklidischen Ebene lokalisiert sind, d.h.~man kann die Städte in einer
fiktiven Landkarte einzeichnen, sodass die Entfernungen zwischen den
Städten proportional zu den Abständen auf der Landkarte sind.  Das ist
sicherlich in der Praxis keine einschränkende Abschwächung des
Problems und zeigt, dass die oben erwähnte Methode nicht immer zum
Erfolg führen muss: Hier bleibt auch das eingeschränkte Problem
$\NP$-vollständig. Aber für diese eingeschränkte \textsf{TSP}-Variante
ist ein Polynomialzeitalgorithmus bekannt, der immer eine Rundreise
berechnet, die höchstens um einen beliebig wählbaren Faktor schlechter
ist, als die optimale Lösung. Ein Chip-Hersteller, der bei der
Bestückung seiner Platinen die Wege der Roboterköpfe minimieren
möchte, kann also beschließen, sich mit einer Tour zufrieden zu geben,
die um 5 \% schlechter ist als die optimale. Für dieses Problem
existiert ein effizienter Algorithmus!  Dieser ist für die Praxis
völlig ausreichend.


% Einige grundlegende Beweistechniken
\special{pdf: out 2 << /Title 
(Einige formale Grundlagen von Beweistechniken) 
/Dest [ @thispage /FitH @ypos ] >>}
\section{Einige formale Grundlagen von Beweistechniken}
\label{sec:proof}
Praktisch arbeitende Informatiker glauben oft völlig ohne (formale)
Beweistechniken auskommen zu können. Dabei meinen sie sogar, dass
formale Beweise keinerlei Berechtigung in der Praxis der Informatik
haben und bezeichnen solches Wissen als "`in der Praxis irrelevantes
Zeug, das nur von und für seltsame Wissenschaftler erfunden
wurde"'. Studenten in den ersten Semestern unterstellen sogar oft,
dass mathematische Grundlagen und Beweistechniken nur als "`Filter"'
dienen, um die Anzahl der Studenten zu reduzieren. Oft stellen sich
beide Gruppen auf den Standpunkt, dass die Korrektheit von
Programmen und Algorithmen durch "`Lassen wir es doch mal laufen und
probieren es aus!"' ($\triangleq$ Testen) belegt werden könne. Diese
Einstellung zeigt sich oft auch darin, dass Programme mit Hilfe einer
IDE schnell "`testweise"' übersetzt werden, in der Hoffnung oder
(schlimmer) in der Überzeugung, dass ein übersetzbares Programm immer
auch semantisch korrekt sei.

Theoretiker, die sich mit den Grundlagen der Informatik beschäftigen,
vertreten oft den Standpunkt, dass die Korrektheit \emph{jedes}
Programms rigoros \emph{bewiesen} werden muss. Wahrscheinlich ist die
Position zwischen diesen beiden Extremen richtig, denn zum einen ist
der formale Beweis von (großen) Programmen oft nicht praktikabel (oder
möglich) und zum anderen kann das Testen mit einer (relativ kleinen)
Menge von Eingaben sicherlich nicht belegen, dass ein Programm
vollständig den Spezifikationen entspricht. Im praktischen Einsatz ist
es dann oft mit Eingaben konfrontiert, die zu einer fehlerhaften
Reaktion führen oder es sogar abstürzen\footnote{Dies wird
eindrucksvoll durch viele Softwarepakete und verbreitete
Betriebssysteme im PC-Umfeld belegt.} lassen. Bei einfacher
Anwendersoftware sind solche Fehler ärgerlich, aber oft zu
verschmerzen. Bei sicherheitskritischer Software (z.B.~bei der
Regelung von Atomkraftwerken, Airbags und Bremssystemen in Autos, in
der Medizintechnik, bei Finanztransaktionssystemen oder bei der 
Steuerung von Raumsonden) gefährden solche Fehler menschliches 
Leben oder führen zu extrem hohen finanziellen Verlusten und müssen 
deswegen unbedingt vermieden werden.

Für den Praktiker bringen Kenntnisse über formale Beweise aber noch
andere Vorteile. Viele Beweise beschreiben direkt den zur Lösung
benötigten Algorithmus, d.h.~eigentlich wird die Richtigkeit einer
Aussage durch die (implizite) Angabe eines Algorithmus gezeigt. Aber
es gibt noch einen anderen Vorteil. Ist der umzusetzende Algorithmus
komplex (z.B.~aufgrund einer komplizierten Schleifenstruktur oder
einer verschachtelten Rekursion), so ist es unwahrscheinlich, eine
korrekte Implementation an den Kunden liefern zu können, ohne die
Hintergründe ($\triangleq$ Beweis) verstanden zu haben. All dies
zeigt, dass auch ein praktischer Informatiker Einblicke in
Beweistechniken haben sollte. Interessanterweise zeigt die persönliche Erfahrung im
praktischen Umfeld auch, dass solches (theoretisches) Wissen über die
Hintergründe oft zu klarer strukturierten und effizienteren Programmen
führt.

Aus diesen Gründen sollen in den folgenden Abschnitten einige
grundlegende Beweistechniken mit Hilfe von Beispielen (unvollständig)
kurz vorgestellt werden.

\special{pdf: out 3 << /Title 
(Direkte Beweise)
/Dest [ @thispage /FitH @ypos ] >>}
\subsection{Direkte Beweise}
Um einen \dindex{direkten Beweis}\index{Beweis!direkt} zu führen,
müssen wir, beginnend von einer initialen Aussage ($\triangleq$
Hypothese), durch Angabe einer Folge von (richtigen) Zwischenschritten
zu der zu beweisenden Aussage ($\triangleq$ Folgerung) gelangen. Jeder
Zwischenschritt ist dabei entweder unmittelbar klar oder muss wieder
durch einen weiteren (kleinen) Beweis belegt werden. Dabei müssen
nicht alle Schritte völlig formal beschrieben werden, sondern es kommt
darauf an, dass sich dem Leser die eigentliche Strategie erschließt.

\goodbreak
\begin{theorem}
\label{ExpoGTSquare}
Sei $n \in \N$. Falls $n \ge 4$, dann ist $2^n \ge n^2$.
\end{theorem}

Wir müssen also, in Abhängigkeit des Parameters $n$, die Richtigkeit
dieser Aussage belegen. Einfaches Ausprobieren ergibt, dass $2^4 = 16
\ge 16 = 4^2$ und $2^5 = 32 \ge 25 = 5^2$, d.h.~intuitiv scheint die
Aussage richtig zu sein. Wir wollen die Richtigkeit der Aussage nun
durch eine Reihe von (kleinen) Schritten belegen:

\begin{proof}

Wir haben schon gesehen, dass die Aussage für $n = 4$ und $n = 5$
richtig ist. Erhöhen wir $n$ auf $n + 1$, so verdoppelt sich der Wert
der linken Seite der Ungleichung von $2^n$ auf $2 \cdot 2^n =
2^{n+1}$. Für die rechte Seite ergibt sich ein Verhältnis von
$(\frac{n+1}{n})^2$. Je größer $n$ wird, desto kleiner wird der Wert
$\frac{n+1}{n}$, d.h.~der maximale Wert ist bei $n = 4$ mit $1.25$
erreicht. Wir wissen $1.25^2 = 1.5625$. D.h.~immer wenn wir $n$ um
eins erhöhen, verdoppelt sich der Wert der linken Seite, wogegen sich
der Wert der rechten Seite um maximal das $1.5625$ fache erhöht. Damit
muss die linke Seite der Ungleichung immer größer als die rechte Seite
sein.\qed
\end{proof}

\bigskip

Dieser Beweis war nur wenig formal, aber sehr ausführlich und wurde
am Ende durch das Symbol "`$\#$"' markiert. Im Laufe der Zeit
hat es sich eingebürgert, das Ende eines Beweises mit einem besonderen
Marker abzuschließen.  Besonders bekannt ist hier
"`$\mathrm{qed}$"'\index{qed=$\mathrm{qed}$}, eine Abkürzung für die
lateinische Floskel "`quod erat demonstrandum"', die mit "`was zu
beweisen war"' übersetzt werden kann. In neuerer Zeit werden statt
"`$\mathrm{qed}$"' mit der gleichen Bedeutung meist die Symbole
"`$\Box$"' oder "`$\#$"' \index{$\#$}\index{$\Box$}\index{qed}
verwendet.

Nun stellt sich die Frage: "`Wie formal und ausführlich muss ein
Beweis sein?"' Diese Frage kann so einfach nicht beantwortet werden,
denn das hängt u.a.~davon ab, welche Lesergruppe durch den Beweis von
der Richtigkeit einer Aussage überzeugt werden soll und wer den Beweis
schreibt. Ein Beweis für ein Übungsblatt sollte auch auf Kleinigkeiten
Rücksicht nehmen, wogegen ein solcher Stil für eine wissenschaftliche
Zeitschrift vielleicht nicht angebracht wäre, da die potentielle
Leserschaft über ganz andere Erfahrungen und viel mehr
Hintergrundwissen verfügt. Nun noch eine Bemerkung zum Thema
"`Formalismus"': Die menschliche Sprache ist unpräzise, mehrdeutig und
Aussagen können oft auf verschiedene Weise interpretiert werden, wie das
tägliche Zusammenleben der Geschlechter eindrucksvoll demonstriert. Diese
Defizite sollen Formalismen\footnote{In diesem Zusammenhang sind
Programmiersprachen auch Formalismen, die eine präzise Beschreibung
von Algorithmen erzwingen und die durch einen Compiler verarbeitet
werden können.} ausgleichen, d.h.~die Antwort muss lauten: "`So viele
Formalismen wie notwendig und so wenige wie möglich!"'. Durch Übung
und Praxis lernt man die Balance zwischen diesen Anforderungen zu
halten und es zeigt sich bald, dass "`Geübte"' die formale
Beschreibung sogar wesentlich leichter verstehen.

\bigskip

Oft kann man andere, schon bekannte, Aussagen dazu verwenden, die
Richtigkeit einer neuen (evtl.~kompliziert wirkenden) Aussage zu belegen.

\goodbreak
\begin{theorem}
\label{ExpoGTSquare2}
Sei $n \in \N$ die Summe von $4$ Quadratzahlen, die größer als $0$
sind, dann ist $2^n \ge n^2$.
\end{theorem}

\begin{proof}
Die Menge der Quadratzahlen ist $\mathbb{S} = \set{0, 1, 4, 9, 16, 25, 36,
  \dots}$, d.h.~$1$ ist die kleinste Quadratzahl, die größer als $0$
ist. Damit muss unsere Summe von $4$ Quadratzahlen größer als $4$
sein. Die Aussage folgt direkt aus Satz \ref{ExpoGTSquare}.
\qed
\end{proof}

\special{pdf: out 4 << /Title 
(Die Kontraposition)
/Dest [ @thispage /FitH @ypos ] >>}
\subsubsection{Die Kontraposition}
\label{KontraPos}
Mit Hilfe von direkten Beweisen haben wir Zusammenhänge der Form
"`Wenn Aussage $H$ richtig ist, dann folgt daraus die Aussage $C$"'
untersucht. Manchmal ist es schwierig einen Beweis für eine solchen
Zusammenhang zu finden. Völlig gleichwertig ist die Behauptung "`Wenn
die Aussage $C$ falsch ist, dann ist die Aussage $H$ falsch"' und oft
ist eine solche Aussage leichter zu zeigen.

Die \dindex{Kontraposition}\index{Beweis!Kontraposition} von 
Satz \ref{ExpoGTSquare} ist also die folgende Aussage: "`Wenn nicht 
$2^n \ge n^2$, dann gilt nicht $n \ge 4$."'. Das entspricht der 
Aussage: "`Wenn $2^n < n^2$, dann gilt $n < 4$."', was offensichtlich 
zu der ursprünglichen Aussage von Satz \ref{ExpoGTSquare} gleichwertig ist.

Diese Technik ist oft besonders hilfreich, wenn man die Richtigkeit
einer Aussage zeigen soll, die aus zwei Teilaussagen zusammengesetzt
und die durch ein "`genau dann wenn"'\footnote{Oft wird "`genau dann
wenn"' durch \emph{\gdw}\index{gdw=\gdw} abgekürzt.} verknüpft sind. In diesem Fall
sind zwei Teilbeweise zu führen, denn zum einen muss gezeigt werden,
dass aus der ersten Aussage die zweite folgt und umgekehrt muss
gezeigt werden, dass aus der zweiten Aussage die erste folgt.

\begin{theorem}
Eine natürliche Zahl $n$ ist durch drei teilbar genau dann, wenn die
Quersumme ihrer Dezimaldarstellung durch drei teilbar ist.
\end{theorem}

\begin{proof}
Für die Dezimaldarstellung von $n$ gilt
\begin{displaymath}
n = \sum_{i=0}^k a_i \cdot 10^i, \text{wobei $a_i \in \set{0,1, \ldots
,9}$ ("`Ziffern"') und $0 \le i \le k$}.
\end{displaymath}
Mit $\mathrm{QS}(n)$ wird die Quersumme von $n$ bezeichnet,
d.h.~$\mathrm{QS}(n) = \sum_{i=0}^{k} a_i$. Mit Hilfe einer einfachen vollständigen
Induktion kann man zeigen, dass für jedes $i \ge 0$ ein $b \in \N$
existiert, sodass $10^i = 9 b + 1$. Damit gilt $n = \sum_{i=0}^k
a_i \cdot 10^i = \sum_{i=0}^k a_i (9 b_i + 1) = \mathrm{QS}(n) +
9 \sum_{i=0}^k a_i b_i$, d.h.~es existiert ein $c \in \N$, so dass $n
= \mathrm{QS}(n) + 9c$.

\bigskip

\noindent "`$\Rightarrow$"': Wenn $n$ durch $3$ teilbar ist, dann muss auch $\mathrm{QS}(n) + 9c$ durch $3$
teilbar sein. Da $9c$ sicherlich durch $3$ teilbar ist, muss auch
$\mathrm{QS}(n) = n - 9c$ durch $3$ teilbar sein.

\medskip

\noindent "`$\Leftarrow$"': Dieser Fall soll durch Kontraposition gezeigt
werden. Sei nun $n$ nicht durch $3$ teilbar, dann darf
$\mathrm{QS}(n)$ nicht durch $3$ teilbar sein, denn sonst wäre $n = 9c
+ \mathrm{QS}(n)$ durch $3$ teilbar.\qed
\end{proof}

\special{pdf: out 3 << /Title 
(Der Ringschluss) /Dest [ @thispage /FitH @ypos ] >>}
\subsection{Der Ringschluss}
\label{RingSchluss}
Oft findet man mehrere Aussagen, die zueinander äquivalent sind. Ein
Beispiel dafür ist Satz \ref{SetProof}. Um die Äquivalenz dieser
Aussagen zu beweisen, müssten jeweils zwei "`genau dann wenn"'
Beziehungen untersucht werden, d.h.~es werden vier Teilbeweise
notwendig. Dies kann mit Hilfe eines so
genannten \emph{Ringschlusses}\index{Ringschluss}\index{Beweis!Ringschluss} 
abgekürzt werden, denn es reicht zu zeigen, dass aus der ersten Aussage die 
zweite folgt, aus der zweiten Aussage die dritte und dass schließlich aus der
dritten Aussage wieder die erste folgt. Im Beweis zu
Satz \ref{SetProof} haben wir deshalb nur drei anstatt vier
Teilbeweise zu führen, was zu einer Arbeitsersparnis führt. Diese
Arbeitsersparnis wird um so größer, je mehr äquivalente Aussagen zu
untersuchen sind. Dabei ist die Reihenfolge der Teilbeweise nicht
wichtig, solange die einzelnen Teile zusammen einen Ring bilden.

\goodbreak
\begin{theorem}
\label{SetProof}
Seien $A$ und $B$ zwei beliebige Mengen, dann sind die folgenden drei
Aussagen äquivalent:
\begin{enumerate}[i)]
%
\item\label{SetProof1} $A \subseteq B$
%
\item\label{SetProof2} $A \cup B = B$
%
\item\label{SetProof3} $A \cap B = A$
%
\end{enumerate}
\end{theorem}

\begin{proof}
Im folgenden soll ein Ringschluss verwendet werden, um die Äquivalenz
der drei Aussagen zu zeigen:

\noindent "`\ref{SetProof1}) $\Rightarrow$ \ref{SetProof2})"': Da nach
Voraussetzung $A \subseteq B$ ist, gilt für jedes Element $a \in A$
auch $a \in B$, d.h.~in der Vereinigung $A \cup B$ sind alle Elemente
nur aus $B$, was $A \cup B = B$ zeigt.

\medskip

\noindent "`\ref{SetProof2}) $\Rightarrow$ \ref{SetProof3})"': Wenn
$A \cup B = B$ gilt, dann ergibt sich durch Einsetzen und mit den
Regeln aus Abschnitt \ref{SetOpSect} (Absorptionsgesetz) direkt
$A \cap B = A \cap (A \cup B) = A$.

\medskip

\noindent "`\ref{SetProof3}) $\Rightarrow$ \ref{SetProof1})"': Sei
nun $A \cap B = A$, dann gibt es kein Element $a \in A$ für das
$a \not\in B$ gilt. Dies ist aber gleichwertig zu der Aussage
$A \subseteq B$. 

Damit hat sich ein Ring von Aussagen "`\ref{SetProof1})
$\Rightarrow$ \ref{SetProof2})"', "`\ref{SetProof2})
$\Rightarrow$ \ref{SetProof3})"' und "`\ref{SetProof3})
$\Rightarrow$ \ref{SetProof1})"' gebildet, was die Äquivalenz aller
Aussagen zeigt.\qed
\end{proof}

\special{pdf: out 3 << /Title 
(Widerspruchsbeweise)
/Dest [ @thispage /FitH @ypos ] >>}
\subsection{Widerspruchsbeweise}
\label{IndirektBeweis}
Obwohl die Technik der
Widerspruchsbeweise\index{Widerspruchsbeweis}\index{Beweis!Widerspruch}
auf den ersten Blick sehr kompliziert erscheint, ist sie meist einfach
anzuwenden, extrem mächtig und liefert oft sehr kurze
Beweise. Angenommen wir sollen die Richtigkeit einer Aussage "`aus der
Hypothese $H$ folgt $C$"' zeigen. Dazu beweisen wir, dass sich ein
Widerspruch ergibt, wenn wir, von $H$ und der Annahme, dass $C$ falsch
ist, ausgehen. Also war die Annahme falsch, und die Aussage $C$ muss
richtig sein.

Anschaulicher wird diese Beweistechnik durch folgendes Beispiel:
Nehmen wir einmal an, dass Alice eine bürgerliche Frau ist und deshalb
auch keine Krone trägt. Es ist klar, dass jede Königin eine Krone
trägt. Wir sollen nun beweisen, dass Alice keine Königin ist. Dazu
nehmen wir an, dass Alice eine Königin ist, d.h.~Alice trägt eine
Krone. Dies ist ein Widerspruch! Also war unsere Annahme falsch, und
wir haben gezeigt, dass Alice keine Königin sein kann.

\goodbreak
\noindent Der Beweis zu folgendem Satz verwendet diese Technik:
\begin{theorem}
Sei $S$ eine endliche Untermenge einer unendlichen Menge $U$. Sei $T$
das Komplement von $S$ bzgl.~$U$, dann ist $T$ eine unendliche Menge.
\end{theorem}

\begin{proof}
Hier ist unsere Hypothese "`$S$ endlich, $U$ unendlich und $T$
Komplement von $S$ bzgl.~$U$"' und unsere Folgerung ist "`$T$ ist
unendlich"'. Wir nehmen also an, dass $T$ eine endliche Menge ist. Da
$T$ das Komplement von $S$ ist, gilt $S \cap T = \emptyset$, also ist
$\cnt(S) + \cnt(T) = \cnt (S \cap T) + \cnt (S \cup T) = \cnt (S \cup
T) = n$, wobei $n$ eine Zahl aus $\N$ ist (siehe
Abschnitt \ref{cntSet}). Damit ist $S \cup T = U$ eine endliche
Menge. Dies ist ein Widerspruch zu unserer Hypothese! Also war die
Annahme "`$T$ ist endlich"' falsch. \qed
\end{proof}

\special{pdf: out 3 << /Title 
(Der Schubfachschluss)
/Dest [ @thispage /FitH @ypos ] >>}
\subsection{Der Schubfachschluss}
\label{Schubfachschluss}
Der \dindex{Schubfachschluss}\index{Beweis!Schubfach} ist auch 
als \dindex{Dirichlets Taubenschlagprinzip}\index{Taubenschlagprinzip} 
bekannt. Werden $n > k$ Tauben auf $k$ Boxen verteilt, so gibt es 
mindestens eine Box in der sich wenigstens zwei Tauben aufhalten. 
Allgemeiner formuliert sagt das Taubenschlagprinzip, dass wenn $n$ 
Objekte auf $k$ Behälter aufgeteilt werden, dann gibt es mindestens 
eine Box die mindestens $\lceil \frac{n}{k} \rceil$ Objekte enthält.

\begin{example}
Auf einer Party unterhalten sich $8$ Personen ($\triangleq$ Objekte),
dann gibt es mindestens einen Wochentag ($\triangleq$ Box) an dem
$\lceil \frac{8}{7} \rceil =2$ Personen aus dieser Gruppe Geburtstag
haben.
\end{example}

\special{pdf: out 3 << /Title 
(Gegenbeispiele)
/Dest [ @thispage /FitH @ypos ] >>}
\subsection{Gegenbeispiele}
Im wirklichen Leben wissen wir nicht, ob eine Aussage richtig oder
falsch ist. Oft sind wir dann mit einer Aussage konfrontiert, die auf
den ersten Blick richtig ist und sollen dazu ein Programm
entwickeln. Wir müssen also entscheiden, ob diese Aussage wirklich
richtig ist, denn sonst ist evtl.~alle Arbeit umsonst und hat hohen
Aufwand verursacht. In solchen Fällen kann man versuchen, ein einziges
Beispiel dafür zu finden, dass die Aussage falsch ist, um so unnötige
Arbeit zu sparen.\index{Gegenbeispiel}\index{Beweis!Gegenbeispiel}

\bigskip

\noindent Wir zeigen, dass die folgenden Vermutungen falsch sind:
\begin{conjecture}
Wenn $p \in \N$ eine Primzahl ist, dann ist $p$ ungerade.
\end{conjecture}

\begin{counterexample}
Die natürliche Zahl 2 ist eine Primzahl und $2$ ist gerade. \qed
\end{counterexample}

\begin{conjecture}
Es gibt keine Zahlen $a,b \in \N$, sodass $a \textrm{ mod } b = b
\textrm{ mod } a$.
\end{conjecture}

\begin{counterexample}
Für $a = b = 2$ gilt $a \textrm{ mod } b = b \textrm{ mod } a = 0$. \qed
\end{counterexample}

\special{pdf: out 3 << /Title 
(Induktionsbeweise und das Induktionsprinzip)
/Dest [ @thispage /FitH @ypos ] >>}
\subsection{Induktionsbeweise und das Induktionsprinzip}
Eine der wichtigsten und nützlichsten Beweismethoden in der Informatik bzw.~Mathematik
ist das \dindex{Induktionsprinzip}. Wir wollen jetzt nachweisen, dass
für jedes $n \in \N$ eine bestimmte Eigenschaft $E$ gilt. Wir
schreiben kurz $E(n)$ für die Aussage "`$n$ besitzt die Eigenschaft
$E$"'. Mit der Schreibweise $E(0)$ drücken\footnote{Mit $E$ wird also
ein Prädikat bezeichnet (siehe Abschnitt \ref{MengenDef})} wir also
aus, dass die erste natürliche Zahl $0$ die Eigenschaft $E$ besitzt.

\noindent\textbf{Induktionsprinzip:} Es gelten
\index{Induktion!Prinzip}
\indudef%
{$E(0)$}% 
{Für $n \ge 0$ gilt, wenn $E(k)$ für $k \le n$ korrekt ist,
dann ist auch $E(n+1)$ richtig.}
\medskip

Dabei ist \textbf{\textsf{IA}} die Abkürzung für
\dindex{Induktionsanfang} und \textbf{\textsf{IS}} ist die Kurzform von
\dindex{Induktionsschritt}. Die Voraussetzung ($\triangleq$ Hypothese)
$E(k)$ ist korrekt für $k \le n$ und wird im Induktionsschritt
als \dindex{Induktionsvoraussetzung} benutzt
(kurz \textbf{\textsf{IV}}). Hat man also den Induktionsanfang und den
Induktionsschritt gezeigt, dann ist es anschaulich, dass jede natürliche Zahl die
Eigenschaft $E$ haben muss.

Es gibt verschiedene Versionen von Induktionsbeweisen. Die bekannteste
Version ist die vollständige Induktion, bei der Aussagen über
natürliche Zahlen gezeigt werden.

\special{pdf: out 4 << /Title 
(Die vollständige Induktion)
/Dest [ @thispage /FitH @ypos ] >>}
\subsubsection{Die vollständige Induktion}

Wie in Piratenfilmen üblich, seien Kanonenkugeln in einer Pyramide mit
quadratischer Grundfläche gestapelt. Wir stellen uns die Frage,
wieviele Kugeln (in Abhängigkeit von der Höhe) in einer solchen
Pyramide gestapelt sind.\index{vollständige Induktion}\index{Induktion!vollständige}\index{Beweis!Induktion!vollständig}

\begin{theorem}
\label{Pyramid}
Mit einer quadratische Pyramide aus Kanonenkugeln der Höhe $n \ge 1$
als Munition, können wir $\frac{n(n+1)(2n+1)}{6}$ Schüsse abgeben.
\end{theorem}

\goodbreak
\begin{proof}
Einfacher formuliert: wir sollen zeigen, dass $\sum\limits_{i=1}^n i^2 =
\frac{n(n+1)(2n+1)}{6}$.
\induproof%
{Eine Pyramide der Höhe $n = 1$ enthält $\frac{1 \cdot 2 \cdot 3}{6} =
  1$ Kugel. D.h.~wir haben die Eigenschaft für $n = 1$ verifiziert.}%
{Für $k \le n$ gilt $\sum\limits_{i=1}^k i^2 = \frac{k(k+1)(2k+1)}{6}$.}%
{%
Wir müssen nun zeigen, dass $\sum\limits_{i=1}^{n+1} i^2 =
\frac{(n+1)((n+1)+1)(2(n+1)+1)}{6}$ gilt und dabei muss die
Induktionsvoraussetzung $\sum\limits_{i=1}^n i^2 = \frac{n(n+1)(2n+1)}{6}$
benutzt werden.  

\begin{displaymath}
\begin{array}{rcl}
\sum\limits_{i=1}^{n+1}i^2 &=& \sum\limits_{i=1}^{n}i^2 + (n + 1)^2\\
&\stackrel{\text{\textbf{\textsf{IV}}}}{=}& \frac{n(n+1)(2n+1)}{6} +
(n^2 + 2n + 1)\\
&=& \frac{2n^3 + 3n^2+n}{6} + (n^2 + 2n + 1)\\
&=& \frac{2n^3 + 9n^2+13n + 6}{6}\\
&=& \frac{(n+1)(2n^2+7n+6)}{6} \quad (\star)\\
&=& \frac{(n+1)(n+2)(2n+3)}{6} \quad (\star\star)\\
&=& \frac{(n+1)((n+1) + 1)(2(n + 1) + 1)}{6}\\
\end{array}
\end{displaymath}
Die Zeile $\star$ (bzw.~$\star\star$) ergibt sich, indem man $2n^3 +
9n^2+13n + 6$ durch $n+1$ teilt (bzw.~$2n^2+7n+6$ durch $n+2$). \qed
}
\end{proof}

Das Induktionsprinzip kann man auch variieren. Dazu soll nun gezeigt
werden, dass die Eigenschaft $E$ für alle Zahlen $k \le n$ erfüllt
ist.

\noindent\textbf{Verallgemeinertes Induktionsprinzip:} Es gelten 
\index{Induktion!Prinzip!verallgemeinert}
\indudef%
{$E(0)$}% 
{Wenn für alle $0 \le k \le n$ die Eigenschaft $E(k)$ gilt, dann ist 
auch $E(n+1)$ richtig.}
\medskip

Damit ist das verallgemeinerte Induktionsprinzip eine
Verallgemeinerung des oben vorgestellten Induktionsprinzips, wie das
folgende Beispiel veranschaulicht:

\begin{theorem}
Jede natürliche Zahl $n \ge 2$ läßt sich als Produkt von Primzahlen 
schreiben.
\end{theorem}
\begin{proof}
Das verallgemeinerte Induktionsprinzip wird wie folgt verwendet:
\induproof{%
Offensichtlich ist $2$ das Produkt von einer Primzahl.
}{%
Jede natürliche Zahl $m$ mit $2 \le m \le n$ kann als Produkt von
Primzahlen geschrieben werden.
}{
Nun wird eine Fallunterscheidung durchgeführt:
\begin{enumerate}[i)]
%
\item Sei $n+1$ wieder eine Primzahl, dann ist nichts zu zeigen, da
$n+1$ direkt ein Produkt von Primzahlen ist. 
%
\item Sei $n+1$ keine Primzahl, dann existieren mindestens zwei Zahlen
$p$ und $q$ mit $2 \le p,q < n+1$ und $p \cdot q = n + 1$. Nach
Induktionsvoraussetzung sind dann $p$ und $q$ wieder als Produkt von
Primzahlen darstellbar. Etwa $p = p_1 \cdot p_2 \cdot \ldots \cdot p_s$
und $q = q_1 \cdot q_2 \cdot \ldots \cdot q_t$. Damit ist aber $n + 1 =
p \cdot q = p_1 \cdot p_2 \cdot \ldots \cdot p_s \cdot q_1 \cdot
q_2 \cdot \ldots \cdot q_t$ ein Produkt von Primzahlen.\qed
}
%
\end{enumerate}

\end{proof}


\noindent Solche Induktionsbeweise treten z.B.~bei der Analyse von Programmen
immer wieder auf.

\special{pdf: out 4 << /Title 
(Induktive Definitionen)
/Dest [ @thispage /FitH @ypos ] >>}
\subsubsection{Induktive Definitionen}

Das Induktionsprinzip kann aber auch dazu verwendet werden,
(Daten-)Strukturen formal zu spezifizieren. Dazu werden in einem
ersten Schritt ($\triangleq$ Induktionsanfang) die "`atomaren"'
Objekte definiert und dann in einem zweiten Schritt die
zusammengesetzten Objekte ($\triangleq$ Induktionsschritt). Diese
Technik ist als \dindex{induktive Definition} bekannt.

\begin{example}
\noindent Die Menge der binären Bäume ist wie folgt definiert:

\medskip

\indudef{Ein einzelner Knoten $w$ ist ein \emph{Baum} und $w$ ist die
  \emph{Wurzel} dieses Baums.}{Seien $T_1, T_2, \dots, T_n$ Bäume mit den
  Wurzeln $\enu{k}{1}{n}$ und $w$ ein einzelner neuer Knoten. Verbinden wir
  den Knoten $w$ mit allen Wurzeln $\enu{k}{1}{n}$, dann entsteht ein neuer Baum
  mit der Wurzel $w$. Nichts sonst ist ein Baum.} 
\end{example}

\begin{example}
\label{induexp}
\noindent Die Menge der arithmetischen Ausdrücke ist wie folgt definiert:

\medskip

\indudef{Jeder Buchstabe und jede Zahl ist ein arithmetischer
Ausdruck.}{Seien $E$ und $F$ Ausdrücke, so sind auch $E + F$, $E * F$
und $[E]$ Ausdrücke. Nichts sonst ist ein Ausdruck.}

\medskip

\noindent D.h.~$x$, $x+y$, $[2*x + z]$ sind arithmetische Ausdrücke,
aber beispielsweise sind $x + $, $yy$, $][x+y$ sowie $x +* z$ keine
Ausdrücke im Sinn dieser Definition.
\end{example}

\begin{example}
\label{indubool}
\noindent Die Menge der aussagenlogischen Formeln ist wie folgt definiert:

\medskip

\indudef{Jede aussagenlogische Variable $x_1, x_2, x_3, \dots $ ist
eine aussagenlogische Formel.}{Seien $H_1$ und $H_2$ aussagenlogische
Formeln, so sind auch $(H_1 \wedge H_2)$, $(H_1 \vee H_2)$, $\neg H_1$,
$(H_1 \leftrightarrow H_2)$, $(H_1 \rightarrow H_2)$ und $(H_1 \oplus H_2)$
aussagenlogische Formeln. Nichts sonst ist eine Formel.}
\end{example}

Bei diesen Beispielen ahnt man schon, dass solche Techniken zur
präzisen Definition von Programmiersprachen und Dateiformaten gute
Dienste leisten. Induktive Definitionen haben noch einen weiteren
Vorteil, denn man kann oft relativ leicht Induktionsbeweise konstruieren, die 
Aussagen über induktiv definierte Objekte belegen / beweisen.

\special{pdf: out 4 << /Title 
(Die strukturelle Induktion)
/Dest [ @thispage /FitH @ypos ] >>}
\subsubsection{Die strukturelle Induktion}

\begin{theorem}
\label{CntBrack}
Die Anzahl der öffnenden Klammern eines arithmetischen Ausdrucks stimmt
mit der Anzahl der schließenden Klammern überein.
\end{theorem}

Es ist offensichtlich, dass diese Aussage richtig ist, denn in
Ausdrücken wie $(x + y) / 2$ oder $x + ((y/2) * z)$ muss ja zu jeder
öffnenden Klammer eine schließende Klammer existieren. Der nächste
Beweis verwendet diese Idee um die Aussage von Satz \ref{CntBrack}
mit Hilfe einer 
\dindex{strukturellen Induktion}\index{Induktion!strukturelle}\index{Beweis!Induktion!strukturell} 
zu zeigen.

\begin{proof}
Wir bezeichnen die Anzahl der öffnenden Klammern eines Ausdrucks $E$
mit $\cnt_[(E)$ und verwenden die analoge Notation $\cnt_](E)$ für die
Anzahl der schließenden Klammern.

\induproof%
{
Die einfachsten Ausdrücke sind Buchstaben und Zahlen. Die Anzahl der
öffnenden und schließenden Klammern ist in beiden Fällen gleich $0$.
}
{
Sei $E$ ein Ausdruck, dann gilt $\cnt_[(E) = \cnt_](E)$.
}
{
 Für einen Ausdruck $E + F$ gilt $\cnt_[(E + F) = \cnt_[(E
) +
 \cnt_[(F) \stackrel{\text{\textbf{\textsf{IV}}}}{=} \cnt_](E) +
 \cnt_](F) = \cnt_](E + F)$. Völlig analog zeigt man dies für $E *
 F$. Für den Ausdruck $[E]$ ergibt sich $\cnt_[([E]) = \cnt_[(E) + 1
 \stackrel{\text{\textbf{\textsf{IV}}}}{=} \cnt_](E) + 1 = \cnt_]([E])$.
 In jedem Fall ist die Anzahl der öffnenden Klammern gleich der Anzahl
 der schließenden Klammern.\qed
}
\end{proof}

\bigskip

Mit Hilfe von Satz \ref{CntBrack} können wir nun leicht ein Programm
entwickeln, das einen Plausibilitätscheck (z.B.~direkt in einem Editor)
durchführt und die Klammern zählt, bevor die Syntax von arithmetischen
Ausdrücken überprüft wird. Definiert man eine vollständige
Programmiersprache induktiv, dann werden ganz ähnliche
Induktionsbeweise möglich, d.h.~man kann die Techniken aus diesem
Beispiel relativ leicht auf die Praxis der Informatik übertragen.

Man überlegt sich leicht, dass die natürlichen Zahlen auch induktiv
definiert werden können. Damit zeigt sich, dass die vollständige
Induktion eigentlich nur ein Spezialfall der strukturellen Induktion
ist.


\begin{center}
\mbox{}
\vfill
$\star \star \star$ \textsc{Ende} $\star \star \star$
\end{center}

% Appendix
\appendix

% Index 
\cleardoublepage
\special{pdf: out 2 << /Title 
(Stichwortverzeichnis) 
/Dest [ @thispage /FitH @ypos ] >>}
\addcontentsline{toc}{section}{Stichwortverzeichnis}
\def\indexname{Stichwortverzeichnis}
\makeatletter
\printindex
\makeatother

% Literatur
\cleardoublepage
\special{pdf: out 2 << /Title 
(Literatur) 
/Dest [ @thispage /FitH @ypos ] >>}
\def\bibname{Literatur}
\addcontentsline{toc}{section}{Literatur}
\bibliography{mbasic}

\end{document}

