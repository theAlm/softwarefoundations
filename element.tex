\special{pdf: out 2 << /Title 
(Grundlagen und Schreibweisen) 
/Dest [ @thispage /FitH @ypos ] >>}
\section{Grundlagen und Schreibweisen}

\special{pdf: out 3 << /Title 
(Mengen) 
/Dest [ @thispage /FitH @ypos ] >>}
\subsection{Mengen}Es ist sehr schwer den fundamentalen Begriff der Menge mathematisch exakt
zu definieren. Aus diesem Grund soll uns hier die von Cantor im Jahr $1895$ gegebene
Erklärung genügen, da sie für unsere Zwecke völlig ausreichend ist:

\begin{definition}[Georg Cantor (\cite{Ca85})]
Unter einer ,\dindex{Menge}' verstehen wir jede Zusammenfassung $M$ von 
bestimmten wohlunterschiedenen Objecten $m$ unsrer Anschauung oder 
unseres Denkens (welche die ,\dindex{Elemente}' von $M$ genannt werden) zu 
einem Ganzen\footnote{Diese Zitat entspricht der originalen
Schreibweise von Cantor.}. 
\end{definition}

\noindent Für die Formulierung "`genau dann wenn"' verwenden wir im Folgenden
die Abkürzung \gdw\index{gdw=gdw.} um Schreibarbeit zu sparen.

\special{pdf: out 4 << /Title 
(Elementbeziehung und Enthaltenseinsrelation) 
/Dest [ @thispage /FitH @ypos ] >>}
\subsubsection{Die Elementbeziehung und die Enthaltenseinsrelation}
Sehr oft werden einfache große lateinische Buchstaben wie $N$, $M$, $A$, $B$ oder $C$ 
als Symbole für Mengen verwendet und kleine Buchstaben für die Elemente einer Menge.
Mengen von Mengen notiert man gerne mit kalligraphischen Buchstaben wie $\mathcal{A}$, 
$\mathcal{B}$ oder $\mathcal{M}$.
\begin{definition}
\label{InclSet}
Sei $M$ eine beliebige Menge, dann ist
\begin{itemize}
%
\item $a \in M$ \gdw\ $a$ ist ein Element der Menge
$M$\index{$\in$},
%
\item $a \not\in M$ \gdw\ $a$ ist kein Element der Menge $M$\index{$\not\in$},
%
\item $M \subseteq N$ \gdw\ aus $a \in M$ folgt $a \in N$ ($M$ ist
\dindex{Teilmenge}\index{Menge!Teil-} von $N$)\index{$\subseteq$},
%
\item $M \not\subseteq N$ \gdw\ es gilt nicht $M \subseteq
N$. Gleichwertig: es gibt ein $a \in M$ mit $a \not\in N$ ($M$ ist
keine Teilmenge von $N$)\index{$\not\subseteq$} und
%
\item $M \subset N$ \gdw\ es gilt $M \subseteq N$ und $M \not= N$ ($M$ ist
echte Teilmenge von $N$)\index{$\subset$}.
%
\end{itemize}
Statt $a \in M$ schreibt man auch $M \ni a$\index{$\ni$}, was in
einigen Fällen zu einer deutlichen Vereinfachung der Notation führt.
\end{definition}

\special{pdf: out 4 << /Title 
(Definition spezieller Mengen) 
/Dest [ @thispage /FitH @ypos ] >>}
\subsubsection{Definition spezieller Mengen}
\label{MengenDef}
Spezielle Mengen können auf verschiedene Art und Weise definiert
werden, wie z.B.
\begin{itemize}
%
\item durch Angabe von Elementen:\ So ist $\set{\enu{a}{1}{n}}$ die Menge,
die aus den Elementen $\enu{a}{1}{n}$ besteht, oder
%
\item durch eine Eigenschaft $E$:\ Dabei ist $\set{a \mid E(a)}$ die Menge
aller Elemente $a$, die die Eigenschaft\footnote{Die Eigenschaft $E$
kann man dann auch als \dindex{Prädikat} bezeichnen.} $E$ besitzen.
%
\end{itemize}
Alternativ zu der Schreibweise $\set{a \mid E(a)}$ wird auch oft 
$\set{a \colon E(a)}$ verwendet.


\begin{example}
\mbox{}

\begin{itemize}
%
\item Mengen, die durch die Angabe von Elementen definiert sind:
\begin{itemize}
  %
  \item $\mathbb{B} \eqd \set{0,1}$
  %  
  \item $\N \eqd \set{0, 1, 2, 3, 4, 5, 6, 7, 8, \dots}$ (Menge der
  \dindex{natürlichen Zahlen}\index{Zahlen!natürlich})\index{$\N$}
  %
  \item $\Z \eqd \set{\dots, -4, -3, -2, -1, 0, 1, 2, 3, 4, \dots}$
  (Menge der \dindex{ganzen Zahlen}\index{Zahlen!ganz})\index{$\Z$}
  %
  \item $2\Z \eqd \set{0, \pm 2, \pm 4, \pm 6, \pm 8, \dots}$
  (Menge der geraden ganzen Zahlen)\index{$2\Z$}
  %
  \item $\PRIM \eqd \set{2, 3, 5, 7, 11, 13, 17, 19, \dots}$
  (Menge der \dindex{Primzahlen})\index{$\PRIM$}
  %
\end{itemize}
%
\item Mengen, die durch eine Eigenschaft $E$ definiert sind:
\begin{itemize}
%
\item $\set{n \mid n \in \N \text{ und $n$ ist durch $3$ teilbar}}$
%
\item $\set{n \mid n \in \N \text{ und $n$ ist Primzahl und $n \le 40$}}$
%
\item $\emptyset \eqd \set{a \mid a \not= a}$ (die leere Menge)\index{$\emptyset$}
%
\end{itemize}
\end{itemize}
Aus Definition \ref{InclSet} ergibt sich, dass die leere Menge
$\emptyset$ Teilmenge jeder Menge ist. Dabei ist zu beachten, dass 
$\set{\emptyset} \not= \emptyset$\index{$\set{\emptyset}$} gilt, denn $\set{\emptyset}$ 
enthält \emph{ein} Element (die leere Menge) und $\emptyset$ enthält \emph{kein} Element.
\end{example}

\special{pdf: out 4 << /Title 
(Operationen auf Mengen) 
/Dest [ @thispage /FitH @ypos ] >>}
\subsubsection{Operationen auf Mengen}
\label{OpSetSect}

\begin{definition}
\label{OpSet}
Seien $A$ und $B$ beliebige Mengen, dann ist
\begin{itemize}
%
\item $A \cap B \eqd \set{a \mid a \in A \text{ und } a \in B}$
(\dindex{Schnitt}\index{Menge!Schnitt-} von $A$ und $B$)\index{$\cap$},
%
\item $A \cup B \eqd \set{a \mid a \in A \text{ oder } a \in B}$
(\dindex{Vereinigung}\index{Menge!Vereinigung-} von $A$ und $B$)\index{$\cup$},
%
\item $A \setminus B \eqd \set{a \mid a \in A \text{ und } a
\not\in B}$ (\dindex{Differenz}\index{Menge!Differenz-} von $A$ und $B$)\index{$\setminus$},
%
\item $\overline{A} \eqd M \setminus A$ (\dindex{Komplement}\index{Menge!Komplement-} von $A$ 
bezüglich einer festen Grundmenge $M$)\index{$\overline{A}$} und
%
\item $\PowerSet{A} \eqd \set{B \mid B \subseteq A}$ 
(\dindex{Potenzmenge}\index{Menge!Potenz-} von $A$)\index{$\PowerSet{A}$}.
%
\end{itemize}
Zwei Mengen $A$ und $B$ mit $A \cap B = \emptyset$ nennt
man \dindex{disjunkt}.

\begin{example}
Sei $A = \set{2, 3, 5, 7}$ und $B = \set{1, 2, 4 , 6}$, dann ist $A
\cap B = \set{2}$, $A \cup B = \set{1, 2, 3, 4, 5, 6, \allowbreak 7}$ und $A
\setminus B = \set{3, 5, 7}$. Wählen wir als Grundmenge die
natürlichen Zahlen, also $M = \N$, dann ist $\overline{A} = \set{n \in \N
\mid n \not= 2 \text{ und } n \not= 3 \text{ und } n
\not= 5 \text{ und } n \not= 7} = \set{1, 4, 6, 8, 9, 10, 11,
\dots}$. 

Als Potenzmenge der Menge $A$ ergibt sich die folgende Menge von Mengen
von natürlichen Zahlen $\PowerSet{A}
= \set{\emptyset,\allowbreak \set{2},\allowbreak \set{3},
\allowbreak \set{5},\allowbreak \set{7}, \allowbreak \set{2,3},
\set{2,5}, \allowbreak \set{2,7}, \allowbreak \set{3,5}, \allowbreak\set{3,7},
\allowbreak \set{5,7},\allowbreak \set{2,\allowbreak 3,5},\allowbreak
\set{2,3,7},\allowbreak \set{2,5,\allowbreak 7}, \allowbreak \set{3,\allowbreak 5,\allowbreak 7},\set{2,3,5,7}}$. 

Offensichtlich ist die Menge $\set{0,2,4,6,8, \dots }$ der geraden
natürlichen Zahlen und die Menge $\set{1,3,5,7,9, \dots }$ der
ungeraden natürlichen Zahlen disjunkt.
\end{example}
\end{definition}

\special{pdf: out 4 << /Title 
(Gesetze für Mengenoperationen) 
/Dest [ @thispage /FitH @ypos ] >>}
\subsubsection{Gesetze für Mengenoperationen}
\label{SetOpSect}
Für die klassischen Mengenoperationen gelten die folgenden Beziehungen:
\begin{displaymath}
\begin{array}{rcll}
A \cap B &=& B \cap A & \text{Kommutativgesetz für den Schnitt}\\
A \cup B &=& B \cup A & \text{Kommutativgesetz für die Vereinigung}\\
A \cap (B \cap C) &=& (A \cap B) \cap C & \text{Assoziativgesetz für
den Schnitt}\\
A \cup (B \cup C) &=& (A \cup B) \cup C & \text{Assoziativgesetz für
die Vereinigung}\\
A \cap (B \cup C) &=& (A \cap B) \cup (A \cap C) & \text{Distributivgesetz}\\
A \cup (B \cap C) &=& (A \cup B) \cap (A \cup C) & \text{Distributivgesetz}\\
A \cap A &=& A & \text{Duplizitätsgesetz für den Schnitt}\\
A \cup A &=& A & \text{Duplizitätsgesetz für die Vereinigung}\\
A \cap (A \cup B) &=& A & \text{Absorptionsgesetz}\\
A \cup (A \cap B) &=& A & \text{Absorptionsgesetz}\\
\overline{A \cap B} &=& (\overline{A} \cup \overline{B}) &
\text{de-Morgansche Regel}\\
\overline{A \cup B} &=& (\overline{A} \cap \overline{B}) &
\text{de-Morgansche Regel}\\
\overline{\overline{A}} &=& A & \text{Gesetz des doppelten Komplements}
\end{array}
\end{displaymath}
Die "`de-Morganschen Regeln"' wurden nach dem englischen
Mathematiker \textsc{Augustus De Morgan}\footnote{\textborn $1806$ in
Madurai, Tamil Nadu, Indien - \textdied $1871$ in London, England}
benannt.

Als Abkürzung schreibt man statt $X_1 \cup X_2 \cup \dots \cup X_n$
(bzw.~$X_1 \cap X_2 \cap \dots \cap X_n$) einfach $\bigcup\limits_{i=1}^n X_i$
(bzw.~$\bigcap\limits_{i=1}^n X_i$). Möchte man alle Mengen $X_i$ mit
$i \in \N$ schneiden (bzw.~vereinigen), so schreibt man kurz
$\bigcap\limits_{i \in \N} X_i$ (bzw.~$\bigcup\limits_{i \in \N} X_i$).

\goodbreak

Oft benötigt man eine Verknüpfung von zwei Mengen, eine solche
Verknüpfung wird allgemein wie folgt definiert:

\begin{definition}["`Verknüpfung von Mengen"']
Seien $A$ und $B$ zwei Mengen und "`$\odot$"' eine beliebige
Verknüpfung zwischen den Elementen dieser Mengen, dann definieren wir
\begin{displaymath}
A \odot B \eqd \set{a \odot b \mid a \in A \text{ und } b \in B}.
\end{displaymath}
\end{definition}

\begin{example}
Die Menge $3\Z = \set{0, \pm 3, \pm 6, \pm 9, \dots}$ enthält alle
Vielfachen\footnote{Eigentlich müsste man statt $3\Z$ die Notation
$\set{3}\Z$ verwenden. Dies ist allerdings unüblich.} von $3$, damit
ist $3\Z + \set{1} = \set{1,\allowbreak 4,\allowbreak -2,\allowbreak
7,\allowbreak -5, 10, -8, \dots}$. Die Menge $3\Z + \set{1}$ schreibt
man kurz oft auch als $3\Z + 1$, wenn klar ist, was mit dieser
Abkürzung gemeint ist.
\end{example}

\special{pdf: out 4 << /Title
(Tupel (Vektoren) und das Kreuzprodukt) 
/Dest [ @thispage /FitH @ypos ] >>}
\subsubsection{Tupel (Vektoren) und das Kreuzprodukt}
Seien $A, A_1, \dots , A_n$ im folgenden Mengen, dann bezeichnet

\begin{itemize}
  % 
  \item $(\enu{a}{1}{n}) \eqd$ die Elemente $\enu{a}{1}{n}$ in genau dieser
  festgelegten \emph{Reihenfolge} und z.B.~$(3,2) \not= (2,3)$. Wir
  sprechen von einem $n$-Tupel\index{Tupel}\index{Tupel=$n$-Tupel}.
  % 
  \item $A_1 \times A_2 \times \dots \times
  A_n \eqd \set{(\enu{a}{1}{n}) \mid a_1 \in A_1, a_2 \in A_2, \dots
  ,a_n \in A_n }$ (Kreuzprodukt der Mengen $A_1, A_2, \dots ,
  A_n$)\index{Kreuzprodukt}\index{$\times$},
  %
  \item $A^n \eqd \underbrace{A \times A \times \dots \times
  A}_{n\text{-mal}}$ ($n$-faches Kreuzprodukt der Menge $A$)\index{$A^n$} und
  %
  \item speziell gilt $A^1 = \set{(a) \mid a \in A}$.
  %
\end{itemize}
Wir nennen $2$-Tupel auch \emph{Paare}\index{Paar}, $3$-Tupel
auch \dindex{Tripel}, $4$-Tupel auch \dindex{Quadrupel} und $5$-Tupel 
\dindex{Quintupel}. Bei $n$-Tupeln ist, im Gegensatz zu Mengen, eine 
Reihenfolge vorgegeben, d.h.~es gilt z.B.~immer $\set{a,b} = \set{b,a}$, aber 
im Allgemeinen $(a,b) \not= (b,a)$.

\begin{example}
Sei $A = \set{1, 2, 3}$ und $B = \set{a, b, c}$, dann bezeichnet das
Kreuzprodukt von $A$ und $B$ die Menge von Paaren $A \times B =
\set{(1,a), (1,b), (1,c), (2,a), (2,b), (2,c),\allowbreak (3,\allowbreak a), (3,b), (3,c)}$.
\end{example}

\special{pdf: out 4 << /Title 
(Die Anzahl von Elementen in Mengen) 
/Dest [ @thispage /FitH @ypos ] >>}
\subsubsection{Die Anzahl von Elementen in Mengen}
\label{cntSet}
Sei $A$ eine Menge, die endlich viele Elemente\footnote{Solche Mengen
werden als \dindex{endliche Mengen}\index{Menge!endliche} bezeichnet.}
enthält, dann ist
\begin{displaymath}
\cnt A \eqd \text{Anzahl der Elemente in der Menge $A$}.
\end{displaymath}
\noindent Beispielsweise ist $\cnt \set{4,7,9} = 3$. Mit dieser Definition gilt

\begin{itemize}
%
\item $\cnt(A^n) = (\cnt A)^n$\index{$\cnt$},
%
\item $\cnt \PowerSet{A} = 2^{\cnt A}$,
%
\item $\cnt A + \cnt B = \cnt(A \cup B) + \cnt (A \cap B)$ und
%
\item $\cnt A = \cnt (A \setminus B) + \cnt(A \cap B)$.
%
\end{itemize}

\special{pdf: out 3 << /Title 
(Relationen und Funktionen) 
/Dest [ @thispage /FitH @ypos ] >>}
\subsection{Relationen und Funktionen}

\special{pdf: out 4 << /Title 
(Eigenschaften von Relationen) 
/Dest [ @thispage /FitH @ypos ] >>}
\subsubsection{Eigenschaften von Relationen}
\label{PropRel}

Seien $\enu{A}{1}{n}$ beliebige Mengen, dann ist $R$ eine
\emph{$n$-stellige Relation}\index{Relation} \gdw 
\ $R \subseteq A_1 \times A_2 \times \dots \times A_n$. Eine
zweistellige Relation nennt man auch \dindex{binäre
Relation}\index{Relation!binär}. Oft werden auch Relationen
$R \subseteq A^n$ betrachtet, diese bezeichnet man dann als
$n$-stellige Relation über der Menge $A$.

\begin{definition}
Sei $R$ eine zweistellige Relation über $A$, dann ist $R$
\begin{itemize}
%
\item \dindex{reflexiv} \gdw \ $(a,a) \in R$ für alle $a \in A$,
%
\item \dindex{symmetrisch} \gdw \ aus $(a,b) \in R$ folgt $(b,a) \in R$,
%
\item \dindex{antisymmetrisch} \gdw \ aus $(a,b) \in R$ und $(b,a) \in R$ folgt $a =
b$,
%
\item \dindex{transitiv} \gdw \ aus $(a,b) \in R$ und $(b,c) \in R$ folgt $(a,c)
\in R$ und 
%
\item \dindex{linear} \gdw \ es gilt immer $(a,b) \in R$ oder $(b, a) \in R$.
%
\item Wir nennen $R$ eine \dindex{Halbordnung} \gdw $R$ ist reflexiv,
antisymmetrisch und transitiv,
%
\item eine \dindex{Ordnung} \gdw $R$ ist eine lineare Halbordnung und
%
\item eine \emph{Äquivalenzrelation}\index{Aquivalenzrelation=Äquivalenzrelation} 
\gdw $R$ reflexiv, transitiv und symmetrisch ist.
%
\end{itemize}
\end{definition}

\begin{example}
Die Teilmengenrelation "`$\subseteq$"' auf allen Teilmengen von $\Z$ ist
eine Halbordnung, aber keine Ordnung. Wir schreiben $a \equiv b \mod
n$, falls es eine ganze Zahl $q$ gibt, für die $a - b = q n$ gilt. Für $n \ge 2$
ist die Relation $R_n(a,b) \eqd \set{(a,b) \mid a \equiv b \mod n} \subseteq
\Z^2$ eine Äquivalenzrelation.
\end{example}

\special{pdf: out 4 << /Title 
(Eigenschaften von Funktionen) 
/Dest [ @thispage /FitH @ypos ] >>}
\subsubsection{Eigenschaften von Funktionen}
\label{PropFunc}
Seien $A$ und $B$ beliebige Mengen. $f$ ist eine \dindex{Funktion} von $A$ nach
$B$ (Schreibweise: $f \colon A \rightarrow B$) \gdw \ $f \subseteq A \times
B$ und für jedes $a \in A$ gibt es \emph{höchstens} ein $b \in B$ mit
$(a, b) \in f$. Ist also $(a,b) \in f$, so schreibt man $f(a) =
b$. Ebenfalls gebrächlich ist die Notation $a \mapsto b$.

\begin{remark}
Unsere Definition von Funktion umfasst auch mehrstellige
Funktionen. Seien $C$ und $B$ Mengen und $A = C^n$ das $n$-fache
Kreuzprodukt von $C$. Die Funktion $f \colon A \rightarrow B$ ist dann
eine $n$-stellige Funktion, denn sie bildet $n$-Tupel aus $C^n$ auf Elemente
aus $B$ ab.
\end{remark}

\begin{definition}
Sei $f$ eine $n$-stellige Funktion. Möchte man die Funktion $f$
benutzen, aber keine Namen für die Argumente vergeben, so
schreibt man auch 
\begin{displaymath}
f(\underbrace{\cdot, \cdot, \ldots , \cdot}_{\text{$n$-mal}})
\end{displaymath}
Ist also der Namen des Arguments einer einstelligen Funktion $g(x)$
für eine Betrachtung unwichtig, so kann man
$g(\cdot)$ \index{$f(\cdot)$} schreiben, um anzudeuten, dass $g$
einstellig ist, ohne dies weiter zu erwähnen.
\end{definition}

Sei nun $R \subseteq A_1 \times A_2 \times \dots \times A_n$ eine
$n$-stellige Relation, dann definieren wir $P^n_R \colon A_1 \times
A_2 \times \dots \times A_n \rightarrow \set{0,1}$ wie folgt:

\begin{displaymath}
P^n_R(\enu{x}{1}{n}) \eqd 
\left\{
\begin{array}{rl}
1,& \text{ falls $(\enu{x}{1}{n}) \in R$}\\
0,& \text{ sonst} 
\end{array}
\right.
\end{displaymath}
Eine solche ($n$-stellige) Funktion, die "`anzeigt"', ob ein Element 
aus $A_1 \times A_2 \times \dots \times A_n$ entweder zu $R$ gehört 
oder nicht, nennt man ($n$-stelliges) \dindex{Prädikat}.

\begin{example}
Sei $\mathbb{P} \eqd \set{n \in \N \mid \text{$n$ ist Primzahl}}$, dann
ist $\mathbb{P}$ eine $1$-stellige Relation über den natürlichen Zahlen. 
Das Prädikat $P^1_{\mathbb{P}}(n)$ liefert für eine natürliche Zahl
$n$ genau dann $1$, wenn $n$ eine Primzahl ist.
\end{example}

Ist für ein Prädikat $P^n_R$ sowohl die Relation $R$ als auch die
Stelligkeit $n$ aus dem Kontext klar, dann schreibt man auch kurz $P$
oder verwendet das Relationensymbol $R$ als Notation für das Prädikat
$P^n_R$. 

\bigskip

\noindent Nun legen wir zwei spezielle Funktionen fest, die oft sehr
hilfreich sind:
\begin{definition}
\label{floorceil}
Sei $\alpha \in \R$ eine beliebige reelle Zahl, dann gilt
\begin{itemize}
%
\item $\lceil x \rceil \eqd \text{die kleinste ganze Zahl, die größer
oder gleich $\alpha$ ist}$ ($\triangleq$ "`Aufrunden"') \index{$\lceil \cdot \rceil$}
%
\item $\lfloor x \rfloor \eqd \text{die größte ganze Zahl, die kleiner
oder gleich $\alpha$ ist}$ ($\triangleq$ "`Abrunden"') \index{$\lfloor \cdot \rfloor$}
%
\end{itemize}
\end{definition}

\begin{definition}
Für eine beliebige Funktion $f$ legen wir fest:
\begin{itemize}
%
\item Der \dindex{Definitionsbereich} von $f$ ist $D_f \eqd
\set{a \mid \text{es gibt ein $b$ mit $f(a) = b$}}$.
%
\item Der \dindex{Wertebereich} von $f$ ist $W_f \eqd
\set{b \mid \text{es gibt ein $a$ mit $f(a) = b$}}$.
%
\item Die Funktion $f \colon A \rightarrow B$ ist \dindex{total} \gdw $D_f
= A$.
% 
\item Die Funktion $f \colon A \rightarrow B$ heißt \dindex{surjektiv} \gdw $W_f = B$.
%
\item Die Funktion $f$ heißt \dindex{injektiv} (oder
eineindeutig\footnote{Achtung: Dieser Begriff wird manchmal
unterschiedlich, je nach Autor, in den Bedeutungen "`bijektiv"' oder
"`injektiv"' verwendet.}) \gdw\ immer wenn $f(a_1)\allowbreak =
f(a_2)$ gilt auch $a_1 = a_2$.
%
\item Die Funktion $f$ heißt \dindex{bijektiv} \gdw $f$ ist injektiv und surjektiv.
\end{itemize}
\end{definition}
Mit Hilfe der Kontraposition (siehe Abschnitt \ref{KontraPos}) kann
man für die Injektivität alternativ auch zeigen, dass immer wenn
$a_1 \not= a_2$, dann muss auch $f(a_1) \not= f(a_2)$ gelten.

\begin{example}
Sei die Funktion $f \colon \N \rightarrow \Z$ durch $f(n) = (-1)^n
\lceil \frac{n}{2} \rceil$ gegeben. Die Funktion $f$ ist surjektiv,
denn $f(0) = 0, f(1) = -1, f(2) = 1, f(3) = -2, f(4) = 2, \dots$, d.h.~die 
ungeraden natürlichen Zahlen werden auf die negativen ganzen Zahlen 
abgebildet, die geraden Zahlen aus $\N$ werden auf die positiven
ganzen Zahlen abgebildet und deshalb ist $W_f = \Z$.

Weiterhin ist $f$ auch injektiv, denn aus\footnote{Für die Definition
der Funktion $\lceil \cdot \rceil$ siehe Definition \ref{floorceil}.}
$(-1)^{a_1} \lceil \frac{a_1}{2} \rceil = (-1)^{a_2}
\lceil \frac{a_2}{2} \rceil$ folgt, dass entweder $a_1$ und $a_2$
gerade oder $a_1$ und $a_2$ ungerade, denn sonst würden auf der linken
und rechten Seite der Gleichung unterschiedliche Vorzeichen
auftreten. Ist $a_1$ gerade und $a_2$ gerade, dann gilt
$\lceil \frac{a_1}{2} \rceil = \lceil \frac{a_2}{2} \rceil$ und auch
$a_1 = a_2$. Sind $a_1$ und $a_2$ ungerade, dann gilt
$-\lceil \frac{a_1}{2} \rceil = -\lceil \frac{a_2}{2} \rceil$, woraus
auch folgt, dass $a_1 = a_2$.
%
Damit ist die Funktion $f$ bijektiv. Weiterhin ist $f$ auch total,
d.h.~$D_f = \N$.
\end{example}

\ifdiscretemath
%
% Remove the subsubsection
%
\else

\special{pdf: out 4 << /Title 
(Permutationen) 
/Dest [ @thispage /FitH @ypos ] >>}
\subsubsection{Permutationen}
\label{Permutationen}
Eine bijektive Funktion $\pi \colon S \rightarrow S$
heißt \dindex{Permutation}\index{$\pi$}. Das bedeutet, dass die
Funktion $\pi$ Elemente aus $S$ wieder auf Elemente aus $S$ abbildet,
wobei für jedes $b \in S$ ein $a \in S$ mit $f(a) = b$ existiert
(Surjektivität) und falls $f(a_1) = f(a_2)$ gilt, dann ist $a_1 = a_2$
(Injektivität). Besonders häufig werden in der
Informatik \emph{Permutationen von endlichen Mengen} benötigt.

Sei nun $S = \set{\range{1}{n}}$ eine endliche Menge und
$\pi \colon \set{\range{1}{n}} \rightarrow \set{\range{1}{n}}$ eine
Permutation. Permutationen dieser Art kann man sehr anschaulich mit
Hilfe einer Matrix aufschreiben:

\begin{displaymath}
\pi = \left( 
\begin{array}{cccc}
1 & 2 & \dots & n\\
\pi(1) & \pi(2) & \dots & \pi(n)
\end{array}
\right)
\end{displaymath}
Durch diese Notation wird klar, dass das Element $1$ der Menge $S$
durch das Element $\pi(1)$ ersetzt wird, das Element $2$ wird mit
$\pi(2)$ vertauscht und allgemein das Element $i$ durch $\pi(i)$ für
$1 \le i \le n$. In der zweiten Zeile dieser Matrixnotation findet
sich also \emph{jedes} (Surjektivität) Element der Menge $S$
genau \emph{einmal} (Injektivität).

\begin{example}
Sei $S = \set{\range{1}{3}}$ eine Menge mit drei Elementen. Dann gibt
es, wie man ausprobieren kann, genau $6$ Permutationen von $S$:

\begin{displaymath}
\begin{array}{rlrlrl}
\pi_1 &= \left( 
\begin{array}{ccc}
1 & 2 & 3\\
1 & 2 & 3
\end{array}
\right)
&
\pi_2 &= \left( 
\begin{array}{ccc}
1 & 2 & 3\\
1 & 3 & 2
\end{array}
\right)
&
\pi_3 &= \left( 
\begin{array}{ccc}
1 & 2 & 3\\
2 & 1 & 3
\end{array}
\right)\\[\bigskipamount]
%
\pi_4 &= \left( 
\begin{array}{ccc}
1 & 2 & 3\\
2 & 3 & 1
\end{array}
\right)
&
\pi_5 &= \left( 
\begin{array}{ccc}
1 & 2 & 3\\
3 & 1 & 2
\end{array}
\right)
&
\pi_6 &= \left( 
\begin{array}{ccc}
1 & 2 & 3\\
3 & 2 & 1
\end{array}
\right)\\
\end{array}
\end{displaymath}
\end{example}

\begin{theorem}
Sei $S$ eine endliche Menge mit $n = |S|$, dann gibt es genau $n!$
(Fakultät) verschiedene Permutationen von $S$.
\end{theorem}

\begin{proof}
Jede Permutation $\pi$ der Menge $S$ von $n$ Elementen kann als Matrix
der Form
\begin{displaymath}
\pi = \left( 
\begin{array}{cccc}
1 & 2 & \dots & n\\
\pi(1) & \pi(2) & \dots & \pi(n)
\end{array}
\right)
\end{displaymath}
aufgeschrieben werden. Damit ergibt sich die Anzahl der Permutationen
von $S$ durch die Anzahl der verschiedenen zweiten Zeilen solcher
Matrizen. In jeder solchen Zeile muss jedes der $n$ Elemente von $S$
genau einmal vorkommen, da $\pi$ eine bijektive Abbildung ist,
d.h.~wir haben für die erste Position der zweiten Zeile der
Matrixdarstellung genau $n$ verschiedene Möglichkeiten, für die zweite
Position noch $n - 1$ und für die dritte noch $n-2$. Für die $n$-te
Position bleibt nur noch $1$ mögliches Element aus $S$
übrig\footnote{Dies kann man sich auch als die Anzahl der
verschiedenen Möglichkeiten vorstellen, die bestehen, wenn man aus
einer Urne mit $n$ numerierten Kugeln alle Kugeln \emph{ohne}
Zurücklegen nacheinander zieht.}. Zusammengenommen haben wir also
$n \cdot (n - 1) \cdot (n - 2) \cdot (n - 3) \multdots 2 \cdot
1 = n!$ verschiedene mögliche Permutationen der Menge $S$.
\qed
\end{proof}

\fi
