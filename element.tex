\ifpdf
\special{pdf: out 2 << /Title 
(Elementare Begriffe und Schreibweisen) 
/Dest [ @thispage /FitH @ypos ] >>}
\fi
\section{Elementare Begriffe und Schreibweisen}

\ifpdf
\special{pdf: out 3 << /Title 
(Mengen) 
/Dest [ @thispage /FitH @ypos ] >>}
\fi
\subsection{Mengen}

\ifpdf
\special{pdf: out 4 << /Title 
(Elementbeziehung und Enthaltenseinsrelation) 
/Dest [ @thispage /FitH @ypos ] >>}
\fi
\subsubsection{Die Elementbeziehung und die Enthaltenseinsrelation}

\begin{definition}
\label{InclSet}
\mbox{}
\begin{itemize}
%
\item $a \in M$ \iffd\ $a$ ist ein Element der Menge $M$
%
\item $a \not\in M$ \iffd\ $a$ ist kein Element der Menge $M$
%
\item $M \subseteq N$ \iffd\ aus $a \in M$ folgt $a \in N$ ($M$ ist
Teilmenge von $N$)
%
\item $M \not\subseteq N$ \iffd\ es gilt nicht $M \subseteq N$ ($M$ ist
keine Teilmenge von $N$)
%
\item $M \subset N$ \iffd\ es gilt $M \subseteq N$ und $M \not= N$ ($M$ ist
echte Teilmenge von $N$)
%
\end{itemize}
\end{definition}


\ifpdf
\special{pdf: out 4 << /Title 
(Definition spezieller Mengen) 
/Dest [ @thispage /FitH @ypos ] >>}
\fi
\subsubsection{Definition spezieller Mengen}

Spezielle Mengen k�nnen auf verschiedene Art und Weise definiert werden:

\begin{itemize}
%
\item durch Angabe von Elementen:\ so ist $\set{\enu{a}{1}{n}}$ die Menge,
die aus den Elementen $\enu{a}{1}{n}$ besteht, oder
%
\item durch eine Eigenschaft $E$:\ dabei ist $\set{a \colon E(a)}$ die Menge
aller Elemente $a$, die die Eigenschaft $E$ besitzen.
%
\end{itemize}

\begin{example}
\mbox{}

\begin{itemize}
%
\item Mengen, die durch die Angabe von Elementen definiert sind:
\begin{itemize}
%
\item $\set{0,1}$
%
\item $\set{2, 3, 5, 7, 11, 13, 17, 19, 13, 19, 31, 37}$
%
\item $\N \eqd \set{0, 1, 2, 3, 4, 5, 6, 7, 8, \dots}$ (Menge der nat�rlichen Zahlen)
%
\item $\Z \eqd \set{\dots, -4, -3, -2, -1, 0, 1, 2, 3, 4, \dots}$ (Menge der ganzen Zahlen)
%
\end{itemize}
%
\item Mengen, die durch eine Eigenschaft $E$ definiert sind:
\begin{itemize}
%
\item $\set{n \colon n \in \N \text{ und $n$ ist durch $3$ teilbar}}$
%
\item $\set{n \colon n \in \N \text{ und $n$ ist Primzahl und $n \le 40$}}$
%
\item $\emptyset \eqd \set{a \colon a \not= a}$ (die leere Menge)
%
\end{itemize}
\end{itemize}
Aus Definition \ref{InclSet} ergibt sich, dass die leere Menge
$\emptyset$ Teilmenge jeder Menge ist.
\end{example}


\ifpdf
\special{pdf: out 4 << /Title 
(Operationen auf Mengen) 
/Dest [ @thispage /FitH @ypos ] >>}
\fi
\subsubsection{Operationen auf Mengen}

\begin{definition}
\label{OpSet}
\mbox{}
\begin{itemize}
%
\item $A \cap B \eqd \set{a \colon a \in A \text{ und } a \in B}$
(Schnitt von $A$ und $B$)
%
\item $A \cup B \eqd \set{a \colon a \in A \text{ oder } a \in B}$
(Vereinigung von $A$ und $B$)
%
\item $A \backslash B \eqd \set{a \colon a \in A \text{ und } a
\not\in B}$ (Differenz von $A$ und $B$)
%
\item $\overline{A} \eqd M \backslash A$ (Komplement von $A$ 
bez�glich einer festen Grundmenge $M$)
%
\item $\PowerSet{A} \eqd \set{B \colon B \subseteq A}$ (Potenzmenge von
$A$)
%
\end{itemize}

\begin{example}
Sei $A = \set{2, 3, 5, 7}$ und $B = \set{1, 2, 4 , 6}$, dann ist $A
\cap B = \set{2}$, $A \cup B = \set{1, 2, 3, 4, 5, 6, \allowbreak 7}$ und $A
\backslash B = \set{3, 5, 7}$. W�hlen wir als Grundmenge die
nat�rlichen Zahlen, also $M = \N$, dann ist $\overline{A} = \set{n
\colon n
\in \N \text{ und } a \not= 2 \text{ und } a \not= 3 \text{ und } a
\not= 5 \text{ und } a \not= 7} = \set{1, 4, 6, 8, 9, 10, 11,
\dots}$. Als Potenzmenge der Menge $A$ ergibt sich die Menge von Mengen
$\set{\emptyset, \set{2},\set{3}, \set{5}, \set{7}, \set{2,3},
\set{2,5}, \allowbreak \set{2,7}, \allowbreak \set{3,5}, \allowbreak \set{3,7},
\allowbreak \set{5,7}, \set{2,3,5},
\set{2,3,7},\set{2,5,7},\set{3,5,7},\set{2,3,5,7}}$.  \end{example}
\end{definition}

\ifpdf
\special{pdf: out 4 << /Title 
(Gesetze f�r Mengenoperationen) 
/Dest [ @thispage /FitH @ypos ] >>}
\fi
\subsubsection{Gesetze f�r Mengenoperationen}
\begin{displaymath}
\begin{array}{rcll}
A \cap B &=& B \cap A & \text{Kommutativgesetz f�r den Schnitt}\\
A \cup B &=& B \cup A & \text{Kommutativgesetz f�r die Vereinigung}\\
A \cap (B \cap C) &=& (A \cap B) \cap C & \text{Assoziativgesetz f�r
den Schnitt}\\
A \cup (B \cup C) &=& (A \cup B) \cup C & \text{Assoziativgesetz f�r
die Vereinigung}\\
A \cap (B \cup C) &=& (A \cap B) \cup (A \cap C) & \text{Distributivgesetz}\\
A \cup (B \cap C) &=& (A \cup B) \cap (A \cup C) & \text{Distributivgesetz}\\
A \cap A &=& A & \text{Duplizit�tsgesetz f�r den Schnitt}\\
A \cup A &=& A & \text{Duplizit�tsgesetz f�r die Vereinigung}\\
A \cap (A \cup B) &=& A & \text{Absorptionsgesetz}\\
A \cup (A \cap B) &=& A & \text{Absorptionsgesetz}\\
\overline{A \cap B} &=& (\overline{A} \cup \overline{B}) &
\text{de-Morgansche Regel}\\
\overline{A \cup B} &=& (\overline{A} \cap \overline{B}) &
\text{de-Morgansche Regel}\\
\overline{\overline{A}} &=& A & \text{Gesetz des doppelten Komplements}
\end{array}
\end{displaymath}

\ifpdf
\special{pdf: out 4 << /Title
(Tupel (Vektoren) und das Kreuzprodukt) 
/Dest [ @thispage /FitH @ypos ] >>}
\fi
\subsubsection{Tupel (Vektoren) und das Kreuzprodukt}

Seien $A, A_1, \dots , A_n$ im folgenden Mengen, dann bezeichnet

\begin{itemize}
%
\item $(\enu{a}{1}{n}) \eqd$ die Elemente $\enu{a}{1}{n}$ in
dieser festgelegten \emph{Reihenfolge}. Wir sprechen von einem
$n$-Tupel.
%
\item $A_1 \times A_2 \times \dots \times A_n \eqd
\set{(\enu{a}{1}{n}) \colon a_1 \in A_1, a_2 \in A_2, \dots ,a_n \in
A_n }$ (Kreuzprodukt der Mengen $A_1, A_2, \dots , A_n$),
%
\item $A^n \eqd \underbrace{A \times A \times \dots \times
A}_{n\text{-mal}}$ ($n$-faches Kreuzprodukt der Menge $A$) und
%
\item speziell gilt $A^1 = \set{(a) \colon a \in A}$.
%
\end{itemize}
Wir nennen $2$-Tupel auch Paare, $3$-Tupel auch Tripel, $4$-Tupel auch
Quadrupel und $5$-Tupel Quintupel. Bei $n$-Tupeln ist, im Gegensatz zu
Mengen, eine Reihenfolge vorgegeben, d.h.~es gilt immer $\set{a,b} =
\set{b,a}$, aber im Allgemeinen $(a,b) \not= (b,a)$.

\begin{example}
Sei $A = \set{1, 2, 3}$ und $B = \set{a, b, c}$, dann bezeichnet das
Kreuzprodukt von $A$ und $B$ die Menge von Paaren $A \times B =
\set{(1,a), (1,b), (1,c), (2,a), (2,b), (2,c), (3,a), (3,b), (3,c)}$.
\end{example}


\ifpdf
\special{pdf: out 4 << /Title 
(Die Anzahl von Elementen in Mengen) 
/Dest [ @thispage /FitH @ypos ] >>}
\fi
\subsubsection{Die Anzahl von Elementen in Mengen}
\label{cntSet}

Sei $A$ eine Menge, die endlich viele Elemente\footnote{Solche Mengen
werden als \emph{endliche Mengen} bezeichnet.} enth�lt, dann ist

\begin{displaymath}
\cnt(A) \eqd \text{Anzahl der Elemente in der Menge $A$}.
\end{displaymath}

\noindent Es gilt
\begin{itemize}
%
\item $\cnt(A^n) = (\cnt A)^n$
%
\item $\cnt \PowerSet{A} = 2^{\cnt A}$
%
\item $\cnt A + \cnt B = \cnt(A \cup B) + \cnt (A \cap B)$
%
\item $\cnt A = \cnt (A \backslash B) + \cnt(A \cap B)$
%
\end{itemize}

\ifpdf
\special{pdf: out 3 << /Title 
(Funktionen und Relationen) 
/Dest [ @thispage /FitH @ypos ] >>}
\fi
\subsection{Funktionen und Relationen}

\ifpdf
\special{pdf: out 4 << /Title 
(Eigenschaften von Funktionen) 
/Dest [ @thispage /FitH @ypos ] >>}
\fi
\subsubsection{Eigenschaften von Funktionen}

Seien $A$ und $B$ beliebige Mengen. $f$ ist eine \emph{Funktion} von $A$ nach
$B$ (Schreibweise: $f \colon A \mapsto B$) \iffd \ $f \subseteq A \times
B$ und f�r jedes $a \in A$ gibt es \emph{h�chstens} ein $b \in B$ mit
$(a, b) \in f$. Ist $(a,b) \in f$, so schreibt man $f(a) = b$.

\begin{enumerate}
%
\item Der \emph{Definitionsbereich} einer Funktion $f$ ist $D_f \eqd
\set{a \colon \text{es existiert ein $b$ mit $f(a) = b$}}$.
%
\item Der \emph{Wertebereich} einer Funktion $f$ ist $W_f \eqd
\set{b \colon \text{es existiert ein $a$ mit $f(a) = b$}}$.
%
\item Eine Funktion $f \colon A \mapsto B$ ist \emph{total} \iffd $D_f
= A$.
% 
\item Eine Funktion $f \colon A \mapsto B$ hei�t \emph{surjektiv} \iffd $W_f = B$.
%
\item Eine Funktion hei�t \emph{injektiv} (oder eineindeutig) \iffd
\ aus $f(a_1) = f(a_2)$ folgt $a_1 = a_2$.
%
\item Eine Funktion hei�t \emph{bijektiv} \iffd $f$ ist injektiv und surjektiv.
\end{enumerate}

\begin{example}
Die Funktion $f \colon \N \mapsto \Z$ gegeben durch $f(n) = (-1)^n
\lceil \frac{n}{2} \rceil$ ist eine Bijektion, denn $f(0) = 0, f(1) = -1
f(2) = 1, f(3) = -2, f(4) = 2, \dots$. Dabei bezeichnen wir mit $\lceil
x \rceil \eqd \text{die kleinste ganze Zahl, die gr��er oder gleich
$x$ ist}$ ($=$ "`Aufrunden"'). Weiterhin ist $f$ auch total. 
\end{example}

\ifpdf
\special{pdf: out 4 << /Title 
(Eigenschaften von Relationen) 
/Dest [ @thispage /FitH @ypos ] >>}
\fi
\subsubsection{Eigenschaften von Relationen}

Seien $\enu{A}{1}{n}$ beliebige Mengen, dann ist $R$ eine
\emph{$n$-stellige Relation} \iffd \ $R \subseteq A_1 \times A_2 \times \dots
\times A_n$. Oft werden auch Relationen $R \subseteq A^n$
betrachtet, diese bezeichnet man dann als $n$-stellige Relation �ber
der Menge $A$.

\noindent  Sei $R$ eine zweistellige Relation �ber $A$, dann ist $R$
\begin{itemize}
%
\item \emph{reflexiv} \iffd \ $(a,a) \in R$ f�r alle $a \in A$,
%
\item \emph{symmetrisch} \iffd \ aus $(a,b) \in R$ folgt $(b,a) \in R$,
%
\item \emph{antisymmetrisch} \iffd \ aus $(a,b) \in R$ und $(b,a) \in R$ folgt $a =
b$,
%
\item \emph{transitiv} \iffd \ aus $(a,b) \in R$ und $(b,c) \in R$ folgt $(a,c)
\in R$ und 
%
\item \emph{linear} \iffd \ es gilt immer $(a,b) \in R$ oder $(b, a) \in R$.
%
\item Wir nennen $R$ eine \emph{Halbordnung} \iffd $R$ ist reflexiv,
antisymmetrisch und transitiv,
%
\item eine \emph{Ordnung} \iffd $R$ ist eine lineare Halbordnung und
%
\item eine \emph{�quivalenzrelation} \iffd $R$ reflexiv, transitiv und
symmetrisch ist.
%
\end{itemize}

\begin{example}
Die Teilmengenrelation "`$\subseteq$"' auf allen Teilmengen von $\Z$ ist
eine Halbordnung, aber keine Ordnung. Wir schreiben $a \equiv b \mod
n$, falls es eine ganze Zahl $q$ gibt, f�r die $a - b = q n$ gilt. F�r $n \ge 2$
ist die Relation $R_n(a,b) \eqd \set{(a,b) \colon a \equiv b \mod n} \subseteq
\Z^2$ eine �quivalenzrelation.
\end{example}
