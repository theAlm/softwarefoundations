\section{Scheduability Analysis Modeling}

% refernces for this section

%@misc{KBK,
%author = {R. Kaiser and K. Beckmann and R. Kröger},
%title = {Echtzeitplanung},
%howpublished = {Handouts},
%note = {found online at \url{https://www.cs.hs-rm.de/~kaiser/1919_ezv/%6_Scheduling-handout.pdf} acessed at the January 07th 2020}, 
%}

%@book{L,
%title = {Real-time systems},
%author= {J. W.S. Liu},
%publisher = {Prentice-Hall, Inc.},
%isbn = {0-13-099651-3}, 
%year = {2000},
%}


%@phdthesis{K,
%  author       = {R. Kasier}, 
%  title        = {Virtualisierung von Mehrprozessorsystemen mit
%Echtzeitanwendungen},
%  school       = {Universität Koblenz-Landau},
%  year         = 2009,
%  month        = 2,
%  day          = 11,
% howpublished = {PHD Thesis},
%}

%@phdthesis{B,
%  author       = {G. Bollella}, 
%  title        = {Slotted Priorities: Supporting Real-Time Computing Within General-Purpose Operating Systems},
%  school       = {Chapel Hill},
%  year         = 1997,
% howpublished = {PHD Thesis},
%}



In this section we want to investigate the encapsulated EDF-scheduler (earliest deadline first scheduler) from the schedulig procedure by Kaiser \cite{K}. It is homogenous multiprocessor systems comprising $m\in \mathbb{N}$ processors.\\

\subsection{Taxanomie of Multiprocessor Scheduling Algorithms}

Let's recap \glspl{real-time system} as given in \cite{KBK}.
And compare it with the calssification given in \cite[section 2.3]{DB2011}.
The {\itshape process} or {\itshape task} is defined as an sequential execution of a {\itshape program}  or {\itshape job} on a processor.
The execution ends after a \emph{finite number} of steps.
Therefore the execution corresponds to a finite execution of machine commands and is not separable.\\

\begin{remark}
	\label{remark:problems}
	Mutiprocessor scheduling can be viewed as attempting  to solve two problems.
	\begin{enumerate}[label=(\roman*)]
		\item The \emph{allocation problem}, or on which processor the task should execute.
		\label{remark:allocationproblem}
		\item The \emph{priority problem}, or in what order with respect to jobs of tasks each job should be executed.
		\label{remark:priorityproblem}
	\end{enumerate}
\end{remark}

\begin{definition}
	 A process is called periodic if it should be restated after a certain time called the period.
	 \begin{enumerate}[label=(\roman*)] 
	\item Otherwise a process is called aperiodic or sporadic.
	\item Furthermore, whenever as process is said to be non-preemptive the execution may not be interrupted between the beginning and ending of the process. 
		A process is called preemptive if it may be interrupted after any instruction.
	\end{enumerate}
	\end{definition}  
 
The slotted priority modell is from Bollela \cite{B97} with  sporadic and periodic processes (\cite{K}).
Due to \cite{B97} the major requirement is said to be as in the following:

\begin{quote}
	`If a system itselves the execution of a real-time and non-real-time thread in alternate intervals the intervals in which real-time threads execute are scheduled to be in every $l$ time unit, then it must be ensured that the interval begin at time $t$ where $kl \leq t \leq kl+\epsilon \quad \forall
 \epsilon \geq 0$'.
\end{quote}

Roughly speaking there should be a covering of the time $t$ with a diameter o
of $\epsilon$ for each positive $\epsilon$.
Moreover there is this requirement labeld by  $\mathcal{B}$ \cite{B97}.
\begin{quote}
`For $L>\epsilon$ (for a suitable $\epsilon$) for which the real-time thread schedule has asserted a real-time thread $\tau$ to be expectet on the CPU, there must be a function of the method by which the minimum number of CPU cycles available to execute the instructions of $\tau$ can be determined.'
\end{quote}
But it is apriory not clear what this number $L$ might be and how this CPU-cycle number is determined.


\paragraph*{Encapsulation of a scheduled process}
 
The scheduler found in \cite{K} is going to be classified for the scope of implementation. 
We follow the paper \cite{DB2011} which gives a survey of hard real-time systems.
It is an encapsulated priority based (see Remark \ref{remark:problems} \ref{remark:priorityproblem}), fixed job priority,   EDF (earlies-deadline-first)) process with scheduling inside the representative process.

\begin{definition}
Let      
\begin{align}
 & \delta_p \in \mathbb{N} \quad \text{be a periodic disruptive process and}\\
 	&\delta_s \in  \mathbb{N} \quad \text{be an asporadic disruptive process.}\\
	 &\text{Let} \quad P := \{1, \cdots, n \} \subset \mathbb{N}^n $\\
	 $\quad \text{be a job, a set of disruptable processes with fixed priority order ascading.}$  \\
 	$&\text{Let} \quad  \delta p_i \in \mathbb{N}, \quad \text{for} \quad i = 1,..,n \quad  \text{denote the period and}  \\
 	&\delta e_i \in \mathbb{N} \quad \text{for} \quad  i = 1,..,n \quad  \text{denote the execution time of a proces}.  
\end{align}   
\end{definition}

Moreover, we define the slotted priority modell by Bollela with Kaiser's sporadic and periodic disruptive process with a discrete time in contrast to these authors.
This chioce is due to the real-time kernel from \cite{PROSA_schedubility_analysis} and the real-time kernel model as in \cite[chp. 5.3]{B97} and the exclusion model \cite[p.12]{B97}.

\begin{definition}
	Time is $\mathbb{N}$.
\end{definition}


\begin{definition}
	Utalization of the i-th process is given by $U(i) = \frac{\delta e_i}{\delta p_i}$
\end{definition}
\begin{lemma}
$U(i)$ is well-defined.
\end{lemma}
\begin{proof}
It is the quotient of two positive numbers.
\end{proof}

\begin{definition}
The utilization is given by $U(P)= \sum_{i=1}^n \frac{\delta e_i}{\delta p_i}$
\end{definition}

According to \cite{B97} due to establishing the exclusion model we denote  a scheduling function by

\begin{definition}
	\begin{align}
	\sigma: \mathbb{N}^n \times \mathbb{N}^n &\longrightarrow \mathbb{N} \\
	(\delta ,\delta e_i)  \mapsto  
	\end{align}
\end{definition}

a scheduler satisfying an EDF-regulation \ref{}. TODO: Add this condition.

\begin{equation}
U(P) = \sum\limits_{i=1}^n \frac{\delta e_i}{\delta p_i} \leq \frac{\Delta e_{sv}}{\Delta p_{sv}}
\end{equation}

\begin{definition}
Let $P_i\in \mathcal{P}$ be a process and 
\begin{equation}
P^j_i = ( \delta p_i, r_i) \in \mathbb{N}_o^2,  
\end{equation}
where $\delta p_i$ is called the period and $r_i$ the residuum. 
\end{definition}
Furthermore, time controlled process allocation due to \cite{B} and \cite[p. 34]{K} is defined as in the following.

\begin{definition}
Let $\sigma: \mathbb{N} \times \mathbb{N}_0 \rightarrow \mathbb{N}_0$ be a function called the time schedule or implementation plan.
\begin{enumerate}
\item $\forall t \in \mathbb{R}^+$ we have $\sigma(t)\in \{1,2,...,n\}$
\item \begin{equation}
		\sigma(t) =
		\begin{cases}
			k > 0 & \text{we say the process $P_k$ is working at time $t$.}\\
			0 & \text{the process is not working.}\\
			  &\quad  \text{Then the idele-process$i$ of the operating system is working.}
		\end{cases}       
\end{equation}
\end{enumerate}
\end{definition}

A depute process is given by $ N = \{1, \cdots, n\}\subset \mathbb{N}$. 

Let $P \subset \{1, \cdots, n\} \subset S$.
$S$ is executed in a periodically repeated timeslot such that $\delta p_{sv}$ is the periodlength and $\delta e_{sv}$ is the size of the time slot.
The remaining period time is contains the  disruptive process $\delta_p$ and $\delta_s$.

We are writing
\begin{align}
	\Delta p_{sv} &\text{ for the period length} \\
	\Delta e_s &\text{ for the executiontime of the sporatic process and}\\ 
	\Delta e_p &\text{ for the executiontime of a periodic task.}
\end{align}

\paragraph{The slotted priority model by Bollela}

Within this model we are having a two-staged procces hirachie. 
The global scheduler is purly time-controlled.
Its' period is given by $\Delta p_{glob}$ and its's to execution task time slot is still to remain $\Delta e_{time}$.
The representative process is the local scheduler. 
It is non interruptable, periodic disruptive process.
We have
\begin{align}
\Delta e_p &:= \Delta p_{glob} - \Delta e\\
\Delta p_{p} &:= \Delta p_{glob}, 
\end{align}
where $\Delta p_{p}$ denotes the period of the desruptive process.

\paragraph{The sporadic disruptive process}
As in \cite{K} the sporadic des
ptive process with cummulated calculation time $\Delta e$ satisfies
\begin{equation}
\Delta p_{p} \leq \lvert{\Delta p_{glob}} \rvert. 
\end{equation}
With the above assumptions we have

\begin{theorem}
\end{theorem}

\paragraph{Model of a sporadic disruptive process}
We say we are talking of a solotted priority model if the 

\begin{remark}
Note that $\sigma$ is a step-function. 
\end{remark}

\begin{lemma}
 $\sigma$ is well-defined.
\end{lemma}
\begin{proof}
Recall the definition of a map \ref{}. We have:
\end{proof}

\paragraph{Related Research Areas}
Besides hard real-time scheduling for homogenous multiprocessor systems there are related ares of research. The these areas are in a  close releation to multi-processor sytems.
\begin{itemize}
\item Worst-case execution time (WCET) an alysis
\item Network/bus scheduling
\item Memory architectures
\item scheduling of uniform and heterogeneous procesors
\item operating systems
\item Power consuption and dissipaation
\item scheduling tasks with soft real-time constraints, and
\item Non-real-time ussues such as load balancing.
\end{itemize}
Some of these research areas are more or less related to muti-processor systems. Some like WCET are very close to multi-processor systems.
Task eexctution time caused by contention of hardware resources due to paralles exectution is a scheduling effect or a WCET \cite[Sec. 10.2]{DB2011}.
 
\subsection{The Akra-Bazzi Theorem}
\label{subsection:The Akra-Bazzi Theorem}
The optimality of algorithms is a very important researche area in many fields.
One may ask if an algorithm was choosen optimal to solve the given problem.
But first of all, we have to determine the problem carefully. And in which sense optimality is required. There are some theorems which determine an straight-forward solution to this optimality problem. A well known-example is application of the master-theorem in search-algorithms e.g. \cite {}TODO: Add reference, master-theorem is linked on wikipedia.\\
For example there is this very powerfull theorem know as Akra-Bazzi-theorem \cite{AB98}.
It is a general solution for a linear devide-and-conquer recurrences of the of the form

 \begin{equation}
 \label{eq:linearDevideAndConquerRecurrence}
 u_n = \sum_{i=0}^k a_i u _{\floor{\frac{n}{b_i}}} + g(n).
 \end{equation}
 
 Linear devide-and-conquer recurrecences is given by a function of the form\begin{equation}
 u_n(t) = 
 	\begin{cases}
			u_0 > 0 & n= 0 \\
			\sum_{i=1}^k a_i u_{\floor{\frac{n}{b_i}}} \ + g(n) \quad &n  \geq 1 
	\end{cases}
 \end{equation}
 where
 \begin{itemize}
 \item $u_0, a_i \in \mathbb{R}^{\star+}$, $\sum_{i=1}^k a_i \geq 1$
 \item $b_{i}, k \in \mathbb{N}, b_i \geq 2, k\geq 1$
 \item $g(x)$ is defined for real values $x$, and is bounded, positive an nondecreasing function $\forall x\geq 0$.
 \item $\forall c > 1, \exists x_1, k_1 >0$ such that $g(\frac{x}{c}) \geq k_1 g(x), \forall x \geq x_1$. 
 \end{itemize}
 
 \begin{theorem}[Akra and Bazzi]
 \label{thm:AkraAndBazzi}
 If $p_0$ is the real solution of the characteristic equation
 \begin{equation}
 \label{eq:characteristicequation}
 \sum_{i=1}^k a_i b_i ^{-p} = 1
 \end{equation}
 (which always exists and is unique and positive), then 
 \begin{equation}
 u_n = \Theta(n^{p_0}) + \Theta \Big( n^{p_0} \int_{n_1}^n \frac{g(u)}{u^{p_0+1}}  du \Big)
 \end{equation}
 for $n_1$ large enough. In particualr,
 \begin{itemize}
 \item If $\exists \epsilon >0$ such that $g(x)=\mathcal{O}(x^{p_0-\epsilon})$, then $u_n = \Theta (n^{p_0})$.
 \item If $\exists \epsilon > 0$ such that $g(x) = \Omega(x^{p_0 + \epsilon})$ and $g(x)/ x^{p_0+\epsilon}$ is a non decreasing function, then $u_n = \Theta(g(n))$.
 \item If $g(x) = \Theta(x^{p_0})$, then $u_{n} = \Theta(n^{p_0} \log n)$. 
 \end{itemize}
 \end{theorem}
 
  An application of this theorem is given by \cite[Theorem 4]{AB98}:
 \begin{theorem}
 Let $f(x)$ be a function defined as in \ref{thm:AkraAndBazzi}. Let $p_0$ be the unique solution of the characteristic equation. Then,
 \begin{enumerate}
 	\item If $\exists \epsilon > 0$ such that $g(x) = \Mathcal{O}(x^{p_0-\epsilon})$, then $f(x) = \Theta(x^{p_0}).$
 	\item If $\exists \epsilon > 0$ such that $g(x) = \Omega( x^{p_{0}+\epsilon})$ and $g(x)/ x^{p_0+\epsilon}$ is a non decreasing function, then $f(x)=\Theta(g(x)).$
 	\item If $g(x) = \Theta(x^{p_{0})$ then $f(x) = \Theta( x^{p_{0}} \log x ).$   
 \end{enumerate}
 \end{theorem}
 Moreover by the application of pointwise convergence (see \cite[Theorem 1]{AB98}) the following intedies are given for non-negative numbers:
 \begin{theorem}
 Let $u_n$ be a seqeunce as in \ref{eq:linearDevideAndConquerRecurrence}.
 Let $f(x)$ be a function defined by 
 \begin{equation}
 f(x) = 
 	\begin{cases}
			u_0 > 0 & x=[0,1) \\
			\sum_{i=1}^k a_i f(x/b_i)+ g(\floor{x}) \quad &x  \in[1,\infty).
	\end{cases} 
 \end{equation}
 Then,
 \begin{enumerate}
 \item $\forall x\geq 0, f(x) = f(\floor{x}).$
 \item $ \forall n \geq 0,f(n)= u_n$. 
 \end{enumerate}
 \end{theorem}