% glossary entries

% example:
%\newglossaryentry{foo}{name={foo},description={},see={bar,baz}}
%\setabbreviationstyle[acronym]{long-short}
%\newacronym{laser}{laser}{light amplification by stimulatedemission of radiation}

\newglossaryentry{Isabelle}{
	name = Isabelle,
	description = {A generic proof assitant \url{https://isabelle.in.tum.de/} }
}

\newglossaryentry{CoC}{
	name = Calculus of Constructions,
	description = {´´The calculus of constructive proofs is a natural deduction style.''
	 	\url{https://doi.org/10.1016/0890-5401(88)90005-3}}
}

\newglossaryentry{Hadoop}{
	name = Hadoop,
	description = {open-source software project for relaiblae, scalable und distrubuted computing.
	\url{https://hadoop.apache.org} }
}

\newglossaryentry{Emacs}{
	name = Emacs,
	description = { a text editor
}}

\newglossaryentry{SAT-solver}{
	name = SAT-solver,
	description = {An algorithm to solve a Boolean satisfubility problem called SAT.},
	plural = {SAT-solver},
}

\newglossaryentry{SMT-solver}{
	name = SMT-solver,
	description = {An algorithm solving the SMT-problem, which is determining if a fomula of first order Logic is staisfiable.},
 	plural = {SMT-solvers},
}

\newglossaryentry{model checker}{
	name = model checker,
	description = {Checking wather a model meets a given specification.}
	plural = {model-checkers}
}

\newglossaryentry{coqtop}{	
	name = coqtop, 
	description = {The Coq Proof Assitant top level system \url{https://coq.inria.fr/refman/practical-tools/coq-commands.html}}
}



%standard library

%Boolean 
%Numbers

%data structs

%hash tables

%module