\subsection{Einführung}
Sehr viele Probleme lassen sich durch Objekte und Verbindungen oder
Beziehungen zwischen diesen Objekten beschreiben. Ein schönes Beispiel
hierfür ist das 
\dindex{Königsberger Brückenproblem}\index{Brückenproblem}, das $1736$ 
von Leonhard Euler\footnote{Der Schweizer Mathematiker Leonhard Euler 
wurde $1707$ in Basel geboren und starb $1783$ in St.~Petersburg.}  
formuliert und gelöst wurde. Zu dieser Zeit hatte 
Königsberg\footnote{Königsberg heißt heute Kaliningrad.} genau sieben 
Brücken, wie die sehr grobe Karte in Abbildung \ref{BProb2} zeigt.

Die verschiedenen Stadtteile sind dabei mit A-D bezeichnet. Euler
stellte sich nun die Frage, ob es möglich ist, einen Spaziergang in
einem beliebigen Stadtteil zu beginnen, jede Brücke \emph{genau
einmal} zu überqueren und den Spaziergang am Startpunkt zu
beenden. Ein solcher Weg soll \emph{Euler-Spaziergang} heißen. Die
Frage lässt sich leicht beantworten, wenn der Stadtplan wie nebenstehend
formalisiert wird.

Die Stadtteile sind bei der Formalisierung zu Knoten geworden und die
Brücken werden durch Kanten zwischen den Knoten
symbolisiert\footnote{Abbildung \ref{BProb2} nennt
man \dindex{Multigraph}, denn hier starten mehrere Kanten von
\emph{einem} Knoten und enden in \emph{einem} anderen Knoten.}.
Angenommen es gäbe in Königsberg einen Euler-Spazier\-gang, dann müsste für jeden
Knoten in Abbildung \ref{BProb2} die folgende Eigenschaft erfüllt
sein: die Anzahl der Kanten die mit einem Knoten verbunden sind ist
gerade, weil für jede Ankunft (über eine Brücke) in einem Stadtteil
ein Verlassen eines Stadtteil (über eine Brücke) notwendig ist.
\begin{figure}
\centering
\subfloat[grober Stadtplan]{\includegraphics[scale=0.5]{BrueckenProb}}
\hfill
\subfloat[formalisierter Stadtplan]{\includegraphics[scale=0.8]{BrueckenProb2}}
\caption{Das Königsberger-Brückenproblem}
\label{BProb2}
\end{figure}

