\ifpdf
\special{pdf: out 3 << /Title 
(Einf�hrung) 
/Dest [ @thispage /FitH @ypos ] >>}
\fi
\subsection{Einf�hrung}

Sehr viele Probleme lassen sich durch Objekte und Verbindungen oder
Beziehungen zwischen diesen Objekten beschreiben. Ein sch�nes Beispiel
hierf�r ist das 
\dindex{K�nigsberger Br�ckenproblem}\index{Br�ckenproblem}, das $1736$ 
von Leonhard Euler\footnote{Der Schweizer Mathematiker Leonhard Euler 
wurde $1707$ in Basel geboren und starb $1783$ in St.~Petersburg.}  
formuliert und gel�st wurde.

Zu dieser Zeit hatte K�nigsberg\footnote{K�nigsberg hei�t heute
Kaliningrad.} genau sieben Br�cken wie die folgende sehr grobe Karte
zeigt:

\begin{figure}[h]
\begin{center}
\includegraphics[scale=0.75]{BrueckenProb.eps}
\end{center}
\end{figure}

\begin{wrapfigure}[15]{r}{8cm}
\begin{center}
\includegraphics[scale=1.0]{BrueckenProb2.eps}
\end{center}
\caption{Der formalisierte Stadtplan.}
\label{BProb2}
\end{wrapfigure}
Die verschiedenen Stadtteile sind dabei mit A-D bezeichnet. Euler
stellte sich nun die Frage, ob es m�glich ist, einen Spaziergang in
einem beliebigen Stadtteil zu beginnen, jede Br�cke \emph{genau
einmal} zu �berqueren und den Spaziergang am Startpunkt zu
beenden. Ein solcher Weg soll \emph{Euler-Spaziergang} hei�en. Die
Frage l�sst sich leicht beantworten, wenn der Stadtplan wie nebenstehend
formalisiert wird.

Die Stadtteile sind bei der Formalisierung zu Knoten geworden und die
Br�cken werden durch Kanten zwischen den Knoten
symbolisiert\footnote{Abbildung \ref{BProb2} nennt
man \dindex{Multigraph}, denn hier starten mehrere Kanten von
\emph{einem} Knoten und enden in \emph{einem} anderen Knoten.}.

Angenommen es g�be in K�nigsberg einen Euler-Spazier\-gang, dann m�sste f�r jeden
Knoten in Abbildung \ref{BProb2} die folgende Eigenschaft erf�llt
sein: die Anzahl der Kanten die mit einem Knoten verbunden sind ist
gerade, weil f�r jede Ankunft (�ber eine Br�cke) in einem Stadtteil
ein Verlassen eines Stadtteil (�ber eine Br�cke) notwendig ist.
