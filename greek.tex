\ifpdf
\special{pdf: out 3 << /Title 
(Gebr�uchliche griechische Buchstaben) 
/Dest [ @thispage /FitH @ypos ] >>}
\fi
\subsection{Gebr�uchliche griechische Buchstaben}
In er Informatik ist es �blich griechische Buchstaben zu
verwenden. Ein Grund hierf�r ist, dass es so m�glich wird mit einer
gr��eren Anzahl von Unbekannten arbeiten zu k�nnen, ohne
un�bersichtliche und oft unhandliche Indices benutzen zu m�ssen.
\index{griechische Buchstaben}\index{Buchstaben!griechische}

\bigskip

\noindent Kleinbuchstaben:
\begin{displaymath}
\begin{array}{c|c||c|c||c|c}
\text{Symbol} & \text{Bezeichnung} & \text{Symbol}
& \text{Bezeichnung} & \text{Symbol} & \text{Bezeichnung}\\
\hline
\alpha & \text{Alpha} & \beta   & \text{Beta}   & \gamma   & \text{Gamma}\\
\hline
\delta & \text{Delta} & \phi    & \text{Phi}    & \varphi  & \text{Phi}\\
\hline
\xi    & \text{Xi}    & \zeta   & \text{Zeta}   & \epsilon & \text{Epsilon}\\         
\hline
\theta & \text{Theta} & \lambda & \text{Lambda} & \pi      & \text{Pi}\\
\hline
\sigma & \text{Sigma} & &
\end{array}
\end{displaymath}

\bigskip

\noindent Grossbuchstaben:
\begin{displaymath}
\begin{array}{c|c||c|c||c|c}
\text{Symbol} & \text{Bezeichnung} & \text{Symbol}
& \text{Bezeichnung} & \text{Symbol} & \text{Bezeichnung}\\
\hline
\Gamma & \text{Gamma} & \Delta & \text{Delta} & \Phi    & \text{Phi}\\
\hline
\Xi    & \text{Xi}    & \Theta & \text{Theta} & \Lambda & \text{Lambda}\\
\hline
\Pi    & \text{Pi}    & \Sigma & \text{Sigma} & \Psi    & \text{Psi}\\
\hline
\Omega & \text{Omega} & &
\end{array}
\end{displaymath}


