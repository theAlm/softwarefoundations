\ifpdf
\special{pdf: out 3 << /Title 
(Gebr�uchliche griechische Buchstaben) 
/Dest [ @thispage /FitH @ypos ] >>}
\fi
\subsection{Gebr�uchliche griechische Buchstaben}
In er Informatik ist es �blich griechische Buchstaben zu
verwenden. Ein Grund hierf�r ist, dass es so m�glich wird mit einer
gr��eren Anzahl von Unbekannten arbeiten zu k�nnen, ohne
un�bersichtliche und oft unhandliche Indices benutzen zu m�ssen.
\index{griechische Buchstaben}\index{Buchstaben!griechische}

\bigskip

\noindent Griechische Kleinbuchstaben:
\begin{itemize}
%
\item $\alpha$ (Alpha)
%
\item $\beta$ (Beta)
%
\item $\gamma$ (Gamma)
%
\item $\delta$ (Delta)
%
\item $\phi$ (Phi)
%
\item $\varphi$ (Phi)
%
\item $\xi$ (Xi)
%
\item $\zeta$ (Zeta)
%
\item $\epsilon$ (Epsilon)
%
\item $\theta$ (Theta)
%
\item $\lambda$ (Lambda)
%
\item $\pi$ (Pi)
%
\item $\sigma$ (Sigma)
%
\end{itemize}

\bigskip

\noindent Griechische Grossbuchstaben:
\begin{itemize}
%
\item $\Gamma$ (Gamma)
%
\item $\Delta$ (Delta)
%
\item $\Phi$ (Phi)
%
\item $\Xi$ (Xi)
%
\item $\Theta$ (Theta)
%
\item $\Lambda$ (Lambda)
%
\item $\Pi$ (Pi)
%
\item $\Sigma$ (Sigma)
%
\item $\Psi$ (Psi)
%
\item $\Omega$ (Omega)
%
\end{itemize}
