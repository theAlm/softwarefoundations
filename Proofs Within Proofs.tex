\section{Proofs Within Proofs}

In Coq and {\itshape} formal mathematics large proofs are often broken into a sequence of theorems and later proofs might refer to more early stated proofs.
Sometimes a proof requires miscellenous trivail or to litte general facts. 
Therefore the facts sometimes do not need a name.

To state intermadiate goals like a little sub theorem during proofs the tactics \lstinline!assert! is used.


\begin{example}
We have can show the previous theorem \lstinline!mult_0_plus!. This is an example using destruct. 
\end{example}

\begin{lstinline}
mult_0_plus': \forall n,m: nat,
...
assert (H: 0+n = n).
...
\end{lstinline}

\end{example}

In listing \ref{lstinline} the tactics assert in line introduces two subgoals. 
The assertion itself is listed with \lstinline!H! the introduction as prefix.
The proof of the assertion is bounded by curly parenthesis.\\
It profides reducability and interactive use of Coq it is more easy to see when the proof of the first subgoal is finished.

The second subgoal is the same as the subgoal in the proof before the subgoal exact of in the context the assuption H was added.
But the previously prooven fact is able to be used to make progress.
 







 
