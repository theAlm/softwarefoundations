\section{Lists - Working with structured Data}

\subsection{Pairs of Numbers}

Using an indictive type definiton each constructor can take any number of arguments. 
Therefore, let's make a pair of numbers.

\begin{lstlisting}[caption = \lstinline!natprod!, label = lst:natprod]
Inductive natprod: Type:=
 | pair (n1 n2: nat).

check(pair 3,5)
\end{lstlisting}
By the decalaration in this listing (\ref{lst:natprod}) there is only one way to construct a pair of numbers; 
Applying the constructor \lstinline!pair! to two argumens of type \lstinline!nat!.
Some simple function on a pair might be given by: 
\begin{lstlisting}[caption = \lstinline!fst!, label = lst:fst]
Definition fst( p: natprod): nat :=
  match p with 
    | pair xy => x (* return 1st component)
  end.
 
 Definition snd ( p: natprod): nat :=
   match p with 
     | pair xy => y (* return 2nd component*)
   end.
\end{lstlisting}

And an abbericvation such that the mathematical standart notation can be used is implemented by :
\begin{lstlisting}[caption=\lstinline!pair!-naotation, label = lst:pairNotation]
Notation "(x,y)" = (pair xy).
\end{lstlisting}

This notation shall be used in pattern matches and expressions as: 
\begin{lstlisting}[caption = \lstinline!fst'! and \lstinline!snd'!, label =lst:newNotation ]
Definiton fst'(p:natprod):nat:=
match p with
  |(x,y) => x
end.

Definition snd' (p:natprod):nat :=
match p with
  |(x,y) => y
end.
\end{lstlisting}




