\section{Introduction}



Lore ipsum \cite{PACGGHSY}

Steffens comments:
Fill later-
 Why do we do this?
 History of Coq
 maybe a little motivating example if you find one
 Mathematical building blocks. Proof technices. Notation to Appendix.\\
 %\gls{functional programming}
 \gls{laser}
We are going to use the interactive mode, because we want to use Coq as a proof-assistant. {\emph Find a reference.}

\subsection{}

Comments, Zeichensatz (ASC II), Data Endings.
Look this up here:\\
https://coq.inria.fr/refman/language/gallina-specification-language.html
\subsection{-forgotten word-}



\paragraph{Proof General}
-explain what this is


\paragraph{Coq IDE}

\paragraph{Coq in the command line}




\subsection{Usage of this Notes}


The listings of Coq-code in this book are meant to be run by the reader in a Coq "forgotten word" while reading.

Find a reference:
\begin{enumerate}
\item syntax check by color highlighting
\item evaluate until point
\end{enumerate}




