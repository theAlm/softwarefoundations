\section{Introduction}


This is a summary of the electronic text-book \cite{PACGGHSY} with comments from the author and based on the peers and supervisors feedback.
This summary aims to be give a clear instruction of the usage and applicability of Coq.\\

    
%%%%%%%%%%%%%%%%%% TODO: %%%%%%%%%%%%%%%%%%%%%%%%%%%%%%%%%%%%%%%%%%%%%%%%%%%%%%%%%%%%%%%%%%%%%%%%%%%%%%%%%%%%%%%%%%%%%%%%%
Steffens comments:
Fill later-
 Why do we do this?
 History of Coq
 maybe a little motivating example if you find one
 Mathematical building blocks. Proof technices. Notation to Appendix.\\
 \gls{functional programming}
We are going to use the interactive mode, because we want to use Coq as a proof-assistant. {\emph Find a reference.}
%%%%%%%%%%%%%%%%%%% TODO END %%%%%%%%%%%%%%%%%%%%%%%%%%%%%%%%%%%%%%%%%%%%%%%%%%%%%%%%%%%%%%%%%%%%%%%%%%%%%%%%%%%%%%%%%%%%%%%%

\subsection{Preface}
Within this introduction the mathematical underpinning of reliable software is given the building blocks are
\begin{itemize}
\item basics concepts of logic (see sec. \ref{} % TODO:)
\item computer assisted theorem prving (see sec. \ref{} %TODO:)
\item Coq-proof assistant (see sec. \ref{} %TODO:
\item functional programming (see sec. \ref{} %TODO)
\item operational semantics (see sec. \ref{} %TODO:)
\item logics for reasoning about programs (see sec. \ref{} %TODO: )
\item static type systems (see sec. \ref{} %TODO:)
\end{itemize} 

\subsection{Overview}
There is a lot of motivation for reliable software. 
First of all the scale, complexity and number of involved people in modern systems is increasing.
Therefore building correct software is extremely difficult.
The information processing is waved into every aspect of society, 
therefore the costs of bugs and insecurities are amplified to multiple levels.\\
Computer scientists and software engineers have responded to improve reliability with a lot of design threats and to improve reliability and mathematical technices for reasoning.
The within this work the it should be contributed to validates these properties. They are:
\begin{enumerate}
\item basic tools from logic for making and justifying precise claims about programs
\item use of proof assistants to construct rigorous logical arguments
\item functional programmings as method of programming and simplifying reasoning about programs as a bridge between programming and logic
\end{enumerate}
%%%%%%%%%%%%%%%%%%%%%% TODO %%%%%%%%%%%%%%%%%%%%%


Comments, Zeichensatz (ASC II), Data Endings.
Look this up here:\\
https://coq.inria.fr/refman/language/gallina-specification-language.html


%%%%%%%%%%%%%%%%%%%%%% TODO end %%%%%%%%%%%%%%%%%


\subsection{Logic}

\begin{quote}
``As a matter of fact, logic has turned out to be significantly more effective in computer science then it has been in mathematics.''
\end{quote}
Volumes have been written about the central role of logic in computer science. 
It's fundamental tool {\itshape inductive proof} is going to be explored very deeply here.

\subsection{Proof Assistants}
In computer science proof assistants are an important tool for helping construct formal proofs of logical propositions.
There are two categories of these tool.
First of all the are automated theorem proofers. These are able of a ''push-button- operation'' which returns true, false'' or ``ran out of time'' given a proposition.
Example applications are SAT-solvers, SMT-solvers or model checkers. 
Second there are proof assistants, which are hybrid tools which automate the more routine-like aspects of a proof, while depending on human guidance. 
Examples are \gls{Isabelle}, Agda, Twelfe, ACL2, PVS or Coq.\\
The Coq proof assistent has been developed since 1988 and gathered a large community in resaerch and the industry.
It provied a rich environment for interactive development of machine-checked code for formal reasoning.\\
It's kernel is a simple proof checker, ensuring that correct decustion of setst are ever performed. Moreover, there are high-level facilites for proof development.
Coq has been applied as critical enabler across computer science and mathematics a platform for modelling programming languages and as an environment for developing formally certifed software and hardware.
 

\subsection{Trivia}
Some French computer scientist have a tradition of naming their software as animal species.
Coq is the French word for roster.
The roster is the national symbol of the French.
Coq sounds like the initial of the \gls{CoC}.
Coq's early developers.-- TODO: Look UP! 

\subsection{Functional Programming}
There are two meanings of the term. It is either refered to programming idioms (something like a pattern)
or something else as in this scope.

Functional promaning refers to a progamiing, which is free of side effects.

\subsection{System Requirements}



\paragraph{Proof General}
-explain what this is
i

\paragraph{Coq IDE}

\paragraph{Coq in the command line}




\subsection{Usage of this Notes}


The listings of Coq-code in this book are meant to be run by the reader in a Coq "forgotten word" while reading.

Find a reference:
\begin{enumerate}
\item syntax check by color highlighting
\item evaluate until point
\end{enumerate}




