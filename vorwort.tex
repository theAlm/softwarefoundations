\special{pdf: out 2 << /Title 
(Vorwort) 
/Dest [ @thispage /FitH @ypos ] >>}
\section{Vorwort}

Viele Studenten haben, besonders in den ersten Semestern, oft
erhebliche Probleme mit mathematischen Notationen und Denkweisen, die
gerade in der (praktischen) Informatik eine extrem große und wichtige
Rolle spielen. Diese Probleme lassen sich nur durch stetiges,
selbstständiges, regelmäßiges und ausdauerndes Üben, "`selber
machen"' und durch selbstständiges Anfertigen von
Vorlesungsmitschriften überwinden. 

Lassen Sie sich nicht entmutigen, wenn das Lösen einer einzigen
Aufgabe manchmal mehrere Stunden benötigt, denn das ist völlig normal!
Leider zeigt es sich oft, dass die Schule solche Kompetenzen nicht
(mehr) vermittelt, d.h.~Sie werden diese am Anfang Ihres Studiums
mühsam, mit Schweiß und oft auch mit Frust erlernen müssen. Das ist 
völlig normal und Sie können sich sicher sein, dass Ihre Dozenten
diese Phase genauso wie Sie erlebt und gemeistert haben! 

Sie erwerben sich durch die Übungsaufgaben u.a.~eine
Problemlösungskompetenz, die für spätere Projekte (z.B. Bachelor- und
Masterarbeit) extrem hilfreich ist. In der beruflichen Praxis werden
Sie sich auch regelmäßig mit sehr komplexen Aufgabenstellungen
beschäftigen, d.h.~diese Problemlösungsstrategien sind eine
Grundvoraussetzung für das Berufsleben eines erfolgreichen
Informatikers. Auch aus diesen Gründen ist es also sehr sinnvoll, die
(wöchentlichen) Übungsblätter oder Praktikumsaufgaben der jeweiligen
Veranstaltung selbstständig und regelmäßig zu bearbeiten! Es reicht
nicht, die in den Übungen präsentierten Lösungen zu konsumieren, denn
ein Lerneffekt stellt sich nur ein, wenn Sie die Aufgaben
selbstständig lösen. In Ihrere beruflichen Praxis werden Sie später
für das (selbstständige) Lösen von Problemen bezahlt, deshalb wird
klar, dass sich Passivität nicht auszahlt. 

\begin{center}
\large Nehmen Sie Ihr Studium selbst in den Hand, denn es ist \emph{Ihres}!
\end{center}
Auf jeden Fall ist es nicht das Studium Ihres Dozenten! 
Dabei ist es auch hilfreich, wenn Sie eine Aufgabe mal nicht lösen,
denn auch bei einer ausdauernden aber nicht erfolgreichen Bearbeitung
einer Aufgabe kommt es zu einem Lerneffekt! Werden die Übungen nicht
selbstständig erarbeitet, so bleibt der Lernerfolg mit an Sicherheit
grenzender Wahrscheinlichkeit aus, was sich sicherlich in der Klausur
zum ersten Mal (aber nicht zum letzten Mal) rächen wird. Leider zeigt
sich, dass diese Bemerkungen durch die Hörer regelmäßig ignoriert
werden, deshalb

\begin{center}
\large Warnung:  diese Schwierigkeiten lassen sich auf gar keinen Fall durch "`drei Tage Lernen vor der Klausur"' lösen!
\end{center}

Auch eine Woche wird dazu erfahrungsgemäß nicht ausreichen! Dieser
Hinweis lässt sich sinngemäß natürlich auch auf (fast) alle andern
Fächer Ihres Studiums übertragen.

\bigskip

In jeder Wissenschaft stellt sich sofort die Frage, wie man (neue)
Erkenntnisse gewinnen kann. Eine mögliche Methode funktioniert wie
folgt: Hat man eine Kette von Aussagen, die \emph{zwingend} und
Schritt für Schritt auseinander hervorgehen, so nennt man das eine
\dindex{Deduktion}. Das Wort Deduktion kommt aus dem Lateinischen und
bedeutet "`ableiten"' (von Wasser) oder "`fortführen"'. Die initiale Aussage
dieser Kette wird \dindex{Hypothese} und die letzte Aussage wird
\dindex{Konklusion} genannt. Ist die Hypothese richtig, so muss auch
die Konklusion richtig sein. Dieses Vorgehen ist als \dindex{deduktive Methode}
\index{Methode!deduktive} bekannt und wird in der Mathematik rigoros 
angewendet, d.h.~\emph{niemals} werden Aussagen oder Definitionen verwendet, die
vorher nicht klar belegt wurden und jeder Schritt in der Kette muss \emph{genauestens}
begründet werden. Dieses Vorgehen ist am Anfang ungewohnt, da 
viele Menschen gewohnt sind einfach (unbelegte) Behauptungen aufzustellen und
daraus dann (die gewünschten) Schlüsse mit Hilfe von unpräzisen Argumenten zu ziehen.  
Im starken Kontrast dazu, stellt die deduktive Methode sicher, dass keine falschen 
Aussagen gemacht werden können (zumindest so lange die Deduktion fehlerfrei ist). 
Folgendes Beispiel gibt einen erste Idee für das Vorgehen der Mathematik:

Wir wissen, dass "`Sokrates ist ein Mensch"' eine wahre Aussage
ist. Dies stellt unsere Hypothese dar. Nun wissen wir auch "`alle
Menschen sind sterblich"'. Wir können daraus unsere Konklusion
folgern, dass "`Sokrates sterblich ist"', wodurch wir eine neue
Erkenntnis gewonnen haben.

Das Vorgehen keine mehrdeutigen Definitionen zu verwenden und jeden
Argumentationsschritt klar und logisch zu begründen wird als
(mathematische) \dindex{Strenge} bezeichnet und wird neben der
Mathematik auch in der Informatik und anderen auf der Mathematik
basierenden Wissenschaften verwendet. So kommentiert der berühmte
Mathematiker David Hilbert $1900$ auf einem Kongress in Paris in
seinem Vortrag "`Mathematische Probleme"' das mathematische Vorgehen
wie folgt:

\begin{quote}
"`$\ldots$ welche berechtigten allgemeinen Forderungen an die Lösung
eines mathematischen Problems zu stellen sind: ich meine vor Allem
die, daß es gelingt, die Richtigkeit der Antwort durch eine endliche
Anzahl von Schlüssen darzuthun und zwar auf Grund einer endlichen
Anzahl von Voraussetzungen, welche in der Problemstellung liegen und
die jedesmal genau zu formuliren sind. Diese Forderung der logischen
Deduktion mittelst einer endlichen Anzahl von Schlüssen ist nichts
anderes als die Forderung der Strenge in der Beweisführung. In der
That die Forderung der Strenge, die in der Mathematik bekanntlich von
sprichwörtlicher Bedeutung geworden ist, entspricht einem allgemeinen
philosophischen Bedürfnis unseres Verstandes und andererseits kommt
durch ihre Erfüllung allein erst der gedankliche Inhalt und die
Fruchtbarkeit des Problems zur vollen Geltung. $\dots$"'
\end{quote}

In der \dindex{Erkenntnistheorie} sind auch andere Methoden des
Erkenntnisgewinns bekannt, wie z.B.~die Induktion. Hier versucht man
bestehende Tatsachen zu verallgemeinern, was (auch) zu nicht korrekten
Aussagen führen \emph{kann}, die man dann genauer untersuchen
muss. Dies könnte beispielsweise so aussehen: "`Alle Fahrzeuge auf dem
Parkplatz sind Autos"' und "`Alle Fahrzeuge sind rot"'. Daraus folgern
wir "`Alle Autos sind rot"'. Diese Vorgehensweise spielt hier (erst
einmal) keine oder eine ehr untergeordnete Rolle.

\bigskip

Weiterhin zeigt sich, dass die mathematischen Beweismethodiken
(Deduktionen) oft eng mit Konzepten aus der Softwareerstellung
verknüpft sind, d.h.~mathematisches Denken hat in der Praxis für
praktische Informatiker einen grossen Wert. Ein schönes Beispiel
hierfür ist die starke Ähnlichkeit von Induktionsbeweisen und
rekursiven Algorithmen. Dies zeigt, dass sich eine Einarbeitung in
mathematische Denkweisen auch für Informatiker lohnt, auch wenn ein
direkter Zusammenhang vielleicht nicht sofort ersichtlich ist.

\bigskip

Das hier vorliegende Dokument ist ein (sehr) kurzer Crashkurs mit einigen
Beispielen, der evtl.~Defizite aus der Schule ausgleichen helfen
soll. Um Hinweise zur Ergänzung dieses Skriptes wird dringend gebeten,
denn Vorlesungsskripte werden für Sie (und nicht für den/die Dozenten)
erstellt, d.h.~es ist in Ihrem eigenen Interesse, dass Verbesserungen und
Erweiterungen eingebaut werden. Beachten Sie auch, dass die Inhalte
der Vorlesungen von den Inhalten der jeweiligen Skripten abweichen
können! Es kann insbesondere Inhalte in einem Skript geben, die nicht
prüfungsrelevant sind, und prüfungsrelevante Inhalte, die nicht im
Skript zur Vorlesung enthalten sind.
