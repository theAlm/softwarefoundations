\section{Vorwort}

Viele Studierende haben, besonders in den ersten Semestern, oft
erhebliche Probleme mit mathematischen Notationen und Denkweisen, die
gerade in der (praktischen) Informatik eine extrem große und wichtige
Rolle spielen. Es ist auch zu beobachten, dass dieser Mangel von den Studierenden gar nicht bemerkt wird, da noch keine tiefere Einsichten in das Gebiet der Informatik  (und Mathematik) gewonnen werden konnten. Als Folge empfinden die Studierenden Mathematik deshalb als schwierig \emph{und} nutzlos. Dieses Vorurteil wird dann erfahrungsgemäß von vielen nicht (mehr) hinterfragt, wodurch sich diese Studierenden die Chance nehmen, tiefliegende Erkenntnisse auf dem Gebiet der Informatik zu sammeln. Weiterhin scheint es auch einen ausgeprägten Widerwillen bzgl.~des Lesens von (linearen) Texten zu geben (Ja, wir sind immer noch im ersten Absatz!), was das Lernen ebenfalls massiv behindert. Deshalb würde ich Ihnen (ja, genau \emph{Sie}) empfehlen, gegen beide Probleme gleich am Anfang Ihres Studiums etwas zu unternehmen. Es lohnt sich!

Alle diese Probleme lassen sich nur durch stetiges,
selbstständiges, regelmäßiges und ausdauerndes Üben, "`selber
machen"' und durch selbstständiges Anfertigen von
Vorlesungsmitschriften überwinden. Lassen Sie sich nicht entmutigen, wenn das Lösen einer einzigen Aufgabe manchmal mehrere Stunden benötigt. Dies ist völlig normal! Auch der Autor dieses Skripts sitzt an seinen "`Übungsaufgaben"' (Forschung) sehr oft, braucht meist sehr viel Zeit (6 Monate sind da keine Seltenheit) und ist bei Misserfolgen frustriert. Kurz: Das muss so sein, damit sich ein signifikanter Lernerfolg einstellt!

Leider zeigt es sich oft, dass viele Schulen solche Kompetenzen nicht
(mehr) vermitteln oder Sie haben einen "`nicht klassischen"' Weg an die Hochschule 
genommen, d.h.~Sie werden diese am Anfang Ihres Studiums
mühsam, mit Schweiß und oft auch mit Frust erlernen müssen. Das ist 
leider völlig normal und Sie können sich sicher sein, dass \emph{alle} Ihre Dozenten
diese Phase in ihrem Studium genauso wie Sie erlebt (und gemeistert) 
haben! Glauben Sie keinen "`didaktischen Heilsversprechen"' (z.B.~eLearning, alternative 
Lehrformen), die angeblich einen einfachen Weg  zur Erkenntnis garantieren. Gäbe es diese Wege, 
so hätte niemand Probleme mit dem mathematischen Denken, das in der Informatik und der 
Alltagswelt sehr nützlich ist. Auch die Grundlagen der Naturwissenschaften wären jedem bestens 
bekannt. Vertrauen Sie lieber auf Ihre \emph{eigenen Fähigkeiten} und akzeptieren Sie die 
Tatsache, dass Lernen  schwer ist  und schwer bleibt,  zumindest wenn es um nichttriviale Inhalte 
geht. Glauben Sie  auch nicht,  dass nur Sie  diese Schwierigkeiten haben! Ideen und Einsichten 
kommen  meist "`von selbst"', wenn Sie  sich (selbst!)  nur lange  genug mit dem Stoff 
beschäftigen, d.h.~das kompromisslose "`Verbeißen"' in ein Thema ist der Schlüssel zum Erfolg. 
Ein Geheimnis für Ihren persönlichen Studienerfolg ist die Einstellung "`ich werde nicht
aufgeben"'. Eine gewisse Leidenschaft für \emph{Ihr} Fach ist also für die
notwendige Motivation unabdingbar.  Erfahrungsgemäß sind "`nicht locker lassen"'
und Motivation fast schon ein Garant für ein erfolgreiches Studium.

Ideen kommen oft zu den komischsten Zeiten, deshalb ist es wichtig
sich dafür (viel) Zeit zu nehmen. Schalten Sie alle Ablenkungen (z.B.~"`Internet"' oder  
"`WhatsApp"') ab! Oft kommen dann Ideen unter der Dusche, beim spazieren gehen, beim 
Ausdauersport, während Tagträumen\footnote{Beschäftigen Sie sich einmal 
mit der Legende von  August Kekulé und der Strukur von Benzol.} oder nach einem 
Mittagsschlaf. Es gibt Hinweise, dass Schlaf für das Lehren extrem wichtig ist, was
mit den persönlichen Erfahrungen des Autors übereinstimmt, deshalb wird das 
Lernen drei Tage vor der Prüfung meist nicht zu einem brauchbaren Erfolg führen!

Sie erwerben sich durch diese Einstellung und die Übungsaufgaben u.a.~eine
\dindex{Problemlösungskompetenz}, die für spätere Projekte (z.B. Bachelor- und
Masterarbeit) extrem hilfreich ist. In der beruflichen Praxis werden
Sie sich (hoffentlich) auch regelmäßig mit sehr komplexen Aufgabenstellungen
beschäftigen\footnote{Stellen Sie sich vor wie langweilig und schlecht bezahlt Ihr Leben 
wäre, wenn Sie immer nur die genau gleiche Tätigkeit ausführen müssten.}, d.h.~diese 
Problemlösungsstrategien sind eine Grundvoraussetzung für das Berufsleben eines 
erfolgreichen Informatikers. Auch aus diesen Gründen ist es also sehr sinnvoll, die
(wöchentlichen) Übungsblätter oder Praktikumsaufgaben der jeweiligen
Veranstaltung selbstständig und regelmäßig zu bearbeiten! Es reicht
nicht, die in den Übungen präsentierten Lösungen zu konsumieren indem man irgendwelche 
Musterlösungen oder Lösungen der Kommilitonen abschreibt, denn
ein Lerneffekt stellt sich nur ein, wenn Sie die Aufgaben
selbstständig vollständig bearbeiten. Es reicht \emph{auch} nicht eine grobe Idee auf einen 
Schmierzettel zu schreiben! Eine gute Lösung ist sauber ausgearbeitet und durchdacht. Die äußere 
Form sollte einem Dritten (z.B.~einem anderen Kommilitonen) davon überzeugen können, dass die 
Idee richtig ist. Bleibt nur ein leiser Zweifel, dann ist die Ausarbeitung noch lückenhaft und muss 
überarbeitet werden. 

Alle diese Tipps  stellen keinen Widerspruch zu der Arbeit in (kleinen) Gruppen dar, wenn die 
einzelnen Schritte alleine gelöst und dann mit anderen Studenten zu einer finalen und  schönen 
Version zusammen gesetzt werden. In Ihrer beruflichen Praxis werden Sie später für das 
selbstständige und vollständige Lösen von Problemen -- auch in Teams -- bezahlt, deshalb 
sollte  klar sein, dass sich  Passivität (im Studium) nicht auszahlt. 
 
Dabei ist es auch hilfreich, wenn Sie eine Aufgabe einmal nicht lösen können,
denn auch bei einer ausdauernden aber nicht erfolgreichen Bearbeitung
einer Aufgabe kommt es zu einem Lerneffekt! Werden die Übungen nicht
selbstständig erarbeitet, so bleibt der Lernerfolg allerdings mit an Sicherheit grenzender Wahrscheinlichkeit aus, was sich sicherlich in der Klausur zum ersten Mal (aber nicht zum letzten Mal) rächen wird. Leider zeigt sich, dass diese Bemerkungen durch die Hörer regelmäßig ignoriert werden, deshalb

\begin{center}
\large \textbf{Warnung:}  diese Schwierigkeiten lassen sich auf gar keinen Fall durch 
"`drei Tage Lernen vor der Klausur"' lösen!
\end{center}

Auch eine Woche wird dazu erfahrungsgemäß nicht ausreichen! Dieser
Hinweis lässt sich sinngemäß natürlich auch auf (fast) alle anderen
Fächer Ihres Studiums übertragen. Deshalb nochmal ein wirklich ernst gemeinter 
Ratschlag:
\begin{center}
	\large Nehmen Sie Ihr Studium selbst in den Hand, denn es ist \emph{Ihres}!
\end{center}
Auf jeden Fall ist es \emph{nicht} das Studium Ihres Dozenten! Beachten Sie auch, dass 
es \emph{nicht} die "`Schuld"' Ihres Dozenten ist, wenn Ihr Studium nicht so erfolgreich
verläuft, wie Sie es sich erhofft haben. Sollten Sie mit einem Dozenten fachlich (oder 
auch menschlich) nicht zurecht kommen, so liegt es in \emph{Ihrer} Verantwortung einen Weg zu finden ohne diesen "`lästigen"', "`uninspirierten"', "`langweiligen"', "`unfähigen"' und  "`unmöglichen"' Menschen erfolgreich zu werden. Auch solche 
Probleme müssen Sie in Ihrem weiteren beruflichen Leben immer wieder meistern, 
d.h.~es nutzt in einem Studium nichts mit seinem Schicksal zu hadern! Auch macht es 
keinen Sinn ein  Fach als "`langweilig"' oder "`nicht relevant"' zu betrachten, bloß weil 
der aktuelle  Dozent (menschlich) nicht passt! Ich erinnere nochmal: "`Es ist Ihr 
Studium!"' Bitte beachten Sie, dass man gerade am Anfang eines Studiums noch gar 
nicht abschätzen kann in welche Richtung\footnote{Der Autor dieses Textes wollte ursprünglich Waffenschmied werden und hatte dann doch (ganz 
fest) ein Studium der "`Chemie"' geplant. Dieser Wunsch wurde dann durch "`Theoretische Physik"' oder "`Teilchenphysik"' abgelöst. Informatik war nur ein "`Betriebsunfall"', der sich als sehr segensreich und spannend erwiesen hat. Inzwischen wäre es kein  Informatikstudium mehr,  sondern ein Studium der Mathematik mit Schwerpunkt auf Zahlentheorie und  algebraische Geometrie.} es gehen soll, d.h.~Sie können noch gar nicht abschätzen welche Fächer (für Sie) wichtig und relevant sind oder werden. Insbesondere werden sich Ihre Interessensgebiete sicherlich verschieben! Deshalb ist es ein guter Tipp sich im Studium breit aufzustellen. Die Spezialisierung kommt im Berufsleben ganz von alleine.

In Ihrer schulischen Laufbahn haben Sie solche Probleme  in diesem Ausmaß sicherlich 
nicht meistern müssen ("`heile Welt"'),  deshalb ist es möglicherweise sinnvoll 
sich, insbesondere am Anfang, helfen zu lassen (z.B.~durch (ältere) Mitstudenten,  Dozenten 
oder die Fachschaft).

\goodbreak
\bigskip

Als Lehrender bekommt man am Anfang einer (eigentlich jeder) Vorlesung \emph{fast immer} die 
Frage "`Gibt es ein Skript oder müssen wir mitschreiben?"' gestellt. Einerseits ist diese Frage
wichtig, aber auf der anderen Seite ist sie falsch gestellt, denn Sie brauchen ein Skript
\emph{und} eine Mitschrift, um eine Vorlesung wirklich erfolgreich zu meistern. Ein Skript 
bringt Ihnen in der Regel wenig für den Studienerfolg, auch wenn alle Lerninhalte drin ganz genau 
beschrieben sind. Ähnlich verhält es sich mit einer Vorlesung. Für das tiefgründige Verständnis des 
Lerninhalts ist die Vor- und Nachbereitung einer Vorlesung essentiell. Dabei ist es entscheidend die
Nachbereitung zeitnah durchzuführen, d.h.~möglichst am gleichen Tag oder notfalls
zumindest in der gleichen Woche. Sie werden schnell lernen bei welchen Fächern Sie eine
unmittelbare Nachbereitung brauchen und bei welchen Fächern Sie die Zügel 
\emph{etwas} schleifen lassen können.

Für die Nacharbeit ist ein Skript hilfreich, da Sie die Lerninhalte dort (hoffentlich) zuverlässig 
nachschlagen können. Noch hilfreicher sind aber entsprechende Lehrbücher, die Sie in der 
Hochschulbibliothek finden können. Überhaupt sollte die Bibliothek einer Ihrer 
Lieblingsaufenthaltsorte werden, denn die Lehr- und Fachbücher enthalten das für Ihr Studium 
notwendige Wissen in ordentlich aufbereiteter Form und machen es dadurch leichter zugänglich. Auf 
keinen Fall reicht es,  Suchmaschinen wie "`Google"'\index{Google} oder "`Bing"'\index{Bing} oder 
Quellen wie "`Wikipedia"'\index{Wikipedia} zu verwenden! Sie  verschwenden \emph{massiv} Zeit, 
wenn Sie sich die benötigten Informationen im Netz zusammen  suchen und Sie bleiben bezüglich 
deren Qualität weiterhin unsicher. Bitte bedenken Sie, dass eine große Portion von Fachwissen benötigt 
wird, um gute und schlechte Informationsquellen zu unterscheiden. Bei guten Fachbüchern können Sie 
recht sicher sein, dass die Autoren ausgewiesene Experten auf ihrem Gebiet sind. Trotzdem sollten Sie 
verschiedene Autoren "`ausprobieren"', um das Buch zu finden, das Ihnen am besten weiterhilft, denn 
jeder Autor hat einen unterschiedlichen Schreib- und Erklärstil. 

Das effektivste und wichtigste Mittel für die Nachbereitung ist eine persönlich erstellte 
\dindex{Mitschrift}, die Sie während der Veranstaltung selbst erstellen. Oft wird gerade von Anfängern 
bezweifelt, dass eine  Mitschrift nützlich ist. Dazu werden allerlei Ausreden hervorgebracht. Für manche 
ist es "`zu viel  Arbeit"' oder sie haben eine "`schlechte Handschrift"'. Auch die Argumente "`ich kann 
nicht  mit der Hand schreiben und zuhören"' oder "`ich kann nicht schnell genug schreiben"' finden 
sich  jedes Semester immer wieder. Damit stellt sich die Frage, warum eine handschriftliche  Mitschrift 
so  wichtig ist. Der Grund ist einfach: Eine Mitschrift enthält die von Ihnen individualisierte Version 
des  (unvertrauten) Lernstoffs und ist  ganz auf Ihr Denken und Ihre Anschauung zugeschnitten. Sie 
enthält Ihre Gedankenbilder, Fragen und die von Ihnen entdecken Zusammenhänge und vielleicht 
sogar kleine Beispiele. Weiterhin gibt es wissenschaftliche Hinweise die nahelegen, dass das Schreiben 
mit der Hand den Wissenserwerb unterstützt (z.B.~\cite{MuBe14}). 

Damit ergibt sich die Frage wie eine "`gute"' Mitschrift aussehen sollte. Die äußere Form ist
auf jeden Fall nicht sonderlich wichtig. Lassen Sie sich nicht von schön gesetzten und formatieren
Texten\footnote{Wie dieser Text zeigt ist dies recht einfach, wenn man das Textsatzsystem
\LaTeX\ verwendet.} blenden! Besser ist \emph{immer} eine Menge von selbst beschriebenen Papier 
mit Kaffeeflecken, Eselsohren, das durch den häufigen Gebrauch\footnote{Der Autor dieses Dokuments 
verwendet Teile seiner alten Vorlesungsmitschriften immer noch. Diese sind zu einer wichtigen Quelle 
geworden, da sich dort Hinweise über damalige Verständnisprobleme finden. Tipp: Heben Sie sich Ihre 
eigenen Ausarbeitungen für später auf, es lohnt sich!} schon ganz speckig geworden ist, als der neuste 
Hochglanzausdruck eines aus dem Internet gesaugten PDF-Files! In Ihrer Mitschrift sollte Wichtiges 
von Unwichtigem klar unterschieden werden. Das benötigt Konzentration, denn Sie müssen dies beim 
Hören der  Vorlesung für sich entscheiden. Mit ein wenig Übung werden Sie hier schnell einen Weg 
finden. Dabei muss der Tafelanschrieb nicht wörtlich übernommen werden, sondern Sie sollten ihn 
geeignet  verkürzen und umformen. Durch diese Denkleistung behält Ihr Gehirn schon einen Teil der 
Informationen, ohne das Sie es merken. Aus diesem Grund reicht es nicht, wenn Sie 
die Mitschrift eines Kommilitonen kopieren. Verwenden Sie, wenn möglich, \emph{eigene}
Formulierungen, kurze Sätze, einzelne Wörter, Symbole und Abkürzungen. Die Gliederung
und die Überschriften übernehmen Sie einfach aus der Vorlesung. Sehr wichtige zentrale
Punkte kann man durch \emph{sparsames} (farbiges) Unterstreichen hervorheben. Schreiben
Sie ein Blatt nicht ganz voll. So bleibt Platz für eigene Anmerkungen, Ideen, Querverweise,
Fragen und Kommentare. Gerade diese sind extrem wertvoll für Ihren Lernprozess.  
Deshalb macht es auch Sinn die Mitschrift aus der Vorlesung (zumindest in Teilen) in der 
Nachbereitungsphase nochmal komplett neu zu schreiben. Ob Sie kompliziertere Fragen gleich oder 
am Anfang der nächsten Vorlesung stellen ist eigentlich egal, allerdings werden Sie bemerken, dass 
sich viele Fragen  von selbst lösen, wenn Sie sie einfach einmal richtig aufgeschrieben haben.

\bigskip

In Ihrem Studium und in jeder Wissenschaft stellt sich sofort die Frage, wie man (neue)
Erkenntnisse gewinnen kann.  Da Sie sich in der Schule mit solchen Fragestellungen
noch nicht beschäftigt haben, sind diese Vorgehensweisen an Hochschulen gerade in der 
ersten Zeit ungewohnt. 

Eine mögliche Methode zur Erzeugung neuen Wissens funktioniert wie folgt: Hat man 
eine Kette von Aussagen, die \emph{zwingend} und
Schritt für Schritt auseinander hervorgehen, so nennt man das eine
\dindex{Deduktion}. Das Wort Deduktion kommt aus dem Lateinischen und
bedeutet "`ableiten"' (von Wasser) oder "`fortführen"'. Die initiale Aussage
dieser Kette wird \dindex{Hypothese} und die letzte Aussage wird
\dindex{Konklusion} genannt. Ist die Hypothese richtig, so muss auch
die Konklusion richtig sein. Dieses Vorgehen ist als \dindex{deduktive Methode}
\index{Methode!deduktive} bekannt und wird in der Mathematik rigoros 
angewendet, d.h.~\emph{niemals} werden Aussagen oder Definitionen verwendet, die
vorher nicht klar belegt wurden und jeder Schritt in der Kette muss \emph{genauestens}
begründet werden. Dieses Vorgehen ist am Anfang ungewohnt, da 
viele Menschen gewohnt sind einfach (unbelegte) Behauptungen aufzustellen und
daraus dann (die gewünschten) Schlüsse mit Hilfe von unpräzisen Argumenten zu ziehen.  
Im starken Kontrast dazu, stellt die deduktive Methode sicher, dass keine falschen 
Aussagen gemacht werden können (zumindest so lange die Deduktion fehlerfrei ist). 
Folgendes Beispiel gibt einen erste Idee für das Vorgehen der Mathematik:

Wir wissen, dass "`Sokrates ist ein Mensch"' eine wahre Aussage
ist. Dies stellt unsere Hypothese dar. Nun wissen wir auch "`alle
Menschen sind sterblich"'. Wir können daraus unsere Konklusion
folgern, dass "`Sokrates ist sterblich"', wodurch wir eine neue
Erkenntnis gewonnen haben. Weiterhin erhalten wir ganz einfach die Einsicht, dass
Sokrates sterben wird.

Das Vorgehen keine mehrdeutigen Definitionen zu verwenden und jeden
Argumentationsschritt klar und logisch zu begründen wird als
(mathematische) \dindex{Strenge} bezeichnet und wird neben der
Mathematik auch in der Informatik und anderen auf der Mathematik
basierenden Wissenschaften verwendet. So kommentiert der berühmte
Mathematiker David Hilbert im Jahr $1900$ auf einem Kongress in Paris in
seinem Vortrag "`Mathematische Probleme"' das mathematische Vorgehen
wie folgt:

\begin{quote}
"`$\ldots$ welche berechtigten allgemeinen Forderungen an die Lösung
eines mathematischen Problems zu stellen sind: ich meine vor Allem
die, daß es gelingt, die Richtigkeit der Antwort durch eine endliche
Anzahl von Schlüssen darzuthun und zwar auf Grund einer endlichen
Anzahl von Voraussetzungen, welche in der Problemstellung liegen und
die jedesmal genau zu formuliren sind. Diese Forderung der logischen
Deduktion mittelst einer endlichen Anzahl von Schlüssen ist nichts
anderes als die Forderung der Strenge in der Beweisführung. In der
That die Forderung der Strenge, die in der Mathematik bekanntlich von
sprichwörtlicher Bedeutung geworden ist, entspricht einem allgemeinen
philosophischen Bedürfnis unseres Verstandes und andererseits kommt
durch ihre Erfüllung allein erst der gedankliche Inhalt und die
Fruchtbarkeit des Problems zur vollen Geltung. $\dots$"'
\end{quote}

In der \dindex{Erkenntnistheorie} sind auch andere Methoden des
Erkenntnisgewinns bekannt, wie z.B.~die Induktion. Hier versucht man
bestehende Tatsachen zu verallgemeinern, was (auch) zu nicht korrekten
Aussagen führen \emph{kann}, die man dann genauer untersuchen
muss. Dies könnte beispielsweise so aussehen: "`Alle Fahrzeuge auf dem
Parkplatz sind Autos"' und "`Alle Fahrzeuge sind rot"'. Daraus folgern
wir "`Alle Autos sind rot"'. Diese Vorgehensweise spielt hier (erst
einmal) keine oder eine eher untergeordnete Rolle.

\bigskip

Weiterhin zeigt sich, dass die mathematischen Beweismethodiken
(Deduktionen) oft eng mit Konzepten aus der Softwareerstellung
verknüpft sind, d.h.~mathematisches Denken hat in der Praxis für
praktische Informatiker einen großen Wert. Ein schönes Beispiel
hierfür ist die starke Ähnlichkeit von Induktionsbeweisen und
rekursiven Algorithmen. Dies zeigt, dass sich eine Einarbeitung in
mathematische Denkweisen auch für Informatiker lohnt, auch wenn ein
direkter Zusammenhang vielleicht nicht sofort ersichtlich ist.

\bigskip

Das hier vorliegende Dokument ist ein (sehr) kurzer Crashkurs mit einigen
Beispielen, der evtl.~Defizite aus der Schule ausgleichen helfen
soll. Um Hinweise zur Ergänzung dieses Skriptes wird dringend gebeten,
denn Vorlesungsskripte werden für Sie (und nicht für den/die Dozenten)
erstellt, d.h.~es ist in Ihrem eigenen Interesse, dass Verbesserungen und
Erweiterungen eingebaut werden. Beachten Sie auch, dass die Inhalte
der Vorlesungen von den Inhalten der jeweiligen Skripten abweichen
können! Es kann insbesondere Inhalte in einem Skript geben, die nicht
prüfungsrelevant sind, und prüfungsrelevante Inhalte, die nicht im
Skript zur Vorlesung enthalten sind.
