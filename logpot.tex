\special{pdf: out 2 << /Title 
(Logarithmieren, Potenzieren und Radizieren) 
/Dest [ @thispage /FitH @ypos ] >>}
\subsection{Logarithmieren, Potenzieren und Radizieren}
Die Schreibweise $a^b$ ist eine Abkürzung für 
\begin{displaymath}
a^b \eqd \underbrace{a \cdot a
\cdot \ldots \cdot a}_{b-\text{mal}} 
\end{displaymath}
und wird als \dindex{Potenzierung} bezeichnet. Dabei wird $a$ als
\dindex{Basis}, $b$ als \dindex{Exponent} und $a^b$ als $b$-te
\dindex{Potenz} von $a$ bezeichnet. \index{Logarithmus} Seien nun 
$r,s,t \in \R$ und $r,t \ge 0$ durch die folgende Gleichung verbunden:

\begin{displaymath}
r^s = t.
\end{displaymath}
Dann läßt sich diese Gleichung wie folgt umstellen und es gelten die
folgenden Rechenregeln\index{Logarithmus}\index{Wurzel}\index{Radizieren}:

\begin{center}
\begin{tabular}{c|c|c}
Logarithmieren & Potenzieren & Radizieren\\
\hline
$\mathbf{s = \log_r t}$ & $\mathbf{t = r^s}$ &
\phantom{$\left(\frac{\frac{c}{d}a}{b}\right)$} $\mathbf{r = \sqrt[\mathbf{s}]{\mathbf{t}}}$\\
\hline
\begin{minipage}[t]{0.38\textwidth}
\begin{enumerate}[i)]
%
\item $\log_r (\frac{u}{v}) = \log_r u - \log_r v$
%
\item $\log_r ({u} \cdot {v}) = \log_r u + \log_r v$
%
\item $\log_r (t^u) = u \cdot \log_r t$
%
\item $\log_r (\sqrt[u]{t}) = \frac{1}{u} \cdot \log_r t$
%
\item $\frac{\log_r t}{\log_r u} = \log_u t$ (Basiswechsel)
%
\end{enumerate}
\end{minipage}
&
\begin{minipage}[t]{0.25\textwidth}
\begin{enumerate}[i)]
%
\item $r^{u} \cdot r^{v} = r^{u + v}$
%
\item $\frac{r^{u}}{r^{v}} = r^{u - v}$
%
\item $u^{s} \cdot v^{s} = (u \cdot v)^{s}$
%
\item $\frac{u^{s}}{v^{s}} = \left(\frac{u}{v}\right)^{s}$
%
\item $(r^{u})^{v} = r^{u \cdot v}$ 
%
\end{enumerate}
\end{minipage}
&
\begin{minipage}[t]{0.28\textwidth}
%
\begin{enumerate}[i)]
%
\item $\sqrt[\leftroot{1} s]{u} \cdot \sqrt[s]{v} = \sqrt[s]{u \cdot v}$
%
\item $\frac{\sqrt[s]{u}}{\sqrt[s]{v}} = \sqrt[s]{\left(\frac{u}{v}\right)}$
%
\item $\sqrt[u]{\sqrt[v]{t}} =
  \sqrt[u \cdot v]{t}$
%
\end{enumerate}
\end{minipage}\\
\end{tabular}
\end{center}
Zusätzlich gilt: Wenn $r > 1$, dann ist $s_1 < s_2$ \gdw $r^{s_1} <
r^{s_2}$ (Monotonie).

Da $\sqrt[s]{t} = t^{\left(\frac{1}{s}\right)}$ gilt, können die
Gesetze für das Radizieren leicht aus den Potenzierungsgesetzen
abgeleitet werden.  Weiterhin legen wir spezielle Schreibweisen für
die Logarithmen zur Basis $10$, $e$ (Eulersche Zahl) und $2$ fest:
$\lg t \eqd \log_{10} t$, $\ln t \eqd \log_{e} t$ und $\mathrm{lb}\,
t \eqd \log_{2} t$.
