\ifpdf
\special{pdf: out 2 << /Title 
(Logarithmieren, Potenzieren und Radizieren) 
/Dest [ @thispage /FitH @ypos ] >>}
\fi
\subsection{Logarithmieren, Potenzieren und Radizieren}
Die Schreibweise $a^b$ ist eine Abk�rzung f�r 
\begin{displaymath}
a^b \eqd \underbrace{a \cdot a
\cdot \dots \cdot a}_{b-\text{mal}} 
\end{displaymath}
und wird als \dindex{Potenzierung} bezeichnet. Dabei wird $a$ als
\dindex{Basis}, $b$ als \dindex{Exponent} und $a^b$ als $b$-te
\dindex{Potenz} von $a$ bezeichnet.  Seien nun $r,s,t \in \R$ und $r,t
\ge 0$ durch die folgende Gleichung verbunden:
\begin{displaymath}
r^s = t.
\end{displaymath}
Dann l��t sich diese Gleichung wie folgt umstellen und es gelten die
folgenden Rechenregeln\index{Logarithmus}\index{Wurzel}\index{Radizieren}:

\begin{center}
\begin{tabular}{c|c|c}
Logarithmieren & Potenzieren & Radizieren\\
\hline
$\mathbf{s = \log_r t}$ & $\mathbf{t = r^s}$ &
\phantom{$\left(\frac{\frac{c}{d}a}{b}\right)$} $\mathbf{r = \sqrt[\mathbf{s}]{\mathbf{t}}}$\\
\hline
\begin{minipage}[t]{0.38\textwidth}
\begin{enumerate}[i)]
%
\item $\log_r (\frac{t_1}{t_2}) = \log_r t_1 - \log_r t_2$
%
\item $\log_r ({t_1} \cdot {t_2}) = \log_r t_1 + \log_r t_2$
%
\item $\log_r (t^u) = u \cdot \log_r t$
%
\item $\log_r (\sqrt[u]{t}) = \frac{1}{u} \cdot \log_r t$
%
\item $\frac{\log_r t}{\log_r u} = \log_u t$ (Basiswechsel)
%
\end{enumerate}
\end{minipage}
&
\begin{minipage}[t]{0.25\textwidth}
\begin{enumerate}[i)]
%
\item $r^{s_1} \cdot r^{s_2} = r^{s_1 + s_2}$
%
\item $\frac{r^{s_1}}{r^{s_2}} = r^{s_1 - s_2}$
%
\item $r_1^{s} \cdot r_2^{s} = (r_1 \cdot r_2)^{s}$
%
\item $\frac{r_1^{s}}{r_2^{s}} = \left(\frac{r_1}{r_2}\right)^{s}$
%
\item $(r^{s_1})^{s_2} = r^{s_1 \cdot s_2}$ 
%
\end{enumerate}
\end{minipage}
&
\begin{minipage}[t]{0.28\textwidth}
%
\begin{enumerate}[i)]
%
\item $\sqrt[\leftroot{1} s]{t_1} \cdot \sqrt[s]{t_2} = \sqrt[s]{t_1 \cdot t_2}$
%
\item $\frac{\sqrt[s]{t_1}}{\sqrt[s]{t_2}} = \sqrt[s]{\left(\frac{t_1}{t_2}\right)}$
%
\item $\sqrt[\uproot{3} s_1]{\sqrt[\uproot{3} s_2]{t}} =
  \sqrt[\uproot{3} s_1 \cdot s_2]{t}$
%
\end{enumerate}
\end{minipage}\\
\end{tabular}
\end{center}
Zus�tzlich gilt: Wenn $r > 1$, dann ist $s_1 < s_2$ \gdw $r^{s_1} <
r^{s_2}$ (Monotonie).

Da $\sqrt[s]{t} = t^{\left(\frac{1}{s}\right)}$ gilt, k�nnen die
Gesetze f�r das Radizieren leicht aus den Potenzierungsgesetzen
abgeleitet werden.  Weiterhin legen wir spezielle Schreibweisen f�r
die Logarithmen zur Basis $10$, $e$ (Eulersche Zahl) und $2$ fest:
$\lg t \eqd \log_{10} t$, $\ln t \eqd \log_{e} t$ und $\mathrm{lb}\, t \eqd \log_{2} t$. 
 

