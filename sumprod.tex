\ifpdf
\special{pdf: out 2 << /Title 
(Summen und Produkte) 
/Dest [ @thispage /FitH @ypos ] >>}
\fi
\section{Summen und Produkte}

\ifpdf
\special{pdf: out 3 << /Title 
(Summen) 
/Dest [ @thispage /FitH @ypos ] >>}
\fi
\subsection{Summen}
Zur abk�rzenden Schreibweise verwendet man f�r Summen das
Summenzeichen $\sum$\index{$\sum$}. Dabei ist
\begin{displaymath}
\sum_{i=1}^n a_i \eqd a_1 + a_2 + \dots + a_n
\end{displaymath}
Mit Hilfe dieser Definition ergeben sich auf elementare Weise die
folgenden Rechenregeln:
\begin{itemize}
%
\item Sei $a_i = a$ f�r $1 \le k \le n$, dann gilt $\sum\limits_{i=1}^n a_i =
  n \cdot a$ (Summe gleicher Summanden).
%
\item $\sum\limits_{i=1}^n a_i = \sum\limits_{i=1}^m a_i +
  \sum\limits_{i = m + 1}^n a_i$, wenn $1 < m < n$ (Aufspalten einer Summe).
%
\item $\sum\limits_{i=1}^n (a_i + b_i + c_i + \dots) =
  \sum\limits_{i=1}^n a_i + \sum\limits_{i=1}^n b_i +
  \sum\limits_{i=1}^n c_i + \dots$ (Addition von Summen gleicher L�nge).
%
\item $\sum\limits_{i=1}^n a_i = \sum\limits_{i=m}^{n+m - 1} a_{i-m+1}$
  und $\sum\limits_{i=1}^n a_i = \sum\limits_{i=l}^{n-m+l} a_{k+m-l}$
  (Umnumerierung von Summen).
%
\item $\sum\limits_{i=1}^n \sum\limits_{j=1}^m a_{i,j} =
  \sum\limits_{j=1}^m \sum\limits_{i=1}^n a_{i,j}$ (Vertauschen der Summationsfolge).
%
\end{itemize}

\ifpdf
\special{pdf: out 3 << /Title 
(Summen) 
/Dest [ @thispage /FitH @ypos ] >>}
\fi
\subsection{Produkte}

Zur abk�rzenden Schreibweise verwendet man f�r Produkte das
Produktzeichen $\prod$\index{$\prod$}. Dabei ist
\begin{displaymath}
\prod_{i=1}^n a_i \eqd a_1 \cdot a_2 \cdot \ldots \cdot a_n
\end{displaymath}
Mit Hilfe dieser Definition ergeben sich auf elementare Weise die
folgenden Rechenregeln:
\begin{itemize}
%
\item Sei $a_i = a$ f�r $1 \le k \le n$, dann gilt $\prod\limits_{i=1}^n a_i =
  a^n$ (Produkt gleicher Faktoren).
%
\item  $\prod\limits_{i=1}^n (c a_i) = c^n \prod\limits_{i=1}^n a_i$
  (Vorziehen von konstanten Faktoren)
%
\item $\prod\limits_{i=1}^n a_i = \prod\limits_{i=1}^m a_i \cdot
  \sum\limits_{i = m + 1}^n a_i$ , wenn $1 < m < n$ (Aufspalten in Teilprodukte).
%
\item $\prod\limits_{i=1}^n (a_i \cdot b_i \cdot c_i \cdot \dots) =
  \prod\limits_{i=1}^n a_i \cdot \prod\limits_{i=1}^n b_i \cdot
  \prod\limits_{i=1}^n c_i \cdot \dots$ (Das Produkt von Produkten).
%
\item $\prod\limits_{i=m}^n a_i = \sum\limits_{i=l}^{n} a_{i+m-l}$
  und $\sum\limits_{i=1}^n a_i = \sum\limits_{i=l}^{n-m+l} a_{k+m-l}$
  (Umnumerierung).
%
\item $\prod\limits_{i=1}^n \prod\limits_{j=1}^m a_{i,j} =
  \prod\limits_{j=1}^m \prod\limits_{i=1}^n a_{i,j}$ (Vertauschen der
  Reihenfolge bei Doppelprodukten).
%
\end{itemize}
