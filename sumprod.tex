\special{pdf: out 3 << /Title 
(Summen und Produkte) 
/Dest [ @thispage /FitH @ypos ] >>}
\subsection{Summen und Produkte}

\special{pdf: out 4 << /Title 
(Summen) 
/Dest [ @thispage /FitH @ypos ] >>}
\subsubsection{Summen}
Zur abkürzenden Schreibweise verwendet man für Summen das
Summenzeichen $\sum$\index{$\sum$}. Dabei ist
\begin{displaymath}
\sum_{i=1}^n a_i \eqd a_1 + a_2 + \dots + a_n.
\end{displaymath}
Mit Hilfe dieser Definition ergeben sich auf elementare Weise die
folgenden Rechenregeln:
\begin{itemize}
%
\item Sei $a_i = a$ für $1 \le i \le n$, dann gilt $\sum\limits_{i=1}^n a_i =
  n \cdot a$ (Summe gleicher Summanden).
%
\item $\sum\limits_{i=1}^n a_i = \sum\limits_{i=1}^m a_i +
  \sum\limits_{i = m + 1}^n a_i$, wenn $1 < m < n$ (Aufspalten einer Summe).
%
\item $\sum\limits_{i=1}^n (a_i + b_i + c_i + \dots) =
  \sum\limits_{i=1}^n a_i + \sum\limits_{i=1}^n b_i +
  \sum\limits_{i=1}^n c_i + \dots$ (Addition von Summen).
%
\item $\sum\limits_{i=1}^n a_i = \sum\limits_{i=l}^{n + l - 1} a_{i-l+1}$
  und $\sum\limits_{i=l}^n a_i = \sum\limits_{i=1}^{n - l + 1} a_{i + l
  - 1}$
  (Umnumerierung von Summen).
%
\item $\sum\limits_{i=1}^n \sum\limits_{j=1}^m a_{i,j} =
  \sum\limits_{j=1}^m \sum\limits_{i=1}^n a_{i,j}$ (Vertauschen der Summationsfolge).
%
\end{itemize}

\special{pdf: out 4 << /Title 
(Produkte) 
/Dest [ @thispage /FitH @ypos ] >>}
\subsubsection{Produkte}

Zur abkürzenden Schreibweise verwendet man für Produkte das
Produktzeichen $\prod$\index{$\prod$}. Dabei ist
\begin{displaymath}
\prod_{i=1}^n a_i \eqd a_1 \cdot a_2 \multdots a_n.
\end{displaymath}
Mit Hilfe dieser Definition ergeben sich auf elementare Weise die
folgenden Rechenregeln:
\begin{itemize}
%
\item Sei $a_i = a$ für $1 \le i \le n$, dann gilt $\prod\limits_{i=1}^n a_i =
  a^n$ (Produkt gleicher Faktoren).
%
\item  $\prod\limits_{i=1}^n (c a_i) = c^n \prod\limits_{i=1}^n a_i$
  (Vorziehen von konstanten Faktoren)
%
\item $\prod\limits_{i=1}^n a_i = \prod\limits_{i=1}^m a_i \cdot
  \prod\limits_{i = m + 1}^n a_i$ , wenn $1 < m < n$ (Aufspalten in Teilprodukte).
%
\item $\prod\limits_{i=1}^n (a_i \cdot b_i \cdot c_i \cdot \ldots) =
  \prod\limits_{i=1}^n a_i \cdot \prod\limits_{i=1}^n b_i \cdot
  \prod\limits_{i=1}^n c_i \cdot \ldots$ (Das Produkt von Produkten).
%
\item $\prod\limits_{i=1}^n a_i = \prod\limits_{i=l}^{n + l - 1} a_{i-l+1}$
  und $\prod\limits_{i=l}^n a_i = \prod\limits_{i=1}^{n - l + 1} a_{i + l
  - 1}$
  (Umnumerierung von Produkten).
%
\item $\prod\limits_{i=1}^n \prod\limits_{j=1}^m a_{i,j} =
  \prod\limits_{j=1}^m \prod\limits_{i=1}^n a_{i,j}$ (Vertauschen der
  Reihenfolge bei Doppelprodukten).
%
\end{itemize}

Oft werden Summen- oder Produktsymbole verwendet bei denen der Startindex 
größer als der Stopindex ist. Solche Summen bzw.~Produkte sind "`leer"', 
d.h.~es wird nichts summiert bzw.~multipliziert. Sind dagegen Start- und
Endindex gleich, so tritt nur genau ein Wert auf, d.h.~das Summen- 
bzw.~Produktsymbol hat in diesem Fall keine Auswirkung. Er gegeben sich
also die folgenden Rechenregeln:
\begin{itemize}
%
\item Seien $n, m \in \Z$ und $n < m$, dann 
\begin{displaymath}
\sum_{i=m}^n a_i = 0  \text{ und } \prod_{i = m}^n a_i = 1
\end{displaymath}
%
\item Sei $n \in \Z$, dann 
\begin{displaymath}
\sum_{i=m}^n a_i = a_i = \prod_{i = m}^n a_i
\end{displaymath}
\end{itemize}


