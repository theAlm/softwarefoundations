\section{Additional Materials}
\label{app:AdditionalMaterials}

In this section selected additional topics about the Coq proof assistant are listed.
They are sought to give example applications of Coq, help or substitute studying and understanding.

\subsection{Coq and Predicate Logic} 
\label{subsec:CoqAndPredicateLogic}

	The semantics of a Coq-\lstinline!theorem! might be interpreted as a logical formula.
	The Coq-\lstinline!Proof.! might be the justification of this formula. 
	For readability in the following premises denoted by capital letters ($P$ and $N$) are introduced which do not appear in Coq. A similar predicate logic based approach is introduced in mathematical standart text books like \cite[Kapitel 1.4.3. Anmerkungen zur Struktur mathematischer Sätze und ihrer Formulierung in der Sparche der Mengenlehre]{Z06}.  
	
	\begin{table}[h]
		\begin{left}
			\begin{tabular}{l|l}
	 			line no.  &  predicate logical translation \\ \hline 
		     	  1-3    % & $P\ n \ m$ means `' \\ 	     	                                                   
		     	         % & $N\ n\ m$ means `$n, m \in \mathbb{N}$' \\ 
		                  & \lstinline!plus_id_example! := $ \forall\ n, \ m\ \in \mathbb{N}: \big(n = m \big) \rightarrow \big( n+n = m+m. \big) $ \\ \hline        
		     	  4       & Proof: \\     \hline	     	      	                      
		                  &   $P( n,\ m):= (n+n = m+m)$    \\
		          5       &   $N( n, \ m) \ := ( n,\ m \in \mathbb{N})$       \\ \hline       
		          6       &   $H(n, \ m) := (n = m)$ \\        
		    	      7   &   $N (n,\ m) :=  H(n, \ m) \rightarrow \Big(P( n, \ m) \leftrightarrow( m+m = m+m) \Big)$\\   \hline 
		          8       &   (* by reflexivity and simplification it is concluded *) \\
		                  & $N(n, \ m) :=  H( n, \ m) \rightarrow P( n, \ m)$.  \\  \hline
		          9       & $\qed.$\\      	          
	        		\end{tabular}
		\end{left}
		\label{tab:CoqAndPreciateLogic}
		\caption{listing \ref{lst:plus_id_example_proof} \lstinline!plus_id_example!'s translation into predicate logic.} 
	\end{table}

\subsection{Coquille - Coqtop Cheat Sheet}

\begin{table}[h]
		\begin{left}
			\begin{tabular}{l|l}
			  \emph{Key}\\ 		
		      F2      & \lstinline!:CoqUndo! \\ 	     	                                                   
		      F3      & \lstinline!:CoqNext! \\         
		      F4      & \lstinline!:CoqToCurser!  
		    \end{tabular}
		\end{left}
		\label{tab:coquilleCheatSheetKeys}
	\end{table}		    
\begin{table}[h]
		\begin{left}		    	
		    \begin{tabular}{l|l}
		    \emph{Coqtop}\\
		    \lstinline!:CoqLaunch!  &   starts the proof checker\\   
		    \lstinline!:CoqRestart! &   restarts the proof checker \\       
		    \lstinline!:CoqKill!    &   quits the proof checker \\        
		    \lstinline!:CoqPrint!   &   prints the command from the information window \\
		    \end{tabular}
		    \label{tab:coquilleCheatSheetCoqTop}
		\end{left} 
	\end{table}	
			    
\begin{table}[h]
   \begin{left}		     	  
     	     \begin{tabular}{l|l}
     	     
     	     \emph{Vernacular commands}\\
		          \lstinline!Eval!  \texttt{term}     & displays the resulting \texttt{term} with its type  \\  
		          \lstinline!Compute! \texttt{term}   & call by value evaluation of the \texttt{term} \\          	          
		          \lstinline!About! \texttt{term}     & returns the namespace of the \texttt{term}
	   		\end{tabular}
	   			\label{tab:coquilleCheatSheetVernacular}
		\end{left} 
	\end{table}		  

	
	
	
	
	
\subsection{Axioms of Kolmogorov}
\label{subsec:AxiomsOfKolmogorov }
 
 This section gives some potential applications of the Coq proof assitent for some more advanced selected topics.
 	
 

 \begin{itemize}
	\item set theory and the scheduler	      
 	\item Borells $\sigma-Algebra$
 	\item $\mathbb{R}$ and $\mathbb{N}$
 	\item Note that $\# \mathcal{M}$ in the "Mathematische Grundlagen" lecture notes \cite[]{} is defined for finite sets only.  
	 \item Die Kardinalität, $|\mathbb{R}| = |\mathbb{N}|$ ? ( See Zorich )
 	 \item $C$ if $f$ is a function  falls es gibt höchstens ein $b\in B$ mit $(a,b)\in f$\\
 		 $A^n = ?$
 	\item Es ist messbar.
 \item   Borel-Cantelli lemma, abhängigkeits Definition$\mapsto$ Zeit-Diskretisierung
 \item Bsp: endlich viele Würfe einer Münze, Lebesgue, Faires Modell\\
 		Zufallsvariable ist messbar abhängig d.h. 
 		\begin{align}
 			\mu(A) = \mathbb{R}(\lambda^{-1}(A)) = \mathbb{P} (\{ X| X(\omega)\} \in A \})
 				 & = \mathbb{P}(\{ X \in A \geq 1| X_n = 1 \} 
 		\end{align}  
 		
 		where $\mu$ denotes the probability measure of $A$\\
 		$\lambda$ denotes the Lebesgue measure\\
 		$\mathbb{P}$ the probability\\
 		$A$ an event in the probability space\\
 	
 					
 \item Die Verteilung von $Y = X-1$ mit $X(\omega) = min \{ \omega \}$.
       \begin{equation}\label{eq:random_walk}
       	\mu(A)= p \sum_{k \in A } (1-p) ^{k-1}    	
       	\end{equation}
 		Jede Verteilfunktion mit den eigenschaften von satz 3.5 ( mein wahrscheinlichkeitstheore \& statistik Skript) ist eine Zufallsvariable.\\
 		Was muss gelten?\\
 		Das suksezzive Gesetzt vom Logarithmus.   
   		Gleichung (\ref{eq:random_walk}) beschreibt die Wahrscheinlichkeitsverteilung eines Random Walks . 
 \end{itemize}
 
 
   There are several publications about set theory and schedubility analysis.\par   
   Keyword: Adjacency Matrix Random Graph Degree Distribution Average Degree Edge Density
   Example Applications:\\
   Partitioning, Random Graphs with General Degree Distributions
 		
