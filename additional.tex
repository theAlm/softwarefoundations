\section{Additional Materials}
\label{app:AdditionalMaterials}

\subsection{Coq and Predicate Logic} 
\label{subsec:CoqAndPredicateLogic}

	The semantics of a Coq-\lstinline!theorem! might be interpreted as a logical formula.
	The Coq-\lstinline!Proof.! might be the justification of this formula. 
	For readability in the following premises denoted by capital letters ($P$ and $N$) are introduced which do not appear in Coq.
	
	\begin{table}[h]
		\begin{center}
			\begin{tabular}{|c|l|}
			    \hline
	 			line no.  &  predicate logical translation \\  \hline
		     	  1-3    % & $P\ n \ m$ means `' \\ 	     	                                                   
		     	         % & $N\ n\ m$ means `$n, m \in \mathbb{N}$' \\ 
		                  & \lstinline!plus_id_example! means `$ \forall\ n\ m\ [n\ , m \in \mathbb{N}]: n = m \rightarrow n+n = m+m.$'\\ \hline        
		     	  4       & Proof: \\     	     	      	                      
		                  &   $ P\ n\ m:= (n+n = m+m)$    \\ \hline
		          5       &   $N \ n \ m \ := ( n,\ m \in \mathbb{N})$       \\ \hline       
		          6       &   $ H\ n\ m :=( n= m)$ \\        
		    	      7       &   $ \forall n \ m\ [N \ n\ m]: H\ n\ m \rightarrow (P\ n\ m \leftrightarrow( m+m = m+m))$\\   \hline 
		          8       &   (* by reflexivity and simplification it is concluded *) \\
		                  & $\forall n\ m\ [ N\ n\ m]: H\ n\ m  \rightarrow P\ n\ m$.  \\  \hline
		          9       & $\qed.$\\ \hline        	          
	        		\end{tabular}
		\end{center}
		\label{tab:CoqAndPreciateLogic}
		\caption{listing \ref{lst:plus_id_example_proof} \lstinline!plus_id_example!'s translation into predicate logic.} 
	\end{table}

TODO: Reference for the logical notation and basics.

\subsection{Coquille - Coqtop Cheat Sheet}

\begin{table}[h]
		\begin{left}
			\begin{tabular}{l|l}
			  \emph{Shortcuts}\\ 		
		      F2      & :CoqUndo \\ 	     	                                                   
		      F3      & :CoqNext \\         
		      F4      & :CoqToCurser  
		    \end{tabular}
		\end{left}
		\label{tab:coquilleCheatSheetKeys}
	\end{table}		    
\begin{table}[h]
		\begin{left}		    	
		    \begin{tabular}{l|l}
		    \emph{Coqtop}\\
		    \lstinline!:CoqRestart! &   restarts the proof checker \\       
		    \lstinline!:CoqKill!    &   quits the proof checker \\        
		    \lstinline!:CoqLaunch!  &   starts the proof checker\\   
		    \lstinline!:CoqPrint!  &   prints the command from the information window \\
		    \end{tabular}
		    \label{tab:coquilleCheatSheetCoqTop}
		\end{left} 
	\end{table}			    
\begin{table}[h]
		\begin{left}		    
     	     \begin{tabular}{l|l}
     	     \emph{Vernacular commands}\\
		          \lstinline!Eval!       & displays the resulting term with its type  \\  
		          \lstinline!Compute!    & call by value evaluation \\          	          
		          \lstinline!About!      & retuns namespace
	   		\end{tabular}
	   			\label{tab:coquilleCheatSheetVernacular}
		\end{left} 
	\end{table}		  
	
