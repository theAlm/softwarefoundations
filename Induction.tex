\section{Induction}

\subsection{Proof by Induction}
In the proof \ref{lst:plus0nPrime} it was shown have , 
that \lstinline!0! is the neutral element from the left if the natural numbers \lstinline!nat! is studied as a group. 

Is should be shown that zero is is the natural element from the right:
\begin{lstlisting}
Theorem plus_n_0_first: for all n: nat,
  n = n + 0.  
\end{lstlisting}

But the tactics, which were introduced till know, are not sufficiently powerful to proof this theorem.

\begin{lstlisting}
Theorem plus_n_O_firsttry : forall n:nat,
  n = n + 0.
\end{lstlisting}

Applying refelxivity can not proof the theorem. In fact, simplifying the expression \lstinline!n + 0 !, leads to nowhere.
Because looking at the definition of \lstinline!plus!, it is obvious.
If \lstinline!n! is an unknown number, the  \lstinline!match! can not be applied.
  
\begin{lstlisting}
Proof.
  intros n.
  simpl. (* Does nothing. *)
Abort.
\end{lstlisting}

Proofing by the \lstinline!destruct!-tactic is going to fail, because in the case \lstinline!n = Sn'! the expression \lstinline!S n' = S n' + 0! can not be simplified by the same reason as above.
\begin{lstlisting}
Proof.
  intros n. destruct n as [| n'] eqn:E.
  - (* n = 0 *)
    reflexivity. 
  - (* n = S n' *)
    simpl.       (*Does nothing.*)
Abort.
\end{lstlisting}

Recall the mathematical principle of induction.
Due to apply induction in Coq the steps are the same and the syntax is simialar to the \lstinline!destruct! tactics.
 

\begin{lstlisting}
Theorem minus_diag : forall n,
  minus n n = 0.
Proof.
  intros n. induction n as [| n' IHn'].
  - (*case: n = 0 *)
    simpl. reflexivity.
  - (*case: n = S n' *)
    simpl. rewrite -> IHn'. reflexivity.  
  Qed.
\end{lstlisting}

The \lstinline!as!-clause has two parts seperated by \lstinline!|!.
\begin{itemize}
	\item In the above statement of \lstinline!induction! the first subgoal is to show the induction basis for \lstinline!n =0!.
	\item And the second subgoal is the induction step for \lstinline!n = Sn'!, since \lstinline!nat! is defined inductivly (see listing  \ref{lst:DefNat}).
	      The assumption \lstinline!n'+0 = n'! is added to Coq's context named as \lstinline!IHn'! as induction hypothesis.
		  It must be shown \lstinline! Sn' = Sn' +0!. \
          Applying \lstinline!simpl.! yields \lstinline!Sn' = S(n'+0)!, which follows from the induction hypothesis. 
          Hence, \lstinline!relfexivity.! finishes this proof.
\end{itemize} 




