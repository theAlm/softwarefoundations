\ifpdf
\special{pdf: out 2 << /Title 
(Einige Grundlagen der Logik) 
/Dest [ @thispage /FitH @ypos ] >>}
\fi
\section{Einige Grundlagen der Logik}
Aussagen sind entweder \dindex{wahr} ($\triangleq 1$) oder
\dindex{falsch}($\triangleq 0$). So ist die Aussage 
\begin{center}
"`Wiesbaden liegt am Mittelmeer"'
\end{center}
sicherlich falsch, wogegen die Aussage
\begin{center}
"`Wiesbaden liegt in Hessen"'
\end{center}
sicherlich wahr ist. Aussagen werden meist durch
\dindex{Aussagenvariablen} ausgedr�ckt, die nur die Werte $0$ oder $1$
annehmen k�nnen. Um die Verkn�pfung von Aussagen auch formal
aufschreiben zu k�nnen, werden die folgenden logischen Operatoren
verwendet

\begin{center}
\begin{tabular}{l|l|l}
Symbol & umgangssprachlicher Name & Name in der Logik\\
\hline
\dindex{$\sand$} & und & Konjunktion\\
\dindex{$\sor$} & oder & Disjunktion / Alternative\\
\dindex{$\sneg$} & nicht & Negation \\
\dindex{$\simpl$} & folgt & Implikation\\
\dindex{$\sequi$} & genau dann wenn (gdw.) & �quivalenz\\
\end{tabular}
\end{center}

Zus�tzlich werden noch die Quantoren $\exists$ ("`es existiert"') und
$\forall$ ("`f�r alle"') verwendet, die z.B.~wie folgt gebraucht
werden k�nnen
\begin{description}
%
\item $\forall x \colon p(x)$ "`F�r alle $x$ gilt die Aussage $p(x)$. 
%
\item $\exists x \colon p(x)$ "`Es existiert ein $x$, f�r das die Aussage
  $p(x) gilt$.
%
\end{description}

\begin{example}
Die Aussage "`Jede gerade nat�rliche Zahl kann als Produkt von $2$ und einer
anderen nat�rlichen Zahl geschrieben werden"' l�sst sich dann wie
folgt schreiben
\begin{displaymath}
(\forall n \in \N \colon n \text{ ist gerade }) \simpl (\exists m
\in \N \colon n = 2 \cdot m) 
\end{displaymath}

F�r die logischen Konnektoren sind die folgenden Wahrheitswertetafeln
festgelegt:

\begin{center}
\begin{tabular}{c|c}
$p$ & $\neg p$\\
\hline
$0$ & $1$\\
$1$ & $0$
\end{tabular}
und
\begin{tabular}{c|c|c|c|c|c}
$p$ & $q$ & $p \wedge q$ & $p \vee q$ & $p \simpl q$ & $p \sequi q$\\
\hline
0 & 0 & 0 & 0 & 1 & 1\\   
0 & 1 & 0 & 1 & 1 & 0\\
1 & 0 & 0 & 1 & 0 & 0\\ 
1 & 1 & 0 & 1 & 1 & 1
\end{tabular}
\end{center}

\end{example}
