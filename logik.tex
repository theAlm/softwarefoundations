\special{pdf: out 2 << /Title 
(Einige (wenige) Grundlagen der elementaren Logik) 
/Dest [ @thispage /FitH @ypos ] >>}
\section{Einige (wenige) Grundlagen der elementaren Logik}
\label{BasisLogik}
Aussagen sind entweder \dindex{wahr} ($\triangleq 1$) oder
\dindex{falsch} ($\triangleq 0$). So sind die Aussagen 
\begin{center}
"`Wiesbaden liegt am Mittelmeer"' und "`$1 = 7$"'
\end{center}
sicherlich falsch, wogegen die Aussagen
\begin{center}
"`Wiesbaden liegt in Hessen"' und "`$11 = 11$"'
\end{center}
sicherlich wahr sind. Aussagen werden meist durch
\dindex{Aussagenvariablen} formalisiert, die nur die Werte $0$ oder $1$
annehmen können. Oft verwendet man auch eine oder mehrere Unbekannte,
  um eine Aussage zu parametrisieren. So könnte "`$P(x)$"' etwa für
"`Wiesbaden liegt im Bundesland $x$"' stehen,
d.h.~"`$P(\text{Hessen})$"' wäre wahr, wogegen "`$P(\text{Bayern})$"'
eine falsche Aussage ist. Solche Konstrukte mit Parameter nennt man auch
\dindex{Prädikat} oder \dindex{Aussageformen}.

Um die Verknüpfung von Aussagen auch formal aufschreiben zu können,
werden die folgenden logischen
Operatoren\index{Operator!logisch}\index{logischer Operator} 
verwendet

\begin{center}
\begin{tabular}{c|l|l}
Symbol & umgangssprachlicher Name & Name in der Logik\\
\hline
\dindex{$\sand$} & und & Konjunktion\\
\dindex{$\sor$} & oder & Disjunktion / Alternative\\
\dindex{$\sneg$} & nicht & Negation \\
\dindex{$\simpl$} & folgt & Implikation\\
\dindex{$\sequi$} & genau dann wenn (\emph{\gdw}\index{gdw=\gdw}) & Äquivalenz\\
\end{tabular}
\end{center}
Zusätzlich werden noch die Quantoren \dindex{$\exists$} ("`es existiert"') und
\dindex{$\forall$} ("`für alle"') verwendet, die z.B.~wie folgt gebraucht
werden können
\begin{description}
%
\item $\forall x \colon P(x)$ bedeutet "`Für alle $x$ gilt die Aussage $P(x)$. 
%
\item $\exists x \colon P(x)$ bedeutet "`Es existiert ein $x$, für das die Aussage
  $P(x)$ gilt.
%
\end{description}
Üblicherweise läßt man sogar den Doppelpunkt weg und schreibt statt $\forall
x \colon P(x)$ vereinfachend $\forall x P(x)$.

\begin{example}
Die Aussage "`Jede gerade natürliche Zahl kann als Produkt von $2$ und einer
anderen natürlichen Zahl geschrieben werden"' lässt sich dann wie
folgt schreiben
\begin{displaymath}
\forall n \in \N \colon ((n \text{ ist gerade}) \simpl (\exists m
\in \N \colon n = 2 \cdot m)) 
\end{displaymath}
Die folgende logische Formel wird wahr \gdw $n$ eine ungerade
natürliche Zahl ist.
\begin{displaymath}
\exists m \in \N \colon (n = 2 \cdot m + 1)
\end{displaymath}
\end{example}
Für die logischen Konnektoren sind die folgenden Wahrheitswertetafeln
festgelegt:

\begin{center}
\begin{tabular}{c||c}
$p$ & $\neg p$\\
\hline
$0$ & $1$\\
$1$ & $0$
\end{tabular}
\hspace*{5em}
und
\hspace*{5em}
\begin{tabular}{c|c||c|c|c|c}
$p$ & $q$ & $p \wedge q$ & $p \vee q$ & $p \simpl q$ & $p \sequi q$\\
\hline
0 & 0 & 0 & 0 & 1 & 1\\   
0 & 1 & 0 & 1 & 1 & 0\\
1 & 0 & 0 & 1 & 0 & 0\\ 
1 & 1 & 1 & 1 & 1 & 1
\end{tabular}
\end{center}
\goodbreak
\noindent Jetzt kann man Aussagen auch etwas komplexer verknüpfen:
\begin{example}
Nun wird der $\sand$-Operator verwendet werden. Dazu soll die Aussage 
"`Für alle natürlichen Zahlen $n$ und $m$ gilt, wenn $n$ kleiner
gleich $m$ und $m$ kleiner gleich $n$ gilt, dann ist $m$ gleich $n$"'
\begin{displaymath}
\forall n,m \in \N\, (((n \le m) \sand (m \le n)) \simpl (n = m))
\end{displaymath}
\end{example}
Oft benutzt man noch den negierten Quantor \dindex{$\nexists$} ("`es existiert kein"').

\setlength{\marginparwidth}{1.7cm}
\marginpar{
\flushleft\sffamily\tiny
Cubum autem in duos cubos, aut quadrato-quadratum in duos
quadrato-quadratos, et generaliter nullam in infinitum ultra quadratum
potestatem in duos eiusdem nominis fas est dividere cuius rei
demonstrationem mirabilem sane detexi. Hanc marginis exiguitas non
caperet.}
\begin{example}["`Großer Satz von Fermat"']
Die Richtigkeit dieser Aussage konnte erst $1994$ nach mehr als $350$
Jahren von Andrew Wiles und Richard Taylor gezeigt werden:
\begin{displaymath}
\forall n\in\N\, \nexists a,b,c \in \N\, (((n > 2) \wedge
(a \cdot b \cdot c \not= 0)) \simpl a^n + b^n
= c^n)  
\end{displaymath}
Für den Fall $n=2$ hat die Gleichung $a^n + b^n
= c^n$ unendlich viele ganzzahlige Lösungen (die so genannten
Pythagoräische Zahlentripel) wie z.B.~$3^2+4^2=5^2$. Diese sind seit 
mehr als $3500$ Jahren bekannt und haben z.B.~geholfen die
Cheops-Pyramide zu bauen.
\end{example}

